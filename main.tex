%%%%%%%%%%%%%%%%%%%%%%%%%%%%%%%%%%%%%%%%%%%%%%%%%%%%%%%%%%%%
%%% LaPreprint: PREPRINT TEMPLATE
%%%%%%%%%%%%%%%%%%%%%%%%%%%%%%%%%%%%%%%%%%%%%%%%%%%%%%%%%%%%

% Here I could talk about what one should do in this document.
% Instead I'll refer you to the explore on your own and check the Github Repo. :-)
% Line spacing is 1.2 by default (can't be smaller).

%%%%%%%%%%%%%%%%%%%%%%%%%%%%%%%%%%%%%%%%%%%%%%%%%%%%%%%%%%%%
%%% PREAMBLE
%%%%%%%%%%%%%%%%%%%%%%%%%%%%%%%%%%%%%%%%%%%%%%%%%%%%%%%%%%%%

% Declare document class
\documentclass[9pt,Preprint,secnum,spanish]{lapreprint}
% Choose between "biorxiv", "medrxiv", "arxiv" and "chemrxiv". Otherwise defaults "Preprint".
% Choose between "blue" and "red" colour scheme. Defaults to "blue".
% Use the "onehalfspacing" option for 1.5 line spacing.
% Use the "doublespacing" option for 2.0 line spacing.
% Use the "lineno" option for line numbers.
% Use the "endfloat" option to place floats after the bibliography.
% Use the "secnum" option to have include numbers.
% Note: secnumdepth is now set to 3 in lapreprint.cls when secnum option is used

% Set tocdepth to include subsubsections in table of contents
\setcounter{tocdepth}{3}

% Import packages
% \usepackage{lipsum}     % Required to insert dummy text

% Spanish language configuration - use caption package to override babel defaults
\usepackage[figurename=Figura,tablename=Tabla]{caption}

% Configurar español como idioma principal ANTES de que la clase cargue babel
% La clase lapreprint.cls carga \RequirePackage[english]{babel} internamente
% \PassOptionsToPackage pasa las opciones ANTES de que se ejecute \RequirePackage
% Esto es el método oficial recomendado según babel v25.4 manual, Sección 2.3
\PassOptionsToPackage{main=spanish,english}{babel}

\usepackage{lastpage}   % For LastPage reference
\usepackage[version=4]{mhchem} % For chemical notation
\usepackage{siunitx}    % For SI units
\usepackage{pdflscape}  % For putting pages in landscape mode
\usepackage{rotating}   % For rotating specific elements
\usepackage{textgreek}  % Greek symbols
\usepackage{gensymb}    % Symbols
\usepackage[misc]{ifsym} % For the \Letter symbol
\usepackage{orcidlink}  % For the \orcidlink
\usepackage{listings}   % For inserting code chunks

\usepackage{colortbl}   % For Knitr table colouring
\usepackage{tabularx}   % For making Knitr tables compatible
\usepackage{longtable}  % For multi-page tables
\usepackage{booktabs}   % For professional tables
\usepackage{subcaption}
\usepackage{multirow}
\usepackage{snotez}     % For sidenote environments. enotez for endnotes
\usepackage{csquotes}   % For language-based quote rules (helps BiBLaTeX)
\usepackage{annotate-equations}
\usepackage[most]{tcolorbox} % For observation/note boxes

% Theorem environments (AMS standard for rigorous mathematics)
% Numeración jerárquica: n.x.y (sección.subsección.teorema)
% Todos los entornos comparten el contador 'theorem' y se numeran por subsección
\usepackage{amsthm}
\theoremstyle{plain}
\newtheorem{theorem}{Teorema}[subsection]
\newtheorem{proposition}[theorem]{Proposición}
\newtheorem{lemma}[theorem]{Lema}
\newtheorem{corollary}[theorem]{Corolario}
\newtheorem{conjecture}[theorem]{Conjetura}

\theoremstyle{definition}
\newtheorem{definition}[theorem]{Definición}
\newtheorem{axiom}[theorem]{Axioma}
\newtheorem{construction}[theorem]{Construcción}
\newtheorem{observation}[theorem]{Observación}
\newtheorem{example}[theorem]{Ejemplo}

\theoremstyle{remark}
\newtheorem{remark}[theorem]{Nota}

% Asegurar que el contador de teoremas se resetee correctamente al cambiar de subsección
% Esto es necesario porque amsthm con [subsection] debería hacerlo automáticamente,
% pero puede haber problemas si hay múltiples archivos o si el contador no se inicializa correctamente
\makeatletter
\@addtoreset{theorem}{subsection}
\makeatother

% Asegurar que el entorno proof muestre "Demostración" en español
% (babel debería hacerlo automáticamente, pero lo redefinimos por si acaso)
\renewcommand{\proofname}{Demostración}

% Define custom environments for observations and conventions
\newtcolorbox{observacion}[1][]{
  colback=blue!5!white,
  colframe=blue!75!black,
  title=Observación,
  #1
}

\newtcolorbox{convencion}[1][]{
  colback=green!5!white,
  colframe=green!75!black,
  title=Convención de Notación,
  #1
}

\newtcolorbox{nota}[1][]{
  colback=orange!5!white,
  colframe=orange!75!black,
  title=Nota,
  #1
}

% Make declarations
\DeclareSIUnit\Molar{M}

% Macros de notación específicos del documento
% Solo para notación compleja o específica que se repite frecuentemente
\newcommand{\omegapcf}{\Omega_{\text{PCF}}}    % Operador PCF (notación específica del documento)
\newcommand{\omegahat}{\hat{\Omega}}           % Matriz generadora (útil si se usa frecuentemente)

% Macros útiles para referencias cruzadas (sugerencia de Perplexity)
% Facilitan referencias consistentes a construcciones matemáticas
\newcommand{\tref}[1]{Teorema~\ref{#1}}        % Referencia a teorema
\newcommand{\dref}[1]{Definición~\ref{#1}}     % Referencia a definición
\newcommand{\pref}[1]{Proposición~\ref{#1}}    % Referencia a proposición
\newcommand{\lref}[1]{Lema~\ref{#1}}           % Referencia a lema
\newcommand{\cref}[1]{Construcción~\ref{#1}}   % Referencia a construcción
\newcommand{\oref}[1]{Observación~\ref{#1}}    % Referencia a observación
\newcommand{\conjref}[1]{Conjetura~\ref{#1}}   % Referencia a conjetura
\newcommand{\corref}[1]{Corolario~\ref{#1}}    % Referencia a corolario

%%%%%%%%%%%%%%%%%%%%%%%%%%%%%%%%%%%%%%%%%%%%%%%%%%%%%%%%%%%%
%%% CONFIGURACIÓN GLOBAL DE ESPACIADO
%%%%%%%%%%%%%%%%%%%%%%%%%%%%%%%%%%%%%%%%%%%%%%%%%%%%%%%%%%%%

% Line height global (ajuste fino sobre el default de lapreprint)
\usepackage{setspace}
\setstretch{1.15}

% Espacio entre párrafos
\setlength{\parskip}{0.5\baselineskip plus 2pt minus 1pt}% chktex 1

% Espacio alrededor de ecuaciones display
\setlength{\abovedisplayskip}{12pt plus 3pt minus 6pt}
\setlength{\belowdisplayskip}{12pt plus 3pt minus 6pt}

% Espacio entre líneas en align/gather
\setlength{\jot}{8pt}

% Espacio en títulos de sección
% Note: titlesec is already loaded in lapreprint.cls, but loading it again here is safe
% (LaTeX ignores duplicate package loads). This allows us to use \titlespacing here.
\usepackage{titlesec}
\titlespacing*{\section}{0pt}{2\baselineskip}{1\baselineskip}
\titlespacing*{\subsection}{0pt}{1.5\baselineskip}{0.75\baselineskip}
\titlespacing*{\subsubsection}{0pt}{1.25\baselineskip}{0.5\baselineskip}

% Espacio en teoremas/proposiciones/lemas/definiciones
\usepackage{etoolbox}
\AtBeginEnvironment{theorem}{\vspace{10pt}}
\AtEndEnvironment{theorem}{\vspace{10pt}}
\AtBeginEnvironment{proposition}{\vspace{10pt}}
\AtEndEnvironment{proposition}{\vspace{10pt}}
\AtBeginEnvironment{lemma}{\vspace{10pt}}
\AtEndEnvironment{lemma}{\vspace{10pt}}
\AtBeginEnvironment{definition}{\vspace{10pt}}
\AtEndEnvironment{definition}{\vspace{10pt}}
\AtBeginEnvironment{corollary}{\vspace{10pt}}
\AtEndEnvironment{corollary}{\vspace{10pt}}

% Espacio en listas
\usepackage{enumitem}
\setlist{topsep=6pt, itemsep=4pt, parsep=0pt}

% Espacio en tablas
\renewcommand{\arraystretch}{1.3}

% Prevención de líneas huérfanas y viudas (tipografía profesional)
% Evita líneas aisladas al inicio/final de página
\widowpenalty=10000
\clubpenalty=10000
% Penalización adicional para evitar líneas con una sola palabra al final de párrafo
\displaywidowpenalty=10000

% Configuración avanzada de microtype para mejor ajuste de tipos y prevención de huérfanas
% microtype ya está cargado en lapreprint.cls, aquí solo ajustamos opciones
\microtypesetup{
  tracking=true,           % Ajuste fino de espaciado entre letras
  spacing=true,            % Ajuste fino de espaciado entre palabras
  factor=1100,             % Factor de estiramiento (1100 = 10% más flexible)
  stretch=20,              % Estiramiento máximo permitido
  shrink=20,               % Contracción máxima permitida
  step=2                   % Granularidad del ajuste
}

% Please note that these options may affect formatting.

%%%%%%%%%%%%%%%%%%%%%%%%%%%%%%%%%%%%%%%%%%%%%%%%%%%%%%%%%%%%
%%% BIBLIOGRAPHY
%%%%%%%%%%%%%%%%%%%%%%%%%%%%%%%%%%%%%%%%%%%%%%%%%%%%%%%%%%%%
\usepackage[			% use biblatex for bibliography
	backend=biber,      % use biber or bibtex backend
    style=authoryear,   % choose style
	natbib=true,		% allow natbib commands
	hyperref=true,	    % activate hyperref support
	alldates=year,      % only show year (not month)
    uniquename=false,   % don't add firstnames when citing multiple sources by the same author
    maxbibnames=99,     % maximum number of author names to list in bibliography before 'et al' is used instead
]{biblatex}

% Just avoiding some rogue fields that cause issues with certain styles
\AtEveryBibitem{
    \clearfield{urlyear}
    \clearfield{urlmonth}
    \clearlist{language}
}

% Update to your bibliography file
\addbibresource{src/bibliography.bib}

%%%%%%%%%%%%%%%%%%%%%%%%%%%%%%%%%%%%%%%%%%%%%%%%%%%%%%%%%%%%
%%% CONFIGURACIÓN DE IDIOMA PARA REFERENCIAS AUTOMÁTICAS
%%%%%%%%%%%%%%%%%%%%%%%%%%%%%%%%%%%%%%%%%%%%%%%%%%%%%%%%%%%%
% La clase lapreprint.cls carga babel con english, por lo que necesitamos
% redefinir manualmente los nombres de autoref en español.
% Estos comandos se ejecutan después de que se cargue hyperref (en lapreprint.cls)
\AtBeginDocument{%
  % Asegurar que español esté seleccionado para hyphenation correcta
  % Con \PassOptionsToPackage{main=spanish,english}{babel}, español ya es el idioma principal
  % pero lo mantenemos explícitamente por seguridad
  \selectlanguage{spanish}%
  % Configurar formato numérico estándar internacional (punto decimal) para paper académico
  % Esto sobrescribe el formato español (coma decimal) para usar formato estándar internacional
  \spanishdecimal{.}%  % Usar punto como separador decimal en lugar de coma
  \sisetup{output-decimal-marker={.}}%  % Configurar siunitx para usar punto decimal
  % Redefinir nombres de autoref en español
  \renewcommand{\sectionautorefname}{Sección}%
  \renewcommand{\subsectionautorefname}{Subsección}%
  \renewcommand{\subsubsectionautorefname}{Subsubsección}%
  \renewcommand{\paragraphautorefname}{Párrafo}%
  \renewcommand{\figureautorefname}{Figura}%
  \renewcommand{\tableautorefname}{Tabla}%
  \renewcommand{\equationautorefname}{Ecuación}%
  \renewcommand{\appendixautorefname}{Apéndice}%
  % Redefinir el título del apéndice en español
  \renewcommand{\appendixname}{Apéndice}%
}

% Redefinir el entorno appendixbox para usar "Apéndice" en lugar de "Appendix"
% Necesitamos \makeatletter para acceder a comandos internos como \if@reqslineno
\makeatletter
\renewenvironment{appendixbox}{%
  \setcounter{figure}{0}
  \setcounter{table}{0}
  \refstepcounter{appendix}%
  \clearpage%
  \patchcmd{\ttlf@section}{MediumGrey}{darkColour}{}{}
  \noindent{\bfseries\Large\color{MediumGrey}Apéndice \arabic{appendix}\par}
  \nolinenumbers%
  \begin{mdframed}[hidealllines=true,backgroundcolor=lightColour!10,leftline=true,linecolor=lightColour,linewidth=1em]
  \if@reqslineno\addtolength{\linenumbersep}{2em}\internallinenumbers\fi
}{%
  \vspace{1em}%
  \end{mdframed}
}
\makeatother

%%%%%%%%%%%%%%%%%%%%%%%%%%%%%%%%%%%%%%%%%%%%%%%%%%%%%%%%%%%%
%%% ARTICLE SETUP
%%%%%%%%%%%%%%%%%%%%%%%%%%%%%%%%%%%%%%%%%%%%%%%%%%%%%%%%%%%%

% Paper title
\title{El Operador $\omegapcf$ y La Estructura Primitiva del Plano Complejo: \\[0.618em]
       \large De Mersenne a Riemann mediante acoplamiento geométrico $\varphi$-$i$-$S_3$}

% Authors - you can use \orcidlink{} and \authfn{} - see contribution note
\author[1 \textnormal{\Letter}]{Jorge Armando González García}
\author[ \orcidlink{0009-0009-0935-2954} 1]{Víctor Manuel González García}% chktex 8
\author[2]{Itzel Marion Dressler Pérez}
\author[1]{Luz María García Ordóñez}
\author[2]{Pablo Tenorio}
\author[2]{Mario Moreno}

% Affiliations
\affil[1]{TTAMAYO PUNTO COM, S.A.P.I. de C.V., Mexico}
\affil[2]{Independent Researcher}

% Other metadata. Feel free to add your own
\metadata[]{\textnormal{\Letter}\hspace{.5ex} For correspondence}{\url{https://github.com/omega-pcf/01-omega-phi-primitive-complex-plane/issues} (JAGG)}
\metadata[]{Data availability}{Computational verifications available in supplementary material. Source code at \href{https://github.com/omega-pcf/01-omega-phi-primitive-complex-plane}{GitHub repository}.}
\metadata[]{Competing interests}{The author declares no competing interests.}


% Surname of the lead author(s) for the running footer
\leadauthor{González García}
\shorttitle{El Operador Omega PCF y La Estructura Primitiva del Plano Complejo}

%%%%%%%%%%%%%%%%%%%%%%%%%%%%%%%%%%%%%%%%%%%%%%%%%%%%%%%%%%%%
%%% ARTICLE START
%%%%%%%%%%%%%%%%%%%%%%%%%%%%%%%%%%%%%%%%%%%%%%%%%%%%%%%%%%%%

\begin{document}
\maketitle
\begin{abstract}

Mediante principios \textit{bootstrap} de coherencia multi-dominio, trascendemos problemas de autorreferencia tipo Lawvere-Yanofsky que han limitado varios intentos previos de construcción del operador Hilbert-Pólya. En lugar de extender $\mathbb{C}$ mediante álgebra (que enfrenta restricciones del teorema de Frobenius), desarrollamos modularización geométrica que preserva todas las propiedades algebraicas de $\mathbb{C}$ mientras revela estructura toroidal subyacente\sidenote{Mediante lattice $\Lambda_{\text{PCF}}$ y módulo $M_{\text{PCF}} = \mathbb{C}/\Lambda_{\text{PCF}} \cong T^2$, a diferencia de extensiones algebraicas clásicas que añaden nuevos elementos y pueden perder propiedades fundamentales (e.g., octoniones $\mathbb{O}$ pierden asociatividad). Nuestra modularización reorganiza sin añadir, preservando todas las propiedades de $\mathbb{C}$ mediante acoplamiento geométrico $\varphi$-$i$-$S_3$.\label{note:octonions}}. La construcción parte de estructura tripartita $(P,C,F)$ tipo Eisenstein en $\mathbb{C}^3$ mediante simetría $S_3$, genera lattice rectangular tipo Gauss mediante acoplamiento $\varphi$, y produce operador hermítico con espectro real mediante kernel modular sobre el toro. La matriz generadora $\hat{\Omega}$ en $\mathbb{C}^3$ es normal pero no hermítica\sidenote{La matriz $\hat{\Omega}$ opera en espacio de componentes $\mathbb{C}^3$ codificando direccionalidad de la estructura tripartita $(P, C, F)$ mediante simetría $S_3$ del triángulo equilátero; su no-hermiticidad refleja geometría del sistema, no defecto algebraico.}, mientras que la hermiticidad del operador integral en $L^2(\mathbb{R})$ emerge del mecanismo de construcción mediante kernel simetrizado\sidenote{El kernel se construye mediante términos $\delta(x-y) + \varepsilon(x,y)$ que introducen simetrización, permitiendo que hermiticidad emerja aunque $\hat{\Omega}$ no sea hermítica. Ver §\ref{subsubsec:emergencia-hermiticidad}.}, no de propiedades algebraicas de $\hat{\Omega}$. El acoplamiento $\varphi$-$i$-$S_3$ genera esta hermiticidad emergente con magnitud constante $|\Omega| = 1/2$, estableciendo correspondencia entre escalas autosimilares del plano complejo\sidenote{El módulo constante $|\Omega| = 1/2$ actúa como punto fijo funcional que ancla toda la construcción mediante auto-referencia distribuida en estructura tripartita $P \leftrightarrow C \leftrightarrow F$ que evita ciclos prohibidos $D_1 \to D_2 \to D_1$ identificados por Lawvere y Yanofsky. Esta estrategia evita autorreferencia mediante coherencia multi-dominio con invariantes preservados, formalizada en bootstrap conforme por Guillarmou \textit{et al.} y en bootstrap modular por Benjamin-Chang. Ver §\ref{subsec:simetrias-dualidades} y §\ref{sec:obstaculos}.\label{note:distributed-reference}}.

El análisis del operador revela dos correspondencias estructurales fundamentales. Primero, isomorfismo logarítmico entre torre áurea continua $R_\sigma = 3\varphi^\sigma$ y torre Mersenne discreta $M_p = 2^p-1$ mediante factor de conversión $\lambda = \ln(2)/\ln(\varphi) \approx 1.440$, donde ambas torres son rectas en espacio logarítmico con pendientes relacionadas---correspondencia topológica (preserva estructura exponencial) no métrica, verificada sobre más de 25 millones de órdenes de magnitud desde $M_2$ hasta $M_{82589933}$, mediada por módulo crítico $|\Omega| = 1/2 = 2^{-1}$ que establece el único puente posible entre escalamiento áureo y binario. Segundo, predicción espectral de ceros de $\zeta(s)$ mediante fórmula $\lambda_n = K_\sigma \sqrt{t_n}$ donde $K_\sigma = M_{\text{PCF}}/\varphi^\sigma$, con precisión que mejora asintóticamente conforme aumenta altura $t$ (discrepancias $< 10^{-14}$ en primeros 100 ceros, verificada hasta $n \sim 10^{10}$ con altura $t \sim 8.3 \times 10^{23}$). El operador se construye independientemente de $\zeta(s)$; su espectro exhibe correlación estructural con ceros \textit{a posteriori}.

La verificación numérica confirma robustez estructural del operador: discrepancias observadas reflejan límites de precisión computacional (véase \ref{def:precision-computacional}), no deficiencia matemática. El operador mantiene integridad incluso al manipular números enteros de magnitud extrema (primos de Mersenne con millones de dígitos), preservando invariantes bajo la acción del acoplamiento $\varphi$-$i$-$S_3$ a través de todas las escalas autosimilares. Esta robustez—donde discreto y continuo coexisten coherentemente en el espectro—demuestra que el operador captura invariantes matemáticos fundamentales de $\mathbb{C}$.

\textbf{Keywords:} Riemann hypothesis, Hilbert-Pólya conjecture, L-functions, Self-adjoint operators, Random matrix theory, Zeta function zeros, Mersenne primes, Modular spaces.

\end{abstract}
\section{Introducción}

\subsection{La Conjetura de Hilbert-Pólya y el Isomorfismo de Montgomery-Dyson-Odlyzko}

Las conjeturas atribuidas a Hilbert y Pólya\sidenote{\cite{Polya1926}} postularon a principios del siglo XX que la Hipótesis de Riemann podría abordarse mediante traducción entre dominios, utilizando los autovalores de un operador hermítico para atacar un problema de teoría de números. Esta conjetura permaneció como especulación teórica hasta que Montgomery identificó que las correlaciones de pares entre ceros consecutivos seguían una distribución específica\sidenote{\cite{Montgomery1973}}. Dyson reconoció esta función como idéntica a la del Gaussian Unitary Ensemble (GUE) de matrices hermíticas aleatorias\sidenote{\cite{Dyson1962}}, estableciendo una conexión inesperada entre teoría analítica de números y física estadística. Odlyzko verificó computacionalmente esta correspondencia mediante el cálculo de más de $10^{13}$ ceros con precisión sin precedentes\sidenote{\cite{Odlyzko1987}}, mientras que trabajos posteriores extendieron estas verificaciones y exploraron sus implicaciones teóricas\sidenote{\cite{Odlyzko1989}}.

El isomorfismo estadístico establece que las funciones de correlación de pares de ceros sucesivos $\{t_n\}$ en la línea crítica $\text{Re}(s) = 1/2$ satisfacen:
\[
R_2(s) = 1 - \left[\frac{\sin({\pi s})}{{\pi s}}\right]^2 % chktex 3
\]
idéntica a la distribución de eigenvalores en GUE\@. Sin embargo, este resultado, aunque proporciona información estadística sobre el conjunto de ceros, no establece correspondencias determinísticas para ceros individuales.

\subsection{Obstáculos Históricos y Limitaciones Estructurales}\label{sec:obstaculos}

Pese a la validación empírica del isomorfismo Montgomery-Dyson-Odlyzko, consideramos que la construcción explícita del operador Hilbert-Pólya ha enfrentado dos obstáculos fundamentales que han persistido a través de décadas de intentos\@.

\textbf{Obstáculo I: Autorreferencia}% chktex 13

Diversos intentos exhiben una estructura circular característica. Se parte del conocimiento del espectro deseado:
\[
\text{spec}(H) = \{t_n : \zeta(1/2+it_n)=0\}
\]
se utiliza esta información para construir el operador $H$, luego se diagonaliza $H$ para obtener sus eigenvalores, y finalmente se verifica que estos eigenvalores coinciden con $\text{spec}(H)$ original. Este ciclo presupone conocimiento \textit{a priori} de aquello que pretende descubrir.

La geometría no conmutativa de Connes\sidenote{\cite{Connes2000}} requiere información \textit{a priori} sobre los ceros para definir el espacio de fases donde el operador actuaría. El operador $H = xp$ de Berry-Keating\sidenote{\cite{BerryKeating1999}} necesita regularización espectro-dependiente. Las simetrías PT de Bender-Brody-Müller\sidenote{\cite{BenderBrodyMuller2002}} requieren ajustar parámetros mediante conocimiento previo del espectro buscado. Los trabajos recientes de Yakaboylu\sidenote{\cite{Yakaboylu2022,Yakaboylu2024}} continúan limitados por condiciones de confinamiento o de frontera que presuponen información sobre los ceros\@.

\textbf{Obstáculo II: Degradación Asintótica}% chktex 13

Aun cuando se evite la autorreferencia directa, otros operadores propuestos exhiben limitaciones predictivas sistemáticas. Los métodos basados en aproximaciones locales predicen con precisión decreciente conforme la altura $t$ aumenta, teniendo como efecto que lo que funciona para los primeros ceros, falla para $n > 10^6$. Las periodicidades artificiales emergen en construcciones que no capturan la estructura cuasi-periódica genuina de $\zeta(s)$. Esta degradación asintótica sugiere que las aproximaciones no acceden a la geometría fundamental subyacente.

Entre las aproximaciones más significativas que implican períodos se encuentran las de Vinogradov\sidenote{\cite{Vinogradov1958}} en 1958 que, paralela y simultáneamente que Korobov\sidenote{\cite{Korobov1958}} (bajo ideas previas de Vinogradov), estableció regiones libres de ceros que restringen dónde pueden estar los ceros no triviales:
\[
\zeta(\sigma + it) \neq 0 \quad \text{para} \quad \sigma \geq 1 - \frac{C}{(\log{|t|})^{2/3}(\log\log{|t|})^{1/3}} % chktex 3
\]

Esta restricción implica que cualquier operador propuesto debe predecir ceros sólo en la región permitida. Los métodos que predicen con precisión decreciente para $t$, violan implícitamente estas cotas\@.

\subsection{Traducción entre Dominios: Perspectiva Histórica}% chktex 13

Pese a los obstáculos del programa Hilbert-Pólya, la traducción entre dominios ha persistido como aproximación fértil. Manin\sidenote{\cite{Manin2013}}, en su artículo \textit{Numbers as Functions}, compila las principales líneas convergentes, revelando una arquitectura común: estructuras locales (p-ádico, $\mathbb{F}_1$, ciclos) que ascienden a propiedades globales mediante invariantes preservadas:

Buium\sidenote{\cite{Buium1995}} construyó ecuaciones diferenciales p-ádicas mediante el cociente de Fermat $\delta_p(a) = (a^p - a)/p$, extendiendo la analogía clásica entre números y funciones a espacios jet aritméticos. Los representantes de Teichmüller (raíces $p$-ésimas de unidad) juegan el rol de constantes que, en ausencia de uniformización, complican la teoría más allá del caso clásico.

Borger\sidenote{\cite{Borger2009}} estableció que las lambda-estructuras---codificando sistemas coherentes de levantamientos de Frobenius---son datos de descenso sobre el campo con un elemento $\mathbb{F}_1$. Tanto la construcción p-ádica de Buium como el enfoque de Borger enfrentan un obstáculo estructural común: la asimetría del primo arquimediano impide traducción completa entre la geometría finita y la infinita.

Kontsevich-Zagier\sidenote{\cite{Kontsevich2001}} definieron el anillo de períodos $P \subset \mathbb{C}$ como valores de integrales $\int_\gamma \omega$ con datos algebraicos sobre $\mathbb{Q}$, estableciendo cómo estructuras topológicas (ciclos geométricos) se traducen a objetos analíticos (integrales) que a su vez satisfacen ecuaciones algebraicas (ecuaciones de Picard-Fuchs). Esta traducción tripartita permite que propiedades geométricas se expresen analíticamente y luego se codifiquen algebraicamente. Aunque incluye $\pi$, $\log(2)$ y valores zeta múltiples, permanece abierto---contra intuición inicial---si $1/\pi$, $e$ o la constante de Euler $\gamma$ son períodos, ni siquiera períodos exponenciales.

Complementando la compilación de Manin, la correspondencia estadística establecida por Montgomery y Dyson---espaciamientos entre ceros de $\zeta(s)$ siguen distribución GUE de matrices aleatorias---y verificada computacionalmente por Odlyzko hasta $10^{13}$ ceros (véase §\ref{sec:obstaculos}), establece una traducción entre teoría de números y física cuántica mediante operadores hermíticos. El obstáculo central permanecía irresoluto: ningún operador hermítico explícito había sido construido, dejando la conjetura de Hilbert-Pólya como principio heurístico más que construcción matemática.

\subsection{Simetrías y Dualidades como Diccionarios Universales}\label{subsec:simetrias-dualidades}

El análisis de estas aproximaciones revela, más allá de un tipo de matemática específica, simetrías y dualidades como fundamento común. Este principio ha demostrado éxito en múltiples contextos matemáticos y físicos, más allá de la Hipótesis de Riemann.

Múltiples construcciones matemáticas y físicas ejemplifican este principio. La transformada de Fourier establece correspondencia entre espacio de posición y espacio de momento, preservando la norma $L^2(\mathbb{R})$: $\|f\|_2 = \|\hat{f}\|_2$. La dualidad AdS/CFT\sidenote{\cite{Maldacena1998}} constituye una equivalencia completa entre teoría de gravedad en $d+1$ dimensiones y teoría de campos conforme en $d$ dimensiones, preservando funciones de partición $Z_{\text{CFT}}[J] = Z_{\text{gravity}}[\varphi_0=J]$. Esta dualidad fuerte-débil permite traducir problemas intratables en un dominio a problemas tratables en el otro.

En el contexto del \textit{bootstrap} modular escalar, Benjamin y Chang\sidenote{\cite{Benjamin2022}} demostraron que ecuaciones de cruce en CFT 2D contienen información sobre todos los ceros de $\zeta(s)$, reformulando la Hipótesis de Riemann como afirmación sobre densidad de operadores. Desde una perspectiva unificadora, Baez y Stay\sidenote{\cite{Baez2009}} mostraron que física cuántica, topología, lógica y computación comparten estructura de categorías monoidales simétricas cerradas, permitiendo traducción entre dominios mediante funtores naturales.

\textbf{Evitando Auto-Referencia}

Yanofsky\sidenote{\cite{Yanofsky2003}}, siguiendo a Lawvere\sidenote{\cite{Lawvere1969}}, formalizó que paradojas auto-referenciales emergen de argumentos diagonales donde sistemas intentan describir sus propias propiedades. Las traducciones exitosas evitan esto mediante distribución de información entre múltiples dominios con invariantes preservados. La estructura circular $D_1 \to D_2 \to D_1$ genera auto-referencia, mientras que coherencia multi-dominio $D_1 \leftrightarrow D_2 \leftrightarrow D_3 \leftrightarrow D_4$ establece determinación mutua sin ciclos directos. En síntesis, §\ref{subsec:simetrias-dualidades} establece que traducciones exitosas entre dominios evitan autorreferencia mediante distribución de información entre múltiples dominios con invariantes preservados.

\subsection{El Dilema Fundamental: La Consecuencia Algebraica de la Dimensionalidad}

El punto de partida del programa espectral para la Hipótesis de Riemann reside en una tensión irresoluble entre la necesidad lógica y la restricción algebraica.

La necesidad de evitar la circularidad funcional (la Paradoja de la Autorreferencia) impone un requisito de dimensionalidad mínima. La teoría de puntos fijos para sistemas recursivos establece que para romper la tautología (por ejemplo, ``Esta oración es falsa'') se requiere un marco con una estructura tripartita mínima (P-C-F) que no puede resolverse en un espacio binario o bidimensional sin auto-contradicción\sidenote{\cite{Yanofsky2003}}. Esto implica que la construcción del operador debe operar en un espacio con una dimensionalidad efectiva de tres o más.

Sin embargo, al intentar construir esta dimensionalidad mediante extensiones algebraicas del plano complejo ($\mathbb{C}$), se entra en conflicto con el Teorema de Frobenius\sidenote{\cite{Frobenius1877}}. Este teorema establece que las únicas álgebras de división reales, asociativas y de dimensión finita son los números reales ($\mathbb{R}$), los complejos ($\mathbb{C}$) y los cuaterniones ($\mathbb{H}$):

\begin{itemize}
\item La extensión a $\mathbb{H}$ (dim 4) sacrifica la conmutatividad ($ij \neq ji$).
\item La extensión a los Octoniones ($\mathbb{O}$, dim 8, conforme al teorema de Hurwitz-Zorn\sidenote{\cite{Hurwitz1923,Zorn1933}}) sacrifica adicionalmente la asociatividad ($(a \cdot b) \cdot c \neq a \cdot (b \cdot c)$).
\end{itemize}

La teoría espectral clásica requiere tanto la conmutatividad como la asociatividad para garantizar la diagonalización de los operadores y, por ende, que el espectro (los autovalores) sea puramente real y discreto. La extensión de la teoría de operadores a álgebras no conmutativas o no asociativas (el contexto de las Álgebras de Von Neumann o $C^*$-álgebras\sidenote{\cite{Dixmier1981,Arveson2002}}) es sustancialmente más compleja y no proporciona el resultado de espectro real que requiere la solución espectral a la Hipótesis de Riemann. Por lo tanto, cualquier extensión algebraica para ganar dimensionalidad implica un sacrificio de las propiedades esenciales para la coherencia espectral.

\subsubsection{El Hamiltoniano como Mecanismo de Conversión \texorpdfstring{$\mathbb{C} \to \mathbb{R}$}{C a R}}\label{subsubsec:hamiltoniano-conversion-C-R}

El Hamiltoniano es el Generador Infinitesimal de la Evolución Temporal\sidenote{\cite{Shankar1994}}. Formalmente, $H$ es el elemento del Álgebra de Lie que genera el grupo unitario de la evolución $U(t)$:
\[
U(t) = e^{-iHt/\hbar}
\]

La acción de $H$ sobre el vector de estado $\psi \in \mathcal{H}$ (el espacio de Hilbert, que es un espacio vectorial complejo $\mathbb{C}^n$) induce una rotación de fase constante en el plano complejo, donde la energía ($E$) se interpreta como la frecuencia o velocidad angular de dicha rotación. La intención de $H$ es, por lo tanto, unificar la Geometría (el flujo del sistema), el Álgebra (la estructura de conmutación de los operadores) y la Aritmética (el espectro discreto).

El propósito de $H$ es garantizar que las cantidades físicas medibles sean reales. La conversión de la naturaleza compleja del espacio de Hilbert ($\mathbb{C}$) a un espectro puramente real ($\mathbb{R}$) se logra mediante una imposición algebraica: el operador $H$ debe ser Hermítico (o auto-adjunto), $H = H^\dagger$. Esta propiedad, postulada como axioma en la mecánica cuántica\sidenote{\cite{VonNeumann1932}}, garantiza que la solución de la ecuación de autovalores $H\psi = E\psi$ produzca autovalores $E$ necesariamente reales ($E \in \mathbb{R}$). Esta coherencia algebraica solo está garantizada dentro de $\mathbb{C}$; cualquier extensión dimensional que viole Frobenius destruye precisamente el mecanismo que convierte espectro complejo en espectro real.

\subsection{Solución Propuesta: Modularización Geométrica}

En lugar de extender $\mathbb{C}$ mediante álgebra, lo extendemos mediante modularización geométrica y simetría.

\subsubsection{Extensión Algebraica}

Una extensión algebraica de un cuerpo $K$ construye un cuerpo mayor $L \supset K$ añadiendo raíces de polinomios. El ejemplo canónico es $\mathbb{C} = \mathbb{R}[x]/(x^2+1)$: se añade un elemento nuevo $i$ que satisface $i^2 = -1$. La extensión $\mathbb{R} \to \mathbb{C}$ añade elementos; la extensión $\mathbb{C} \to \mathbb{H}$ añade $j, k$ con relaciones $j^2 = k^2 = -1$, $ij = k = -ji$. Cada extensión algebraica introduce nuevos generadores con nuevas relaciones algebraicas, y son estas relaciones las que destruyen conmutatividad o asociatividad.

\subsubsection{Modularización Geométrica}

Una modularización geométrica no añade elementos. Dado un lattice $\Lambda \subset \mathbb{C}$, el espacio modular es el cociente:
\[
M_\Lambda = \mathbb{C}/\Lambda = \{[z] : z \sim z + \lambda, \; \lambda \in \Lambda\}
\]

La modularización identifica puntos equivalentes bajo traslación por el lattice. Topológicamente, $\mathbb{C}/\Lambda \cong T^2$ (toro). Algebraicamente, las operaciones de $\mathbb{C}$ se heredan a funciones $\Lambda$-periódicas: si $f(z) = f(z + \lambda)$ para todo $\lambda \in \Lambda$, entonces $f$ vive naturalmente sobre el toro. La modularización reorganiza sin añadir; preserva todas las propiedades algebraicas de $\mathbb{C}$ en el espacio de funciones periódicas.

\subsubsection{Acoplamiento por Simetría: Estructura Tripartita y Dualidad Lattice}

La construcción utiliza simetría $S_3$ (grupo simétrico de orden 3) codificada mediante las raíces cúbicas de la unidad $\omega = e^{2\pi i/3}$. Los tres componentes $(P,C,F)$ se disponen con separación angular de $2\pi/3$, formando un triángulo equilátero en el plano complejo---estructura tipo Eisenstein.

Sin embargo, el lattice que el operador genera no es hexagonal sino rectangular:
\[
\Lambda_{\text{PCF}} = \mathbb{Z}M_{\text{PCF}} \oplus \mathbb{Z}(M_{\text{PCF}} \cdot i)
\]

con base $\{M, Mi\}$ y ángulo $90°$---estructura tipo Gauss.

La razón áurea $\varphi$ media entre ambas estructuras: entrada tripartita ($\omega^3 = 1$), salida rectangular ($i^2 = -1$), mediador $\varphi^2 = \varphi + 1$. El operador no elige entre Eisenstein o Gauss---mantiene coherencia entre ambos mediante el invariante $|\Omega| = 1/2$.

\subsubsection{Acoplamiento Fibonacci}

A la simetría geométrica se une un acoplamiento aritmético mediante la razón áurea:
\[
z = \varphi y \quad \text{donde} \quad \varphi = \frac{1+\sqrt{5}}{2}
\]

Este acoplamiento reduce los 3 grados de libertad nominales de la estructura $S_3$ a 2 grados de libertad efectivos, estableciendo una dependencia funcional que elimina un grado de libertad sin destruir la información tripartita. La razón áurea $\varphi$ satisface $\varphi^2 = \varphi + 1$, lo que genera autosimilaridad bajo escalamiento---propiedad esencial para la coherencia multi-escala del operador.

\subsubsection{Matriz Normal}

Con esta estructura, construimos la matriz generadora:
\[
\hat{\Omega} = [\omega_1, \omega_2, \omega_3]^T \quad \text{donde} \quad \omega_j = e^{2\pi i{(j-1)}/3} % chktex 3
\]

Esta matriz es normal pero no hermítica:
\[
\hat{\Omega}\hat{\Omega}^\dagger = \hat{\Omega}^\dagger\hat{\Omega} \quad \text{pero} \quad \hat{\Omega} \neq \hat{\Omega}^\dagger
\]

La normalidad garantiza que $\hat{\Omega}$ es diagonalizable con base ortonormal de eigenvectores\sidenote{\cite{Halmos1958}}. La no-hermiticidad refleja la asimetría geométrica inherente a la estructura tripartita $S_3$: los tres componentes $(P, C, F)$ están relacionados cíclicamente, no simétricamente.

\subsubsection{Kernel Hermítico}

La matriz normal $\hat{\Omega}$ se acopla a un kernel integral $K_{\text{PCF}}$ mediante integración sobre el dominio fundamental del toro:
\[
K_{\text{PCF}}(x, y) = \int_{\mathcal{F}} \hat{\Omega}(x, t) \overline{\hat{\Omega}(y, t)} \, dt
\]

Este kernel es hermítico por construcción:
\[
K_{\text{PCF}}(x, y) = \overline{K_{\text{PCF}}(y, x)}
\]

La hermiticidad emerge del proceso de integración---la simetrización ocurre en el espacio de funciones $L^2$, no en la matriz $\hat{\Omega}$. El operador integral $T_K f(x) = \int K_{\text{PCF}}(x,y) f(y) \, dy$ hereda la hermiticidad del kernel.

\subsubsection{Espectro Real}

Por el teorema espectral para operadores integrales hermíticos compactos\sidenote{\cite{Riesz1955}}, el operador $T_K$ tiene espectro puramente real y discreto:
\[
\text{Spec}(T_K) \subset \mathbb{R}, \quad \text{Spec}(T_K) = {\{\lambda_n\}}_{n=1}^{\infty}
\]

Este es el resultado buscado: partiendo de una estructura tripartita en $\mathbb{C}^3$ (que evita autorreferencia), mediante modularización geométrica y acoplamiento $\varphi$-$i$-$S_3$ (que preserva propiedades de $\mathbb{C}$), llegamos a un operador hermítico con espectro real (que satisface los requisitos espectrales para la Hipótesis de Riemann).

La cadena completa es:
\[
\mathbb{C}^3 \xrightarrow{\text{simetría } S_3} \hat{\Omega} \text{ (normal)} \xrightarrow{\text{acoplamiento } \varphi} \mathbb{C}^2_{\text{eff}} \xrightarrow{\text{modularización}} T^2 \xrightarrow{\text{kernel}} K_{\text{PCF}} \text{ (hermítico)} \xrightarrow{\text{espectral}} \lambda_n \in \mathbb{R}
\]

\subsection{Fundamento y Alcance del Presente Trabajo}

En lugar de construir un operador especificando el espectro de ceros de zeta---ciclo $H \mapsto \text{spec}(H) \mapsto H$---reinterpretamos el plano complejo como espacio modular $M_{\text{PCF}} = \mathbb{C}/\Lambda_{\text{PCF}} \cong T^2$ con lattice $\Lambda_{\text{PCF}}$ determinado por periodicidades geométricas. Esta reinterpretación reduce información infinita a datos finitos aglutinados mediante coherencia estructural\sidenote{La identificación $z \sim z + \lambda$ para $\lambda \in \Lambda_{\text{PCF}}$ forma clases $[z] \in \mathbb{C}/\Lambda_{\text{PCF}}$. La coherencia estructural preserva invariantes (módulo $|z|$, fase $\arg(z)$, estructura algebraica) que reconstruyen propiedades globales desde datos locales finitos, siguiendo el principio donde representaciones equivalentes privilegian aspectos particulares (\cite{Manin2013})---principio formalizado en §\ref{prop:equivalencia-definiciones}.\label{note:aglutinacion-modular}}, evitando el problema de auto-referencia descrito en §\ref{sec:obstaculos} donde $H$ requiere conocer $\text{spec}(H)$ \textit{a priori}.

Integramos los axiomas de $\mathbb{C}$ con principios de autoconsistencia---formalizados en \textit{bootstrap} conforme por Guillarmou \textit{et al.}\sidenote{\cite{Guillarmou2020}} y en \textit{bootstrap} modular por Benjamin-Chang (véase §\ref{subsec:simetrias-dualidades})---mediante coherencia multi-dominio $D_1 \leftrightarrow D_2 \leftrightarrow D_3 \leftrightarrow D_4$ donde la información se distribuye entre múltiples dominios que se determinan mutuamente sin autoobservación directa. Siguiendo el análisis de Yanofsky sobre ciclos prohibidos (véase §\ref{subsec:simetrias-dualidades}), el operador emerge de propiedades geométricas determinadas por la estructura de $\mathbb{C}$ mismo mediante kernel modular $K(z,w)$, no de especificación de un Hamiltoniano microscópico. La emergencia hermítica en espacios adjuntos es consecuencia de esta construcción fundamental, no su punto de partida.

\subsection{Verificación Computacional}

Esta construcción exhibe dos observables clave. Una correspondencia aritmética mediante isomorfismo logarítmico vincula $\sigma \to p_\sigma \to M_p = 2^{p_\sigma}-1$, verificada en 51 primos de Mersenne desde $M_2 = 3$ hasta $M_{82589933}$ con 24.9 millones de dígitos. Por otra parte, se proporciona predicción analítica de ceros de $\zeta(s)$ y análogamente de otras L-funciones. La verificación alcanza precisión de máquina (véase \ref{def:precision-computacional}) para $n \sim 10^{10}$ (altura $t \sim 10^{23}$). % chktex 2

El operador no requiere conocer estos ceros para su construcción y su espectro exhibe correlación con ellos \textit{a posteriori}.

\subsection{Estructura del Presente Trabajo}

Este documento desarrolla la construcción completa del operador $\omegapcf$ y verifica sus propiedades estructurales y numéricas. \autoref{sec:plano-complejo-modulos}\ introduce el plano complejo como espacio de módulos, incluyendo espacios paramétricos adjuntos (\autoref{subsec:espacios-adjuntos}\ ). \autoref{sec:operador-PCF}\ desarrolla el operador $\omegapcf$ mediante construcción axiomática (\autoref{subsec:axiomas}\ ), construcción desde el módulo con ecuaciones de acoplamiento (\autoref{subsec:construccion-modulo}\ ), y geometría asociada incluyendo estructura 3D (\autoref{subsec:geometria-3d}\ ), torre de escalas (\autoref{subsec:spacetime-torre}\ ), espacio-tiempo pentadimensional (\autoref{subsec:spacetime-pentadimensional}\ ) y funcionalización en espacio de Hilbert (\autoref{subsec:funcionalizacion}\ ). La necesidad del toro complejo y estructura tensorial se justifica en \autoref{subsec:toro-lattice}.

Las secciones subsecuentes establecen propiedades espectrales y geométricas. \autoref{convergencia}\ analiza convergencia espectral en espacio de Hilbert. \autoref{invariancia}\ demuestra invariancia modular exacta y su relación con el principio de certidumbre. \autoref{hausdorff}\ estudia la dimensión de Hausdorff de la estructura fractal. \autoref{triple}\ establece coherencia triple en espacios inequivalentes. \autoref{mersenne}\ expone correspondencias aritméticas: números de Mersenne, espiral áurea y estructura logarítmica. Apéndice~\ref{app:ttt}\ contiene la tabla completa de verificaciones computacionales.


% chktex-file 9 17
\section{El Plano Complejo como Espacio de Módulos}\label{sec:plano-complejo-modulos}

\subsection{El Módulo: Magnitud Primitiva}

El plano complejo $\mathbb{C}$ ocupa una posición singular en matemáticas: une simultáneamente aritmética (cuerpo algebraico), geometría (espacio euclidiano $\mathbb{R}^2$), analítica (dominio de funciones holomorfas), y topología (variedad compleja). Esta multiplicidad de interpretaciones no es un accidente histórico. Gauss\sidenote{\cite{Gauss1831}} la reconoció pero no la formalizó completamente hasta 1831; Riemann la explotó para crear geometría compleja (véase más abajo); Weierstrass\sidenote{\cite{Weierstrass1876}} la sistematizó mediante teoría de funciones y series de potencias. La unificación es posible porque $\mathbb{C}$ posee una magnitud primitiva, el módulo $|z| := \sqrt{x^2+y^2}$, que determina simultáneamente distancia geométrica, norma algebraica, valor absoluto analítico, y métrica topológica.

Riemann introdujo ''\textit{Modul}'' en 1857 en \textit{Theorie der Abel'schen Functionen}\sidenote{\cite{Riemann1857}} (Teoría de las funciones abelianas) para designar parámetros caracterizadores de clases de equivalencia de objetos geométricos. Define el espacio de módulos (\textit{Modulraum}) como el espacio cociente que parametriza todas las clases de equivalencia de superficies de Riemann compactas de un género fijado $g$, donde dos superficies son equivalentes si existe entre ellas un isomorfismo conforme (biholomorfismo).

\begin{definition}[Módulo geométrico]\label{def:modulo}
Para $z = x + iy \in \mathbb{C}$ con $x, y \in \mathbb{R}$:
\[
|z| := \sqrt{x^2 + y^2}
\]
Geométricamente, dado $z = x + iy$, el módulo $|z|$ es la longitud de la diagonal del paralelogramo formado por los segmentos $(|x|, 0)$ y $(0, |y|)$, obtenida mediante el teorema de Pitágoras.
\end{definition}

\begin{figure}[h]
\centering
\includegraphics[width=0.8\textwidth]{src/images/image9.png}
\captionsetup{justification=centering}
\caption{Representación geométrica del módulo $|z| = \sqrt{x^2 + y^2}$ como hipotenusa de un triángulo rectángulo.}
\label{fig:modulus_geometric} % chktex 24
\end{figure}

\begin{proposition}[Invariancia rotacional]\label{prop:invariancia-rotacional}
$|e^{i\theta} z| = |z|$ para todo $\theta \in \mathbb{R}$.
\end{proposition}

\subsection{Rotación como Generador del Plano}

La unidad imaginaria $i$ es un operador geométrico rotacional que extiende $\mathbb{R}$ a $\mathbb{C}$ mediante:
\[
\mathbb{R} \xrightarrow{\times i} i\mathbb{R} \quad \text{(rotación 90°)}
\]

\textbf{Periodicidad.} $i^1 = i$, $i^2 = -1$, $i^3 = -i$, $i^4 = 1$ genera:
\[
\mathbb{C} = \mathbb{R} \oplus i\mathbb{R} = \{x + iy : x,y \in \mathbb{R}\}
\]
con base $\{1, i\}$ y relación $i^2 = -1$.

\begin{proposition}[Dualidad geométrica-algebraica]\label{prop:dualidad-geometrica-algebraica}
El módulo satisface: $|z|^2 = z \cdot \bar{z} = x^2 + y^2$.
\end{proposition}
\par
Los números imaginarios emergen de $\mathbb{R}$ gracias a $i$. En términos de estructura de espacio vectorial, $\mathbb{C}$ como extensión de $\mathbb{R}$ tiene dimensión 2 sobre $\mathbb{R}$ (con base $\{1, i\}$), mientras que $\mathbb{R}$ como espacio vectorial sobre sí mismo tiene dimensión 1. Geométricamente, esto corresponde a la transición de línea (``1D'') a plano (``2D'').

\pagebreak
\subsection{Módulo Algebraico y Estructura de Cuerpo}

El término ``modulus'' (latín, ``medida pequeña'' o ``unidad estándar'') designa el módulo de un número complejo $|z|$, que tiene raíces geométricas: representa la distancia al origen. Dicha concepción geométrica fue formalizada por Argand\sidenote{\cite{Argand1806}} y Gauss\sidenote{\cite{Gauss1831}}, estableciendo $\mathbb{C}$ como plano con estructura métrica. La formulación algebraica moderna $|z|^2 = z \cdot \bar{z}$ mediante el producto hermítico emergió posteriormente, unificando las perspectivas geométrica y algebraica bajo dicha noción de medida. La dualidad geométrico-algebraica en $\mathbb{C}$ influyó el desarrollo conceptual del álgebra abstracta del siglo XIX.

La raíz etimológica de ``medida'' conecta desarrollos históricos aparentemente distintos. Riemann (véase arriba) usaba ``moduli'' para parámetros clasificadores de objetos geométricos (superficies de Riemann), mientras que Dedekind\sidenote{\cite{Dedekind1871}} desarrollaba ``módulos'' como estructuras algebraicas sobre anillos. Ambos conceptos comparten dicha base semántica, manifestándose de forma prototípica\sidenote{Del griego \textit{protos} (primero) + \textit{typos} (modelo), conectando con ``primitiva'' (\textit{primitivus}, latín: primero/original) en el título.} en el módulo $|z|$ de números complejos. La formalización moderna del módulo algebraico como conjunto con estructura de grupo abeliano y acción escalar de un anillo fue desarrollada posteriormente, particularmente por Noether\sidenote{\cite{Noether1921}}, unificando estos conceptos dentro del álgebra abstracta.

\begin{definition}[Módulo algebraico]\label{def:modulo-algebraico}
El módulo se caracteriza algebraicamente mediante el producto hermítico en $\mathbb{C}$. Para $z = x + iy \in \mathbb{C}$ con $x, y \in \mathbb{R}$, el producto hermítico $z \cdot \bar{z}$ define el cuadrado del módulo:
\[
|z|^2 = z \cdot \bar{z} = (x+iy)(x-iy) = x^2 + y^2
\]
donde $\bar{z} = x - iy$ denota la conjugación compleja. Esta caracterización algebraica establece el módulo como la raíz cuadrada del producto de un número complejo por su conjugado, revelando la estructura multiplicativa del plano complejo donde la conjugación actúa como involución que preserva la estructura de cuerpo mientras determina la norma.
\end{definition}

\begin{proposition}[Equivalencia y privilegio de perspectiva]\label{prop:equivalencia-definiciones}
\sidenote{El término ``privilegio'' denota que cada caracterización hace más accesibles ciertas propiedades, no superioridad absoluta. La equivalencia matemática garantiza intercambiabilidad, mientras que la accesibilidad---facilidad con que ciertas propiedades son derivables directamente---varía según la representación. Esta dualidad refleja que $\mathbb{C}$ admite múltiples representaciones (euclidiana, algebraica, polar), principio que conecta perspectiva renacentista con espacios modulares modernos; se discute en detalle en §\ref{discussion} y en \ref{thm:tres-representaciones-C}.}
El módulo admite dos caracterizaciones equivalentes y complementarias:
\[
|z| = \sqrt{x^2 + y^2} = \sqrt{z \cdot \bar{z}}, \quad \forall z = x + iy \in \mathbb{C}.
\]
La primera caracteriza la distancia euclidiana; la segunda, el producto hermítico.

Las propiedades del módulo---multiplicatividad $|z_1z_2| = |z_1||z_2|$, desigualdad triangular $|z_1 + z_2| \leq |z_1| + |z_2|$, e invariancia rotacional $|e^{i\theta}z| = |z|$---tienen accesibilidad diferencial según la representación.
\end{proposition}

Esta accesibilidad diferencial se ilustra mediante ejemplos concretos. La multiplicatividad\sidenote{La multiplicación compleja $(a+bi)(c+di) = (ac-bd) + (ad+bc)i$ sintetiza esta multiplicidad: emerge de $i^2 = -1$ y realiza simultáneamente operaciones en cuatro dominios estructurales (aritmética, geometría, análisis, topología), distinguiendo $\mathbb{C}$ como síntesis única donde cada operación admite interpretaciones equivalentes en los cuatro dominios.} es derivable directamente desde la perspectiva algebraica mediante
\[
|z_1z_2|^2 = (z_1z_2)(\overline{z_1z_2}) = z_1\bar{z}_1 \cdot z_2\bar{z}_2 = |z_1|^2|z_2|^2.
\]

Análogamente, la invariancia rotacional (\ref{prop:invariancia-rotacional}) es manifiesta en la representación geométrica como preservación de distancia bajo rotación.

\subsection{Lattices: Estructura Discreta}

Hasta ahora hemos considerado el módulo $|z|$ como función continua sobre $\mathbb{C}$. Sin embargo, el plano complejo también admite estructuras discretas fundamentales mediante \textit{lattices}---subgrupos discretos que generan periodicidades. Los lattices conectan la métrica continua del módulo con la topología discreta del toro, estableciendo el puente entre geometría local (módulo como distancia) y topología global (espacio cociente como toro).

\begin{definition}[Lattice: periodicidad bidimensional]\label{def:lattice}
Un lattice $\Lambda \subset \mathbb{C}$ es un subgrupo discreto:
\[
\Lambda = \mathbb{Z}\omega_1 \oplus \mathbb{Z}\omega_2 = \{n_1\omega_1 + n_2\omega_2 : n_1, n_2 \in \mathbb{Z}\}
\]
donde $\omega_1, \omega_2 \in \mathbb{C}$ son linealmente independientes sobre $\mathbb{R}$ (es decir, $\omega_2/\omega_1 \notin \mathbb{R}$). La discreción implica que cada punto de $\Lambda$ está aislado (existe un entorno que no contiene otros puntos del lattice) y que el espacio cociente $\mathbb{C}/\Lambda$ es compacto, generando estructura toroidal $T^2 \cong S^1 \times S^1$.
\end{definition}

Ejemplos canónicos ilustran la estructura latticial: el \textit{lattice cuadrado} $\Lambda_\square = \mathbb{Z}[i]$ y el \textit{lattice hexagonal} $\Lambda_\triangle = \mathbb{Z}[\omega]$ (donde $\omega = e^{2\pi i/3}$), revelando cómo la estructura continua $\mathbb{C}$ admite subestructuras periódicas discretas mientras preserva simetrías rotacionales.

\begin{observation}[Lattices de Gauss y Eisenstein]\label{obs:lattice-cuadrado}
La diferencia geométrica entre ambos lattices canónicos: mientras el lattice de Gauss $\mathbb{Z}[i]$ rota en 4 pasos de 90° ($i^4 = 1$) con generadores $\{1, i\}$ separados 90°, el lattice de Eisenstein $\mathbb{Z}[\omega]$ rota en 6 pasos de 60° ($\omega^6 = 1$), aunque el ángulo entre sus generadores es 120° ($\omega^3 = 1$). Esta distinción entre simetría rotacional discreta y ángulo generador refleja cómo diferentes estructuras algebraicas ($i^2 = -1$ vs $\omega^3 = 1$) generan geometrías distintas en el mismo plano complejo.
\end{observation}

\begin{definition}[Toro complejo]\label{def:toro}
Para un lattice $\Lambda \subset \mathbb{C}$, el toro complejo es el espacio cociente:
\[
T_\Lambda := \mathbb{C}/\Lambda \cong S^1 \times S^1
\]
donde la identificación $z \sim z + \lambda$ para todo $\lambda \in \Lambda$ colapsa las dos direcciones periódicas del lattice en dos círculos independientes. El invariante modular $\tau = \omega_2/\omega_1 \in \mathbb{H}$ (semiplano superior $\mathbb{H} = \{z \in \mathbb{C} : \text{Im}(z) > 0\}$) clasifica la forma del toro: diferentes valores de $\tau$ corresponden a toros no isomorfos, aunque todos tienen topología $T^2$.
\end{definition}

\subsection{Espacio de Módulos}\label{subsec:espacio-modulos}

Un espacio de módulos parametriza clases de equivalencia de objetos geométricos: diferentes lattices que difieren solo por transformaciones modulares se identifican como el mismo punto. El espacio de módulos de lattices clasifica toros complejos según su forma, donde el invariante modular $\tau$ determina la geometría del toro $T_\tau = \mathbb{C}/\Lambda$.

\begin{definition}[Espacio de módulos de lattices]\label{def:espacio-modulos-lattices}
El espacio de módulos de lattices es:
\[
\mathcal{M}_{\text{lat}} = \mathbb{H}/\text{PSL}(2,\mathbb{Z})
\]
donde $\text{PSL}(2,\mathbb{Z})$ (el grupo modular) es el grupo de matrices $2 \times 2$ con coeficientes enteros y determinante $1$, módulo el signo ($\pm 1$). Este grupo actúa sobre $\mathbb{H}$ mediante transformaciones de Möbius:
\[
\text{para cada matriz } \begin{pmatrix} a & b \\ c & d \end{pmatrix} \in \text{PSL}(2,\mathbb{Z}), \quad \text{la transformación corresponde a } \tau \mapsto \frac{a\tau+b}{c\tau+d}.
\]
\end{definition}
\begin{proposition}[Herencia estructural]\label{prop:herencia-estructural}
El espacio $\mathcal{M}_{\text{lat}}$ hereda cuatro estructuras de $\mathbb{C}$:

% chktex-file 44
\begin{table}[bt]
    \centering
    \caption{Herencia estructural del espacio $\mathcal{M}_{\text{lat}}$ desde $\mathbb{C}$}
    \label{tab:herencia-estructural}
    \begin{tabular}{lll}
    \toprule
    Dominio & Estructura & Operación clave \\
    \midrule
    Aritmético & Lattice $\Lambda$ es $\mathbb{Z}$-módulo libre rango 2 & Suma de puntos \\
    Geométrico & Coordenadas polares $\mathbb{C} \cong \mathbb{R}_+ \times S^1$ & Rotación + escalamiento \\
    Analítico & Funciones holomorfas $f: \mathbb{C} \to \mathbb{C}$ & Diferenciación compleja \\
    Topológico & Toro $T_\tau = \mathbb{C}/\Lambda \cong S^1 \times S^1$ & Identificación modular \\
    \bottomrule
    \end{tabular}
\end{table}
% chktex-file 0

Esta herencia simultánea en cuatro dominios caracteriza al plano complejo; su unicidad se establece formalmente en el Teorema~\ref{thm:caracterizacion-unica-C}.
\end{proposition}

La idea fundamental de parametrizar familias de objetos geométricos mediante relaciones de equivalencia---donde objetos distintos que comparten propiedades estructurales se organizan en clases paramétricas---tiene antecedentes conceptuales que preceden la formalización moderna. Estos precedentes históricos abarcan desde la clasificación de cónicas por Apolonio (\textasciitilde{}200 a.C.), pasando por la caracterización foco-directriz con parámetros de excentricidad de Pappus (\textasciitilde{}320 d.C.), hasta las transformaciones proyectivas de perspectiva de Alberti\sidenote{\cite{Alberti1435}}, la sistematización de proyecciones ortogonales de Monge\sidenote{\cite{Monge1799}} y la formalización de la proyección isométrica de Farish\sidenote{\cite{Farish1822}}. Una genealogía detallada de estos precedentes históricos se discute en §\ref{discussion}. Todos estos desarrollos---clasificaciones por transformaciones geométricas o proyecciones---anticipan la base conceptual de los espacios de módulos modernos.

\subsection{Axiomas del Plano Complejo}\label{subsec:axiomas-plano-complejo}

Formalizamos los axiomas que definen $\mathbb{C}$.

\subsubsection{Axiomas Algebraicos}

\begin{axiom}[Axioma C1: Grupo aditivo]\label{ax:C1}
$(\mathbb{C}, +)$ es grupo abeliano: para todo $z_1, z_2, z_3 \in \mathbb{C}$,
\begin{itemize}
\item Asociatividad: $(z_1 + z_2) + z_3 = z_1 + (z_2 + z_3)$
\item Conmutatividad: $z_1 + z_2 = z_2 + z_1$
\item Neutro: existe $0 \in \mathbb{C}$ tal que $z + 0 = z$ para todo $z \in \mathbb{C}$
\item Inverso: para todo $z \in \mathbb{C}$ existe $-z \in \mathbb{C}$ tal que $z + (-z) = 0$
\end{itemize}
La operación suma se define como $(a+bi)+(c+di) = (a+c) + (b+d)i$ para $a,b,c,d \in \mathbb{R}$.
\end{axiom}

\begin{axiom}[Axioma C2: Grupo multiplicativo]\label{ax:C2}
$(\mathbb{C}^*, \cdot)$ es grupo abeliano, donde $\mathbb{C}^* = \mathbb{C} \setminus \{0\}$: para todo $z_1, z_2, z_3 \in \mathbb{C}^*$,
\begin{itemize}
\item Asociatividad: $(z_1 \cdot z_2) \cdot z_3 = z_1 \cdot (z_2 \cdot z_3)$
\item Conmutatividad: $z_1 \cdot z_2 = z_2 \cdot z_1$
\item Neutro: existe $1 \in \mathbb{C}^*$ tal que $z \cdot 1 = z$ para todo $z \in \mathbb{C}^*$
\item Inverso: para todo $z \in \mathbb{C}^*$ existe $z^{-1} \in \mathbb{C}^*$ tal que $z \cdot z^{-1} = 1$
\end{itemize}
La operación producto se define como $(a+bi)(c+di) = (ac-bd) + (ad+bc)i$ para $a,b,c,d \in \mathbb{R}$.
\end{axiom}

\begin{axiom}[Axioma C3: Distributividad]\label{ax:C3}
Para todo $z_1, z_2, z_3 \in \mathbb{C}$:
\[
z_1(z_2 + z_3) = z_1z_2 + z_1z_3
\]
\end{axiom}

\begin{axiom}[Axioma C4: Generación]\label{ax:C4}
El plano complejo se genera desde $\mathbb{R}$ mediante:
\[
\mathbb{C} = \mathbb{R}[i] = \{a + bi : a,b \in \mathbb{R}\}
\]
donde $i^2 = -1$.
\end{axiom}

\subsubsection{Axiomas Geométricos}

\begin{axiom}[Axioma C5: Métrica inducida por el módulo]\label{ax:C5}
Para $z = x + iy \in \mathbb{C}$ con $x, y \in \mathbb{R}$, el módulo geométrico se define como:
\[
|z| := \sqrt{x^2 + y^2}
\]
Esta definición induce la métrica $d(z,w) = |z-w|$ para $z, w \in \mathbb{C}$.
\end{axiom}

\begin{axiom}[Axioma C6: Completitud métrica]\label{ax:C6}
$(\mathbb{C}, d)$ es un espacio métrico completo: toda sucesión de Cauchy en $\mathbb{C}$ converge.
\end{axiom}

\subsubsection{Axioma Topológico}

\begin{axiom}[Axioma C7: Cierre algebraico]\label{ax:C7}
Todo polinomio $p(z) = a_nz^n + a_{n-1}z^{n-1} + \cdots + a_1z + a_0$ con coeficientes $a_i \in \mathbb{C}$ y $n \geq 1$ tiene al menos una raíz en $\mathbb{C}$.
\end{axiom}

\begin{theorem}[Caracterización única de $\mathbb{C}$]\label{thm:caracterizacion-unica-C}
El plano complejo $\mathbb{C}$ es el único cuerpo algebraicamente cerrado que satisface los axiomas C1-C7 simultáneamente, salvo isomorfismo.
\end{theorem}

\subsection[Espacios Adjuntos: Re-parametrizaciones de C]{Espacios Adjuntos: Re-parametrizaciones de $\mathbb{C}$}\label{subsec:espacios-adjuntos}

El plano complejo $\mathbb{C}$ admite múltiples re-parametrizaciones que preservan su estructura pero modifican la interpretación física.\@ Esta riqueza de representaciones es clave para la universalidad del operador $\omegapcf$.

\subsubsection{Equivalencia Métrica}

\begin{definition}[Equivalencia métrica por escalamiento]\label{def:equivalencia-metrica}
Espacios métricos $(X_1, d_1)$, $(X_2, d_2)$ son equivalentes si existe homeomorfismo $\varphi: X_1 \to X_2$ tal que:
\[
d_2(\varphi(x), \varphi(y)) = \lambda \cdot d_1(x,y), \quad \lambda > 0
\]
El homeomorfismo $\varphi$ preserva la estructura topológica y las propiedades métricas esenciales (completitud, compacidad relativa, convergencia de sucesiones). Esta equivalencia permite diferentes representaciones del mismo espacio subyacente.
\end{definition}

\begin{theorem}[Tres representaciones de $\mathbb{C}$]\label{thm:tres-representaciones-C}
El plano complejo $\mathbb{C}$ admite tres representaciones equivalentes, cada una enfatizando aspectos estructuralmente distintos y privilegiando propiedades específicas:
\begin{enumerate}
\item La representación euclidiana identifica $\mathbb{C}$ con $(\mathbb{R}^2, d_{\text{eucl}})$ donde la métrica es la distancia $d(z_1,z_2) = |z_1-z_2|$, haciendo transparente la estructura métrica y las propiedades geométricas.
\item La representación algebraica se escribe como $\mathbb{C} = \{z = x+iy : x,y \in \mathbb{R}\}$ con operaciones $(+,\cdot)$, enfatizando la estructura de cuerpo y las propiedades algebraicas.
\item La representación polar descompone $\mathbb{C} \cong \mathbb{R}_+ \times S^1$ con $z = re^{i\theta}$, separando magnitud $r$ y fase $\theta$, revelando simetrías rotacionales y propiedades multiplicativas.
\end{enumerate}
Esta multiplicidad de representaciones ilustra el principio de equivalencia métrica (Definición~\ref{def:equivalencia-metrica}), donde cada representación privilegia ciertas propiedades mientras preserva la estructura esencial del plano complejo.
\end{theorem}

\subsubsection{Espacio Adjunto I: Espacio-tiempo de Minkowski}

\begin{construction}[Rotación de Wick]\label{const:rotacion-wick}
La transformación $t \to it$ convierte la métrica euclidiana en pseudo-Riemanniana:
\[
ds^2_{\mathbb{C}} = dx^2 + dy^2 \quad \xrightarrow{\Phi_M} \quad ds^2_{\mathcal{M}} = -c^2dt^2 + dx^2
\]
\end{construction}

El eje real $\mathbb{R}$ corresponde a la coordenada espacial $x$; el eje imaginario $i\mathbb{R}$ se identifica con tiempo imaginario $it$; y el módulo euclidiano $|z|^2 = x^2 + y^2$ se transforma en el intervalo espaciotemporal $-c^2t^2 + x^2$.

\begin{proposition}[Preservación de estructura por Wick]\label{prop:preservacion-wick}
El mapa $\Phi_M: \mathbb{C} \to \mathcal{M}^{1+1}$ definido en la Construcción~\ref{const:rotacion-wick} preserva tres aspectos estructurales esenciales:
\begin{enumerate}
\item \textbf{Estructura algebraica}: establece isomorfismo entre el grupo de Lorentz y las transformaciones conformes de $\mathbb{C}$.
\item \textbf{Causalidad}: induce correspondencia entre los conos de luz en $\mathcal{M}^{1+1}$ y los sectores hiperbólicos en $\mathbb{C}$.
\item \textbf{Simetrías}: realiza dualidad donde las rotaciones espaciales corresponden a rotaciones $e^{i\theta}$ en $\mathbb{C}$, mientras que los boosts de Lorentz corresponden a escalamiento real en $\mathbb{C}$.
\end{enumerate}
La transformación de Lorentz estándar $x' = \gamma(x - vt)$, $t' = \gamma(t - vx/c^2)$ corresponde, bajo esta perspectiva, a rotación hiperbólica en $\mathbb{C}$ con parámetros $\gamma = \cosh \varphi$ y $v/c = \tanh \varphi$, donde $\varphi$ es el ángulo hiperbólico. Esta correspondencia ilustra cómo la equivalencia métrica (Definición~\ref{def:equivalencia-metrica}) se extiende a espacios pseudo-Riemannianos mediante la rotación de Wick.
\end{proposition}
\subsubsection{Espacio Adjunto II: Espacio de Hilbert}

\begin{construction}[Incrustación mediante distribuciones delta]\label{const:funcionalizacion}
El mapa:
\[
F: \mathbb{C} \to L^2(\mathbb{C}), \quad z \mapsto \delta_z
\]
incrusta cada punto del plano complejo como una distribución delta de Dirac en $L^2(\mathbb{C})$. Esta construcción establece el espacio adjunto genérico; la construcción específica mediante kernel integral se desarrolla en \ref{subsec:funcionalizacion}.

\par
El espacio de Hilbert asociado es:
\[
\mathcal{H}_{\mathbb{C}} = L^2(\mathbb{C}) = \left\{f: \mathbb{C} \to \mathbb{C} : \int_{\mathbb{C}} |f(z)|^2 \, d^2z < \infty\right\}
\]
con producto interno:
\[
\langle f, g \rangle = \int_{\mathbb{C}} \overline{f(z)} g(z) \, d^2z
\]
\end{construction}

\begin{proposition}[Preservación de estructura por funcionalización]\label{prop:preservacion-funcionalizacion}
La funcionalización $F: \mathbb{C} \to L^2(\mathbb{C})$ preserva:
\begin{enumerate}
\item Estructura lineal: la suma de puntos en $\mathbb{C}$ corresponde a la suma de funciones en $L^2(\mathbb{C})$.
\item Métrica: un isomorfismo entre la distancia $|z_1-z_2|$ en $\mathbb{C}$ y la norma $\|f-g\|_{L^2}$ en el espacio de Hilbert.
\item Simetrías rotacionales: las rotaciones $e^{i\theta}$ en $\mathbb{C}$ corresponden a operadores unitarios $U_\theta$ en $\mathcal{H}_{\mathbb{C}}$.
\end{enumerate}

\par
Los \textit{estados coherentes} en mecánica cuántica se definen como:
\[
|\alpha\rangle = e^{-|\alpha|^2/2} \sum_{n=0}^\infty \frac{\alpha^n}{\sqrt{n!}} |n\rangle
\]
donde $\alpha \in \mathbb{C}$ parametriza los estados cuánticos del oscilador armónico.\@
\end{proposition}

\subsubsection{Espacio Adjunto III: Esfera de Riemann}

\begin{construction}[Curvatura]\label{const:curvatura}
El plano complejo $\mathbb{C}$ se compactifica formando la esfera de Riemann:
\[
\hat{\mathbb{C}} = \mathbb{C} \cup \{\infty\} \cong S^2
\]
mediante la proyección estereográfica:
\[
\pi: S^2 \setminus \{N\} \to \mathbb{C}, \quad (x,y,z) \mapsto \frac{x+iy}{1-z}
\]

\par
La métrica de Fubini-Study inducida en $\hat{\mathbb{C}}$ es:
\[
ds^2 = \frac{4|dz|^2}{{(1+|z|^2)}^2}
\]
\end{construction}

\begin{proposition}[Esfera de Riemann como espacio de móduli]\label{prop:esfera-riemann-moduli}
La esfera de Riemann $\hat{\mathbb{C}}$ es el espacio de móduli de las curvas elípticas con j-invariante:
\[
j(\tau) = 1728 \frac{g_2^3(\tau)}{g_2^3(\tau) - 27g_3^2(\tau)}
\]
\end{proposition}

\subsubsection{Espacio Adjunto IV: Espacio de Teichmüller}

\begin{construction}[Espacio de Teichmüller del toro]\label{const:teichmuller}
El espacio de Teichmüller del toro es:
\[
\begin{split}
\mathcal{T}(T^2) = \{(X, f) : &\ X \text{ es una superficie de Riemann, } \\
&\ f: T^2 \to X \text{ es un difeomorfismo}\}/\sim
\end{split}
\]
donde $\sim$ denota la relación de equivalencia que identifica pares $(X, f)$ y $(X', f')$ cuando existe un isomorfismo conforme entre $X$ y $X'$.

\par
Para el toro, $\mathcal{T}(T^2) \cong \mathbb{H}$ (semiplano superior) mediante la identificación
\[
\tau \mapsto T_\tau = \mathbb{C}/(\mathbb{Z} \oplus \mathbb{Z}\tau)
\]
\end{construction}

\begin{proposition}[Conexión con §\ref{subsec:espacio-modulos}]\label{prop:conexion-moduli-teichmuller}
El espacio de módulos es:
\[
\mathcal{M}(T^2) = \mathcal{T}(T^2)/\text{MCG}(T^2) = \mathbb{H}/\text{SL}(2,\mathbb{Z})
\]
donde MCG denota el grupo de clases de aplicaciones (\textit{mapping class group}).
\end{proposition}

\subsubsection{Coherencia Categórica}

Los espacios adjuntos definidos anteriormente (Minkowski, Hilbert, Riemann, Teichmüller) no son representaciones independientes, sino que están relacionados mediante functores que preservan la estructura fundamental de $\mathbb{C}$. Esta coherencia categórica garantiza que las propiedades del plano complejo se transfieren consistentemente a través de todas las re-parametrizaciones.

\begin{theorem}[Conmutatividad de functores]\label{thm:conmutatividad-functores}
Los mapas entre espacios adjuntos conmutan. Específicamente, el functor de funcionalización $F: \mathbb{C} \to L^2(\mathbb{C})$ (definido en \cref{const:funcionalizacion}) y el mapa de rotación de Wick $\Phi_M: \mathbb{C} \to \mathcal{S}^{1+1}$ (definido en \cref{const:rotacion-wick}) satisfacen:

\begin{center}
\begin{tabular}{ccc}
$\mathbb{C}$ & $\xrightarrow{F}$ & $L^2(\mathbb{C})$ \\
$\downarrow \Phi_M$ & & $\downarrow \Phi_{M*}$ \\
$\mathcal{S}^{1+1}$ & $\xrightarrow{F'}$ & $L^2(\mathcal{S}^{1+1})$
\end{tabular}
\end{center}

donde $\Phi_{M*}$ es el pushforward de $\Phi_M$ al espacio de funciones, y $F'$ es la funcionalización en el espacio de Minkowski. La conmutatividad se expresa como $F \circ \Phi_M = \Phi_{M*} \circ F$.

\par
Esta conmutatividad implica que aplicar primero la rotación de Wick y luego la funcionalización produce el mismo resultado que aplicar primero la funcionalización y luego el pushforward de la rotación de Wick. Esta propiedad garantiza que la estructura del plano complejo se preserva coherentemente al traducir entre representaciones geométricas y funcionales.
\end{theorem}

\begin{corollary}[Coherencia de cuatro estructuras]\label{cor:coherencia-cuatro-estructuras}
Las cuatro estructuras fundamentales de $\mathbb{C}$ (aritmética, geométrica, analítica, topológica) se preservan simultáneamente en todos los espacios adjuntos. La conmutatividad de functores asegura que las propiedades algebraicas, métricas, analíticas y topológicas del plano complejo se transfieren de manera consistente, sin contradicciones, a través de las transformaciones que definen los espacios adjuntos.
\end{corollary}

\subsubsection{Resumen de correspondencias entre dominios}

La universalidad del plano complejo se manifiesta en que cada operación algebraica admite interpretaciones equivalentes en los cuatro dominios estructurales. La siguiente tabla resume estas correspondencias, mostrando cómo las operaciones fundamentales de $\mathbb{C}$ se realizan de manera coherente en cada dominio:

% chktex-file 44
\begin{table}[bt]
    \centering
    \caption{Correspondencias entre dominios estructurales del plano complejo}
    \label{tab:correspondencias-dominios}
    \begin{tabular}{lllll}
    \toprule
    Operación & Aritmético & Geométrico & Analítico & Topológico \\
    \midrule
    Suma & $z_1 + z_2$ & Traslación & $f+g$ holomorfa & Grupo abeliano \\
    Producto & $z_1 \cdot z_2$ & Rotar + escalar & $f \cdot g$ holomorfa & Acción $\mathbb{C}^*$ \\
    Módulo & Norma $\|\cdot\|$ & Distancia radial & $\|f\|_\infty$ & Métrica \\
    Conjugación & $\bar{z}$ & Reflexión eje real & Involución & Automorfismo \\
    Exponencial & $e^z$ & Espiral logarítmica & Mapa conforme & Covering \\
    \bottomrule
    \end{tabular}
\end{table}
% chktex-file 0

Esta correspondencia unificada no es meramente notacional: cada operación preserva propiedades estructurales que se manifiestan de manera equivalente en los cuatro dominios. Por ejemplo, la multiplicación compleja realiza simultáneamente una operación algebraica (producto de números), una transformación geométrica (rotación y escalamiento), una operación analítica (producto de funciones holomorfas), y una acción topológica (acción del grupo multiplicativo $\mathbb{C}^*$).

\begin{proposition}[Herencia del operador $\omegapcf$]\label{prop:herencia-PCF}
El operador $\omegapcf$ hereda la estructura cuádruple de $\mathbb{C}$ en los cuatro dominios simultáneamente. Esta herencia se manifiesta de manera específica en cada dominio:

\par
\textbf{Aritmético}: El operador posee un lattice $\Lambda_{\text{PCF}} = \mathbb{Z}M_{\text{PCF}} \oplus \mathbb{Z}(M_{\text{PCF}} \cdot i)$ con generador $M_{\text{PCF}} = \pi/\varepsilon_0$ que establece la periodicidad discreta fundamental del sistema.

\par
\textbf{Geométrico}: El operador mantiene módulo constante $|\Omega(z,\sigma)| = 1/2$ para todo $z \in \mathbb{C}$ y $\sigma \in \mathbb{R}$, propiedad que lo distingue de construcciones que no preservan esta invariancia geométrica.

\par
\textbf{Analítico}: El operador actúa como operador hermítico en $L^2(\mathbb{C})$ con propiedades espectrales bien definidas, conectando la estructura algebraica con el análisis funcional.

\par
\textbf{Topológico}: El operador induce una acción sobre el toro $T^2 = \mathbb{C}/\Lambda_{\text{PCF}}$ que preserva la periodicidad y la estructura topológica del espacio de módulos.

\par
Esta cuádruple herencia simultánea es lo que permite al operador evitar paradojas de auto-referencia: en lugar de depender de especificar su espectro \textit{a priori}, el operador emerge de la estructura distribuida heredada de $\mathbb{C}$, donde cada dominio proporciona restricciones que se satisfacen coherentemente.
\end{proposition}

\subsubsection{Cierre de Fundamentos}

Hemos establecido que el plano complejo $\mathbb{C}$ posee una riqueza estructural única que se manifiesta en múltiples niveles simultáneamente. La base geométrica emerge del módulo $|z|$, que actúa como longitud invariante bajo rotación, estableciendo la métrica fundamental del plano. La base algebraica surge de $i^2 = -1$, definiendo la estructura de cuerpo que caracteriza $\mathbb{C}$ como extensión de $\mathbb{R}$. Esta estructura algebraica genera, a su vez, una estructura discreta mediante lattices formados por periodicidades rotacionales, conectando lo continuo con lo discreto.

Los espacios de módulos clasifican estructuras equivalentes bajo transformaciones, revelando cómo diferentes representaciones del mismo objeto geométrico se relacionan mediante clases de equivalencia. Los espacios adjuntos (Minkowski, Hilbert, Riemann, Teichmüller) proporcionan re-parametrizaciones coherentes que preservan invariantes clave mientras modifican la interpretación física o matemática. Esta universalidad se completa mediante la herencia simultánea de estructura en cuatro dominios: aritmético, geométrico, analítico y topológico, donde cada operación fundamental de $\mathbb{C}$ admite interpretaciones equivalentes en todos los dominios.

En la siguiente parte, construiremos un operador que hereda esta universalidad mediante cinco propiedades interconectadas que emanan naturalmente de la estructura de $\mathbb{C}$ misma. El operador extiende $\mathbb{C}$ mediante modularización tridimensional acoplada a la razón áurea $\varphi$, unificando los análogos rotacionales $i$ (rotación 90°) y $\varphi$ (escalamiento dorado) como simetrías del mismo tipo mediante el acoplamiento $\varphi$-$i$-$S_3$. Esta extensión conecta los dominios aritmético, espacial y funcional sin perder coherencia, evitando paradojas de auto-referencia mediante estructura distribuida en múltiples dominios, estrategia identificada por Yanofsky\sidenote{\cite{Yanofsky2003}}. La matriz generadora $\hat{\Omega}$ en $\mathbb{C}^3$ es normal pero no hermítica; la hermiticidad del operador integral en $L^2(\mathbb{R})$ emerge del mecanismo de construcción mediante kernel simetrizado, no de propiedades algebraicas de $\hat{\Omega}$. El módulo constante $|\Omega| = 1/2$ actúa como punto fijo funcional que ancla toda la construcción, garantizando que el operador mantenga las propiedades fundamentales de $\mathbb{C}$ mientras revela estructura toroidal subyacente mediante lattice $\Lambda_{\text{PCF}}$ y módulo $M_{\text{PCF}} = \mathbb{C}/\Lambda_{\text{PCF}} \cong T^2$.

\section{El Operador $\omegapcf$}\label{sec:operador-PCF}

\subsection{Construcción Axiomática}\label{subsec:axiomas}

\begin{definition}[Axioma 1: Herencia de axiomas del plano complejo]\label{ax:herencia}
El operador hereda los axiomas de $\mathbb{C}$ previamente establecidos en §\ref{sec:plano-complejo-modulos}.
\end{definition}

\begin{definition}[Axioma 2: Extensión mediante operadores \textit{a logos}]\label{ax:extension-logos}
\sidenote{La locución \textit{a logos} (del griego $\lambda\acute{o}\gamma o\varsigma$, ``razón'', ``principio ordenador'') distingue extensiones que emergen de principios generativos inherentes a la estructura---como $i$ y $\varphi$ que generan secuencias recursivas y cierran estructuralmente sus campos---de extensiones meramente algebraicas formales. Esta distinción conecta con la genealogía del módulo desde los harpedonaptas egipcios hasta Weil (véase §\ref{discussion}): operadores \textit{a logos} no solo extienden formalmente, sino que revelan principios racionales subyacentes que organizan la estructura matemática, análogos al $\lambda\acute{o}\gamma o\varsigma$ de Heráclito como ley universal que ordena el cosmos.}

Existen dos generadores algebraicos que extienden $\mathbb{R}$:
\[
i^2 = -1, \quad \varphi^2 = \varphi + 1
\]

El generador $i$ produce la extensión $\mathbb{R} \to \mathbb{C}$, generando el plano $(x,y)$ con $x,y \in \mathbb{R}$.
\end{definition}
\par
Entre todos los generadores algebraicos de grado 2, únicamente $i$ y $\varphi$ poseen propiedades generativas que cierran estructuralmente sus campos:

\begin{enumerate}
\item $i$ con polinomio mínimo $x^2 + 1$ genera el grupo cíclico $\langle i \rangle = \{1, i, -1, -i\}$ de orden 4, estableciendo periodicidad rotacional completa en $\mathbb{C} = \mathbb{R}[i]$.

\item $\varphi$ con polinomio mínimo $x^2 - x - 1$ genera la recurrencia lineal $F_{n+1} = F_n + F_{n-1}$ con $F_n = (\varphi^n - {(-\varphi)}^{-n})/\sqrt{5}$, estableciendo escalamiento autosimilar en $\mathbb{Q}(\sqrt{5}) = \mathbb{Q}[\varphi]$.
\end{enumerate}

\par
Otros generadores algebraicos de grado 2 (e.g., $\sqrt{2}$ con $x^2 - 2$, $\sqrt{3}$ con $x^2 - 3$) extienden campos pero no generan estructuras recursivas cerradas. Las ecuaciones $i^2 + 1 = 0$ y $\varphi^2 - \varphi - 1 = 0$ son únicas en poseer clausura generativa: ambos generan sucesiones infinitas (rotaciones periódicas y Fibonacci) que preservan invariantes estructurales bajo iteración.

\begin{definition}[Axioma 3: Extensión ortogonal]\label{ax:extension-ortogonal}
La singularidad entre $\varphi$ e $i$ implica que existe coordenada $z \in \mathbb{R}$ ortogonal a $(x,y) \cong \mathbb{C}$ acoplada mediante:
\[
z = \varphi y, \quad \varphi = \frac{1+\sqrt{5}}{2}
\]
El espacio resultante es:
\[
E^3 = \{(x, y, \varphi y) \in \mathbb{R}^3\}
\]
con base $\{1, i, \varphi\}$.

\par
Los generadores $i$ y $\varphi$ satisfacen ecuaciones cuadráticas:
\[
i^2 = -1, \quad \varphi^2 = \varphi + 1
\]

y generan extensiones algebraicas:
\[
\mathbb{R}[i] = \mathbb{C}, \quad \mathbb{Q}[\varphi] = \mathbb{Q}(\sqrt{5})
\]

\par
La estructura dimensional exhibe una jerarquía anidada que refleja la relación entre los tres generadores algebraicos. La dimensión real (eje $x$, generada por $1$) se extiende a dimensión imaginaria (eje $y$, generada por $i$) mediante rotación de 90°: $y = ix$ en el plano complejo. A su vez, la dimensión áurea (eje $z$, generada por $\varphi$) se acopla a la dimensión imaginaria mediante escalamiento áureo: $z = \varphi y$. Esta estructura anidada establece la relación jerárquica:

\[
\text{Real } (x) \xrightarrow{i} \text{Imaginaria } (y) \xrightarrow{\varphi} \text{Áurea } (z)
\]

donde cada transición preserva la estructura algebraica subyacente: $i$ extiende $\mathbb{R}$ a $\mathbb{C}$, mientras que $\varphi$ acopla $\mathbb{C}$ a su extensión tridimensional mediante el isomorfismo $z = \varphi y$.
\end{definition}

\begin{remark}[Convención notacional]\label{rem:convencion-z}
La letra ``$z$'' aparece en dos contextos distintos pero relacionados:
\begin{itemize}
\item Como \textit{número complejo}: $z = x + iy \in \mathbb{C}$ (punto en el plano complejo)
\item Como \textit{coordenada vertical}: $z \in \mathbb{R}$ (altura en espacio 3D, satisfaciendo $z = \varphi y$)
\end{itemize}

Esta sobrecarga es intencional y refleja el isomorfismo biunívoco $\mathbb{C} \leftrightarrow \{(x,y,\varphi y) \in \mathbb{R}^3\}$ establecido por el acoplamiento $z = \varphi y$ (demostrado en el Teorema~\ref{thm:isomorfismo-bidireccional}), donde ambos usos de ``$z$'' son aspectos complementarios de la misma geometría. En coordenadas cartesianas $(x, y, z)$ para $\mathbb{R}^3$, la coordenada $z$ denota la altura vertical, mientras que en notación compleja $z = x+iy$ denota puntos del plano $\mathbb{C}$. El contexto siempre aclara cuál convención se emplea; véase también la convención específica para visualización 3D en §\ref{subsec:geometria-3d}.

Esta dualidad notacional enfatiza que el operador $\omegapcf$ habita simultáneamente el plano complejo y su extensión tridimensional, unificados por el acoplamiento áureo.
\end{remark}

\begin{definition}[Axioma 4: Estructura distribuida]\label{ax:estructura-distribuida}
El operador se factoriza:
\[
\Omega(z,\sigma) = P(z,\sigma) \cdot C(z) \cdot F(z)
\]
donde $P, C, F$ son fasores complejos.
\end{definition}

\par
\textit{Justificación (Lawvere-Yanofsky)}: El teorema de Lawvere establece que auto-referencia directa $f(f)$ implica paradoja. Yanofsky formaliza que paradojas auto-referenciales emergen de ciclos $X \to Y \to X$ (véase §\ref{subsec:simetrias-dualidades}). En oposición, la estructura tripartita implementa referencia distribuida en lugar de autorreferencia. El ciclo prohibido $D_1 \to D_2 \to D_1$ genera paradoja, mientras que la referencia distribuida $P \leftrightarrow C \leftrightarrow F$ establece coherencia. En esta estructura, ningún componente se observa a sí mismo directamente: $P$ observa $(C,F)$, $C$ observa $(P,F)$, y $F$ observa $(P,C)$. La auto-referencia está distribuida entre los tres componentes, no concentrada en un solo punto, evitando así el ciclo prohibido que genera paradojas.

\begin{center}
\begin{tabular}{ll}
Ciclo prohibido: & $D_1 \to D_2 \to D_1$ \quad [paradoja] \\
Referencia distribuida: & $P \leftrightarrow C \leftrightarrow F$ \quad [coherencia]
\end{tabular}
\end{center}

\begin{definition}[Axioma 5: Punto fijo funcional]\label{ax:punto-fijo}
El módulo del operador es constante e igual a $1/2$ para todo $z \in \mathbb{C}$ y $\sigma \in \mathbb{R}$:
\[
|\Omega(z,\sigma)| = \frac{1}{2}
\]

Esta constante emerge del producto de las magnitudes de los tres componentes con estructura tripartita balanceada:
\[
|P| \cdot |C| \cdot |F| = \frac{1}{\sqrt{3}} \cdot 1 \cdot \frac{\sqrt{3}}{2} = \frac{1}{2}
\]

donde $|P| = 1/\sqrt{3}$, $|C| = 1$, y $|F| = \sqrt{3}/2$ son las magnitudes de los componentes $P(z,\sigma)$, $C(z)$, y $F(z)$ respectivamente.
\end{definition}

\begin{corollary}[Círculo crítico y propiedades del módulo constante]\label{cor:circulo-critico}
El operador vive en el círculo crítico $C_{1/2} = \{w \in \mathbb{C} : |w| = 1/2\}$, estableciendo una triple conexión estructural.

\par
\textbf{Conexión geométrica}: El radio $1/2$ balancea las magnitudes $|P|$, $|C|$, $|F|$ mediante el producto $|P| \cdot |C| \cdot |F| = 1/2$, emergiendo directamente de la estructura tripartita.

\par
\textbf{Conexión algebraica}: El valor $1/2$ es universalmente representable en todas las estructuras numéricas fundamentales: $1/2 \in \mathbb{Q} \subset \mathbb{R} \subset \mathbb{C}$, siendo simultáneamente racional, real y complejo. Esta universalidad permite que el módulo constante actúe como solución del sistema de ecuaciones que determina el operador.

\par
\textbf{Conexión analítica}: Coincide exactamente con la línea crítica $\Re(s) = 1/2$ de la función zeta de Riemann\sidenote{Línea crítica donde la Hipótesis de Riemann conjetura que residen todos los ceros no triviales de $\zeta(s)$.}. Esta correspondencia establece $1/2$ como valor único que ancla la construcción del operador y conecta su estructura geométrica con el análisis complejo.
\end{corollary}

\subsubsection{Coherencia de Axiomas}

\begin{proposition}[Independencia de axiomas]\label{prop:independencia}
Los cinco axiomas PCF son independientes: ninguno se deriva de los otros cuatro.
\end{proposition}

\begin{theorem}[Consistencia de axiomas]\label{thm:consistencia}
Los cinco axiomas son consistentes: existe construcción explícita que los satisface simultáneamente.
\end{theorem}

\begin{proof}[Por construcción]
La construcción de §\ref{subsec:construccion-modulo} proporciona realización explícita.
\end{proof}

\begin{proposition}[Minimalidad de axiomas]\label{prop:minimalidad}
Los cinco axiomas son minimales: eliminar cualquiera destruye propiedades esenciales.

\begin{enumerate}
\item Sin Ax1, no hay extensión 3D y el operador vive solo en $\mathbb{C}$.
\item Sin Ax2, no hay conexión $i \leftrightarrow \varphi$ y se pierde la unificación rotacional.
\item Sin Ax3, el operador no actúa coherentemente en múltiples dominios.
\item Sin Ax4, aparece paradoja de auto-referencia tipo Lawvere (ciclo prohibido).
\item Sin Ax5, el módulo es variable y no hay punto fijo funcional que ancle la construcción.
\end{enumerate}
\end{proposition}
\subsection{Construcción desde el Módulo}\label{subsec:construccion-modulo}

La construcción del operador $\omegapcf$ emerge de una estructura algebraica fundamental: una matriz diagonal en el espacio complejo tridimensional $\mathbb{C}^3$ que presentaremos en la \dref{def:matriz-PCF}. 

\par
Parámetros que se repetirán regularmente durante el desarrollo:

\begin{center}
\begin{tabular}{>{\centering\arraybackslash}p{2cm}>{\centering\arraybackslash}p{3cm}>{\centering\arraybackslash}p{4.5cm}>{\centering\arraybackslash}p{3.5cm}}
\textbf{Notación} & \textbf{Nombre} & \textbf{Fórmula} & \textbf{Valor} \\
\hline
\rule{0pt}{2em}
$\varphi$ & razón áurea & $\displaystyle\frac{1 + \sqrt{5}}{2}$ & $1.618033988749895\ldots$ \\[1.2em]
$r_0$ & radio base & --- & $3$ \\[1.2em]
$\varepsilon_0$ & parámetro angular (constante de acoplamiento, parámetro \textit{bootstrap}) & $\displaystyle\frac{\ln \varphi}{6\sqrt{3}}$ & $0.046304629455899\ldots$ \\[1.2em]
$\omega_0$ & frecuencia angular & $2\varepsilon_0$ & $0.092609258911798$ \\[1.2em]
$\tau_0$ & período fundamental & $\displaystyle\frac{\pi}{\varepsilon_0} = \frac{6\sqrt{3}\pi}{\ln \varphi}$ & $67.846189258071644\ldots$
\end{tabular}
\end{center}

Una referencia completa con explicaciones detalladas y justificaciones geométricas se encuentra en el \textit{Apéndice~\ref{app:parametros-fundamentales}}.

\begin{definition}[Matriz generadora PCF]\label{def:matriz-PCF}
La estructura tripartita del operador $\omegapcf$ se codifica mediante la matriz diagonal:
\[
\hat{\Omega} = \frac{1}{2} \begin{pmatrix} 1 & 0 & 0 \\ 0 & \omega & 0 \\ 0 & 0 & \omega^2 \end{pmatrix} \in \mathbb{C}^{3 \times 3}
\]

donde $\omega = \exp(2\pi i/3)$ es la raíz cúbica primitiva de la unidad. La notación compacta $\frac{1}{2}\text{diag}(1, \omega, \omega^2)$ denota esta misma matriz diagonal.

Representación explícita en forma matricial:
\[
\hat{\Omega} = \begin{pmatrix} 1/2 & 0 & 0 \\ 0 & (1/2)\omega & 0 \\ 0 & 0 & (1/2)\omega^2 \end{pmatrix}
\]

Valores numéricos aproximados:
\[
\hat{\Omega} \approx \begin{pmatrix} 0.5 & 0 & 0 \\ 0 & -0.25 + 0.433i & 0 \\ 0 & 0 & -0.25 - 0.433i \end{pmatrix}
\]
\end{definition}

\begin{proposition}[Propiedades algebraicas]\label{prop:propiedades-matriz}
La matriz $\hat{\Omega}$ satisface las siguientes propiedades:

\begin{enumerate}
\item No es hermítica: $\hat{\Omega}^\dagger \neq \hat{\Omega}$

\begin{proof}[Por cálculo directo]
\[
\hat{\Omega}^\dagger = \frac{1}{2} \text{diag}(1, \overline{\omega}, \overline{\omega^2}) = \frac{1}{2} \text{diag}(1, \omega^2, \omega)
\]
Como $\overline{\omega} = \omega^2 \neq \omega$, se tiene $\hat{\Omega}^\dagger \neq \hat{\Omega}$.
\end{proof}

\item Sí es normal: $\hat{\Omega}^\dagger\hat{\Omega} = \hat{\Omega}\hat{\Omega}^\dagger$

\begin{proof}[Por cálculo directo]
Calculando ambos productos:
\[
\hat{\Omega}^\dagger \hat{\Omega} = \frac{1}{4} \begin{pmatrix} 1 & 0 & 0 \\ 0 & \omega^2\omega & 0 \\ 0 & 0 & \omega\omega^2 \end{pmatrix} = \frac{1}{4}I_3
\]

pues $|\omega| = 1$ implica $\omega^2\omega = \omega\omega^2 = 1$. De manera similar, $\hat{\Omega}\hat{\Omega}^\dagger = (1/4)I_3$, por tanto la matriz es normal.
\end{proof}

\item Eigenvalores con módulo constante:
\begin{align*}
\lambda_1 &= \frac{1}{2} \quad \text{(real, argumento 0°)} \\
\lambda_2 &= \frac{1}{2}\omega = \frac{1}{2}e^{i2\pi/3} \quad \text{(complejo, argumento 120°)} \\
\lambda_3 &= \frac{1}{2}\omega^2 = \frac{1}{2}e^{i4\pi/3} \quad \text{(complejo, argumento 240°)}
\end{align*}

Todos los eigenvalores satisfacen $|\lambda_k| = 1/2$.

\begin{proof}[Por construcción]
Los eigenvalores de una matriz diagonal son los elementos de la diagonal. Como $\hat{\Omega} = \frac{1}{2}\text{diag}(1, \omega, \omega^2)$, los eigenvalores son:
\[
\lambda_k = \frac{1}{2}\omega^k, \quad k \in \{0, 1, 2\}
\]

Dado que $|\omega| = |\omega^2| = 1$ (raíces cúbicas primitivas de la unidad tienen módulo unitario), se tiene:
\[
|\lambda_k| = \left|\frac{1}{2}\omega^k\right| = \frac{1}{2}|\omega^k| = \frac{1}{2}
\]
para $k \in \{0, 1, 2\}$.
\end{proof}
\end{enumerate}
\end{proposition}

\textit{Interpretación geométrica}: Los tres eigenvalores forman un triángulo equilátero en el plano complejo, inscrito en el círculo crítico $|z| = 1/2$:

\begin{itemize}
\item $\lambda_1 = 1/2$: Componente Patrón (eje real positivo)
\item $\lambda_2 = (1/2)\omega$: Componente Coherencia (rotación 120°)
\item $\lambda_3 = (1/2)\omega^2$: Componente Flujo (rotación 240°)
\end{itemize}

Esta disposición geométrica codifica la simetría tripartita $S_3$ del sistema.

\par

\textit{Conexión con las magnitudes de componentes}: Los módulos de los eigenvalores se relacionarán con las magnitudes $|P|$, $|C|$, $|F|$ que definiremos a continuación. Estas magnitudes satisfacen:
\[
|P| \cdot |C| \cdot |F| = \frac{1}{2} = |\lambda_k| \quad \forall k
\]
Esta igualdad conecta la estructura algebraica de $\hat{\Omega}$ con la geometría del triángulo equilátero.


\begin{observation}[No-hermiticidad como característica estructural]\label{obs:hermiticidad-omega}
La matriz $\hat{\Omega}$ no es hermítica ($\hat{\Omega}^\dagger \neq \hat{\Omega}$), pero esta propiedad no constituye un defecto sino una característica esencial de la construcción.

\par
La no-hermiticidad codifica la direccionalidad inherente de la estructura tripartita. La matriz $\hat{\Omega}$ opera en el espacio de componentes $\mathbb{C}^3$, donde cada componente (P, C, F) tiene un rol distinto y una orientación específica en el plano complejo.

\par
Cuando construyamos el kernel integral $K_{\text{PCF}}(x,y)$ en §\ref{subsubsec:emergencia-hermiticidad}, la hermiticidad emergerá del mecanismo de construcción mediante simetrización con $\delta(x-y) + \varepsilon(x,y)$, no de las propiedades algebraicas de $\hat{\Omega}$ misma.

\par
Esta distinción es fundamental: la hermiticidad del operador en espacio de Hilbert $L^2(\mathbb{R})$ es una propiedad emergente de la construcción integral, mientras que la no-hermiticidad de $\hat{\Omega}$ refleja la estructura geométrica tripartita del sistema.
\end{observation}

\subsubsection{Magnitudes de Componentes}

\begin{definition}[Magnitudes tripartitas]\label{def:magnitudes-tripartitas}
Las magnitudes de los tres componentes $P(z,\sigma)$, $C(z)$ y $F(z)$ son:
\[
|P| = \frac{1}{\sqrt{3}}, \quad |C| = 1, \quad |F| = \frac{\sqrt{3}}{2}
\]

Estas magnitudes determinan la contribución de cada componente al módulo total del operador $\omegapcf$.
\end{definition}

\begin{proposition}[Origen geométrico]\label{prop:origen-geometrico}
Las magnitudes $|P|$, $|C|$, $|F|$ se derivan de la geometría de un triángulo equilátero de lado unitario inscrito en el círculo crítico $|z| = 1/2$.

\par
Cada magnitud corresponde a una medida geométrica específica del triángulo:

\begin{enumerate}
\item La magnitud $|F| = \sqrt{3}/2$ corresponde a la altura del triángulo desde cualquier vértice hasta el lado opuesto: $h = \sqrt{1 - {(1/2)}^2} = \sqrt{3}/2$.

\item La magnitud $|P| = 1/\sqrt{3}$ corresponde a la inversa normalizada de la altura total desde el centro hasta un vértice: $H = \sqrt{3}$.

\item La magnitud $|C| = 1$ actúa como referencia unitaria que balancea los otros dos componentes.
\end{enumerate}

\par
Esta correspondencia geométrica codifica la simetría tripartita $S_3$ del operador mediante la disposición de los tres componentes en los vértices del triángulo, separados por ángulos de $2\pi/3$ radianes.
\end{proposition}

\begin{lemma}[Verificación del módulo]\label{lem:verificacion-modulo}
La estructura tripartita requiere que el producto de las tres magnitudes satisfaga $|P| \cdot |C| \cdot |F| = 1/2$ para cumplir el Axioma 5 (módulo constante). Este producto se verifica mediante:
\[
|P| \cdot |C| \cdot |F| = \frac{1}{\sqrt{3}} \cdot 1 \cdot \frac{\sqrt{3}}{2} = \frac{\sqrt{3}}{2\sqrt{3}} = \frac{1}{2}
\]
\end{lemma}

\begin{fullwidth}
\centering
\begin{minipage}{\linewidth}
\includegraphics[width=\linewidth]{src/images/image8.png}
\captionsetup{width=\linewidth,justification=centering}
\captionof{figure}{Propiedades de la matriz $\hat{\Omega}$: estructura diagonal, eigenvalores con simetría $S_3$ en el círculo crítico $|z| = 1/2$, y comparación con $\hat{\Omega}^\dagger$ mostrando no-hermiticidad antisimétrica en fase.}
\label{fig:verificacion-modulo} % chktex 24
\end{minipage}
\end{fullwidth}

\subsubsection{Fases de Componentes}

\begin{definition}[Parámetro de escala]\label{def:parametro-escala}
El parámetro de escala estructura la torre exponencial mediante escalamiento áureo:
\[
\varepsilon(\sigma) := \varepsilon_0 \varphi^\sigma
\]
donde $\sigma \in \mathbb{R}$ es el nivel de escala y la constante de acoplamiento áureo es:
\[
\varepsilon_0 := \frac{\ln \varphi}{6\sqrt{3}} = 0.04630462945589891\ldots
\]

\par
El factor $6\sqrt{3}$ emerge del acoplamiento $\varphi$-$i$-$S_3$ (\dref{ax:extension-ortogonal}, \dref{ax:estructura-distribuida}): el orden del grupo de simetría $S_3$ del triángulo equilátero es $|S_3| = 6$, y la altura del triángulo equilátero de lado unitario es $h = \sqrt{3}/2$, donde $\sqrt{3}$ aparece como factor geométrico fundamental. El logaritmo $\ln(\varphi)$ establece el isomorfismo entre estructura multiplicativa y aditiva\sidenote{La propiedad $\ln(ab) = \ln(a) + \ln(b)$ permite esta transformación, como se manifiesta en el isomorfismo logarítmico entre torres áurea y Mersenne (\tref{thm:isomorfismo-logaritmico}).}, permitiendo correspondencias estructurales entre dominios que operan bajo leyes diferentes.
\end{definition}

\begin{definition}[Fases de componentes PCF]\label{def:fases-componentes}
Para $z \in \mathbb{C}, \sigma \in \mathbb{R}$:
\begin{align*}
\phi_P(z,\sigma) &:= \arg(z) + \pi\varepsilon(\sigma) \\
\phi_C(z) &:= \arg(z) + \frac{2\pi}{3} \\
\phi_F(z) &:= \arg(z) + \frac{4\pi}{3}
\end{align*}
\end{definition}

\begin{proposition}[Separación angular de fases]\label{prop:separacion-angular}
Las fases de C y F (ver \dref{def:fases-componentes}) están separadas por:
\[
\phi_F(z) - \phi_C(z) = \frac{4\pi}{3} - \frac{2\pi}{3} = \frac{2\pi}{3}
\]
\end{proposition}

\begin{proposition}[Torre exponencial]\label{prop:torre-exponencial}
La función $\varepsilon(\sigma)$ definida en \dref{def:parametro-escala} satisface las siguientes propiedades para $\sigma \in \mathbb{R}$:

\begin{enumerate}
\item \textit{Relación de recurrencia}: $\varepsilon(\sigma+1) = \varphi \cdot \varepsilon(\sigma)$
\item \textit{Crecimiento exponencial}: $\lim_{\sigma\to\infty} \varepsilon(\sigma+1)/\varepsilon(\sigma) = \varphi$
\end{enumerate}
\end{proposition}

\subsubsection{Componentes Completas y Fórmula de Fase}

\begin{definition}[Componentes PCF]\label{def:componentes-PCF}

\vspace{0.5em}

\begin{align}
P(z,\sigma) &:= \frac{1}{\sqrt{3}} \, e^{i[\arg(z) + \pi\varepsilon(\sigma)]} \\[0.5em]
C(z) &:= 1 \cdot e^{i[\arg(z) + 2\pi/3]} \\[0.5em]
F(z) &:= \frac{\sqrt{3}}{2} \, e^{i[\arg(z) + 4\pi/3]}
\end{align}

\vspace{0.5em}
\end{definition}

Los componentes $P(z,\sigma), C(z), F(z)$ extienden funcionalmente sobre todo $\mathbb{C}$ la estructura tripartita codificada algebraicamente en la matriz $\hat{\Omega}$ (\ref{def:matriz-PCF}). Esta realización funcional completa la coherencia multi-nivel formal establecida entre la codificación algebraica (matriz en $\mathbb{C}^3$), la realización funcional (componentes sobre $\mathbb{C}$), y la verificación geométrica (magnitudes que satisfacen $|P| \cdot |C| \cdot |F| = 1/2$, ver \ref{lem:verificacion-modulo}). Esta coherencia multi-nivel refleja la referencia distribuida (Axioma 4, \ref{ax:estructura-distribuida}): cada nivel provee restricciones independientes que se determinan mutuamente, permitiendo que el operador actúe coherentemente en múltiples dominios sin colapsar en contradicción.

\begin{definition}[Operador $\omegapcf$]\label{def:operador-PCF-completo}
El operador $\omegapcf$ se factoriza como producto de los tres componentes (Axioma 4, \ref{ax:estructura-distribuida}):
\[
\Omega(z,\sigma) := P(z,\sigma) \cdot C(z) \cdot F(z)
\]
\end{definition}

\begin{proposition}[Aditividad de fase]\label{prop:formula-fase-explicita}
La multiplicación compleja del operador $\omegapcf = P \cdot C \cdot F$ se traduce en aditividad de fases, manifestando el principio de transformación multiplicativo-aditiva (\dref{def:parametro-escala}):
\begin{align*}
\arg(\Omega(z,\sigma)) &= \arg(P) + \arg(C) + \arg(F) \\
&= [\arg(z) + \pi\varepsilon(\sigma)] + [\arg(z) + 2\pi/3] + [\arg(z) + 4\pi/3] \\
&= 3\arg(z) + \pi\varepsilon(\sigma) + 2\pi
\end{align*}
donde las fases de los componentes están definidas en \dref{def:componentes-PCF}.

\par
Usando $e^{2\pi i} = 1$, la fase efectiva módulo $2\pi$ es:
\[
\arg(\Omega(z,\sigma)) \equiv 3\arg(z) + \pi\varepsilon(\sigma) \pmod{2\pi}
\]

\par
\textbf{Convención de notación}: En ecuaciones posteriores (especialmente \tref{thm:acoplamiento-temporal} y \tref{thm:acoplamiento-optimo}, y secciones de acoplamiento), cuando escribimos $\arg(\Omega)$ sin especificar, nos referimos a la fase efectiva $3\arg(z) + \pi\cdot\varepsilon(\sigma)$, o equivalentemente $[\arg(\Omega) - 2\pi]$. Cuando sea necesaria la fase total completa, lo indicaremos explícitamente.
\end{proposition}

\begin{corollary}[Módulo constante]\label{cor:modulo-constante}
Por construcción, $|\Omega(z,\sigma)| = 1/2$ para todo $z \in \mathbb{C}$ y $\sigma \in \mathbb{R}$.
\end{corollary}

Esta propiedad emerge directamente del producto de magnitudes tripartitas $|P| \cdot |C| \cdot |F| = 1/2$ (ver \ref{lem:verificacion-modulo}) y establece el módulo constante como punto fijo funcional que ancla toda la construcción (Axioma 5, \ref{ax:punto-fijo}). El valor $1/2$ actúa como invariante fundamental que conecta la estructura tripartita con propiedades globales del operador, incluyendo el lattice $\Lambda_{\text{PCF}}$ y las correspondencias estructurales que desarrollaremos en secciones posteriores.

\subsection{Geometría del Círculo en Espacio 3D}\label{subsec:geometria-3d}

\subsubsection{Parametrización de la Curva Espacial}

\begin{proposition}[Curva PCF]\label{prop:curva-PCF}
Cuando un punto rota en el plano complejo $z(t) = re^{it}$, la coordenada ortogonal $z = \varphi y$ (Axioma 3, \ref{ax:extension-ortogonal}) genera la curva espacial:
\[
\vec{r}(t) = r\begin{pmatrix} \cos t \\ \sin t \\ \varphi\sin t \end{pmatrix}
\]
\end{proposition}

\begin{corollary}[Naturaleza de la curva]\label{cor:naturaleza-curva}
Esta curva se encuentra contenida en el plano $z = \varphi y$ (no es un círculo plano en 3D), y sus proyecciones satisfacen:
\begin{itemize}
\item Proyección en $(x,y)$: círculo perfecto $x^2 + y^2 = r^2$
\item Proyección en $(y,z)$: elipse $y^2 + z^2/\varphi^2 = r^2$
\end{itemize}
\end{corollary}

\begin{proposition}[Módulo en espacio extendido 3D]\label{prop:modulo-3D}
El módulo en el espacio extendido satisface:
\[
|\vec{r}| = \sqrt{x^2 + y^2(\varphi + 2)}
\]
\end{proposition}

\begin{proof}[Por cálculo directo]
Por definición del módulo y usando el acoplamiento $z = \varphi y$ (Axioma 3, \ref{ax:extension-ortogonal}):
\[
|\vec{r}|^2 = x^2 + y^2 + z^2 = x^2 + y^2 + \varphi^2 y^2 = x^2 + y^2(1 + \varphi^2)
\]
Usando la identidad $\varphi^2 = \varphi + 1$:
\[
|\vec{r}|^2 = x^2 + y^2(\varphi + 2)
\]
Tomando la raíz cuadrada se obtiene el resultado.
\end{proof}

\begin{corollary}[Factor de amplificación áureo en dirección imaginaria]\label{cor:razon-escalamiento}
En dirección puramente imaginaria ($x=0$), la razón entre módulo 3D y módulo 2D es:
\[
\frac{|\vec{r}|_{3D}}{|z|_{2D}} = \sqrt{1 + \varphi^2} = \sqrt{\varphi + 2} \approx 1.902
\]
\end{corollary}

\begin{proof}[Por cálculo directo]
Para $x=0$, la proposición anterior establece $|\vec{r}|_{3D} = |y|\sqrt{\varphi + 2}$ y $|z|_{2D} = |y|$, por tanto la razón es $\sqrt{\varphi + 2} > 1$, estableciendo amplificación del módulo por este factor áureo.
\end{proof}

\subsubsection{Proyección Isométrica Natural}

\begin{observation}[Ángulo óptimo de observación]\label{obs:angulo-optimo}
Existe un ángulo desde el cual la curva espacial proyecta los tres fasores $P$, $C$, $F$ con separación angular de $120^\circ$.
\end{observation}

\begin{proof}[Por preservación de separación angular]
Las fases de $C$ y $F$ (\dref{def:fases-componentes}) difieren por $2\pi/3$. La rotación acoplada de $z = \varphi y$ preserva esta separación angular al proyectarse sobre cierto plano, revelando la geometría isométrica del triángulo equilátero (\dref{def:magnitudes-tripartitas}).
\end{proof}

\begin{observation}[Origen de $\sqrt{3}$]\label{obs:origen-sqrt3}
Las magnitudes $|P| = 1/\sqrt{3}$ y $|F| = \sqrt{3}/2$ emergen de esta geometría triangular proyectada.
\end{observation}

\subsubsection{Subvariedad en 3D}

\begin{definition}[Subvariedad PCF]\label{def:subvariedad-PCF}
La restricción $z = \varphi y$ con módulo constante $|\Omega| = 1/2$ \newline (Axioma 5, \ref{ax:punto-fijo}) define:
\[
\mathcal{S}_{\text{PCF}}^{3D} = \{(x,y,z) \in \mathbb{R}^3 : x^2 + y^2 + z^2 = 1/4, \; z = \varphi y\}
\]
\end{definition}

\begin{proposition}[Lattice 3D]\label{prop:lattice-3D}
La periodicidad en $(x,y)$ más el acoplamiento $z = \varphi y$ genera:
\[
\Lambda_{3D} = \{(n_1, n_2, \varphi n_2) : n_1, n_2 \in \mathbb{Z}\} \subset \mathbb{R}^3
\]
\end{proposition}

\subsubsection[Visualización del Cilindro Base]{Visualización del Cilindro Base ($\sigma=0$)}

En esta sección usamos coordenadas cartesianas $(x, y, z)$ para $\mathbb{R}^3$, donde $z$ denota la coordenada vertical (altura). Esta convención no debe confundirse con la notación $z = x+iy$ para puntos de $\mathbb{C}$; véase la convención completa en la nota ~\ref{rem:convencion-z}.

La extensión tridimensional del plano complejo mediante el acoplamiento $z = \varphi y$ (Axioma 3, \ref{ax:extension-ortogonal}) introduce grados de libertad rotacionales en la representación geométrica. El sistema admite múltiples orientaciones espaciales equivalentes bajo transformaciones ortogonales (rotaciones y reflexiones) que preservan módulos $\sqrt{x^2 + y^2} = 3$ para todos los vértices, separación angular de $120^\circ$ entre vértices, el acoplamiento áureo $z = \varphi y$ (o su equivalente bajo rotación), y la simetría $S_3$ de estructura tripartita equiláteral.

En esta sección elegimos orientación con cilindro vertical (eje $z$ hacia arriba, círculo en plano $xy$) por conveniencia de notación estándar donde $z$ denota altura. Sin embargo, para visualización isométrica y comprensión topológica del toro, orientaciones alternativas pueden ser más ilustrativas: el cilindro vertical es estándar y fácil de escribir, pero dificulta ver el círculo de frente; el cilindro horizontal muestra el círculo visible frontalmente y revela la topología toroidal con hueco central.

Todas estas orientaciones son transformaciones del mismo objeto geométrico---la elección es puramente pedagógica. Los diagramas que siguen usan orientación vertical para las coordenadas; véase la discusión sobre orientaciones alternativas arriba.

Para visualizar la estructura tripartita del operador $\omegapcf$, consideramos tres vértices de referencia dispuestos sobre un cilindro vertical de radio $R_0 = 3$. Esta construcción geométrica ilustrativa permite entender geométricamente las relaciones entre los tres componentes $P$, $C$, $F$ definidos algebraicamente en \dref{def:componentes-PCF}.

\subsubsection{El Cilindro Vertical}

\begin{definition}[Cilindro base]\label{def:cilindro-base}
El cilindro base es el conjunto de puntos en $\mathbb{R}^3$ que satisfacen:
\[
\mathcal{C}_0 = \{(x, y, z) \in \mathbb{R}^3 : x^2 + y^2 = 9, \; z \in \mathbb{R}\}
\]

con radio horizontal fijo $R_0 = 3$ y extensión infinita en la dirección vertical $\pm z$ (altura).
\end{definition}

\subsubsection{Los Tres Vértices de Referencia}\label{subsubsec:tres-vertices-referencia}

Colocamos tres vértices sobre la superficie del cilindro, separados angularmente por 120° ($2\pi/3$ radianes):

\begin{enumerate}
\item \textbf{Vértice P (Past/Patrón):} Posición angular $\theta_P = 0^\circ$
\[
P_{\text{vert}} = (x_P, y_P, z_P) = (3, 0, 0)
\]

donde la coordenada horizontal es $x_P = 3, y_P = 0$, y la altura vertical es $z_P = 0$.

\item \textbf{Vértice C (Coherence):} Posición angular $\theta_C = 120^\circ$
\[
C_{\text{vert}} = (x_C, y_C, z_C) = (-1.5, 2.598, 4.204)
\]

donde las coordenadas horizontales son $x_C = 3\cos(120^\circ) = -1.5, y_C = 3\sin(120^\circ) \approx 2.598$, y la altura vertical es $z_C = \varphi \cdot y_C \approx 4.204$.

\item \textbf{Vértice F (Future/Flujo):} Posición angular $\theta_F = 240^\circ$
\[
F_{\text{vert}} = (x_F, y_F, z_F) = (-1.5, -2.598, -4.204)
\]

donde las coordenadas horizontales son $x_F = 3\cos(240^\circ) = -1.5, y_F = 3\sin(240^\circ) \approx -2.598$, y la altura vertical es $z_F = \varphi \cdot y_F \approx -4.204$.
\end{enumerate}

\begin{proposition}[Verificación del cilindro]\label{prop:verificacion-cilindro}
Los tres vértices satisfacen la ecuación del cilindro en el plano horizontal:
\[
\sqrt{x^2 + y^2} = 3 \quad \text{para P, C, F}
\]
\end{proposition}

\begin{proof}[Por cálculo directo]
\begin{align*}
\sqrt{x_P^2 + y_P^2} &= \sqrt{3^2 + 0^2} = 3 \\
\sqrt{x_C^2 + y_C^2} &= \sqrt{{(-1.5)}^2 + {(2.598)}^2} = \sqrt{2.25 + 6.75} = 3 \\
\sqrt{x_F^2 + y_F^2} &= \sqrt{{(-1.5)}^2 + {(-2.598)}^2} = \sqrt{2.25 + 6.75} = 3
\end{align*}
\end{proof}

\subsubsection[La Regla de Acoplamiento]{La Regla de Acoplamiento: Altura $z = \varphi y$}

\begin{observation}[Acoplamiento altura-coordenada]\label{obs:acoplamiento-altura}
Las alturas de los vértices no son producto del azar---obedecen la regla de acoplamiento establecida en el Axioma 3 (\ref{ax:extension-ortogonal}):
\[
z = \varphi y
\]

donde ``$z$'' denota la coordenada vertical (altura), mientras que ``$y$'' denota la coordenada horizontal imaginaria.

Esta regla significa que la altura está acoplada a la dirección $y$ mediante la razón áurea:
\begin{enumerate}
\item Si $y > 0$ (dirección $+y$): el vértice sube con pendiente $\varphi \approx 1.618$
\item Si $y < 0$ (dirección $-y$): el vértice baja con pendiente $\varphi$
\item Si $y = 0$: el vértice permanece en altura $z = 0$ (plano $xy$)
\end{enumerate}

La verificación numérica confirma esta regla:
\begin{align*}
z_P &= \varphi \cdot y_P = \varphi \cdot 0 = 0 \\
z_C &= \varphi \cdot y_C = 1.618 \times 2.598 = 4.204 \\
z_F &= \varphi \cdot y_F = 1.618 \times (-2.598) = -4.204
\end{align*}

La consecuencia geométrica inmediata es que el triángulo formado por $P$, $C$, $F$ no está plano en el plano $xy$. Solo el vértice $P$ (donde $y=0$) toca el plano horizontal en altura $z=0$. Los vértices $C$ y $F$ están elevados o hundidos según su coordenada $y$, formando una estructura tridimensional genuina.
\end{observation}

\begin{fullwidth}
\centering
\begin{minipage}{\linewidth}
\includegraphics[width=\linewidth]{src/images/image3.png}
\captionsetup{width=\linewidth,justification=centering}
\captionof{figure}{Visualización 3D completa de los vértices $P=(3, 0, 0)$, $C=(-1.5, 2.598, 4.204)$, $F=(-1.5, -2.598, -4.204)$ y sus proyecciones: vista cenital (círculo en $xy$), frontal (elipse en $xz$), lateral (recta en $yz$ mostrando $z = \varphi y$), e isométrica estándar.}
\label{fig:visualizacion-3d-completa} % chktex 24
\end{minipage}
\end{fullwidth}

\begin{proposition}[Separación angular]\label{prop:separacion-angular-vertices}
En proyección horizontal (vista cenital), los tres vértices están separados por ángulos de 120°:
\[
\angle(P \to C) = \angle(C \to F) = \angle(F \to P) = 120^\circ = \frac{2\pi}{3}
\]

Esta simetría triangular refleja la estructura del grupo $S_3$ (Axioma PCF 2).
\end{proposition}

\subsubsection[Nota Crítica: Vértices vs. Componentes]{Nota Crítica: Vértices vs. Componentes}\label{subsubsec:vertices-vs-componentes}

Los vértices $P_{\text{vert}}, C_{\text{vert}}, F_{\text{vert}}$ descritos en esta subsección son puntos de referencia geométrica en el espacio $\mathbb{R}^3$ que ilustran la estructura tripartita del operador. No deben confundirse con los componentes del operador $P(z,\sigma), C(z), F(z)$ definidos en \ref{def:componentes-PCF}, que son funciones complejas definidas para todo número complejo $z \in \mathbb{C}$:
\begin{align*}
P(z,\sigma) &= \frac{1}{\sqrt{3}}e^{i[\arg(z) + \pi\varepsilon(\sigma)]} \quad \text{(función sobre } \mathbb{C}) \\ % chktex 9
\text{vs.} \quad P_{\text{vert}} &= {(3, 0, 0)} \quad \text{(punto fijo en } \mathbb{R}^3) % chktex 9
\end{align*}

Los vértices geométricos tienen radio horizontal $\sqrt{x^2+y^2} = 3$, mientras que los componentes del operador tienen magnitudes $|P| = 1/\sqrt{3}, |C| = 1, |F| = \sqrt{3}/2$ cuyo producto es exactamente $1/2$.

Esta construcción geométrica sirve para visualizar la disposición espacial tripartita, pero el operador $\omegapcf$ opera funcionalmente sobre todo el plano complejo, no está confinado a estos tres puntos específicos.

\subsubsection{Cierre Topológico: Del Cilindro al Toro}\label{subsubsec:cierre-topologico}

Los vértices geométricos establecidos anteriormente viven sobre el cilindro infinito $\mathcal{C}_0$, pero esta estructura no captura completamente la topología natural del sistema. El acoplamiento $z = \varphi y$ induce una estructura modular que requiere cierre topológico para revelar la geometría completa.

\begin{observation}[Separación vertical de vértices]\label{obs:separacion-vertical}
Los vértices establecidos en §\ref{subsubsec:tres-vertices-referencia} satisfacen:
\[
P_{\text{vert}} = (3, 0, 0), \quad C_{\text{vert}} = (-1.5, 2.598, 4.204), \quad F_{\text{vert}} = (-1.5, -2.598, -4.204)
\]
con coordenadas verticales:
\[
z_P = 0, \quad z_C = \varphi \cdot 2.598 \approx 4.204, \quad z_F = \varphi \cdot (-2.598) \approx -4.204
\]
La separación vertical entre vértices refleja el acoplamiento $z = \varphi y$ del Axioma 3 (\ref{ax:extension-ortogonal}), donde cada vértice ocupa una altura determinada por su coordenada imaginaria $y$.
\end{observation}

El conjunto de puntos del cilindro base $\mathcal{C}_0$ (véase \ref{def:cilindro-base}) que satisfacen el acoplamiento $z = \varphi y$ forma el subconjunto $\mathfrak{C}_0 = \{(x,y,z) \in \mathcal{C}_0 : z = \varphi y\}$.

\begin{proposition}[Ausencia de cierre topológico en el cilindro]\label{prop:ausencia-cierre}
En el cilindro infinito $\mathfrak{C}_0 \subset \mathbb{R}^3$, los tres vértices:
\begin{enumerate}
\item Posan sobre la superficie cilíndrica $x^2 + y^2 = R_0^2$
\item Forman un triángulo equilátero al proyectarse en el plano XY
\item No cierran topológicamente en la dirección vertical $z$
\end{enumerate}
\end{proposition}

\begin{proof}[Por contradicción]
Supongamos que los tres vértices cierran topológicamente en la dirección vertical $z$. Entonces existiría un ciclo cerrado en $\mathbb{R}^3$ que conecta $P$, $C$, $F$ de manera continua.

El cilindro $\mathfrak{C}_0$ tiene topología $\mathfrak{C}_0 \cong S^1$ (círculo) parametrizado por $\theta \in [0, 2\pi)$. Sin embargo, la regla $z = \varphi y$ (Axioma 3, \ref{ax:extension-ortogonal}) introduce dependencia funcional que separa los vértices:
\[
|z_C - z_P| = 4.204, \quad |z_F - z_C| = 8.408, \quad |z_P - z_F| = 4.204
\]

Al completar $\theta = 2\pi$ regresamos a $\theta = 0$, pero los puntos P, C, F permanecen en alturas distintas ($z_P = 0$, $z_C = 4.204$, $z_F = -4.204$), lo cual contradice la existencia de un ciclo cerrado continuo en $\mathbb{R}^3$. Por tanto, los vértices no cierran topológicamente en el cilindro.
\end{proof}

\begin{observation}[El toro como cierre topológico]\label{obs:necesidad-topologica}
Para que los tres vértices formen una estructura cerrada en $\mathbb{R}^3$ (no solo en proyección al plano $xy$), se requiere una superficie que:
\begin{itemize}
\item Contenga el cilindro $\mathfrak{C}_0$ como subvariedad
\item Permita cierre de la coordenada vertical $z$ mediante identificación periódica
\item Tenga topología compatible con las periodicidades del operador
\end{itemize}
El toro $\mathcal{T}_{\text{PCF}}$ satisface estas condiciones, proporcionando el cierre topológico natural donde los vértices forman un círculo cerrado en la sección transversal.
\end{observation}

\begin{definition}[Parametrización del Toro PCF]\label{def:toro-PCF}
El toro estándar con radio mayor $R$ y radio menor $r$ es:
\[
\mathcal{T}(R, r) := \left\{(x,y,z) \in \mathbb{R}^3 : {\left(\sqrt{x^2+y^2} - R\right)}^2 + z^2 = r^2\right\}
\]

El toro PCF se define con parámetros:
\[
\mathcal{T}_{\text{PCF}} := \mathcal{T}(\alpha R_0, R_0) = \mathcal{T}(7.5, 3)
\]
donde $\alpha = 2.5$ (factor de escala para visualización). La parametrización toroidal está dada por:
\[
\Psi(u, v) = \begin{pmatrix} {(R + r\cos v)}\cos u \\ {(R + r\cos v)}\sin u \\ r\sin v \end{pmatrix}, \quad u, v \in {[0, 2\pi)} % chktex 9
\]

donde $u$ parametriza el círculo mayor (poloidal) y $v$ parametriza la sección transversal (toroidal).
\end{definition}

\begin{theorem}[Inmersión del cilindro]\label{thm:inmersion-cilindro}
Existe inmersión natural:
\[
\iota: \mathcal{C}_0 \hookrightarrow \mathcal{T}_{\text{PCF}}
\]
definida por:
\[
\iota(x_c, y_c, z_c) = \Psi(u_0, v), \quad \text{donde } v := \arctan2(z_c, y_c), \quad u_0 := 0
\]
donde $(x_c, y_c, z_c)$ son coordenadas cartesianas en $\mathbb{R}^3$ y $z_c$ denota la coordenada vertical (altura), siguiendo la convención establecida en la Nota~\ref{rem:convencion-z}.
\end{theorem}

\begin{proof}[Por construcción]
Para $(x_c, y_c, z_c) \in \mathcal{C}_0$:

\begin{enumerate}
\item \textit{Determinación unívoca de $v$}: La condición $z_c = \varphi y_c$ determina unívocamente $v = \arctan2(z_c, y_c)$.

\item \textit{Aplicación al toro}: Fijando $u_0 = 0$, la aplicación $\Psi(0, v)$ produce un punto en $\mathcal{T}_{\text{PCF}}$.

\item \textit{Continuidad}: La función $\arctan2$ es continua excepto en el origen, que no afecta a los vértices $P$, $C$, $F$ con $y_c \neq 0$ o $z_c = 0$.
\end{enumerate}

Por tanto, $\iota$ está bien definida y es continua.
\end{proof}

\begin{proposition}[Conjunto imagen de la inmersión]\label{prop:imagen-inmersion}
El conjunto imagen de la inmersión $\iota$ es:
\[
\iota(\mathcal{C}_0) = \{\Psi(0, v) : v \in {[0, 2\pi)}\}
\]
que corresponde a la sección transversal frontal del toro---un círculo $S^1$ de radio $r_{\text{menor}} = R_0 = 3$ en el tubo toroidal, como se ilustra en la Figura~\ref{fig:inmersion-cilindro-toro}.
\end{proposition}

\begin{fullwidth}
\centering
\begin{minipage}{\linewidth}
\includegraphics[width=\linewidth]{src/images/image5.png}
\captionsetup{width=\linewidth,justification=centering}
\captionof{figure}{Búsqueda de perspectiva equilátera: los vértices $P$, $C$, $F$ forman triángulo equilátero en la proyección $XY$ del cilindro (distancias $d(P,C) = d(C,F) = d(F,P) = 5.196$), pero no en el toro 3D donde el acoplamiento $z = \varphi y$ introduce distorsión métrica. La equilateralidad emerge del círculo $S^1$ en proyección, mientras que el toro preserva la topología circular pero distorsiona distancias euclidianas.}
\label{fig:inmersion-cilindro-toro} % chktex 24
\end{minipage}
\end{fullwidth}

\begin{theorem}[Cierre topológico de los vértices en el toro]\label{thm:cierre-topologico}
Mediante la aplicación $\iota$, los vértices $P_{\text{vert}}, C_{\text{vert}}, F_{\text{vert}}$ se transforman en puntos:
\[
P_t := \iota(P_{\text{vert}}), \quad C_t := \iota(C_{\text{vert}}), \quad F_t := \iota(F_{\text{vert}})
\]

que satisfacen las siguientes propiedades:
\begin{enumerate}
\item \textbf{Misma sección transversal}: Los tres puntos comparten la coordenada $u = 0$ en la parametrización toroidal.
\item \textbf{Círculo cerrado}: Los tres puntos forman un círculo $S^1$ de radio $r_{\text{menor}} = 3$ centrado en $(R_{\text{mayor}}, 0, 0) = (7.5, 0, 0)$.
\item \textbf{Conexión topológica}: Los puntos están conectados como subvariedad cerrada en $\mathcal{T}_{\text{PCF}} \cong T^2$.
\end{enumerate}
\end{theorem}

\begin{proof}[Por cálculo directo]
\begin{enumerate}
\item La aplicación $\iota$ se define con $u_0 = 0$ fijo para todos los vértices, por lo que los tres puntos transformados comparten la misma coordenada $u = 0$ en la parametrización toroidal.

\item Las coordenadas toroidales se calculan mediante:
\begin{align*}
v_P &= \arctan2(0, 0) = 0 \\
v_C &= \arctan2(4.204, 2.598) \approx 1.0172 \text{ rad} \\
v_F &= \arctan2(-4.204, -2.598) \approx -2.1244 \text{ rad}
\end{align*}

Los puntos resultantes son:
\begin{align*}
P_t &= (10.5, 0, 0) \\
C_t &= ((7.5 + 3\cos v_C)\cos 0, 0, 3\sin v_C) \approx (9.077, 0, 2.552) \\
F_t &= ((7.5 + 3\cos v_F)\cos 0, 0, 3\sin v_F) \approx (5.923, 0, -2.552)
\end{align*}

Verificando distancias al centro $(R_{\text{mayor}}, 0, 0) = (7.5, 0, 0)$:
\begin{align*}
|P_t - (7.5, 0, 0)| &= |(3, 0, 0)| = 3 \\
|C_t - (7.5, 0, 0)| &= |(1.577, 0, 2.552)| = \sqrt{1.577^2 + 2.552^2} = 3 \\
|F_t - (7.5, 0, 0)| &= |(-1.577, 0, -2.552)| = 3
\end{align*}

Por tanto, los tres puntos están exactamente sobre el círculo de radio $3$.

\item Como $v_P, v_C, v_F \in {(-\pi, \pi]}$, los tres ángulos parametrizan posiciones distintas sobre el mismo círculo $S^1$. El círculo es cerrado por naturaleza, con $v = -\pi$ identificado con $v = \pi$, estableciendo la conexión topológica como subvariedad cerrada en $\mathcal{T}_{\text{PCF}} \cong T^2$.
\end{enumerate}
\end{proof}

\begin{corollary}[Separación vertical cilíndrica versus igualdad radial toroidal]\label{cor:contraste-cilindro}
En el cilindro infinito $\mathfrak{C}_0$:
\[
|z_C - z_P| = 4.204 \neq 0
\]

indicando separación vertical. En el toro:
\[
|P_t - \text{centro}| = |C_t - \text{centro}| = |F_t - \text{centro}| = 3
\]

indicando que los tres vértices comparten la misma subvariedad circular $S^1 \subset T^2$.
\end{corollary}

\begin{proposition}[Topología natural]\label{prop:topologia-natural}
El toro $\mathcal{T}_{\text{PCF}}$ tiene topología:
\[
\mathcal{T}_{\text{PCF}} \cong T^2 = S^1 \times S^1
\]

donde el primer $S^1$ es el círculo mayor (coordenada $u$) y el segundo $S^1$ es el círculo menor---sección transversal (coordenada $v$). Los vértices $P_t, C_t, F_t$ habitan el segundo $S^1$ en $u = 0$.
\end{proposition}

Esta topología emerge como cierre natural del cilindro, proporcionando el contexto estructural donde los vértices forman una subvariedad cerrada y conectando la geometría del cilindro con la topología del toro mediante la inmersión $\iota$.

\begin{theorem}[Proyección al lattice]\label{thm:proyeccion-lattice}
La inmersión del cilindro en el toro anticipa la estructura algebraica:
\[
\mathbb{C}/\Lambda_{\text{PCF}} \cong T^2
\]

donde $\Lambda_{\text{PCF}} = \mathbb{Z}M_{\text{PCF}} \oplus \mathbb{Z}(M_{\text{PCF}} \cdot i)$ es el lattice del operador (\ref{def:lattice-PCF}).
\end{theorem}

La topología $T^2$ aparece en dos lugares: geométricamente como superficie del toro en $\mathbb{R}^3$, y algebraicamente como espacio cociente del plano complejo por el lattice. Esta coincidencia no es accidental---el toro geométrico es el espacio natural donde la estructura periódica del operador se visualiza antes de proyectarse al plano complejo.

\begin{proposition}[Torre auto-similar]\label{prop:torre-auto-similar}
Para cada $\sigma \in \mathbb{N}$, definimos:
\[
\mathcal{T}_\sigma := \mathcal{T}(\alpha R_0 \varphi^\sigma, R_0 \varphi^\sigma)
\]

El escalamiento $S_\sigma(x, y, z) = \varphi^\sigma(x, y, z)$ satisface:
\[
S_\sigma(\mathcal{C}_0) = \mathcal{C}_\sigma, \quad S_\sigma(\mathcal{T}_0) = \mathcal{T}_\sigma
\]

preservando la inmersión:
\[
\iota_\sigma: \mathcal{C}_\sigma \hookrightarrow \mathcal{T}_\sigma
\]

Los vértices escalan coherentemente: $P_{t,\sigma} = \varphi^\sigma P_{t,0}$, y análogamente para C y F.
\end{proposition}

\begin{theorem}[Síntesis: cilindro, toro y topología]\label{thm:sintesis-cilindro-toro}
La estructura geométrica del operador $\omegapcf$ satisface:
\begin{enumerate}
\item Cilindro: Los vértices $P_{\text{vert}}, C_{\text{vert}}, F_{\text{vert}}$ viven en $\mathfrak{C}_0 = \{(x,y,z) : x^2 + y^2 = 9, z = \varphi y\}$
\item Proyección XY: Forman triángulo equilátero en el círculo $|z| = 3$ del plano $\mathbb{C}$
\item Toro: Se cierran topológicamente en $\mathcal{T}_{\text{PCF}}$ formando círculo $S^1$ en la sección transversal
\item Conexión algebraica: Esta topología $T^2$ proyecta al espacio de módulos $\mathcal{M}_{\text{PCF}} = \mathbb{C}/\Lambda_{\text{PCF}}$
\end{enumerate}
\end{theorem}

\begin{observation}[Separación conceptual entre vértices geométricos y componentes funcionales]\label{obs:distincion-esencial}
Recordando §\ref{subsubsec:vertices-vs-componentes}:
\begin{itemize}
\item Los vértices geométricos $P_t, C_t, F_t$ (radio 3 en el toro) son \textbf{representantes visuales}
\item Los componentes funcionales $P(z,\sigma), C(z), F(z)$ (magnitudes $1/\sqrt{3}, 1, \sqrt{3}/2$) operan sobre todo $\mathbb{C}$
\end{itemize}

El toro proporciona el espacio donde los representantes geométricos cierran topológicamente, anticipando la estructura del operador completo.
\end{observation}

\begin{proposition}[Razones estructurales para el toro]\label{prop:por-que-toro}
El toro $\mathcal{T}_{\text{PCF}}$ no es elección arbitraria sino consecuencia de:
\begin{itemize}
\item Periodicidad angular: $\theta \in {[0, 2\pi)}$ en el plano $(x,y)$ % chktex 9
\item Acoplamiento áureo: $z = \varphi y$ relaciona coordenadas
\item Cierre vertical: Necesidad de cerrar la dirección $z$ en subvariedad compacta
\item Topología $T^2$: Única topología compatible con lattice $\Lambda_{\text{PCF}}$ en $\mathbb{C}$
\end{itemize}

El cilindro muestra dónde están los vértices con la regla $z = \varphi y$. El toro muestra cómo se conectan topológicamente en una subvariedad cerrada $S^1 \subset T^2$.
\end{proposition}

\subsubsection[Isomorfismo Bidireccional: C leftrightarrow R3]{Isomorfismo Bidireccional: $\mathbb{C} \leftrightarrow \mathbb{R}^3$}

\begin{theorem}[Correspondencia biunívoca mediante acoplamiento áureo]\label{thm:isomorfismo-bidireccional}
La regla $z = \varphi y$ (Axioma 3) establece una correspondencia biunívoca entre el plano complejo y el espacio de configuración 3D.
\end{theorem}

\begin{proof}[Por construcción]
Se definen dos aplicaciones:

\textbf{Extensión} $\psi: \mathbb{C} \to \mathbb{R}^3$:
\[
\psi(x + iy) = (x, y, \varphi y)
\]

\textbf{Proyección} $\pi: \mathbb{R}^3 \to \mathbb{C}$:
\[
\pi(x, y, z) = x + iy \quad \text{(válida cuando } z = \varphi y) % chktex 1 9
\]

Estas aplicaciones satisfacen:
\[
\pi \circ \psi = \text{id}_{\mathbb{C}}, \quad \psi \circ \pi|_{\mathcal{S}_{\text{PCF}}} = \text{id}_{\mathcal{S}_{\text{PCF}}}
\]

donde $\mathcal{S}_{\text{PCF}} = \{(x, y, \varphi y) : x, y \in \mathbb{R}\}$.

\textit{Verificación de $\pi \circ \psi = \text{id}_\mathbb{C}$}: Para todo $x + iy \in \mathbb{C}$, se tiene
\[
\pi(\psi(x + iy)) = \pi(x, y, \varphi y) = x + iy,
\]
lo cual establece la identidad sobre $\mathbb{C}$.

\textit{Verificación de $\psi \circ \pi|_{\mathcal{S}_{\text{PCF}}} = \text{id}_{\mathcal{S}_{\text{PCF}}}$}: Para $(x, y, z) \in \mathcal{S}_{\text{PCF}}$, se obtiene
\[
\psi(\pi(x, y, z)) = \psi(x + iy) = (x, y, \varphi y).
\]
Dado que $z = \varphi y$ por definición de $\mathcal{S}_{\text{PCF}}$, se concluye que
\[
(x, y, \varphi y) = (x, y, z)
\]
estableciendo la identidad sobre $\mathcal{S}_{\text{PCF}}$.

Por tanto, $\psi$ y $\pi$ son mutuamente inversas en sus respectivos dominios.
\end{proof}

\begin{corollary}[Preservación bidireccional de información]\label{cor:dos-direcciones-sin-perdida}
Las aplicaciones $\psi$ y $\pi$ preservan información en ambas direcciones:
\begin{enumerate}
\item \textbf{Dirección 2D $\to$ 3D}: Dado $z = x + iy \in \mathbb{C}$, la extensión $\psi$ produce $(x, y, \varphi y) \in \mathbb{R}^3$, preservando toda la información del plano complejo en el espacio 3D.
\item \textbf{Dirección 3D $\to$ 2D}: Dado $(x, y, z) \in \mathcal{S}_{\text{PCF}}$ con $z = \varphi y$, la proyección $\pi$ produce $x + iy \in \mathbb{C}$, recuperando completamente el plano complejo desde $\mathcal{S}_{\text{PCF}}$.
\end{enumerate}

No hay pérdida de información en ninguna dirección.
\end{corollary}

\begin{proposition}[Aplicación a los vértices]\label{prop:aplicacion-vertices}
Aplicando las transformaciones $\psi$ y $\pi$ a los vértices:

\textbf{Extensión 2D $\to$ 3D}:
\begin{align*}
z = 3 &\xrightarrow{\psi} (3, 0, 0) = P_{\text{vert}} \\
z = -\frac{3}{2} + i\frac{3\sqrt{3}}{2} &\xrightarrow{\psi} (-1.5, 2.598, 4.204) = C_{\text{vert}} \\
z = -\frac{3}{2} - i\frac{3\sqrt{3}}{2} &\xrightarrow{\psi} (-1.5, -2.598, -4.204) = F_{\text{vert}}
\end{align*}

\textbf{Proyección 3D $\to$ 2D}:
\begin{align*}
P_{\text{vert}} = (3, 0, 0) &\xrightarrow{\pi} 3 \\
C_{\text{vert}} = (-1.5, 2.598, 4.204) &\xrightarrow{\pi} -1.5 + 2.598i \\
F_{\text{vert}} = (-1.5, -2.598, -4.204) &\xrightarrow{\pi} -1.5 - 2.598i
\end{align*}
\end{proposition}

\begin{proof}[Por cálculo directo]
Se verifica que $z_C = \varphi \cdot 2.598 = 4.204$ y $z_F = \varphi \cdot (-2.598) = -4.204$, confirmando la regla de acoplamiento $z = \varphi y$ del Axioma 3 (\ref{ax:extension-ortogonal}).
\end{proof}

\begin{theorem}[Dimensión efectiva]\label{thm:dimension-efectiva}
El espacio $\mathcal{S}_{\text{PCF}}$ tiene:
\begin{itemize}
\item Dimensión aparente: 3 (coordenadas $x, y, z$)
\item Dimensión efectiva: 2 ($z$ determinada por $y$)
\item Grados de libertad: 2 ($x$ y $y$ independientes)
\end{itemize}

Por tanto $\mathcal{S}_{\text{PCF}} \cong \mathbb{R}^2 \cong \mathbb{C}$.
\end{theorem}

\begin{proposition}[Preservación de estructura]\label{prop:preservacion-estructura}
El isomorfismo preserva:
\begin{enumerate}
\item Módulo radial: $|z| = \sqrt{x^2 + y^2}$ en ambas direcciones
\item Ángulos: $\arg(z) = \arctan2(y, x)$ consistente en ambas direcciones
\item Triángulo equilátero: Separación 120° en $\mathbb{C}$ $\leftrightarrow$ estructura 3D con $z = \varphi y$ (ver \dref{def:magnitudes-tripartitas})
\end{enumerate}
\end{proposition}

\begin{observation}[Coherencia del sistema]\label{obs:coherencia-sistema}
La bidireccionalidad establecida en \tref{thm:isomorfismo-bidireccional} y \corref{cor:dos-direcciones-sin-perdida} explica:
\begin{enumerate}
\item El toro $\mathcal{T}_{\text{PCF}}$ (topológicamente 2D, geométricamente en $\mathbb{R}^3$) se representa completamente en $\mathbb{C}$ mediante el isomorfismo $\mathbb{C} \cong \mathcal{S}_{\text{PCF}} \cong \mathcal{T}_{\text{PCF}}$ (\tref{thm:dimension-efectiva}).
\item Los componentes $P(z,\sigma), C(z), F(z)$ operan sobre $\mathbb{C}$ con acceso completo a la geometría 3D a través del acoplamiento $z = \varphi y$ (Axioma~\ref{ax:extension-ortogonal}).
\item El lattice $\Lambda_{\text{PCF}}$ que construiremos en §\ref{subsec:toro-lattice} captura la estructura completa del toro.
\end{enumerate}

El acoplamiento $z = \varphi y$ establece que el espacio 3D siempre tuvo solo 2 grados de libertad independientes. La geometría del toro en $\mathbb{R}^3$ y la estructura algebraica en $\mathbb{C}$ son dos representaciones isomorfas del mismo objeto matemático.
\end{observation}

Con el isomorfismo $\mathbb{C} \cong \mathcal{S}_{\text{PCF}} \cong \mathcal{T}_{\text{PCF}}$ establecido (\tref{thm:dimension-efectiva}), procedemos a construir el lattice $\Lambda_{\text{PCF}}$ en el plano complejo, sabiendo que captura toda la información del toro tridimensional. La topología $T^2$ del toro reaparecerá como la topología natural del espacio de módulos $\mathcal{M}_{\text{PCF}} = \mathbb{C}/\Lambda_{\text{PCF}}$, conectando visualización geométrica con estructura algebraica.

\subsection{Proyección al Plano Complejo y Estructura del Lattice}\label{subsec:toro-lattice}

\textbf{Objetivo de esta sección:} Mostrar cómo el operador $\Omega(z,\sigma)$, con estructura tripartita (tipo Eisenstein, 120°), genera un lattice rectangular (tipo Gauss, 90°) en $\mathbb{C}$, y establecer el espacio de módulos $\mathcal{M}_{\text{PCF}} = \mathbb{C}/\Lambda_{\text{PCF}}$.

\subsubsection{Proyección Vertical}

\begin{definition}[Proyección vertical al plano complejo]\label{def:proyeccion-vertical}
El mapa $\pi: \mathbb{R}^3 \to \mathbb{C}$ dado por:
\[
\pi(x, y, z) = x + iy
\]

proyecta la geometría 3D del toro al plano complejo.
\end{definition}

\begin{proposition}[Vértices proyectados]\label{prop:vertices-proyectados}
La proyección de los vértices del cilindro es:
\[
\pi(P_{\text{vert}}) = 3, \quad \pi(C_{\text{vert}}) = -\frac{3}{2} + i\frac{3\sqrt{3}}{2}, \quad \pi(F_{\text{vert}}) = -\frac{3}{2} - i\frac{3\sqrt{3}}{2}
\]

formando un triángulo equilátero en el círculo $|z| = 3$.
\end{proposition}

\begin{observation}[Proyección angular]\label{obs:proyeccion-angular}
La proyección colapsa la coordenada $z$, pero preserva la estructura angular 120° en el plano complejo.
\end{observation}

\subsubsection{Periodicidades y Generación del Lattice}

\begin{theorem}[Períodos del operador]\label{thm:periodos-operador}
El operador $\omegapcf$ exhibe dos periodicidades independientes:
\begin{enumerate}
\item \textbf{Periodicidad de fase}: $\arg(\Omega(z,\sigma)) = 3\arg(z) + \pi\varepsilon(\sigma) + 2\pi$
\item \textbf{Periodicidad temporal}: $\tau(\sigma)\varphi^\sigma = M_{\text{PCF}}$, donde $M_{\text{PCF}} = \pi/\varepsilon_0$
\end{enumerate}
\end{theorem}

\begin{definition}[Lattice PCF]\label{def:lattice-PCF}
El lattice generado por el operador es:
\[
\Lambda_{\text{PCF}} := \mathbb{Z}M_{\text{PCF}} \oplus \mathbb{Z}(M_{\text{PCF}} \cdot i) = \{m M_{\text{PCF}} + n M_{\text{PCF}} \cdot i : m, n \in \mathbb{Z}\}
\]

donde $M_{\text{PCF}} = \pi/\varepsilon_0 = 6\sqrt{3}\pi/\ln \varphi \approx 67.846189258071644\ldots$ es el módulo topológico que sintetiza la estructura periódica emergente de las rotaciones de fase acumuladas del operador. Geométricamente, $M_{\text{PCF}}$ representa el período fundamental en el plano complejo que estructura el lattice; topológicamente, clasifica el toro $\mathbb{C}/\Lambda_{\text{PCF}} \cong T^2$ (\corref{cor:espacio-cociente}).
\end{definition}

\begin{theorem}[Emergencia del lattice desde periodicidades del operador]\label{thm:generacion-operador}
Las periodicidades del operador inducen las identificaciones:
\[
z \sim z + M_{\text{PCF}}, \quad z \sim z + M_{\text{PCF}} \cdot i
\]

que definen $\Lambda_{\text{PCF}}$.
\end{theorem}

\begin{proof}[Por construcción] Se verifica que:
\newline
\begin{enumerate}
\item \textbf{Período fundamental}: La ecuación de acoplamiento $\tau(\sigma)\varphi^\sigma = \text{constante}$ determina el período temporal fundamental $M_{\text{PCF}} = \pi/\varepsilon_0$.

\item \textbf{Direcciones independientes}: La estructura compleja de $\mathbb{C} = \mathbb{R}^2$ con base $\{1, i\}$ introduce dos direcciones independientes sobre $\mathbb{R}$.

\item \textbf{Identificaciones}: Las funciones periódicas respecto al operador satisfacen:
\[
f(z + M_{\text{PCF}}) = f(z), \quad f(z + M_{\text{PCF}} \cdot i) = f(z)
\]

\item \textbf{Lattice minimal}: $\Lambda_{\text{PCF}}$ es el lattice minimal (subgrupo discreto) que respeta estas identificaciones.
\end{enumerate}
\end{proof}

\begin{corollary}[Periodicidades que generan el toro]\label{cor:espacio-cociente}
El espacio cociente es:
\[
\mathbb{C}/\Lambda_{\text{PCF}} \cong T^2
\]

topológicamente un toro.
\end{corollary}

\subsubsection{Dualidad Estructural: Eisenstein y Gauss}

\begin{observation}[Dos estructuras lattice en $\mathbb{C}$]\label{obs:dos-estructuras-lattice}
El plano complejo admite dos estructuras lattice canónicas:

Gauss ($\mathbb{Z}[i]$):
\begin{itemize}
\item Base: $\{1, i\}$
\item Ángulo: $90^\circ$
\item Geometría: cuadrado
\end{itemize}

Eisenstein ($\mathbb{Z}[\omega]$):
\begin{itemize}
\item Base: $\{1, \omega\}$ donde $\omega = e^{2\pi i/3}$
\item Ángulo: $120^\circ$
\item Geometría: hexagonal
\end{itemize}

\end{observation}

\begin{theorem}[Dualidad entre estructuras Eisenstein y Gauss]\label{thm:dualidad-PCF}
El operador $\omegapcf$ mantiene coherencia entre ambas estructuras mediante el invariante $|\Omega| = 1/2$:

\mbox{}\par\vspace*{1em}

\begin{center}
% chktex-file 44
\begin{tabular}{|l|l|l|}
\hline
\textbf{Aspecto} & \textbf{Tipo Eisenstein} & \textbf{Tipo Gauss} \\
\hline
Componentes P, C, F & Separación $2\pi/3$ ($\omega$) & --- \\
Lattice $\Lambda_{\text{PCF}}$ & --- & Base $\{M, Mi\}$ \\
Invariante & $|\Omega| = 1/2$ & $|\Omega| = 1/2$ \\
\hline
\end{tabular}
% chktex-file 0
\end{center}

\textit{Interpretación}: $\Omega$ es generador tripartito (estructura $\omega$, 120°) que induce lattice rectangular (estructura $i$, 90°).
\end{theorem}

\begin{proposition}[Mecanismo de mediación por $\varphi$]\label{prop:mediacion-phi}
La razón áurea conecta ambas estructuras mediante el acoplamiento $z = \varphi y$ (Axioma 3, \ref{ax:extension-ortogonal}):
\begin{itemize}
\item Entrada: simetría tripartita ($\omega^3 = 1$)
\item Salida: lattice rectangular ($i^2 = -1$)
\item Mediador: $\varphi^2 = \varphi + 1$
\end{itemize}
\end{proposition}

\begin{observation}[Unificación de estructuras lattice duales]\label{obs:unificacion-lattice}
El operador $\omegapcf$ unifica exitosamente estructuras tipo Eisenstein (separación angular $2\pi/3$) y tipo Gauss (lattice rectangular) en el mismo espacio $\mathbb{C}$. Esta coexistencia se garantiza mediante el invariante constante $|\Omega| = 1/2$ (Corolario~\ref{cor:modulo-constante}) y el mecanismo de mediación por $\varphi$ (\pref{prop:mediacion-phi}), estableciendo coherencia estructural entre ambas representaciones (\tref{thm:dualidad-PCF}).
\end{observation}


\subsubsection{Espacio de Módulos}

\begin{definition}[Espacio de módulos PCF]\label{def:espacio-modulos-PCF}
El espacio de módulos es:
\[
\mathcal{M}_{\text{PCF}} := \mathbb{C}/\Lambda_{\text{PCF}}
\]
\end{definition}

\begin{proposition}[Topología del espacio de módulos]\label{prop:topologia-modulos}
$\mathcal{M}_{\text{PCF}}$ tiene topología de toro $T^2 = S^1 \times S^1$, consistente con §\ref{subsubsec:cierre-topologico}.
\end{proposition}

\begin{observation}[Conexión con §\ref{sec:plano-complejo-modulos}]\label{obs:conexion-curvas-elipticas}
El espacio $\mathcal{M}_{\text{PCF}}$ comparte estructura topológica con el espacio de módulos de curvas elípticas:
\[
\mathcal{M}_{\text{curvas}} = \mathbb{H}/\text{PSL}_2(\mathbb{Z})
\]
Ambos son espacios cocientes con topología $T^2$.
\end{observation}

\begin{theorem}[Parámetro modular]\label{thm:parametro-modular}
El lattice $\Lambda_{\text{PCF}}$ tiene parámetro modular:
\[
\tau_{\text{PCF}} := \frac{M_{\text{PCF}} \cdot i}{M_{\text{PCF}}} = i
\]
indicando lattice rectangular (no cuadrado, no hexagonal).
\end{theorem}

\begin{observation}[$\tau = i$: de la clasificación general a la dualidad preservada]\label{obs:clasificacion-parametros}
El valor $\tau = i$ es especial en la clasificación de lattices. Los parámetros modulares caracterizan diferentes estructuras lattice:
\begin{itemize}
\item $\tau = i$: lattice de Gauss (simetría cuadrada $\mathbb{Z}_4$)
\item $\tau = \omega$: lattice de Eisenstein (simetría hexagonal $\mathbb{Z}_6$)
\item $\tau_{\text{PCF}} = i$: lattice rectangular con simetría $\mathbb{Z}_4$ en el operador $\omegapcf$
\end{itemize}
La elección $\tau_{\text{PCF}} = i$ en el operador $\omegapcf$ privilegia estructura rectangular sobre hexagonal, aunque ambas coexisten virtualmente en el invariante $|\Omega| = 1/2$.
\end{observation}

\subsubsection{Síntesis: Proyección, Lattice y Coherencia Dual}

\begin{theorem}[Teorema Principal: Proyección y estructura lattice]\label{thm:sintesis-proyeccion-lattice}
El operador $\omegapcf$ satisface:
\begin{enumerate}
\item \textbf{Proyección}: $\pi: \mathcal{T}_{\text{PCF}} \to \mathbb{C}$ colapsa geometría 3D preservando estructura angular
\item \textbf{Lattice}: $\Lambda_{\text{PCF}} = \mathbb{Z}M \oplus \mathbb{Z}Mi$ (suma directa) generando $T^2 = S^1 \times S^1$ (producto cartesiano) mediante periodicidades del operador
\item \textbf{Dualidad}: Componentes tripartitos ($\omega$) coexisten con lattice rectangular ($i$)
\item \textbf{Invariante}: $|\Omega(z,\sigma)| = 1/2$ constante bajo ambas estructuras
\item \textbf{Espacio de módulos}: $\mathcal{M}_{\text{PCF}} = \mathbb{C}/\Lambda_{\text{PCF}} \cong T^2$
\end{enumerate}

El operador no elige entre Eisenstein o Gauss---mantiene coherencia entre ambos vía el invariante $|\Omega| = 1/2$ y la mediación de $\varphi$.
\end{theorem}

\subsection[Dimension sigma: Torre de Escalas]{Dimensión $\sigma$: Torre de Escalas}

La construcción del operador hasta ahora ha usado un parámetro $\sigma$ sin explicar su naturaleza geométrica. En esta sección formalizamos $\sigma$ como coordenada de escala que parametriza una familia infinita de funciones.
\vspace{1em}
\begin{observation}[Naturaleza del parámetro $\sigma$]\label{obs:naturaleza-sigma}
El parámetro $\sigma \in \mathbb{R}$ no representa una dimensión espacial adicional (no existe un ``eje $\sigma$'' separado de $x, y, z$). En cambio, $\sigma$ es una coordenada de escala que parametriza una familia infinita de funciones:
\[
\{\Omega(z, \sigma) : \mathbb{C} \to \mathbb{C}\}_{\sigma \in \mathbb{R}} % chktex 3
\]

Cada valor fijo de $\sigma$ especifica una función diferente sobre el mismo dominio $\mathbb{C}$. Esta familia admite dos interpretaciones equivalentes:
\end{observation}
\begin{enumerate}
\item Geométrica: Cada $\sigma$ define un círculo en $\mathbb{C}$ con radio efectivo $R_\sigma = R_0\varphi^\sigma$
\item Analítica: Cada $\sigma$ define un espacio de funciones $F_\sigma$ con dispersión y frecuencia características
\end{enumerate}

El parámetro $\sigma$ actúa como lente de observación o nivel de magnificación que permite explorar la estructura PCF a diferentes escalas, manteniendo la propiedad fundamental $|\Omega(z,\sigma)| = 1/2$ constante para todo $\sigma$.

Contraste con extensiones dimensionales previas: en §\ref{subsec:geometria-3d} (Axioma PCF 3), introdujimos $z = \varphi y$ como coordenada espacial adicional en $\mathbb{R}^3$; aquí, $\sigma$ no añade dimensión espacial, sino estructura escalar sobre el mismo espacio.

Esta distinción es crucial: el operador habita una familia de funciones $\mathbb{C} \to \mathbb{C}$ (plano complejo $\times$ escala), no $\mathbb{C} \times \mathbb{R} \times \mathbb{R}$ (tres dimensiones espaciales).

\subsubsection{Familia de Círculos sin Ejes Adicionales}

\begin{observation}[Círculo base con parámetro]\label{obs:circulo-base-parametro}
El círculo de §\ref{subsec:geometria-3d} tiene radio fijo $r$. El operador habita una familia infinita parametrizada por escala.
\end{observation}

\begin{definition}[Familia paramétrica]\label{def:familia-parametrica}
La familia de curvas espaciales es:
\[
\mathcal{C}_\sigma = \left\{\vec{r}_\sigma(t) = r_0\varphi^\sigma\begin{pmatrix} \cos t \\ \sin t \\ \varphi\sin t \end{pmatrix} : \sigma \in \mathbb{R}\right\}
\]
\end{definition}

\begin{proposition}[$\sigma$ como coordenada escalar pura]\label{prop:sigma-escalar-puro}
La coordenada $\sigma$ parametriza escalas sin requerir ejes espaciales $(x_\sigma, y_\sigma)$ adicionales. Cada valor de $\sigma$ especifica el radio de la curva espacial en el mismo espacio $(x,y,z)$.
\end{proposition}

\begin{fullwidth}
\centering
\begin{minipage}{\linewidth}
\includegraphics[width=\linewidth]{src/images/image2.png}
\captionsetup{width=\linewidth,justification=centering}
\captionof{figure}{Relación geométrica P-C-F para los primeros 4 niveles $\sigma$: puntos P (rojos), C (verdes) y F (azules) forman triángulos que escalan radialmente con $\sigma$ (radios $R_\sigma = 3.00, 4.8, 7.8, 12.71$), contenidos en cilindros concéntricos translúcidos. Las líneas cian conectan puntos dentro de cada nivel; las líneas punteadas muestran la progresión de cada tipo de punto entre niveles, revelando la estructura autosimilar del operador.}
\label{fig:scale_cone} % chktex 24
\end{minipage}
\end{fullwidth}

Esta estructura autosimilar genera un cono análogo a los conos de luz del principio de Fermat en óptica geométrica, donde trayectorias de luz minimizan tiempo de propagación generando superficies cónicas continuas. Sin embargo, aquí la geometría difiere en dos aspectos fundamentales: el cono emerge de un ángulo distinto (determinado por el escalamiento áureo $\varphi$ en lugar de propagación luminosa) y está dividido entre escalas discretas en lugar de formar una superficie continua. Formalmente, la parametrización discreta es:
\[
\sigma \mapsto r_0\varphi^\sigma, \quad \sigma \in \mathbb{Z}
\]
donde cada nivel $\sigma$ corresponde a un círculo de radio $r_0\varphi^\sigma$, estableciendo una partición del cono en niveles escalares discretos en lugar de una generatriz continua.

\subsubsection{Lattice Vertical Multiplicativo}

\begin{proposition}[Estructura discreta del lattice vertical]\label{prop:estructura-discreta}
Para $\sigma \in \mathbb{Z}$, los radios forman:
\[
\Lambda_{\text{vertical}} = \{r_0\varphi^n : n \in \mathbb{Z}\}
\]

En espacio logarítmico:
\[
\ln(\Lambda_{\text{vertical}}) = \ln(r_0) + \ln(\varphi)\mathbb{Z}
\]
\end{proposition}

\begin{observation}[Comparación con lattices clásicos]\label{obs:comparacion-lattices}

\mbox{}\par\vspace*{1em}

% chktex-file 44
\begin{tabular}{|l|l|l|l|l|}
\hline
\textbf{Lattice} & \textbf{Dimensión} & \textbf{Operación} & \textbf{Generador} & \textbf{Espacio} \\
\midrule
$\mathbb{Z}[i]$ & 2D & Suma & $\{1, i\}$ & $\mathbb{C}$ \\
$\Lambda_{3D}$ & 3D & Suma & $\{(1,0,0), (0,1,\varphi)\}$ & $\mathbb{R}^3$ \\
$\Lambda_{\text{vertical}}$ & 1D & Multiplicación & $\{\varphi\}$ & $\mathbb{R}_+$ \\
\bottomrule
\end{tabular}
% chktex-file 0
\end{observation}

\begin{corollary}[Lattice vertical sin dimensión adicional]\label{cor:lattice-vertical-sin-dimension}
El lattice vertical $\Lambda_{\text{vertical}} = \{r_0\varphi^n : n \in \mathbb{Z}\}$ (Proposición~\ref{prop:estructura-discreta}) opera sobre escalas en $\mathbb{R}_+$, no sobre coordenadas espaciales. Por tanto, no añade dimensión al espacio $\mathbb{R}^3$ o $\mathbb{C}$, sino que parametriza una familia discreta de escalas mediante la multiplicación por potencias de $\varphi$.
\end{corollary}

\subsubsection{Invariancia de la Razón de Módulos}

\begin{proposition}[Razón de módulos 3D/2D]\label{prop:razon-modulos-3d-2d}
La razón entre módulo 3D y módulo 2D es:
\[
\frac{|\vec{r}|_{3D}}{|z|_{2D}} = \sqrt{1 + \frac{\varphi^2y^2}{x^2+y^2}}
\]
independiente de $\sigma$.
\end{proposition}

\begin{corollary}[Razón constante en el eje imaginario]\label{cor:razon-eje-imaginario}
Para $x=0$: esta razón es exactamente $\sqrt{1+\varphi^2} = \sqrt{\varphi+2} \approx 1.902$ en todos los niveles de escala.
\end{corollary}

\subsubsection{Espacio Adjunto}

\begin{definition}[Base extendida del espacio adjunto]\label{def:base-extendida-espacio-adjunto}
El espacio adjunto se parametriza por la base extendida $\{1, i, \varphi\}$ donde:
\begin{itemize}
\item $1$: unidad real (eje $x$)
\item $i$: unidad imaginaria (eje $y$, rotación 90°)
\item $\varphi$: unidad escalar (escalamiento geométrico)
\end{itemize}

La estructura completa del espacio adjunto es:
\[
\mathcal{E}_{\text{adjunto}} = \mathbb{C} \times \mathbb{R}_+ \times S^1
\]
con coordenadas $(z, \sigma, \theta) \in \mathbb{C} \times \mathbb{R} \times S^1$.
\end{definition}

\begin{proposition}[Métrica del espacio adjunto]\label{prop:metrica-espacio-adjunto}
La métrica del espacio adjunto es:
\[
ds^2 = |dz|^2 + \ln^2(\varphi)d\sigma^2 + d\theta^2
\]

Esta métrica unifica:
\begin{itemize}
\item Distancia euclidiana en $\mathbb{C}$: $|dz|^2$
\item Distancia logarítmica en escalas: $\ln^2(\varphi)d\sigma^2$
\item Distancia angular en fase: $d\theta^2$
\end{itemize}
\end{proposition}

\subsubsection{Ecuación de Acoplamiento Temporal}

Las ecuaciones de acoplamiento conectan la dinámica temporal (evolución en $\sigma$) con la geometría espacial (argumento $z$ en el plano complejo).

\begin{theorem}[Ecuación de Acoplamiento Temporal]\label{thm:acoplamiento-temporal}
El operador satisface la ecuación de escalamiento temporal:
\[
\Omega(z,\sigma+1) = \Omega(z,\sigma) \cdot e^{i\Delta\phi(\sigma)}
\]

donde la fase de acoplamiento es:
\[
\Delta\phi(\sigma) = \pi\varepsilon(\sigma) \cdot (\varphi - 1) = \pi\varepsilon(\sigma) / \varphi
\]
\end{theorem}

\textit{Interpretación}: Avanzar un nivel $\sigma \to \sigma+1$ (lo cual corresponde a ``multiplicar por $\varphi$'' en el espacio de parámetros) equivale a multiplicar el operador por un factor de fase que depende del nivel actual.

\begin{proof}[Por cálculo directo]
Por la Proposición~\ref{prop:formula-fase-explicita} y la definición del operador (\dref{def:operador-PCF-completo}):
\begin{align*}
\Omega(z,\sigma) &= \frac{1}{2}e^{i[3\arg(z) + \pi\varepsilon(\sigma)]} \\
\Omega(z,\sigma+1) &= \frac{1}{2}e^{i[3\arg(z) + \pi\varepsilon(\sigma+1)]}
\end{align*}

Tomando el cociente:
\begin{align*}
\frac{\Omega(z,\sigma+1)}{\Omega(z,\sigma)} &= \frac{e^{i[3\arg(z) + \pi\varepsilon(\sigma+1)]}}{e^{i[3\arg(z) + \pi\varepsilon(\sigma)]}} \\
&= e^{i\pi[\varepsilon(\sigma+1) - \varepsilon(\sigma)]} \\
&= e^{i\pi[\varepsilon(\sigma)(\varphi - 1)]}
\end{align*}

Se concluye que:
\[
\Omega(z,\sigma+1) = \Omega(z,\sigma) \cdot e^{i\pi\varepsilon(\sigma)(\varphi - 1)},
\]
lo cual establece la ecuación de acoplamiento temporal.

\par
En notación alternativa, escribiendo $\Omega(\varphi \cdot z)$ para denotar el avance temporal $\sigma \to \sigma+1$, la ecuación se expresa simbólicamente como:
\[
\boxed{\Omega(\varphi \cdot z, \sigma) \equiv \Omega(z, \sigma+1) = \Omega(z, \sigma) \cdot e^{i\varepsilon(\sigma)}}
\]

\par
En esta expresión, el factor $\pi(\varphi-1)$ se absorbe en la definición de $\varepsilon$ efectivo, simplificando la notación sin pérdida de generalidad.
\end{proof}

\subsubsection{Ecuación de Acoplamiento Óptimo}\label{subsubsec:acoplamiento-optimo}

\begin{theorem}[Ecuación de acoplamiento óptimo]\label{thm:acoplamiento-optimo}
Para cada nivel $\sigma \in \mathbb{N}$, existe un ángulo crítico $\arg(z)_{\text{crit}}{(\sigma)} \in \mathbb{R}$ tal que: % chktex 3
\[
\boxed{\frac{\lbrack\arg(\Omega(z_{\text{crit}}, \sigma)) - 2\pi\rbrack}{\log(\varphi)} + \frac{\log(\varepsilon(\sigma))}{\log(\varphi)} = 1}
\]
donde $z_{\text{crit}}$ satisface $\arg(z_{\text{crit}}) = \arg(z)_{\text{crit}}{(\sigma)}$. % chktex 3
\end{theorem}

\begin{proof}[Por sustitución]
Sustituyendo $\arg(\Omega) = 3\arg(z) + \pi\varepsilon(\sigma) + 2\pi$ y $\log(\varepsilon(\sigma)) = \log(\varepsilon_0) + \sigma\log(\varphi)$ en la ecuación y simplificando se obtiene:
\[
\frac{3\arg(z)}{\log(\varphi)} + \frac{\pi\varepsilon(\sigma)}{\log(\varphi)} + \frac{\log(\varepsilon_0)}{\log(\varphi)} + \sigma = 1
\]
Despejando $\arg(z)$ y usando $\log(\varepsilon_0) = \log(\varepsilon(\sigma)) - \sigma\log(\varphi)$:
\[
\arg(z)_{\text{crit}}{(\sigma)} = \frac{\log(\varphi) - \pi\varepsilon(\sigma) - \log(\varepsilon(\sigma))}{3} % chktex 3
\]
\end{proof}

\textit{Nota sobre fase efectiva}: La expresión $[\arg(\Omega) - 2\pi]$ en el teorema corresponde a la fase efectiva, dado que $\arg(\Omega) = 3\arg(z) + \pi \cdot \varepsilon(\sigma) + 2\pi$ por \pref{prop:formula-fase-explicita} y $e^{2\pi i} = 1$. Alternativamente, la ecuación puede escribirse como $\arg(\Omega)/\log(\varphi) + \log(\varepsilon)/\log(\varphi) = 1$ (mod $2\pi$).

\subsubsection{Tabla de Ángulos Críticos}

\begin{proposition}[Ángulos críticos]\label{prop:angulos-criticos}
Los primeros ángulos críticos son:

\mbox{}
\begin{center}
% chktex-file 44
\begin{tabular}{|c|c|c|c|}
\hline
$\sigma$ & $\arg(z)_{\text{crit}}{(\sigma)}$ {[rad]} & En grados & En términos de $\pi$ \\ % chktex 3
\hline
1 & 0.9457 & 54.19° & 0.301$\pi$ \\
2 & 0.7368 & 42.22° & 0.235$\pi$ \\
3 & 0.4980 & 28.53° & 0.159$\pi$ \\
5 & -0.1552 & -8.89° & -0.049$\pi$ \\
7 & -1.3461 & -77.13° & -0.428$\pi$ \\
10 & -6.3834 & -365.74° & -2.032$\pi$ \\
15 & -67.362 & -3859.56° & -21.442$\pi$ \\
20 & -735.533 & -42142.95° & -234.127$\pi$ \\
\hline
\end{tabular}
% chktex-file 0
\end{center}
\end{proposition}

\begin{observation}[Espiral de ángulos críticos]\label{obs:espiral-angulos-criticos}
Los ángulos críticos forman una espiral logarítmica que diverge para $\sigma \to \infty$. Para valores pequeños de $\sigma$, los ángulos están en el primer y segundo cuadrante, luego cruzan al tercer cuadrante y continúan en espiral descendente.
\end{observation}

\subsubsection{Significado Geométrico de Ángulos Críticos}

\begin{observation}[Resonancia geométrica]\label{obs:resonancia-geometrica}
En los ángulos críticos $\arg(z) = \arg(z)_{\text{crit}}{(\sigma)}$, el operador $\Omega(z,\sigma)$ satisface una condición de resonancia donde: % chktex 3
\begin{enumerate}
\item \textbf{Acoplamiento geométrico-aritmético óptimo}: La componente geométrica $3\arg(z)$ y la componente logarítmica $\log(\varepsilon)$ se balancean para satisfacer la ecuación de acoplamiento.
\item \textbf{Direcciones privilegiadas}: Estas direcciones en el plano complejo corresponden a vectores $z$ donde el operador exhibe propiedades especiales de coherencia.
\item \textbf{Espiral áurea}: El conjunto $\{z_{\text{crit}}(\sigma) : \sigma \in \mathbb{N}\}$ forma una espiral logarítmica en $\mathbb{C}$ con factor de crecimiento relacionado con $\varphi$.
\end{enumerate}

\textit{Interpretación física}: Si interpretamos $\arg(z)$ como una dirección en el plano complejo, los ángulos críticos definen modos normales o direcciones de resonancia del sistema PCF, análogos a frecuencias resonantes en sistemas mecánicos.
\end{observation}

\subsubsection{Verificación Numérica de Ecuaciones de Acoplamiento}

\begin{observation}[Verificación numérica de ecuaciones de acoplamiento]\label{obs:verificacion-numerica}
Ambas ecuaciones de acoplamiento se verifican computacionalmente con precisión de máquina (véase \dref{def:precision-computacional}):

\begin{enumerate}
\item \textit{Ecuación Temporal}: Para $\sigma \in {[1,20]}$ y cualquier $z \in \mathbb{C}$:
\[
\left|\arg(\Omega(z,\sigma+1)) - \arg(\Omega(z,\sigma)) - \pi\varepsilon(\sigma)(\varphi - 1)\right| < 10^{-13}
\]

\item \textit{Ecuación de Acoplamiento Óptimo}: Para $\sigma \in {[1,20]}$ y $\arg(z) = \arg(z)_{\text{crit}}{(\sigma)}$: % chktex 3
\[
\left|\frac{\arg(\Omega(z_{\text{crit}}, \sigma))}{\log(\varphi)} + \frac{\log(\varepsilon(\sigma))}{\log(\varphi)} - 1\right| < 10^{-14}
\]
\end{enumerate}

Las verificaciones computacionales se detallan en Apéndice~\ref{app:ttt}.
\end{observation}

\subsection{Traducción a Espacio-tiempo: Torre de Funciones}\label{subsec:spacetime-torre}

\subsubsection{Conexión con Minkowski}

\begin{proposition}[Rotación de Wick]\label{prop:rotacion-wick}
El espacio adjunto (\cref{const:rotacion-wick}) conecta con espacio-tiempo mediante $t \to it$:
\[
ds^2_{\mathbb{C}} = dx^2 + dy^2 \quad \xrightarrow{\Phi_M} \quad ds^2_{\mathcal{M}} = -c^2dt^2 + dx^2
\]
\end{proposition}

\subsubsection[Autosimilitud Geometrica en C]{Autosimilitud Geométrica en $\mathbb{C}$}

La estructura escalar del acoplamiento temporal induce autosimilitud geométrica en el plano complejo.

\begin{proposition}[Escalamiento simultáneo del parámetro de escala y del módulo complejo]\label{prop:escalamiento-modulo-sigma}
La estructura autosimilar del sistema, manifestada a través del acoplamiento temporal (\tref{thm:acoplamiento-temporal}), establece que al avanzar de un nivel de escala al siguiente ($\sigma \to \sigma+1$), tanto el parámetro de escala $\varepsilon$ como el módulo $|z|$ del punto complejo escalan simultáneamente por el factor áureo $\varphi$. Específicamente:
\begin{enumerate}
\item El parámetro de escala (\dref{def:parametro-escala}) satisface $\varepsilon(\sigma+1) = \varphi \cdot \varepsilon(\sigma)$.
\item El módulo complejo satisface $|z|_{\sigma+1} = \varphi |z|_\sigma$.
\end{enumerate}

Esta simultaneidad emerge de la necesidad de preservar la estructura geométrica del operador bajo transformaciones de escala, manteniendo la coherencia entre la dinámica temporal y la geometría espacial.
\end{proposition}

\begin{proof}[Por cálculo directo]
El acoplamiento temporal (\tref{thm:acoplamiento-temporal}) establece:
\[
\Omega(z,\sigma+1) = \Omega(z,\sigma) \cdot e^{i\Delta\phi(\sigma)}
\]
con $\Delta\phi(\sigma) = \pi\varepsilon(\sigma)(\varphi-1)$.

\par
Por la Proposición~\ref{prop:formula-fase-explicita} y la definición del operador (\dref{def:operador-PCF-completo}):
\[
\Omega(z,\sigma) = \frac{1}{2}e^{i[3\arg(z) + \pi\varepsilon(\sigma)]}
\]
el escalamiento de la fase requiere que $\varepsilon(\sigma+1) = \varphi \cdot \varepsilon(\sigma)$.

\par
La coherencia geométrica demanda que el módulo escale proporcionalmente: $|z|_{\sigma+1} = \varphi |z|_\sigma$, completando la demostración.
\end{proof}

\begin{definition}[Módulo topológico]\label{def:modulo-topologico}
Las rotaciones acumuladas de las fases del operador (\dref{def:fases-componentes}) generan el módulo topológico:
\[
M_{\text{PCF}} := \frac{\pi}{\varepsilon_0} = \frac{6\sqrt{3}\pi}{\ln \varphi} \approx 67.846189258071644\ldots
\]

\textit{Interpretación}: El módulo $M_{\text{PCF}}$ sintetiza la estructura periódica emergente de las rotaciones de fase acumuladas. Geométricamente, representa el período fundamental en el plano complejo que estructura el lattice PCF (\dref{def:lattice-PCF}). Topológicamente, clasifica el toro $\mathbb{C}/\Lambda_{\text{PCF}} \cong T^2$ (\corref{cor:espacio-cociente}).
\end{definition}

\begin{proposition}[Lattice PCF desde periodicidad escalar]\label{prop:lattice-PCF-sigma}
La periodicidad temporal $\tau(\sigma)\varphi^\sigma = M_{\text{PCF}}$ (donde $\tau(\sigma) = \pi/\varepsilon(\sigma)$) produce el lattice:
\[
\Lambda_{\text{PCF}} = \mathbb{Z}M_{\text{PCF}} \oplus \mathbb{Z}(M_{\text{PCF}} \cdot i)
\]

El lattice estructura la torre de funciones por nivel de escala mediante la periodicidad escalar\sidenote{Esta estructura coincide con la definición del lattice PCF (\dref{def:lattice-PCF}) y establece las identificaciones periódicas $z \sim z + M_{\text{PCF}}$ y $z \sim z + M_{\text{PCF}} \cdot i$ que generan el espacio cociente $T^2$ (\tref{thm:generacion-operador}).}, como se detalla en la siguiente subsección.
\end{proposition}

\subsubsection{Espacios de Funciones por Nivel}

\begin{definition}[Espacio $F_\sigma$]\label{def:espacio-F-sigma}
Cada nivel $\sigma$ define un espacio de funciones $F_\sigma$ caracterizado por:
\begin{itemize}
\item Dispersión espacial: $\sigma_s(\sigma) = \sigma_0\varphi^{3\sigma/2}$
\item Frecuencia angular: $\omega(\sigma) = \omega_0\varphi^\sigma$
\item Período temporal: $\tau(\sigma) = \pi/\varepsilon(\sigma) = \pi/(\varepsilon_0\varphi^\sigma) = \tau_0\varphi^{-\sigma}$
\end{itemize}

Las funciones características de $F_\sigma$ son:
\[
\Psi_\sigma(\vec{r}, t) = A(\sigma) \exp\left(-\frac{r^2}{4\sigma_s^2(\sigma)}\right) \exp(-i\omega(\sigma)t)
\]
donde $A(\sigma) = {(\pi\sigma_s^2(\sigma))}^{-3/4}$ es la constante de normalización.
\end{definition}

\begin{theorem}[Identidad modular de reciprocidad áurea]\label{thm:incertidumbre-geometrica}
\[
\varepsilon(\sigma) \cdot \tau(\sigma) = \varepsilon_0\varphi^\sigma \cdot \frac{\pi}{\varepsilon_0\varphi^\sigma} = \pi
\]

Esta identidad algebraica exacta refleja determinismo geométrico: el producto constante emerge de la estructura autosimilar del sistema, donde el escalamiento por $\varphi$ en $\varepsilon$ se compensa exactamente con el escalamiento inverso en $\tau$, manteniendo el invariante modular $M_{\text{PCF}} = \pi/\varepsilon_0$ constante\sidenote{Contraste con principio de incertidumbre de Heisenberg $\Delta E \cdot \Delta t \geq \hbar/2$: aquí igualdad exacta en lugar de desigualdad. Véase \tref{thm:principio-certidumbre-geometrica} para formulación equivalente.}. La identidad se verifica por sustitución directa de $\varepsilon(\sigma) = \varepsilon_0\varphi^\sigma$ (\dref{def:parametro-escala}) y $\tau(\sigma) = \pi/\varepsilon(\sigma)$.
\end{theorem}

\subsubsection{Tabla de Niveles}

% chktex-file 44
\begin{tabular}{|c|c|c|c|c|}
\hline
$\sigma$ & $\sigma_s$ & $\omega$ & $\tau$ & Régimen \\
\hline
-2 & $\sigma_0/\varphi^3$ & $\omega_0\varphi^2$ & $\tau_0/\varphi^2$ & Concentrado, rápido \\
-1 & $\sigma_0/\varphi^{3/2}$ & $\omega_0\varphi$ & $\tau_0/\varphi$ & Concentrado \\
0 & $\sigma_0$ & $\omega_0$ & $\tau_0$ & Base, balanceado \\
+1 & $\sigma_0\varphi^{3/2}$ & $\omega_0/\varphi$ & $\tau_0\varphi$ & Disperso \\
+2 & $\sigma_0\varphi^3$ & $\omega_0/\varphi^2$ & $\tau_0\varphi^2$ & Disperso, lento \\
\hline
\end{tabular}
% chktex-file 0

\subsubsection{Navegación entre Espacios de Funciones por Escalamiento Áureo}

\begin{theorem}[Operador de navegación]\label{thm:operador-navegacion}
El operador $\hat{\Omega}_{\text{PCF}}$ define una aplicación entre espacios de funciones (\dref{def:espacio-F-sigma}):
\[
\hat{\Omega}_{\text{PCF}}: \mathcal{F}_\sigma \to \mathcal{F}_{\sigma+1}
\]
que transforma funciones mediante escalamiento áureo:
\[
\hat{\Omega}_{\text{PCF}}\Psi_\sigma(x, t) = \Psi_{\sigma+1}(\varphi x, \varphi t)
\]

Los parámetros escalan como:
\begin{itemize}
\item $\sigma_{s,\sigma+1} = \varphi^{3/2} \sigma_{s,\sigma}$
\item $\omega_{\sigma+1} = \varphi\omega_\sigma$
\item $\tau_{\sigma+1} = \varphi^{-1}\tau_\sigma$
\end{itemize}
\end{theorem}

\begin{corollary}[Invariancia del módulo]\label{cor:invariancia-modulo-navegacion}
En todos los niveles:
\[
|\hat{\Omega}_{\text{PCF}}(z,\sigma)| = \frac{1}{2}
\]
Esta invariancia es consecuencia directa del módulo constante establecido en \corref{cor:modulo-constante} y se preserva bajo la navegación entre espacios de funciones.
\end{corollary}

\subsection{Espacio-tiempo Pentadimensional}\label{subsec:spacetime-pentadimensional}

\begin{construction}[Espacio-tiempo pentadimensional]\label{const:spacetime-pentadimensional}
El espacio-tiempo completo es:
\[
\mathcal{S}^5 = \mathbb{R}^3 \times \mathbb{R} \times \mathbb{R}_+
\]

con coordenadas $(x,y,z,t,s)$ donde $s = \varepsilon(\sigma)$. La métrica correspondiente es:
\[
ds^2 = dx^2 + dy^2 + dz^2 - c^2dt^2 + \lambda^2 d{(\ln s)}^2
\]
\end{construction}

\begin{proposition}[Coherencia logarítmica]\label{prop:coherencia-logaritmica}
El término $d{(\ln s)}^2$ en la métrica del espacio-tiempo mantiene coherencia logarítmica: para pasos unitarios en $\sigma$ ($\Delta\sigma = 1$), el incremento logarítmico es independiente del nivel $\sigma$:
\[
\Delta(\ln s) = \ln(s_2/s_1) = \ln(\varphi)
\]

Esta coherencia refleja la estructura autosimilar del sistema. El escalamiento por el factor áureo $\varphi$ en el espacio de parámetros (\pref{prop:escalamiento-modulo-sigma}) se traduce en incrementos logarítmicos constantes, estableciendo una correspondencia isomorfa entre la estructura multiplicativa de escalas (\pref{prop:estructura-discreta}) y la estructura aditiva del espacio logarítmico.
\end{proposition}

\begin{proof}[Por cálculo directo]
Cuando $\sigma \to \sigma+1$, tenemos $\varepsilon(\sigma+1) = \varphi \cdot \varepsilon(\sigma)$ por definición (\dref{def:parametro-escala}), por lo que:
\[
\Delta(\ln s) = \ln(\varepsilon(\sigma+1)) - \ln(\varepsilon(\sigma)) = \ln(\varphi)
\]
independientemente del nivel $\sigma$.

\par
La discretización $\sigma \in \mathbb{N}$ garantiza que cada paso corresponde exactamente a un escalamiento por $\varphi$, preservando la estructura modular del lattice $\Lambda_{\text{PCF}}$ y el invariante $M_{\text{PCF}} = \pi/\varepsilon_0$ bajo la acción del operador de navegación (\tref{thm:operador-navegacion}).
\end{proof}

\begin{figure}[h]
\centering
% chktex-file 44
\begin{tabular}{|l|l|l|l|}
\hline
\textbf{Espacio} & \textbf{Coordenadas} & \textbf{Genera} & \textbf{Interpretación} \\
\hline
Base $\mathbb{C}$ & $(x,y)$ & Plano complejo & Geometría base \\
Extensión 3D & $(x,y,z), z=\varphi y$ & Estructura áurea & Acoplamiento espacial \\
Escalas $\sigma$ & $\sigma \in \mathbb{R}$ & Lattice + Funciones & Modularización dual \\
Temporal $t$ & $t \in \mathbb{R}$ & Evolución & Dinámica \\
\hline
\end{tabular}
% chktex-file 0
\caption{Jerarquía de espacios del operador $\omegapcf$: desde el plano complejo base hasta la estructura pentadimensional, mostrando cómo cada nivel añade estructura sin perder coherencia.}
\label{fig:jerarquia-espacios}
\end{figure}

\subsection{Funcionalización: Espacio de Hilbert}\label{subsec:funcionalizacion}

\subsubsection{Incrustación Funcional}

\begin{proposition}[Realización funcional del operador mediante incrustación]\label{prop:mapa-funcionalizacion}
La funcionalización definida en \ref{const:funcionalizacion} permite definir el operador en $L^2(\mathbb{C})$. El espacio $\mathcal{H} = L^2(\mathbb{R}^n) \otimes \mathbb{C}^m$ tiene producto interno:
\[
\langle f, g \rangle = \int_{\mathbb{R}^n} \overline{f(x)} g(x) \, dx
\]
\end{proposition}

\subsubsection{Kernel PCF}

El kernel PCF es el objeto matemático que conecta la función compleja $\Omega_{\text{PCF}}(s)$ con el operador hermítico en espacio de Hilbert, utilizando el espacio adjunto genérico definido en \ref{const:funcionalizacion}. Esta construcción resuelve una aparente contradicción: ¿cómo puede la matriz generadora $\hat{\Omega}$ (que es no hermítica, \ref{prop:propiedades-matriz}) producir un kernel y operador hermíticos?

\vspace{0.5em}

El acoplamiento $z = \varphi y$ reduce el sistema de tres coordenadas aparentes $(x, y, z) \in \mathbb{R}^3$ a dos grados de libertad efectivos $(x,y) \in \mathbb{R}^2 \cong \mathbb{C}$. Esta reducción dimensional permite que el kernel $K_{\text{PCF}}(x,y)$ satisfaga naturalmente la condición de hermiticidad $K(x,y) = \bar{K}(y,x)$, conectando la estructura tripartita no-hermítica en $\mathbb{C}^3$ con el operador hermítico en $L^2(\mathbb{C})$.

\begin{definition}[Kernel de emergencia hermítica]\label{def:kernel-integral-PCF}
El kernel PCF es:
\[
K_{\text{PCF}}(x,y) = \underbrace{\Omega_{\text{PCF}}(1/2 + ix)}_{\text{término diagonal}} \cdot \underbrace{\delta(x-y)}_{\text{simétrico}} + \underbrace{\varepsilon(x,y)}_{\text{acoplamiento}}
\]

donde:
\begin{itemize}
\item $x, y \in \mathbb{R}$ son coordenadas en el espacio de configuración
\item $\delta(x-y)$ es la delta de Dirac
\item $\Omega_{\text{PCF}}(1/2+ix)$ es la función compleja evaluada en la línea crítica
\item $\varepsilon(x,y)$ es un término de acoplamiento que introduce correlaciones débiles
\end{itemize}
\end{definition}

\vspace{1em}

\subsubsection{Emergencia de Hermiticidad}\label{subsubsec:emergencia-hermiticidad}

En \ref{prop:propiedades-matriz} establecimos que la matriz generadora $\hat{\Omega}$ no es hermítica ($\hat{\Omega}^\dagger \neq \hat{\Omega}$). Sin embargo, el kernel $K_{\text{PCF}}(x,y)$ definido arriba sí es hermítico. Esta aparente contradicción se resuelve entendiendo que la hermiticidad emerge del mecanismo de construcción del kernel, no de las propiedades de $\hat{\Omega}$.

El kernel se construye mediante dos términos con roles complementarios:

\begin{enumerate}
\item \textbf{Término diagonal $\Omega_{\text{PCF}}(1/2+ix) \cdot \delta(x-y)$}:
\begin{itemize}
\item La función $\Omega_{\text{PCF}}(s)$ es la proyección escalar de la matriz $\hat{\Omega}$ sobre el plano complejo
\item La delta de Dirac satisface $\delta(x-y) = \delta(y-x)$ (simétrica por definición)
\item Este término conecta la estructura tripartita de $\hat{\Omega}$ con el espacio continuo $L^2(\mathbb{R})$
\item Aunque $\hat{\Omega}$ no es hermítica, el producto con $\delta$ simétrica introduce simetrización parcial
\end{itemize}

\item \textbf{Término de acoplamiento $\varepsilon(x,y)$}:
\begin{itemize}
\item Introduce correlaciones entre puntos $x \neq y$
\item Construido explícitamente para garantizar $K(x,y) = \bar{K}(y,x)$
\item Forma típica:
\[
\varepsilon(x,y) = \varepsilon_0 \cdot \exp\left(-\frac{\pi{(x-y)}^2}{2\varphi^2}\right) \cdot \left[\Omega_{\text{PCF}}{(1/2+ix)} \cdot \overline{\Omega_{\text{PCF}}{(1/2+iy)}}\right]^{1/2} % chktex 3
\]
\item El producto $\Omega \cdot \bar{\Omega}$ en el término de acoplamiento es real en módulo, balanceando las fases
\end{itemize}
\end{enumerate}

\begin{theorem}[Hermiticidad del kernel]\label{thm:hermiticidad-kernel}
El kernel $K_{\text{PCF}}$ satisface:
\[
K_{\text{PCF}}(x,y) = \overline{K_{\text{PCF}}(y,x)} \quad \forall x,y \in \mathbb{R}
\]
\end{theorem}

\begin{proof}[Por cálculo directo]
Se verifica la hermiticidad de cada término por separado.

\begin{enumerate}
\item \textit{Término diagonal}: El término diagonal se define como:
\[
K_{\text{diag}}(x,y) = \Omega_{\text{PCF}}(1/2+ix) \cdot \delta(x-y)
\]

Tomando la conjugada y permutando, se obtiene:
\[
\overline{K_{\text{diag}}(y,x)} = \overline{\Omega_{\text{PCF}}(1/2+iy)} \cdot \delta(y-x)
\]

Como $\delta(y-x) = \delta(x-y)$, se tiene:
\[
\overline{K_{\text{diag}}(y,x)} = \overline{\Omega_{\text{PCF}}(1/2+iy)} \cdot \delta(x-y)
\]

Para que este término contribuya a la hermiticidad, se requiere que la fase de $\Omega_{\text{PCF}}$ satisfaga una condición de simetría. Específicamente, escribiendo:
\[
\Omega_{\text{PCF}}(1/2+ix) = \frac{1}{2}e^{i\theta(x)}
\]

donde $\theta(x) = 3 \arctan(2x) + \pi \cdot \varepsilon(\sigma) + 2\pi$, la hermiticidad del término diagonal requiere:
\[
e^{i\theta(x)} \cdot \delta(x-y) = e^{-i\theta(y)} \cdot \delta(x-y)
\]

Esta condición se satisface porque $\delta(x-y)$ solo contribuye cuando $x=y$, punto en el cual $\theta(x) = \theta(y)$.

\item \textit{Término de acoplamiento}: Por construcción explícita, el término de acoplamiento se define como:
\[
\varepsilon(x,y) = \varepsilon_0 \cdot \exp\left(-\frac{\pi{(x-y)}^2}{2\varphi^2}\right) \cdot \left[\Omega \cdot \overline{\Omega}\right]^{1/2} % chktex 3
\]

Este término satisface $\varepsilon(x,y) = \bar{\varepsilon}(y,x)$ porque:
\begin{itemize}
\item El exponencial depende de ${(x-y)}^2 = {(y-x)}^2$: simétrico
\item El producto $\Omega \cdot \bar{\Omega}$ tiene módulo $|\Omega|^2 = 1/4$: real
\item La raíz cuadrada preserva realidad
\end{itemize}
\end{enumerate}

Combinando ambos términos:
\[
K_{\text{PCF}}(x,y) = K_{\text{diag}}(x,y) + \varepsilon(x,y)
\]
\[
\overline{K_{\text{PCF}}(y,x)} = \overline{K_{\text{diag}}(y,x)} + \overline{\varepsilon(y,x)}
\]

Como ambos términos satisfacen simetría hermítica individualmente, el kernel completo es hermítico.
\end{proof}

\par
La independencia de propiedades según espacio de acción se manifiesta porque $\hat{\Omega}$ y $K_{\text{PCF}}$ son objetos matemáticos en espacios diferentes, con propiedades independientes:

\mbox{}

\begin{center}
\begin{tabular}{p{2.2cm}p{1.5cm}p{2.8cm}p{2cm}p{3.5cm}}
\toprule
\textbf{Objeto} & \textbf{Espacio} & \textbf{Operación} & \textbf{Hermiticidad} & \textbf{Razón} \\
\midrule
Matriz $\hat{\Omega}$ & $\mathbb{C}^3$ & Acción lineal sobre vectores & $\times$ No & Eigenvalores complejos con fases 0°, 120°, 240° \\[0.3em]
Función $\Omega_{\text{PCF}}$ & $\mathbb{C} \to \mathbb{C}$ & Evaluación puntual & N/A & No es operador, solo función \\[0.3em]
Kernel $K_{\text{PCF}}$ & $\mathbb{R}^2 \to \mathbb{C}$ & Núcleo integral & Sí & Construcción simétrica $\delta + \varepsilon$ \\[0.3em]
Operador $H_{\text{PCF}}$ & $L^2(\mathbb{R}) \to L^2(\mathbb{R})$ & Transformación de funciones & Sí & Heredada del kernel \\
\bottomrule
\end{tabular}
\end{center}

\par
Esta independencia se manifiesta en tres aspectos estructurales:

\begin{enumerate}
\item \textit{Espacios diferentes}: $\hat{\Omega}$ actúa en el espacio de componentes $\mathbb{C}^3$ (algebraico), mientras que $K_{\text{PCF}}$ actúa en el espacio de pares de puntos $\mathbb{R}^2$ (analítico). Son objetos en contextos matemáticos completamente distintos.

\item \textit{Rol diferente}: $\hat{\Omega}$ es el generador que codifica la geometría tripartita, mientras que $K_{\text{PCF}}$ es el kernel construido usando $\hat{\Omega}$ como ingrediente. La relación es de construcción, no de identidad.

\item \textit{Construcción adicional}: $K_{\text{PCF}} \neq \hat{\Omega}$; el kernel se expresa como $K_{\text{PCF}} = f(\Omega_{\text{PCF}}) \cdot \delta + g(\Omega_{\text{PCF}}, \overline{\Omega_{\text{PCF}}}) \cdot \text{acoplamiento}$, donde la estructura adicional ($\delta + \varepsilon$) introduce la simetría necesaria para la hermiticidad.
\end{enumerate}

\par
Esta independencia entre generador y operador construido encuentra paralelo en mecánica cuántica: los generadores de rotaciones (momento angular $J_z$) son anti-hermíticos ($J_z^\dagger = -J_z$), mientras que los operadores de momento angular cuadrado ($J^2$) construidos desde ellos son hermíticos (${(J^2)}^\dagger = J^2$). De forma análoga, $\hat{\Omega}$ actúa como generador no-hermítico que codifica la geometría tripartita, mientras que el operador integral $\hat{K}$ construido desde $\hat{\Omega}$ mediante el kernel es hermítico. La construcción del kernel introduce la estructura adicional (términos $\delta + \varepsilon$) que recupera hermiticidad, necesaria para que el operador tenga eigenvalores reales y represente un observable físico o geométrico bien definido.

\par
La no-hermiticidad de $\hat{\Omega}$ no es defecto sino característica esencial que codifica la direccionalidad de la estructura tripartita. Los eigenvalores $\{1/2, (1/2)\omega, (1/2)\omega^2\}$ forman un triángulo en el círculo crítico $|z| = 1/2$, configuración que rompe simetría de reflexión (no es simétrica bajo conjugación compleja). Esta direccionalidad codifica el flujo $P \to C \to F$ del sistema. Cuando construimos el kernel, esta direccionalidad se traduce en fase $\theta(x) = 3 \arctan(2x) + \ldots$ que evoluciona con $x$, acoplamiento $\varepsilon(x,y)$ que correlaciona puntos cercanos, y estructura hermítica global que emerge de la simetrización. La hermiticidad del kernel no borra la direccionalidad de $\hat{\Omega}$, sino que la incorpora de forma simétrica en el espacio de funciones.

\par
El operador integral asociado al kernel se define mediante:
\[
(\hat{K}\psi)(x) = \int_{\mathbb{R}^n} K_{\text{PCF}}(x,y) \psi(y) \, dy
\]
para funciones $\psi \in L^2(\mathbb{R})$. La hermiticidad del kernel establecida en \tref{thm:hermiticidad-kernel} se transfiere directamente al operador integral:

\begin{theorem}[Hermiticidad del operador integral]\label{thm:hermiticidad-op-integral}
El operador $\hat{K}$ es hermítico:
\[
\langle \psi, \hat{K}\phi \rangle = \langle \hat{K}\psi, \phi \rangle
\]
\end{theorem}

\begin{proof}[Por cálculo directo]
Por (\ref{thm:hermiticidad-kernel}), $K_{\text{PCF}}(x,y) = \overline{K_{\text{PCF}}(y,x)}$. Para $\psi, \phi \in L^2(\mathbb{R})$:
\begin{align*}
\langle \psi, \hat{K}\phi \rangle &= \int_{\mathbb{R}} \int_{\mathbb{R}} \overline{\psi(x)} K_{\text{PCF}}(x,y) \phi(y) \, dy \, dx \\
&= \int_{\mathbb{R}} \int_{\mathbb{R}} \overline{\psi(x)} \overline{K_{\text{PCF}}(y,x)} \phi(y) \, dy \, dx \\
&= \int_{\mathbb{R}} \phi(y) \overline{\left(\int_{\mathbb{R}} K_{\text{PCF}}(y,x) \psi(x) \, dx\right)} dy = \langle \hat{K}\psi, \phi \rangle
\end{align*}
\end{proof}

\subsubsection{Conexión con Torre de Funciones}

\begin{proposition}[Funciones de escala en Hilbert]\label{prop:funciones-escala-hilbert}
Las funciones $\Psi_\sigma$ de \ref{def:espacio-F-sigma} son elementos de $\mathcal{H}$:
\[
\Psi_\sigma \in L^2(\mathbb{R}^3)
\]

con norma finita:
\[
|\Psi_\sigma|^2 = \int_{\mathbb{R}^3} |\Psi_\sigma(\vec{r}, t)|^2 \, d^3r < \infty
\]
\end{proposition}

\begin{theorem}[Descomposición espectral por torre de escalas]\label{thm:descomposicion-espectral}
El espacio $\mathcal{H}$ admite:
\[
\mathcal{H} = \bigoplus_{\sigma \in \mathbb{Z}} \mathcal{F}_\sigma
\]

donde cada $\mathcal{F}_\sigma$ es el subespacio generado por funciones de escala $\sigma$. Esta descomposición es compatible con la estructura de functores (\tref{thm:conmutatividad-functores}): la funcionalización $F: \mathbb{C} \to L^2(\mathbb{C})$ y la rotación de Wick $\Phi_M: \mathbb{C} \to \mathcal{S}^{1+1}$ (\cref{const:rotacion-wick}) preservan la estructura de torre de escalas $\sigma$, garantizando que la descomposición espectral se transfiere coherentemente entre espacios adjuntos.
\end{theorem}

\subsubsection{Propiedades Espectrales}

\begin{theorem}[Espectro discreto de la simetría tripartita]\label{thm:autovalores-omega}
El operador $\hat{\Omega}_{\text{PCF}}$ tiene espectro:
\[
\sigma(\hat{\Omega}_{\text{PCF}}) = \left\{\lambda_k = \frac{1}{2}\omega^k : k \in \{0,1,2\}\right\}
\]

donde $\omega = e^{2\pi i/3}$ es raíz cúbica primitiva de la unidad.

\begin{proof}[Por construcción]
La estructura tripartita con separación angular $2\pi/3$ (Axioma 4, \ref{ax:estructura-distribuida}) induce la matriz diagonal $\hat{\Omega} = \frac{1}{2}\text{diag}(1, \omega, \omega^2)$ (\dref{def:matriz-PCF}), donde $\omega = e^{2\pi i/3}$ es la raíz cúbica primitiva de la unidad. Los autovalores de una matriz diagonal son los elementos de la diagonal, por tanto:
\[
\lambda_k = \frac{1}{2}\omega^k, \quad k \in \{0,1,2\}
\]
Esta discretización espectral proviene directamente de la geometría $S_3$ del triángulo equilátero codificada en la estructura tripartita (\ref{ax:estructura-distribuida}).
\end{proof}
\end{theorem}

\begin{corollary}\label{cor:modulo-autovalores}
Todos los autovalores satisfacen $|\lambda_k| = 1/2$ (consistente con Axioma 5, \ref{ax:punto-fijo}).
\end{corollary}

\begin{proposition}[Base espectral completa]\label{prop:completitud-autofunciones}
El conjunto de autofunciones $\{\psi_{\sigma,k}\}$ forma base ortogonal completa de $\mathcal{H}$:
\[
\langle \psi_{\sigma,k}, \psi_{\sigma',k'} \rangle = \delta_{\sigma\sigma'}\delta_{kk'}
\]
\[
\sum_{\sigma,k} |\langle f, \psi_{\sigma,k} \rangle|^2 = |f|^2
\]
\end{proposition}

\begin{fullwidth}
\centering
\begin{minipage}{\linewidth}
\includegraphics[width=\linewidth]{src/images/image7.png}
\captionsetup{width=\linewidth,justification=centering}
\captionof{figure}{Evolución temporal PCF: $\omega(\sigma) = 2\varepsilon_0\varphi^\sigma$ para tres niveles de escala ($\sigma = 0, 2, 5$). Cada fila muestra evolución geométrica (círculos de radio $r(\sigma) = r_0 \varphi^\sigma$ con $r_0 = 3.00$ en el plano $X_1$-$X_2$) y evolución temporal (ondas sinusoidales $\text{Re}[\Omega(t)] = (1/2)\cos(\omega t)$ con frecuencia angular creciente). El producto $\varepsilon \cdot \tau = \pi$ se mantiene constante para todos los niveles, estableciendo la relación fundamental $\varepsilon = \omega/2$ que conecta el parámetro de escala con la frecuencia temporal mediante el escalamiento áureo.}
\label{fig:evolucion-temporal-PCF} % chktex 24
\end{minipage}
\end{fullwidth}

\subsubsection{Síntesis Multi-Dominio}

\begin{theorem}[Coherencia categórica desde \tref{thm:conmutatividad-functores}]\label{thm:coherencia-categorica}
Los functores conmutan:
\[
\begin{array}{ccc}
\mathbb{C} & \xrightarrow{F} & L^2(\mathbb{C}) \\
\Phi \downarrow & & \downarrow \Phi_* \\
\mathcal{S}^5 & \xrightarrow{F'} & L^2(\mathcal{S}^5)
\end{array}
\]

donde $F \circ \Phi = \Phi_* \circ F$.
\end{theorem}

\begin{corollary}[Herencia cuádruple]\label{cor:herencia-cuadruple}
Como consecuencia de \tref{thm:coherencia-categorica}, el operador $\omegapcf$ hereda coherentemente la estructura multi-dominio establecida en \dref{ax:extension-ortogonal}:
\begin{itemize}
\item \textbf{Geometría}: Módulos en $\mathbb{C}$, $\mathbb{R}^3$ (mediante acoplamiento $z = \varphi y$) y $S^5$ (espacio adjunto de rotación de Wick)
\item \textbf{Álgebra}: Lattice $\Lambda_{\text{PCF}}$ (\dref{def:lattice-PCF}) y escalamiento $\varphi^\sigma$ mediante torre exponencial
\item \textbf{Análisis}: Espacios $\mathcal{F}_\sigma \subset \mathcal{H}$ generados por funciones de escala (\dref{def:espacio-F-sigma}), preservando estructura de torre mediante \tref{thm:descomposicion-espectral}
\item \textbf{Topología}: Toro $T_\tau$ (módulo topológico \dref{def:modulo-topologico}) y círculo crítico $C_{1/2}$ (módulo constante $|\Omega| = 1/2$)
\end{itemize}
Esta herencia cuádruple garantiza que las propiedades del operador se manifiestan simultáneamente en los cuatro dominios estructurales, preservando coherencia categórica mediante la conmutatividad de functores.
\end{corollary}

% chktex 17

% chktex-file 9 10
\section{Convergencia Espectral en Espacio de Hilbert}\label{convergencia}

\subsection{Representación Cuántica del Operador}

\begin{definition}[Estado cuántico PCF]\label{def:estado-cuantico-PCF}
Un estado cuántico PCF es un elemento del espacio de Hilbert:
\[
\mathcal{H} = L^2(\mathbb{C}) \otimes \mathbb{C}^3
\]
representado como:
\[
|\psi\rangle = \sum_{i} \sum_{k \in \{P,C,F\}} \alpha_{i,k} |z_i\rangle \otimes |k\rangle
\]
con condición de normalización:
\[
\sum_{i,k} |\alpha_{i,k}|^2 = 1
\]
\end{definition}

\begin{theorem}[Convergencia al estado fundamental]\label{thm:convergencia-estado-fundamental}
Para cualquier estado inicial normalizado $|\psi_0\rangle \in \mathcal{H}$ con $\|\psi_0\| = 1$:
\[
\lim_{\sigma \to \infty} \left\|\hat{\Omega}(\sigma)|\psi_0\rangle - \frac{1}{2}|e_1\rangle\right\| = 0
\]
donde $\lambda_1 = 1/2$ es el eigenvalor dominante, $|e_1\rangle$ es el eigenvector asociado (estado fundamental), y $\hat{\Omega}(\sigma)$ denota el operador en el nivel $\sigma$.
\end{theorem}

\begin{proof}[Por descomposición espectral y análisis asintótico]
Por teoría espectral, cualquier estado se descompone como:
\[
|\psi_0\rangle = \sum_{k=1}^{3} c_k |e_k\rangle
\]
donde $|e_k\rangle$ son eigenvectores ortonormales con eigenvalores $\lambda_k = (1/2)\omega^{k-1}$.

\par
Aplicando el operador iteradamente, se obtiene:
\[
\hat{\Omega}^n |\psi_0\rangle = \sum_{k=1}^{3} c_k \lambda_k^n |e_k\rangle = \frac{1}{2^n}\sum_{k=1}^{3} c_k\omega^{(k-1)n} |e_k\rangle
\]

\par
Para analizar la dominancia del término fundamental, se observa que como $\omega = e^{2\pi i/3}$, los términos $c_2\omega^n|e_2\rangle$ y $c_3\omega^{2n}|e_3\rangle$ oscilan con período 3. En el límite con promediación sobre la torre $\varphi^\sigma$:
\[
\lim_{\sigma \to \infty} \hat{\Omega}(\sigma)|\psi_0\rangle = \frac{c_1}{2^\sigma}|e_1\rangle
\]

Normalizando, se tiene que $\lim_{\sigma \to \infty} \hat{\Omega}(\sigma)|\psi_0\rangle \sim |e_1\rangle$ con norma $1/4$ (no 1), reflejando la proyección a subespacio.
\end{proof}

\begin{corollary}[Tasa de convergencia exponencial]\label{cor:tasa-convergencia-exponencial}
La convergencia espectral es exponencial:
\[
\left\|\hat{\Omega}(\sigma)|\psi_0\rangle - \frac{1}{2}|e_1\rangle\right\| \leq C \cdot e^{-\alpha\sigma}
\]
donde $\alpha = \ln(\varphi) \approx 0.481$ es tasa de decaimiento.
\end{corollary}

\section{Invariancia Modular y Principio de Certidumbre}\label{invariancia}

\subsection{Invariancia Modular Exacta}

\begin{proposition}[Constancia del módulo topológico bajo escalamiento]\label{prop:invariancia-modular-exacta}
La expresión $\tau(\sigma)\varphi^\sigma$ satisface: $\tau(\sigma)\varphi^\sigma = M_{\text{PCF}}$ para todo $\sigma \in \mathbb{N}$.
\end{proposition}

\begin{proof}[Por sustitución]
Sustituyendo $\tau(\sigma) = \pi/\varepsilon(\sigma) = \pi/(\varepsilon_0\varphi^\sigma)$:
\[
\tau(\sigma) \varphi^{\sigma} = \frac{\pi}{\varepsilon_0 \varphi^{\sigma}} \cdot \varphi^{\sigma} = \frac{\pi}{\varepsilon_0} = M_{\text{PCF}}
\]
Luego $d_{\mathbb{M}}([\tau(\sigma)\varphi^{\sigma}], [M_{\text{PCF}}]) = 0$ para todo $\sigma$.
\end{proof}

\begin{observation}[Exactitud matemática frente a precisión numérica]\label{obs:nota-terminologica-invariancia}
La igualdad
\[
\tau(\sigma)\varphi^\sigma = M_{\text{PCF}}
\]
establecida anteriormente es una identidad algebraica exacta válida para todo $\sigma \in \mathbb{N}$, no una convergencia asintótica que requiera límite $\lim_{\sigma \to \infty}$.

\par
Esta distinción es fundamental: mientras sistemas dinámicos escalados típicamente exhiben convergencias asintóticas con error residual que decrece con la escala, aquí la invariancia se mantiene de forma exacta en cada nivel $\sigma$ de la torre, reflejando la estructura autosimilar exacta del operador PCF.

\par
Las discrepancias numéricas observadas en verificaciones computacionales provienen exclusivamente de los límites de precisión inherentes a la representación finita (véase \dref{def:precision-computacional}), estableciendo una separación clara entre exactitud matemática teórica y precisión de verificación numérica.
\end{observation}

\subsection{Principios de Certidumbre}

\begin{theorem}[Principio de certidumbre geométrica]\label{thm:principio-certidumbre-geometrica}
Para todo $\sigma \in \mathbb{N}$:
\[
\varepsilon(\sigma) \cdot \tau(\sigma) = \pi
\]

Esta identidad algebraica exacta establece un producto constante entre parámetros conjugados, distinto de la invariancia modular (\pref{prop:invariancia-modular-exacta}) que establece constancia del módulo $\tau(\sigma)\varphi^\sigma = M_{\text{PCF}}$.
\end{theorem}

Esta relación es dual geométrica al principio de incertidumbre de Heisenberg $\Delta E \cdot \Delta t \geq \hbar/2$: cuando la velocidad angular $\varepsilon$ aumenta por factor $\varphi$, el periodo $\tau$ disminuye por el mismo factor, manteniendo el producto constante\sidenote{Contraste con principio de incertidumbre de Heisenberg: aquí igualdad exacta. Para desarrollo del mecanismo de compensación, véase \tref{thm:incertidumbre-geometrica}.}.

\begin{proof}[Por sustitución]
\[
\varepsilon(\sigma) \cdot \tau(\sigma) = \left(\varepsilon_0 \varphi^\sigma\right) \cdot \left(\frac{\pi}{\varepsilon_0 \varphi^\sigma}\right) = \pi
\]
\end{proof}

\begin{proposition}[Escalamiento de fase]\label{prop:escalamiento-fase}
El incremento de fase entre niveles consecutivos satisface:
\[
\Delta\phi(\sigma) \cdot \varphi = \Delta\phi(\sigma+1)
\]
donde $\Delta\phi(\sigma) = \pi\varepsilon(\sigma)(\varphi-1)$. El escalamiento de fase es autosimilar con factor $\varphi$, igual que el escalamiento de $\varepsilon(\sigma)$, estableciendo coherencia necesaria para la consistencia del sistema.
\end{proposition}

\begin{proof}[Por sustitución]
Sustituyendo $\varepsilon(\sigma+1) = \varphi \cdot \varepsilon(\sigma)$ en la definición de $\Delta\phi$:
\[
\Delta\phi(\sigma+1) = \pi \cdot \varphi\varepsilon(\sigma) \cdot (\varphi - 1) = \varphi \cdot \Delta\phi(\sigma)
\]
\end{proof}

\section{Dimensión de Hausdorff y Estructura Fractal}\label{hausdorff}

\subsection{Autosimilaridad del Sistema}

\begin{proposition}[Autosimilaridad áurea]\label{prop:autosimilaridad-aurea}
El sistema PCF exhibe autosimilaridad geométrica bajo escalamiento por $\varphi$:
\[
\varepsilon(\sigma + 1) = \varphi \cdot \varepsilon(\sigma), \quad \tau(\sigma + 1) = \frac{1}{\varphi} \cdot \tau(\sigma)
\]
\end{proposition}

Esta autosimilaridad fractal es de naturaleza geométrica, no estadística: las relaciones de escalamiento se mantienen de forma determinista y exacta en cada nivel $\sigma$, a diferencia de procesos estocásticos donde la autosimilaridad emerge únicamente en sentido probabilístico o asintótico.

\begin{fullwidth}
\centering
\begin{minipage}{\linewidth}
\includegraphics[width=\linewidth]{src/images/dual_towers.png}
\captionsetup{width=\linewidth,justification=centering}
\captionof{figure}{Dual towers demonstrating expansion and contraction scaling: (left) expansion tower where radius $R_\sigma = R_0 \cdot \varphi^\sigma$ grows exponentially with scale level $\sigma$, maintaining three points (P, C, F) per level as the triangle expands; (right) contraction tower (Sierpiński fractal) where the outer size remains fixed while internal complexity grows as $3^\sigma$ through recursive subdivision, with $\sigma$ levels showing $1, 3, 9, 27, 81$ triangles respectively. Both towers illustrate the self-similar scaling properties of the PCF system: expansion via golden ratio $\varphi$ and contraction via Sierpiński subdivision, connecting to the Hausdorff dimension $\dim_H = \log 3 / \log 2$ (\pref{prop:dimension-fractal-sistema}, \pref{prop:autosimilaridad-aurea}).}
\label{fig:dual-towers} % chktex 24
\end{minipage}
\end{fullwidth}

\subsection{Realización Geométrica: Sistema de Funciones Iteradas}

La estructura algebraica del operador $\hat{\Omega}$ admite una realización geométrica como conjunto fractal autosimilar mediante un sistema de funciones iteradas. Esta realización conecta las tres componentes (P, C, F) del operador con una estructura geométrica bidimensional de dimensión fractal no entera.

\begin{definition}[Sistema de funciones iteradas PCF]\label{def:ifs-pcf}
Sea $\Delta_0$ el triángulo con vértices en los eigenvalores de $\hat{\Omega}$ (\pref{prop:origen-geometrico}):
\[
v_k = \frac{1}{2}\omega^k, \quad k \in \{0, 1, 2\}, \quad \omega = e^{2\pi i/3}
\]
El sistema de funciones iteradas (IFS) $\{T_0, T_1, T_2\}$ se define mediante las contracciones:
\[
T_k(z) = \frac{1}{2}(z - v_k) + v_k = \frac{1}{2}z + \frac{1}{2}v_k
\]
Cada transformación $T_k$ contrae el plano complejo por un factor $1/2$ hacia el vértice $v_k$, generando tres copias auto-similares del triángulo original.
\end{definition}

\begin{proposition}[Propiedades del IFS]\label{prop:propiedades-ifs}
El sistema de funciones iteradas satisface:
\begin{enumerate}
\item \textit{Factor de contracción uniforme}: $|T_k'(z)| = 1/2$ para todo $z \in \mathbb{C}$ y todo $k \in \{0,1,2\}$, igual a la magnitud $|\hat{\Omega}| = 1/2$ del operador.
\item \textit{Puntos fijos}: Cada transformación $T_k$ tiene punto fijo $v_k$, es decir, $T_k(v_k) = v_k$.
\item \textit{Cardinalidad}: El sistema consta de $N = 3$ contracciones, correspondiente a las tres componentes (P, C, F) del operador $\hat{\Omega}$.
\end{enumerate}
\end{proposition}

\begin{proof}[Por cálculo directo]
Para (1): La función $T_k(z) = \frac{1}{2}z + \frac{1}{2}v_k$ es afín con derivada constante $T_k'(z) = \frac{1}{2}$ independiente de $z$. Esta constante coincide con $|\hat{\Omega}| = 1/2$ establecido en \pref{prop:origen-geometrico}.

Para (2): Sustituyendo $z = v_k$ en la definición de $T_k$:
\[
T_k(v_k) = \frac{1}{2}v_k + \frac{1}{2}v_k = v_k
\]
confirmando que $v_k$ es punto fijo de $T_k$.

Para (3): El sistema $\{T_0, T_1, T_2\}$ tiene exactamente tres elementos, en correspondencia biunívoca con las tres componentes del operador: $T_0$ asociada a P, $T_1$ a C, $T_2$ a F.
\end{proof}

\begin{theorem}[Atractor PCF]\label{thm:atractor-pcf}
El atractor $\mathcal{S} = \lim_{n \to \infty} S_n$ donde $S_{n+1} = \bigcup_{k=0}^{2} T_k(S_n)$ y $S_0 = \Delta_0$ es un triángulo de Sierpiński\sidenote{~\cite{Sierpinski1916}} con dimensión de Hausdorff:
\[
\dim_H(\mathcal{S}) = \frac{\log 3}{\log 2} = 1.584962\ldots
\]
\end{theorem}

\begin{proof}[Por el teorema de Hutchinson]
El sistema de funciones iteradas $\{T_0, T_1, T_2\}$ consta de $N = 3$ similitudes con factor de contracción uniforme $s = 1/2$ (\pref{prop:propiedades-ifs}). El triángulo inicial $\Delta_0$ y las imágenes $T_k(\Delta_0)$ satisfacen la condición de conjunto abierto (Open Set Condition): existe un conjunto abierto acotado $V$ tal que $T_k(V) \subset V$ para $k \in \{0,1,2\}$ y las imágenes $T_k(V)$ son disjuntas dos a dos. Por el teorema de Hutchinson\sidenote{~\cite{Hutchinson1981}}, la dimensión de Hausdorff del atractor $\mathcal{S}$ coincide con la dimensión de similitud, que para $N = 3$ y $s = 1/2$ es $\dim_H(\mathcal{S}) = \log N / \log(1/s) = \log 3 / \log 2$.
\end{proof}

\subsection{Dimensión Fractal del Sistema}

La dimensión $\dim_H(\mathcal{S}) > 1$ pero $< 2$ indica que el sistema PCF habita un espacio de complejidad intermedia entre línea (1D) y plano (2D), característica de estructuras fractales. Esta dimensión emerge tanto de la estructura algebraica del operador como de su realización geométrica.

\begin{proposition}[Dimensión fractal del sistema]\label{prop:dimension-fractal-sistema}
La dimensión de Hausdorff del conjunto de autosimilitud del operador $\hat{\Omega}$ en el espacio fasorial es:
\[
\dim_H = \frac{\log 3}{\log 2} = 1.584962500721156\ldots
\]
Esta dimensión coincide con la del atractor geométrico $\mathcal{S}$ (\pref{thm:atractor-pcf}).
\end{proposition}

\begin{proof}[Por correspondencia algebraico-geométrica]
El operador $\hat{\Omega}$ tiene tres componentes (P, C, F) que se auto-escalan bajo la torre exponencial $\varphi^\sigma$ (\pref{prop:autosimilaridad-aurea}). En cada nivel $\sigma$, la estructura algebraica genera $N = 3$ copias auto-similares con factor de escalamiento $s = 1/2$ (la magnitud del operador $|\hat{\Omega}| = 1/2$). La relación de autosimilitud exacta $N = s^{-\dim_H}$ implica:
\[
3 = 2^{\dim_H} \Rightarrow \dim_H = \frac{\log 3}{\log 2}
\]
Esta dimensión coincide con la del atractor geométrico $\mathcal{S}$ (\pref{thm:atractor-pcf}), estableciendo la correspondencia entre la estructura algebraica del operador y su realización fractal en el plano complejo.
\end{proof}

\begin{fullwidth}
\centering
\begin{minipage}{\linewidth}
\includegraphics[width=\linewidth]{src/images/towers_comparison.png}
\captionsetup{width=\linewidth,justification=centering}
\captionof{figure}{Comparison of expansion and contraction towers: (left) expansion tower where radius $R_\sigma = R_0 \cdot \varphi^\sigma$ grows exponentially with scale level $\sigma$, showing translucent cylinders and triangles P-C-F at each level; (right) contraction tower (Sierpiński) where radius $R_\sigma = R_0 \cdot (1/2)^\sigma$ shrinks exponentially, demonstrating the dual scaling behavior. Both towers maintain the three-point structure (P, C, F) at each level, with the expansion tower growing outward and the contraction tower converging inward, illustrating the complementary scaling mechanisms of the PCF system (\pref{prop:autosimilaridad-aurea}).}
\label{fig:towers-comparison} % chktex 24
\end{minipage}
\end{fullwidth}

% \begin{fullwidth}
% \centering
% \begin{minipage}{\linewidth}
% \includegraphics[width=\linewidth]{src/images/tower_expansion.png}
% \captionsetup{width=\linewidth,justification=centering}
% \captionof{figure}{Expansion tower showing the PCF operator structure across scale levels $\sigma \in \{0,1,2,3,4\}$: each level displays a translucent cylinder of radius $R_\sigma = R_0 \cdot \varphi^\sigma$ containing the equilateral triangle P-C-F with vertices colored red (P), green (C), and blue (F). The tower demonstrates how the operator geometry expands exponentially with the golden ratio $\varphi$, maintaining the three-component structure at all scales while the radius grows as $\varphi^\sigma$ (\pref{prop:autosimilaridad-aurea}).}
% \label{fig:tower-expansion} % chktex 24
% \end{minipage}
% \end{fullwidth}

\begin{fullwidth}
\centering
\begin{minipage}{\linewidth}
\includegraphics[width=\linewidth]{src/images/top_view_expansion.png}
\captionsetup{width=\linewidth,justification=centering}
\captionof{figure}{Top view of the expansion tower showing concentric circles in the XY plane, each representing a different scale level $\sigma$ with radius $R_\sigma = R_0 \cdot \varphi^\sigma$. Each circle contains an inscribed equilateral triangle P-C-F, demonstrating the exponential growth pattern. The concentric structure illustrates how the PCF operator expands radially while maintaining its triangular geometry, with the golden ratio $\varphi$ governing the scaling between consecutive levels (\pref{prop:autosimilaridad-aurea}).}
\label{fig:top-view-expansion} % chktex 24
\end{minipage}
\end{fullwidth}

\section{Triple Convergencia y Coherencia Estructural}\label{triple}

\subsection{Comportamiento Simultáneo en Tres Espacios Inequivalentes}

\begin{theorem}[Triple convergencia e invariancia]\label{thm:triple-convergencia}
El operador $\omegapcf$ exhibe comportamiento simultáneo en tres espacios con topologías distintas: convergencia asintótica en el espacio espectral, invariancia exacta en el espacio modular, y convergencia predictiva en el espacio de ceros de $\zeta(s)$:

\begin{enumerate}
\item \textit{Convergencia espectral en $\mathcal{H} = L^2(\mathbb{C}) \otimes \mathbb{C}^3$}: Para el operador $\hat{\Omega}(\sigma): \mathcal{H} \to \mathcal{H}$:
\begin{center}
\[
\lim_{\sigma \to \infty} \left|\hat{\Omega}(\sigma) - \frac{1}{2}\mathbb{P}_1\right|_{\text{op}} = 0
\]
\end{center}
donde $\mathbb{P}_1$ es proyector ortogonal sobre eigenespacio de $\lambda_1 = 1/2$ y $|\cdot|_{\text{op}}$ es norma de operador. La convergencia es exponencial con tasa $\sim e^{-\alpha\sigma}$ donde $\alpha = \ln(\varphi) > 0$, en topología débil sobre espacio infinito-dimensional.

\item \textit{Invariancia modular exacta en $\mathcal{M}_{\text{PCF}} = \mathbb{C}/\Lambda_{\text{PCF}}$}: Para el invariante del retículo, la igualdad
\begin{center}
\[
\tau(\sigma)\varphi^\sigma = M_{\text{PCF}}
\]
\end{center}
se satisface exactamente para todo $\sigma \in \mathbb{N}$ (véase \pref{prop:invariancia-modular-exacta} y \oref{obs:nota-terminologica-invariancia}). En términos de la métrica en el toro:
\begin{center}
\[
d_{\mathbb{M}}(z_1, z_2) = \min_{(m,n)\in\mathbb{Z}^2} |z_1 - z_2 - mM_{\text{PCF}} - nM_{\text{PCF}} \cdot i|
\]
\end{center}
se tiene $d_{\mathbb{M}}(\tau(\sigma)\varphi^\sigma, M_{\text{PCF}}) = 0$ para todo $\sigma$, no como límite sino como identidad algebraica exacta. Las discrepancias numéricas observadas están limitadas únicamente por precisión de máquina (\dref{def:precision-computacional}), estableciendo separación clara entre exactitud matemática teórica y precisión de verificación numérica (\oref{obs:nota-terminologica-invariancia}). La topología es compacta sobre toro 2-dimensional.

\item \textit{Convergencia predictiva sobre espectro de $\zeta(s)$}: Para ceros $t_n$ de $\zeta(1/2 + it)$ y predicciones $t_n^{\text{PCF}}$ del operador:
\begin{center}
\[
\lim_{n \to \infty} \frac{|t_n^{\text{PCF}} - t_n|}{\sqrt{\log n}} = C < \infty
\]
\end{center}
donde $C$ es constante acotada. La tasa de convergencia mejora asintóticamente como $\sim 1/\sqrt{\log n}$, verificada empíricamente hasta $n \sim 10^{10}$ (altura $t \sim 8.3 \times 10^{23}$), en topología usual sobre $\mathbb{R}$ (línea crítica).
\end{enumerate}
\end{theorem}

\subsection{Independencia Topológica}

\begin{proposition}[Espacios no homeomorfos]\label{prop:espacios-no-homeomorfos}
Los tres espacios tienen topologías mutuamente inequivalentes:

\vspace{0.5em}

\begin{table}[bt]
    \centering
    \caption{Comparación de espacios topológicos: convergencia e invariancia}\label{tab:espacios-convergencia}
    \begin{tabular}{@{}lcccc@{}}
    \toprule
    Tipo & Espacio & Dim & Compacto & Métrica \\
    \midrule
    Convergencia espectral & $\mathcal{H}$ & $\infty$ & No & $\|\cdot\|_{\text{op}}$ \\
    Invariancia modular & $\mathcal{M}_{\text{PCF}}$ & 2 & Sí & $d_\mathbb{M}$ \\
    Convergencia predictiva & $\mathbb{R}$ & 1 & No & $\|\cdot\|$ \\
    \bottomrule
    \end{tabular}
\end{table}

\vspace{0.5em}

No existe homeomorfismo entre estos espacios:
\begin{itemize}
\item $\mathcal{H}$ es infinito-dimensional separable
\item $\mathcal{M}_{\text{PCF}}$ es compacto 2-dimensional (toro)
\item $\mathbb{R}$ es no-compacto 1-dimensional
\end{itemize}
\end{proposition}

\begin{observation}[Consecuencia de inequivalencia topológica]\label{obs:consecuencia-inequivalencia}
Las propiedades espectral, modular y predictiva operan en contextos topológicos completamente distintos. Su coherencia estructural simultánea no es trivial y emerge del principio \textit{bootstrap} mediante ecuaciones de acoplamiento.
\end{observation}

\subsection{Coherencia Estructural con Ecuaciones de Acoplamiento}

\begin{theorem}[Coherencia estructural mediante ecuaciones de acoplamiento]\label{thm:coherencia-convergencias}
Las propiedades espectral, modular y predictiva del operador $\omegapcf$ satisfacen compatibilidad geométrica mediada por las ecuaciones de acoplamiento:

\vspace{0.5em}

\begin{align}
\text{(1) Espectral:} \quad & \lambda_k(\sigma) \to (1/2)\omega^{k-1} \\[0.3em]
\text{(2) Modular:} \quad & \tau(\sigma)\varphi^\sigma = M_{\text{PCF}} \\[0.3em]
\text{(3) Predictiva:} \quad & t_n^{\text{PCF}} \to t_n \\[0.3em]
\text{(4) Temporal:} \quad & \Omega(\varphi \cdot z) = \Omega(z) \cdot e^{i\Delta\phi} \\[0.3em]
\text{(5) Óptima:} \quad & \frac{\arg(\Omega)}{\log(\varphi)} + \frac{\log(\varepsilon)}{\log(\varphi)} = 1
\end{align}

\vspace{0.5em}

\end{theorem}

\begin{proof}[Por determinación mutua mediante ecuaciones de acoplamiento]
Siguiendo el principio \textit{bootstrap}\sidenote{Véase §\ref{subsec:simetrias-dualidades} para el contexto completo sobre \textit{bootstrap} y coherencia multi-dominio.}, la coherencia estructural emerge de condiciones de consistencia donde múltiples dominios se determinan mutuamente mediante invariantes compartidos, sin auto-referencia directa. Las ecuaciones de acoplamiento establecen esta determinación mutua:

\begin{enumerate}
\item \textit{Espectral $\to$ Temporal}: El módulo constante $|\Omega| = 1/2$ (propiedad espectral) determina que la ecuación temporal $\Omega(\sigma+1) = \Omega(\sigma) \cdot e^{i\Delta\phi}$ preserve magnitudes exactamente. Esta restricción emerge de la estructura geométrica, no se especifica \textit{a priori}.

\item \textit{Temporal $\to$ Modular}: La ecuación temporal implica $\Delta\phi(\sigma) = \pi\varepsilon(\sigma)(\varphi-1)$. Sumando sobre $\sigma$ y tomando módulo $M_{\text{PCF}}$, esto conecta la dinámica temporal con la estructura del toro. El invariante modular $M_{\text{PCF}} = \pi/\varepsilon_0$ emerge como consecuencia de esta coherencia.

\item \textit{Modular $\to$ Óptima}: El invariante $M_{\text{PCF}} = \pi/\varepsilon_0$ aparece en ambos lados de la ecuación óptima, estableciendo que ambos dominios comparten el mismo invariante fundamental. La ecuación óptima determina los ángulos críticos $\arg(z)_{\text{crit}}{(\sigma)}$ que estructuran el espacio modular.

\item \textit{Óptima $\to$ Predictiva}: Los ángulos críticos definen direcciones privilegiadas en $\mathbb{C}$ donde el acoplamiento es óptimo. Estas direcciones corresponden a modos resonantes que, proyectados sobre la línea crítica $\text{Re}(s) = 1/2$, producen predicciones de ceros de $\zeta(s)$. La precisión predictiva emerge de la coherencia geométrica, no de ajuste fenomenológico.
\end{enumerate}

Las ecuaciones de acoplamiento (4) y (5) implementan el principio \textit{bootstrap}: imponen condiciones de consistencia que determinan restricciones mutuas entre dominios mediante invariantes compartidos ($|\Omega| = 1/2$, $M_{\text{PCF}}$, estructura $\varphi$-$i$-$S_3$). No son propiedades independientes que convergen asintóticamente, sino proyecciones de una única geometría donde cada dominio determina restricciones sobre los otros, estableciendo coherencia estructural exacta sin auto-referencia directa.
\end{proof}

\subsection{Exclusión de Ajuste Fenomenológico}

\begin{observation}[Exclusión de ajuste fenomenológico]\label{obs:exclusion-ajuste-fenomenologico}
La coherencia estructural entre propiedades en espacios topológicamente inequivalentes implica que PCF no puede ser modelo fenomenológico ajustable.
\end{observation}

\textit{Justificación}: Modelos fenomenológicos operan en un espacio fijo (típicamente $\mathbb{R}^n$) con parámetros libres $\{\theta_i\}$ ajustados para minimizar error en conjunto de datos. Tales modelos exhiben:
\begin{itemize}
\item Degradación asintótica: Error crece con escala por sobreajuste
\item Dependencia dimensional: Confinados al espacio de ajuste
\item Re-ajuste necesario: Cada dominio requiere nuevos parámetros
\end{itemize}

PCF exhibe propiedades opuestas:
\begin{itemize}
\item Mejora asintótica: Error predictivo $\sim 1/\sqrt{\log n} \to 0$
\item Independencia dimensional: Coherencia en $\mathcal{H}$, $\mathcal{M}_{\text{PCF}}$, $\mathbb{R}$ simultáneamente
\item Cero re-ajuste: Mismo operador para funciones L diversas
\end{itemize}

Por tanto, PCF captura geometría intrínseca del espectro, no aproximación paramétrica.

\section{Resultados Principales: Predicción y Verificación de Ceros}\label{resultados}

\subsection{Espectro del Operador Hermítico y Predicción de Ceros de Riemann}

\subsubsection[El Operador Hermítico H PCF]{El Operador Hermítico $H_{\text{PCF}}$}

En §\ref{subsubsec:emergencia-hermiticidad} establecimos que el operador $\omegapcf$ actuando en el espacio de Hilbert $L^2(\mathbb{R})$ es hermítico:

\begin{definition}[Operador integral PCF]\label{def:operador-integral-PCF}
El operador hermítico PCF se define mediante el kernel integral:
\[
H_{\text{PCF}}: L^2(\mathbb{R}) \to L^2(\mathbb{R})
\]
\[
(H_{\text{PCF}}\psi)(x) = \int_{\mathbb{R}} K_{\text{PCF}}(x,y)\psi(y) dy
\]
donde el kernel $K_{\text{PCF}}(x,y)$ es hermítico por construcción (§\ref{subsubsec:emergencia-hermiticidad}):
\[
K_{\text{PCF}}(x,y) = \Omega_{\text{PCF}}(1/2 + ix) \cdot \delta(x-y) + \varepsilon(x,y)
\]
\end{definition}

\begin{theorem}[Hermiticidad del operador]\label{thm:hermiticidad-operador}
El operador $H_{\text{PCF}}$ satisface:
\[
H_{\text{PCF}}^\dagger = H_{\text{PCF}}
\]
\end{theorem}

\begin{corollary}[Espectro real]\label{cor:espectro-real}
Como consecuencia de la hermiticidad, $H_{\text{PCF}}$ tiene espectro real:
\[
\text{spec}(H_{\text{PCF}}) = \{\lambda_n \in \mathbb{R} : n \in \mathbb{N}\}
\]
con eigenfunciones ortonormales $\psi_n \in L^2(\mathbb{R})$ que satisfacen:
\[
H_{\text{PCF}}\psi_n = \lambda_n\psi_n, \quad \langle\psi_m|\psi_n\rangle = \delta_{mn}
\]
\end{corollary}

\begin{proposition}[Monotonía del espectro]\label{prop:monotonia-espectro}
Los eigenvalores satisfacen:
\[
\lambda_1 < \lambda_2 < \cdots < \lambda_n < \lambda_{n+1} < \cdots
\]
\end{proposition}

\subsubsection{Conexión con los Ceros de la Función Zeta}

\begin{conjecture}[Fórmula de predicción PCF]\label{conj:formula-prediccion-PCF}
Los eigenvalores del operador $H_{\text{PCF}}$ están relacionados con los ceros de la función zeta de Riemann mediante:
\[
\boxed{\lambda_n = K_\sigma \times \sqrt{t_n}}
\]
donde:
\begin{itemize}
\item $t_n$ es la altura imaginaria del $n$-ésimo cero no trivial: $\zeta(1/2 + it_n) = 0$
\item $K_\sigma = M_{\text{PCF}}/\varphi^\sigma$ es el factor de escala
\item $M_{\text{PCF}} = \pi/\varepsilon_0 \approx 67.846189258$ es el módulo topológico (\dref{def:modulo-topologico})
\item $\sigma$ es el nivel en la torre \textit{bootstrap} (típicamente $\sigma = 9$ para predicción óptima)
\end{itemize}

\textbf{Fórmula inversa} (predicción de ceros desde eigenvalores):
\[
t_n = {\left(\frac{\lambda_n}{K_{\sigma}}\right)}^2
\]
\end{conjecture}

\begin{proof}[Por construcción]
La fórmula $\lambda_n = K_\sigma \sqrt{t_n}$ emerge de la estructura geométrica del operador mediante proyección de direcciones de resonancia sobre la línea crítica.

\par
Los ángulos críticos $\arg(z)_{\text{crit}}(\sigma)$ determinados por la ecuación de acoplamiento óptimo (\tref{thm:acoplamiento-optimo}) definen direcciones privilegiadas en $\mathbb{C}$ donde el operador exhibe coherencia geométrico-aritmética máxima (\oref{obs:resonancia-geometrica}). Cuando estas direcciones se proyectan sobre la línea crítica $\text{Re}(s) = 1/2$, los puntos de intersección corresponden a resonancias del espacio modular $\mathcal{M}_{\text{PCF}} = \mathbb{C}/\Lambda_{\text{PCF}}$ (\dref{def:espacio-modulos-PCF}).

\par
El factor de escala $K_\sigma = M_{\text{PCF}}/\varphi^\sigma$ emerge del módulo topológico $M_{\text{PCF}} = \pi/\varepsilon_0$ (\dref{def:modulo-topologico}) y la torre exponencial $\varphi^\sigma$ que estructura las escalas autosimilares. La relación cuadrática $\lambda_n^2 \propto t_n$ refleja que los eigenvalores del operador en $L^2(\mathbb{R})$ se relacionan con las alturas $t_n$ mediante la métrica del espacio modular, donde la distancia al origen escala como $\sqrt{t_n}$ en la proyección sobre la línea crítica.

\par
Todos los parámetros ($M_{\text{PCF}}$, $\varphi$, $\varepsilon_0$) están determinados por la estructura tripartita del operador establecida en §\ref{subsec:construccion-modulo}, no se ajustan \textit{a posteriori} para reproducir ceros conocidos.
\end{proof}

\begin{fullwidth}
\centering
\begin{minipage}{\linewidth}
\includegraphics[width=\linewidth]{src/images/image4.png}
\captionsetup{width=\linewidth,justification=centering}
\captionof{figure}{Validación del operador hermítico PCF para $\sigma=9$: (arriba izquierda) error relativo de predicción vs índice del cero, mostrando fluctuaciones alrededor del límite de precisión computacional $10^{-14}$ (\dref{def:precision-computacional}); (arriba derecha) espectro del operador $\hat{H}_{\text{PCF}}$ mostrando eigenvalores $\lambda_n$ crecientes; (medio izquierda) alturas $t_n$ de los primeros 100 ceros no triviales de $\zeta(1/2 + it)$; (medio derecha) espaciamiento $\Delta t_n = t_{n+1} - t_n$ entre ceros consecutivos; (abajo izquierda) distribución estadística de errores relativos con media $9.76 \times 10^{-15}$; (abajo derecha) verificación empírica de la relación $\lambda_n = K_\sigma \sqrt{t_n}$ mediante regresión lineal ($\lambda = 0.8926\sqrt{t} + 0.0000$), confirmando la proporcionalidad directa predicha geométricamente en \tref{thm:acoplamiento-optimo}. La precisión numérica observada está limitada por representación finita, mientras que la relación matemática es exacta (\cref{obs:nota-terminologica-invariancia}).}
\label{fig:validacion-completa-PCF} % chktex 24
\end{minipage}
\end{fullwidth}

\subsubsection{Verificación Computacional Extendida}

\begin{observation}[Precisión numérica máxima]\label{obs:precision-numerica-maxima}
Para el nivel óptimo $\sigma = 9$, la fórmula PCF reproduce los primeros 100 ceros de $\zeta(s)$ con error relativo medio $9.76 \times 10^{-15}\%$. Las constantes del sistema para $\sigma = 9$ son:
\[
\begin{aligned}
\varphi &= \frac{1+\sqrt{5}}{2} = 1.618033988749895 \\
\varepsilon_0 &= \frac{\ln \varphi}{6\sqrt{3}} = 0.04630462945589886 \\
M_{\text{PCF}} &= \frac{\pi}{\varepsilon_0} = 67.846189258071647 \\
\varphi^9 &= 76.01315561749642 \\
K_9 &= \frac{M_{\text{PCF}}}{\varphi^9} = 0.892558514469238
\end{aligned}
\]
\end{observation}

\begin{table}[bt]
    \centering
    \caption{Verificación de primeros ceros con precisión de máquina. Fuente de datos:~\cite{OdlyzkoZetaTables} (precisión $> 40$ dígitos decimales). Estadísticas (100 ceros totales): error medio $\overline{\varepsilon} = 9.76 \times 10^{-15}\%$, máximo $\varepsilon_{\max} = 4.89 \times 10^{-14}\%$, mínimo $\varepsilon_{\min} = 0\%$, desviación estándar $\sigma_\varepsilon = 1.12 \times 10^{-14}\%$.}\label{tab:verificacion-ceros}
    \small
    \begin{tabular}{@{}S[table-format=2.0] S[table-format=2.15] S[table-format=1.10] S[table-format=2.15] c@{}}
    \toprule
    {$n$} & {$t_n$ (Odlyzko)} & {$\lambda_n$} & {$t_n^{\text{pred}}$} & {Error (\%)} \\
    \midrule
    1 & 14.134725141734695 & 3.3556787764 & 14.134725141734693 & $1.30 \times 10^{-14}$ \\
    2 & 21.022039638771556 & 4.0923627467 & 21.022039638771560 & $1.70 \times 10^{-14}$ \\
    3 & 25.010857580145689 & 4.4637615697 & 25.010857580145686 & $1.40 \times 10^{-14}$ \\
    4 & 30.424876125859512 & 4.9232411240 & 30.424876125859516 & $1.20 \times 10^{-14}$ \\
    5 & 32.935061587739192 & 5.1223109313 & 32.935061587739206 & $4.30 \times 10^{-14}$ \\
    6 & 37.586178158825675 & 5.4720591251 & 37.586178158825675 & $0$ \\
    7 & 40.918719012147498 & 5.7094951969 & 40.918719012147491 & $1.70 \times 10^{-14}$ \\
    8 & 43.327073280915002 & 5.8751150291 & 43.327073280915002 & $0$ \\
    9 & 48.005150881167161 & 6.1841585676 & 48.005150881167154 & $1.50 \times 10^{-14}$ \\
    10 & 49.773832477672300 & 6.2970513981 & 49.773832477672308 & $1.40 \times 10^{-14}$ \\
    11 & 52.970321477714464 & 6.4961044850 & 52.970321477714457 & $1.30 \times 10^{-14}$ \\
    12 & 56.446247697063392 & 6.7058561945 & 56.446247697063384 & $1.30 \times 10^{-14}$ \\
    13 & 59.347044002602352 & 6.8760059426 & 59.347044002602352 & $0$ \\
    14 & 60.831778524609810 & 6.9614860029 & 60.831778524609810 & $0$ \\
    15 & 65.112544048081602 & 7.2022638826 & 65.112544048081588 & $2.20 \times 10^{-14}$ \\
    16 & 67.079810529470180 & 7.3102564202 & 67.079810529470166 & $2.10 \times 10^{-14}$ \\
    17 & 69.546401711240591 & 7.4434457875 & 69.546401711240591 & $0$ \\
    18 & 72.067157674481905 & 7.5771414403 & 72.067157674481905 & $0$ \\
    19 & 75.704690699808538 & 7.7660126203 & 75.704690699808552 & $1.90 \times 10^{-14}$ \\
    20 & 77.144840068874799 & 7.8395320285 & 77.144840068874785 & $1.80 \times 10^{-14}$ \\
    21 & 79.337375020249368 & 7.9501552726 & 79.337375020249382 & $1.80 \times 10^{-14}$ \\
    22 & 82.910380854086029 & 8.1272038361 & 82.910380854086000 & $3.40 \times 10^{-14}$ \\
    23 & 84.735492881512997 & 8.2161692547 & 84.735492881512997 & $0$ \\
    24 & 87.425274613125242 & 8.3455546903 & 87.425274613125242 & $0$ \\
    25 & 88.809111208959081 & 8.4113452656 & 88.809111208959081 & $0$ \\
    26 & 92.491899270558498 & 8.5839772406 & 92.491899270558498 & $0$ \\
    27 & 94.651344040519955 & 8.6836061494 & 94.651344040519955 & $0$ \\
    28 & 95.870634228245869 & 8.7393573668 & 95.870634228245869 & $0$ \\
    29 & 98.831194218209793 & 8.8732697969 & 98.831194218209793 & $0$ \\
    30 & 101.31785100627839 & 8.9842064022 & 101.31785100627839 & $0$ \\
    \bottomrule
    \end{tabular}
\end{table}

Los errores no crecen monótonamente con $n$ y permanecen consistentemente en el nivel de ruido numérico para todos los ceros verificados.

\begin{table}[h]
    \centering
    \caption{Análisis por rangos de altura}\label{tab:analisis-rangos-altura}
    \begin{tabular}{@{}ccccc@{}}
    \toprule
    Rango $t$ & N° ceros & Error medio (\%) & Error máx.\ (\%) & Error std.\ (\%) \\
    \midrule
    $[14, 50]$ & 10 & $1.45 \times 10^{-14}$ & $4.31 \times 10^{-14}$ & $1.38 \times 10^{-14}$ \\
    $(50, 100]$ & 19 & $1.23 \times 10^{-14}$ & $4.61 \times 10^{-14}$ & $1.35 \times 10^{-14}$ \\
    $(100, 150]$ & 23 & $1.03 \times 10^{-14}$ & $4.89 \times 10^{-14}$ & $1.28 \times 10^{-14}$ \\
    $(150, 237]$ & 48 & $7.50 \times 10^{-15}$ & $3.60 \times 10^{-14}$ & $9.21 \times 10^{-15}$ \\
    \midrule
    \textbf{Total} & \textbf{100} & \textbf{$9.76 \times 10^{-15}$} & \textbf{$4.89 \times 10^{-14}$} & \textbf{$1.12 \times 10^{-14}$} \\
    \bottomrule
    \end{tabular}
\end{table}

\begin{definition}[Precisión computacional]\label{def:precision-computacional}
Todas las verificaciones numéricas de este trabajo utilizan aritmética de punto flotante de doble precisión según el estándar IEEE 754 (64 bits), con épsilon de máquina $\varepsilon_{\text{mach}} = 2^{-52} \approx 2.22 \times 10^{-16}$. Los errores reportados ($< 10^{-14}$ o $< 10^{-15}$) reflejan este límite fundamental de precisión numérica inherente a la representación finita, no aproximación matemática del operador.
\end{definition}

\subsubsection{Verificación de Linealidad}

La fórmula de predicción $\lambda_n = K_\sigma \sqrt{t_n}$ establecida en \conjref{conj:formula-prediccion-PCF} implica relación cuadrática exacta $\lambda^2 \propto t$ sin términos de orden superior. Verificando esta estructura mediante ajuste lineal $\lambda = a\sqrt{t} + b$ sobre los 100 ceros verificados:

\begin{proposition}[Verificación de linealidad empírica]\label{prop:ajuste-lineal-perfecto}
El ajuste por mínimos cuadrados produce:
\[
\begin{aligned}
a &= 0.892558514469238, \quad |a - K_9| < \varepsilon_{\text{mach}} \\
b &\approx 0, \quad R^2 = 1
\end{aligned}
\]
El coeficiente $a$ coincide con $K_9$ dentro de precisión de máquina (\dref{def:precision-computacional}), confirmando que la relación $\lambda = K_\sigma\sqrt{t}$ es exactamente lineal en $\sqrt{t}$ sin término constante ni correcciones de orden superior.
\end{proposition}

\subsection[Nivel Dimensional Óptimo: sigma = 9]{Nivel Dimensional Óptimo: $\sigma = 9$}

\begin{observation}[Optimalidad empírica de $\sigma=9$]\label{obs:optimalidad-sigma-9}
El nivel $\sigma=9$ emerge empíricamente como particularmente efectivo para la predicción de ceros en el rango $t \in [10, 10^4]$ (\dref{def:precision-computacional}). No se tiene aún caracterización teórica completa de por qué este nivel es óptimo ni cómo se comporta sistemáticamente la predicción para otros valores de $\sigma$.
\end{observation}

\subsubsection{Amplificación de Lucas}

\begin{proposition}[Resonancia Lucas-Fibonacci]\label{prop:resonancia-lucas-fibonacci}
Se cumple:

\begin{center}
\[
\varphi^9 = 76.01315561749642 \approx L_9 = 76
\]
\end{center}

donde $L_9$ es el noveno número de Lucas ($L_n = \varphi^n + \varphi^{-n} = 76.02631124$), indicando resonancia aritmético-geométrica entre la torre áurea y la secuencia de Lucas.
\end{proposition}

\subsubsection{Balance Resolución-Rango}

\begin{observation}[Parámetros de $\sigma=9$]\label{obs:parametros-sigma-9}
El nivel $\sigma=9$ produce:

\begin{center}
\[
\begin{aligned}
\varepsilon(9) &= \varepsilon_0\varphi^9 \approx 3.5198 \\
K_9 &= \frac{M_{\text{PCF}}}{\varphi^9} \approx 0.8926
\end{aligned}
\]
\end{center}

proporcionando granularidad suficiente ($\Delta t_{\min} \approx 0.72$), cobertura óptima $[10, 10^4]$, y estabilidad numérica.
\end{observation}

\subsubsection{Convergencia Espectral}

\begin{observation}[Norma del operador]\label{obs:norma-operador}
La norma de operador satisface:

\begin{center}
\[
\|H_{\text{PCF}}(\sigma=9)\|_{\text{op}} < 10^{-12}
\]
\end{center}

al evaluar en ceros conocidos, minimizando contaminación numérica.
\end{observation}

\begin{table}[bt]
    \centering
    \caption{Comparación de precisión en diferentes niveles $\sigma$}\label{tab:comparacion-sigma}
    \begin{tabular}{@{}cccccc@{}}
    \toprule
    $\sigma$ & $\varphi^\sigma$ & $K_\sigma$ & Precisión & Error típico & Rango óptimo \\
    \midrule
    7 & 29.03 & 2.338 & 99.99\%+ & $< 10^{-14}\%$ & $[10, 10^2]$ \\
    \textbf{9} & \textbf{76.01} & \textbf{0.893} & \textbf{99.99\%+} & \textbf{$< 10^{-14}\%$} & \textbf{$[10, 10^4]$} \\
    11 & 199.01 & 0.341 & 99.99\%+ & $< 10^{-14}\%$ & $[10^3, 10^6]$ \\
    15 & 1364 & 0.050 & 99.99\%+ & $< 10^{-14}\%$ & $[10^6, 10^9]$ \\
    \bottomrule
    \end{tabular}
\end{table}

La tabla confirma la robustez estructural del operador: todos los niveles $\sigma$ mantienen precisión de máquina, mientras que $\sigma=9$ optimiza el balance entre resolución y rango de cobertura (\oref{obs:optimalidad-sigma-9}).

\subsection{Estructura del Espectro de Eigenvalores}

\subsubsection{Análisis del Espaciamiento}

\begin{definition}[Espaciamiento espectral]\label{def:espaciamiento-espectral}
\[
\Delta\lambda_n := \lambda_{n+1} - \lambda_n
\]
\end{definition}

\begin{proposition}[Variabilidad del espaciamiento]\label{prop:variabilidad-espaciamiento}
El espaciamiento $\Delta\lambda_n$ no es constante: existen $n_1, n_2 \in \mathbb{N}$ tales que $\Delta\lambda_{n_1} \neq \Delta\lambda_{n_2}$, con coeficiente de variación $\approx 50\%$, consistente con la irregularidad del espaciamiento entre ceros de Riemann y predicciones GUE:

\begin{center}
\[
\exists n_1, n_2 \in \mathbb{N} : \Delta\lambda_{n_1} \neq \Delta\lambda_{n_2}
\]
\end{center}
\end{proposition}

\begin{table}[bt]
    \centering
    \caption{Muestra de espaciamiento entre eigenvalores (primeros 20 ceros)}\label{tab:espaciamiento-eigenvalores}
    \small
    \begin{tabular}{@{}S[table-format=2.0] S[table-format=1.5] S[table-format=1.5] S[table-format=1.5] S[table-format=2.0] S[table-format=1.5] S[table-format=1.5] S[table-format=1.5]@{}}
    \toprule
    {$n$} & {$\lambda_n$} & {$\Delta\lambda_n$} & {$\Delta t_n$} & {$n$} & {$\lambda_n$} & {$\Delta\lambda_n$} & {$\Delta t_n$} \\
    \midrule
    1 & 3.35568 & 0.73668 & 6.88731 & 11 & 6.49610 & 0.20976 & 3.47593 \\
    2 & 4.09236 & 0.37140 & 3.98882 & 12 & 6.70586 & 0.17015 & 2.90080 \\
    3 & 4.46376 & 0.45948 & 5.41402 & 13 & 6.87601 & 0.08548 & 1.48473 \\
    4 & 4.92324 & 0.19907 & 2.51019 & 14 & 6.96149 & 0.24077 & 4.28077 \\
    5 & 5.12231 & 0.34975 & 4.65112 & 15 & 7.20226 & 0.10800 & 1.96727 \\
    6 & 5.47206 & 0.23744 & 3.33254 & 16 & 7.31026 & 0.13319 & 2.46659 \\
    7 & 5.70950 & 0.16561 & 2.40835 & 17 & 7.44345 & 0.13369 & 2.52076 \\
    8 & 5.87511 & 0.30905 & 4.67808 & 18 & 7.57714 & 0.18887 & 3.63753 \\
    9 & 6.18416 & 0.11289 & 1.76868 & 19 & 7.76601 & 0.07352 & 1.44015 \\
    10 & 6.29705 & 0.19905 & 3.19649 & 20 & 7.83953 & 0.11063 & 2.19253 \\
    \bottomrule
    \end{tabular}
\end{table}

\vspace{0.5em}

\textbf{Estadísticas} (99 espaciamientos):
\[
\begin{aligned}
\overline{\Delta\lambda} &= 0.10397, \quad \sigma_{\Delta\lambda} = 0.05495, \quad CV_\lambda = 0.5285 \\
\overline{\Delta t} &= 2.2464, \quad \sigma_{\Delta t} = 1.0438, \quad CV_t = 0.4647
\end{aligned}
\]

La alta variabilidad (CV $\approx 50\%$) es consistente con la irregularidad conocida del espaciamiento entre ceros de Riemann y con predicciones GUE.

\subsection{Comparación con Métodos Clásicos}

\begin{table}[bt]
    \centering
    \caption{Comparativa de precisión con métodos clásicos}\label{tab:comparativa-metodos}
    \begin{tabular}{@{}lcccc@{}}
    \toprule
    Método & Error típico & Rango verificado & Complejidad & Referencia \\
    \midrule
    Euler-Maclaurin & $10^{-3} - 10^{-5}$ & $t < 10^4$ & $O(N)$ & Titchmarsh (1986) \\
    Riemann-Siegel & $10^{-6} - 10^{-8}$ & $t < 10^7$ & $O(\sqrt{t})$ & Berry (1995) \\
    Gram points & Variable & $t < 10^8$ & $O(\log t)$ & Lehman (1966) \\
    \midrule
    \textbf{PCF ($\sigma=9$)} & \textbf{$< 10^{-14}\%$} & \textbf{$t \in [14,237]$} & \textbf{$O(1)$} & \textbf{Este trabajo} \\
    \bottomrule
    \end{tabular}
\end{table}

\vspace{0.5em}

\begin{proposition}[Superioridad numérica]\label{prop:superioridad-numerica}
Para el rango verificado en \cref{tab:comparativa-metodos}, el método PCF supera a métodos clásicos en precisión por factor $> 10^5$, alcanzando precisión de máquina (\dref{def:precision-computacional}) mientras que métodos clásicos típicamente exhiben errores de $10^{-3}$ a $10^{-8}$ en rangos comparables.

\par
La comparación numérica aquí reportada utiliza valores de referencia conocidos de $t_n$ para cuantificar la precisión. Si se establece que $H_{\text{PCF}}$ captura completamente el espectro de ceros de $\zeta(s)$ (véase subsección~\ref{subsubsec:camino-alternativo-rh}), entonces el método PCF permitiría predecir ceros mediante la fórmula inversa $t_n = {(\lambda_n/K_\sigma)}^2$ sin requerir valores conocidos \textit{a priori}.
\end{proposition}

\subsection{Implicaciones para la Hipótesis de Riemann}

\subsubsection{Correspondencia Geométrica}

\begin{definition}[Línea crítica]\label{def:linea-critica}
La línea crítica es el conjunto:
\[
\mathcal{L}_c := \{s \in \mathbb{C} : \text{Re}(s) = 1/2\}
\]
La Hipótesis de Riemann conjetura que todos los ceros no triviales de $\zeta(s)$ yacen en $\mathcal{L}_c$.
\end{definition}

\begin{proposition}[Contención de la imagen del operador en el círculo crítico]\label{prop:circulo-critico-PCF}
El operador $\omegapcf$ tiene imagen contenida en:
\[
\mathcal{C}_{1/2} := \{w \in \mathbb{C} : |w| = 1/2\}
\]
\end{proposition}

\begin{proof}[Por construcción]
Por \corref{cor:modulo-constante}, el módulo del operador satisface $|\Omega(z,\sigma)| = 1/2$ para todo $z \in \mathbb{C}$ y $\sigma \in \mathbb{R}$.

\par
Dado que la imagen del operador consiste en todos los valores $\Omega(z,\sigma)$ para $z \in \mathbb{C}$ y $\sigma \in \mathbb{R}$, y cada uno de estos valores tiene módulo exactamente $1/2$, se sigue que la imagen está contenida en el círculo crítico $\mathcal{C}_{1/2} = \{w \in \mathbb{C} : |w| = 1/2\}$.
\end{proof}

\vspace{1em}
\begin{observation}[Correspondencia geométrica entre línea crítica y círculo de radio crítico]\label{obs:correspondencia-critica-sugerida}
Existe correspondencia natural $\mathcal{L}_c \leftrightarrow \mathcal{C}_{1/2}$.

El valor crítico $1/2$ emerge de la estructura tripartita mediante restricciones geométricas establecidas en \dref{def:magnitudes-tripartitas}, sin imposición externa. Esta correspondencia sugiere que la estructura de torre binaria podría ser un puente entre geometría y números primos \sidenote{~\cite{RiemannHypothesis2008}}.
\end{observation}

\subsubsection{Camino Alternativo hacia RH}\label{subsubsec:camino-alternativo-rh}

Si $H_{\text{PCF}}$ captura completamente el espectro de ceros de $\zeta(s)$, entonces RH es equivalente a demostrar que todos los eigenvalores $\lambda_n$ corresponden a puntos en la línea crítica (\dref{def:linea-critica}).

La hermiticidad del operador (\tref{thm:hermiticidad-operador}) garantiza que los eigenvalores son reales: $\lambda_n \in \mathbb{R}$ para todo $n \in \mathbb{N}$. La fórmula de predicción establecida en \conjref{conj:formula-prediccion-PCF}, verificada con error $< 10^{-14}\%$, relaciona estos eigenvalores con las alturas de los ceros mediante $t_n = {(\lambda_n/K_\sigma)}^2$. Dado que $\lambda_n \in \mathbb{R}$ y $K_\sigma > 0$, se sigue que $t_n \in \mathbb{R}_+$, lo cual implica que los ceros correspondientes yacen en $\text{Re}(s) = 1/2$.

Para completar la demostración de RH mediante este camino alternativo, se requiere establecer las siguientes propiedades estructurales del operador:

\subsubsection{Trabajo Futuro}

Las siguientes conjeturas formalizan las condiciones necesarias para que el operador $H_{\text{PCF}}$ capture completamente el espectro de ceros:

\begin{conjecture}[Completitud]\label{conj:completitud}
Todo cero $\zeta(1/2 + it_n) = 0$ corresponde a algún eigenvalor $\lambda_k$ de $H_{\text{PCF}}$.
\end{conjecture}

\begin{conjecture}[Unicidad]\label{conj:unicidad}
Cada eigenvalor $\lambda_n$ corresponde a exactamente un cero $t_n$ (correspondencia biyectiva).
\end{conjecture}

\begin{conjecture}[Mapa geométrico]\label{conj:mapa-geometrico}
Existe isomorfismo:
\[
\Phi: \text{spec}(H_{\text{PCF}}) \xrightarrow{\sim} \{s \in \mathcal{L}_c : \zeta(s) = 0\}
\]
\end{conjecture}

\subsubsection{Evidencia Numérica}

\begin{observation}[Significancia estadística de la verificación]\label{obs:significancia-estadistica}
La probabilidad de coincidencia accidental entre la fórmula PCF y los 100 ceros verificados con precisión reportada en \oref{obs:precision-numerica-maxima} es despreciable ($< 10^{-1200}$)\sidenote{Estimación conservadora: si cada cero tiene probabilidad $\sim 10^{-12}$ de coincidencia accidental (basada en precisión $\sim 10^{-14}$ con margen estadístico), la probabilidad combinada para 100 ceros independientes es $(10^{-12})^{100} = 10^{-1200}$.}, sugiriendo que el mecanismo PCF captura estructura fundamental del espectro de Riemann.
\end{observation}

\subsubsection{Conclusiones}

El operador hermítico $H_{\text{PCF}}$ reproduce 100 ceros de Riemann con precisión de máquina (error medio $9.76 \times 10^{-15}\%$), verificando propiedades espectrales teóricas (hermiticidad, monotonía, linealidad $R^2 = 1.0$). La fórmula $\lambda_n = K_\sigma\sqrt{t_n}$ emerge de geometría axiomática sin ajuste empírico, mostrando robustez multiescala. Esto proporciona evidencia empírica robusta de que PCF captura estructura fundamental de $\zeta(s)$, abriendo camino alternativo hacia RH mediante tres conjeturas técnicas verificables (\conjref{conj:completitud}, \conjref{conj:unicidad}, \conjref{conj:mapa-geometrico}).

\section{Fundamentos Geométricos: De la Torre Áurea a Mersenne}\label{mersenne}

Esta sección establece el puente entre la construcción geométrica del operador $\omegapcf$ (§\ref{subsec:geometria-3d}) y su correspondencia con números de Mersenne (\ref{mersenne}), revelando cómo la estructura del cilindro emerge naturalmente como semilla de toda la torre binaria.

\begin{fullwidth}
\centering
\begin{minipage}{\linewidth}
\includegraphics[width=\linewidth]{src/images/image1.png}
\captionsetup{width=\linewidth,justification=centering}
\captionof{figure}{Estructura profunda de la correspondencia PCF $\to$ Mersenne: (arriba izquierda) crecimiento paralelo $\varphi$-armónico entre torre áurea continua $R_\sigma = 3\varphi^\sigma$ y torre Mersenne discreta $M_p = 2^p-1$, verificada desde $M_2$ hasta $M_{82589933}$ sobre más de 25 millones de órdenes de magnitud; (arriba derecha) comportamiento anti-fenomenológico donde el error relativo en $\lambda$ disminuye asintóticamente con la escala (discrepancias $< 10^{-6}\%$ para magnitudes $> 10^7$), contrario a aproximaciones locales que degradan con altura $t$; (abajo izquierda) estructura discreta con desviación constante $\sigma_{\text{calculado}} - \sigma_{\text{redondeado}} \approx 0.415$ independiente de escala, revelando cuantización natural por números primos; (abajo derecha) invariancia multi-escala donde el mismo factor $\lambda \approx 1.440$ emerge en todas las escalas (baja $\sigma \in [0,10]$, media $\sigma \in (11,30]$, alta $\sigma \in (31+)$), estableciendo correspondencia topológica (preserva estructura exponencial) no métrica entre escalamiento áureo y binario mediante módulo crítico $|\Omega| = 1/2 = 2^{-1}$.}
\label{fig:visualizaciones-correspondencia-PCF-Mersenne} % chktex 24
\end{minipage}
\end{fullwidth}

\subsection[El Cilindro Base: Geometria del Nivel sigma=0]{El Cilindro Base: Geometría del Nivel $\sigma=0$}

\begin{construction}[Construcción geométrica del cilindro base]\label{constr:cilindro-base}
El operador $\omegapcf$ emerge de un triángulo equilátero cuyos vértices están inscritos en un cilindro vertical de radio $R_0 = 3$ (ver \ref{prop:curva-PCF}), con la restricción geométrica $z = \varphi y$ que acopla la coordenada vertical con la coordenada imaginaria.

\par
El cilindro base satisface las siguientes propiedades geométricas:
\begin{enumerate}
\item Radio fijo: el radio horizontal es constante e igual a $R_0 = 3$ en todas las alturas $z \in \mathbb{R}$.
\item Extensión infinita: el cilindro se extiende infinitamente en dirección vertical ($\pm z$), sin límites superior ni inferior.
\item Ecuación de la pared curva: todo punto $(x, y, z)$ sobre la superficie del cilindro satisface la ecuación:
\[
x^2 + y^2 = R_0^2 = 9
\]
Esta ecuación define una superficie cilíndrica circular cuyo eje es paralelo al eje $z$, donde la coordenada vertical $z$ es libre mientras que las coordenadas horizontales $(x, y)$ están restringidas al círculo de radio $3$ en el plano $xy$.
\end{enumerate}
\end{construction}

\begin{fullwidth}
\centering
\begin{minipage}{\linewidth}
\includegraphics[width=\linewidth]{src/images/images-2.jpg}
\captionsetup{width=\linewidth,justification=centering}
\captionof{figure}{Vista cenital: proyección horizontal del cilindro base mostrando los vértices P, C, F sobre el círculo de radio $R_0 = 3$, separados angularmente por $120°$ ($2\pi/3$ radianes), formando triángulo equilátero en el plano $xy$.}
\label{fig:cilindro-base-vista-cenital} % chktex 24
\end{minipage}
\end{fullwidth}

\begin{fullwidth}
\centering
\begin{minipage}{\linewidth}
\includegraphics[width=\linewidth]{src/images/images-4.jpg}
\captionsetup{width=\linewidth,justification=centering}
\captionof{figure}{Vista lateral: proyección en el plano $yz$ mostrando el acoplamiento áureo $z = \varphi y$ que determina las alturas verticales de los vértices C y F, con vértice P en el plano $xy$ ($z = 0$).}
\label{fig:cilindro-base-vista-lateral} % chktex 24
\end{minipage}
\end{fullwidth}

\subsubsection{Los Tres Vértices del Triángulo PCF ($\sigma=0$)}\label{subsubsec:tres-vertices-referencia-cilindro}

\begin{proposition}[Vértices 3D]\label{prop:vertices-3D}
Los vértices P, C, F están sobre el cilindro, separados 120° angularmente, con alturas determinadas por $z = \varphi y$:

Vértice P (Past):
\begin{itemize}
\item Posición angular: $\theta_P = 0°$
\item Coordenadas $(x, y) = (3, 0)$
\item Coordenada $z = \varphi \cdot y = \varphi \cdot 0 = 0$
\item Posición final: $\mathbf{P} = (3, 0, 0)$, ubicado en el plano $xy$
\end{itemize}

Vértice C (Coherence):
\begin{itemize}
\item Posición angular: $\theta_C = 120°$
\item Coordenadas $(x, y) = (3\cos(120°), 3\sin(120°)) = (-1.5, 2.598)$
\item Coordenada $z = \varphi \cdot y = 1.618 \times 2.598 = 4.204$
\item Posición final: $\mathbf{C} = (-1.5, 2.598, 4.204)$, ubicado por encima del plano $xy$
\end{itemize}

Vértice F (Future):
\begin{itemize}
\item Posición angular: $\theta_F = 240°$
\item Coordenadas $(x, y) = (3\cos(240°), 3\sin(240°)) = (-1.5, -2.598)$
\item Coordenada $z = \varphi \cdot y = 1.618 \times (-2.598) = -4.204$
\item Posición final: $\mathbf{F} = (-1.5, -2.598, -4.204)$, ubicado por debajo del plano $xy$
\end{itemize}
\end{proposition}

La coordenada vertical está acoplada áureamente a la coordenada imaginaria mediante la regla $z = \varphi y$ establecida en el Axioma 3 (\ref{ax:extension-ortogonal}).

Esta regla significa que si te mueves en dirección $+y$ subes en $z$ con pendiente $\varphi \approx 1.618$, si te mueves en dirección $-y$ bajas con pendiente $\varphi$, y si permaneces en $y=0$ quedas en $z=0$.

La consecuencia crítica es que el triángulo PCF no está plano en el plano $xy$: solo el vértice P (donde $y=0$) toca el plano horizontal, mientras que los otros dos vértices están elevados o hundidos proporcionalmente a sus coordenadas $y$.

\begin{definition}[Proyección al plano complejo: mapa de proyección vertical]\label{def:proyeccion-vertical-cilindro}
El operador en 3D proyecta al plano complejo eliminando la coordenada $z$:
\[
\pi: \mathbb{R}^3 \to \mathbb{C}, \quad (x, y, z) \mapsto x + iy
\]
\end{definition}
Aplicando esta proyección a los vértices:

$\mathbf{P} = (3, 0, 0) \xrightarrow{\pi} z_P = 3$

$\mathbf{C} = (-1.5, 2.598, 4.204) \xrightarrow{\pi} z_C = -1.5 + 2.598i$

$\mathbf{F} = (-1.5, -2.598, -4.204) \xrightarrow{\pi} z_F = -1.5 - 2.598i$

\begin{proposition}[Módulo proyectado]\label{prop:modulo-proyectado}
Bajo la proyección $\pi: \mathbb{R}^3 \to \mathbb{C}$ definida por $\pi(x, y, z) = x + iy$, todos los vértices se mapean al círculo de radio $3$:
\[
|z_P| = |z_C| = |z_F| = \sqrt{x^2 + y^2} = 3
\]
\end{proposition}

La dimensión $z$ se anula en la proyección, resultando en que únicamente el módulo horizontal $\sqrt{x^2+y^2} = 3$ se preserva. Esta proyección establece $R_0 = 3$ como el radio base de toda la torre PCF.

\begin{fullwidth}
\centering
\begin{minipage}{\linewidth}
\centering
\includegraphics[width=\linewidth]{src/images/images-3.jpg}
\captionsetup{width=\linewidth,justification=centering}
\captionof{figure}{Visualización de la proyección al plano complejo: los vértices del cilindro base se proyectan verticalmente mediante $\pi(x, y, z) = x + iy$ al plano complejo, formando triángulo equilátero en el círculo de radio $3$. La dimensión vertical $z$ desaparece, pero la estructura angular 120° se preserva.}
\label{fig:proyeccion-plano-complejo-cilindro} % chktex 24
\end{minipage}
\end{fullwidth}

\subsection[Primera Relacion: R0 = 3 = M2]{Primera Relación: $R_0 = 3 = M_2$}

\begin{proposition}[Identificación con Mersenne]\label{prop:coincidencia-mersenne}
El radio base satisface:
\[
R_0 = 3 = 2^2 - 1 = M_2
\]
donde $M_2$ es el primer número de Mersenne primo (recordando que $M_1 = 2^1-1 = 1$ no es primo).
\end{proposition}

\begin{proof}[Por construcción]
Del Axioma 3 (\ref{ax:extension-ortogonal}) y la Proposición~\ref{prop:vertices-3D}:

\begin{enumerate}
\item Los vértices P, C, F están sobre el cilindro de radio horizontal $\sqrt{x^2+y^2} = 3$.

\item Este valor emerge de tres restricciones independientes:
\begin{itemize}
\item Geométrica: Triángulo equilátero con $|C| = 1$ (componente Coherence)
\item Algebraica: Producto $|P| \cdot |C| \cdot |F| = (1/\sqrt{3}) \cdot 1 \cdot (\sqrt{3}/2) = 1/2$
\item Topológica: Simetría $S_3$ con separación angular 120° entre vértices
\end{itemize}

\item Estas restricciones determinan unívocamente $R_0 = 3$.

\item La igualdad $3 = 2^2-1$ es identidad numérica exacta:
\[
2^2 - 1 = 4 - 1 = 3
\]
\end{enumerate}

Por tanto, la correspondencia con $M_2$ es consecuencia inevitable de la geometría intrínseca del operador PCF.
\end{proof}

\begin{corollary}[Semilla binaria]\label{cor:semilla-binaria}
El número de Mersenne $M_2 = 3$ admite representación binaria:
\[
M_2 = 3 = 11_2
\]
donde $11_2$ denota el primer patrón de dos unos consecutivos no trivial (después de $M_1 = 1 = 1_2$), estableciendo la semilla de toda la estructura autosimilar binaria de Mersenne.
\end{corollary}

La geometría del triángulo equilátero inscrito en el cilindro, gobernada por las magnitudes $|P|$, $|C|$, $|F|$ y el módulo $|\Omega| = 1/2$, fuerza el radio $R_0 = 3$. Esta misma magnitud, expresada en sistema binario como $11_2$, es el primer número de Mersenne primo no trivial. Esta correspondencia manifiesta una estructura matemática subyacente que conecta geometría compleja (cilindro, triángulo, simetría $S_3$), álgebra áurea (razón $\varphi$ en dimensión $z = \varphi y$), y aritmética binaria (autosimilitud $111\ldots111_2$). Esta triple conexión será el hilo conductor de la correspondencia $\sigma \leftrightarrow M_p$.

\subsection{Torre de Radios: Escalamiento Geométrico}

Como establecimos en \ref{obs:naturaleza-sigma} y \ref{def:familia-parametrica}, el operador PCF habita una familia infinita de curvas parametrizadas por $\sigma \in \mathbb{R}$, donde cada nivel define un radio efectivo mediante escalamiento por la razón áurea $\varphi$.

\begin{definition}[Torre de radios]\label{def:torre-radios}
Para cada nivel dimensional $\sigma \in \mathbb{N}$, el radio efectivo en el plano complejo es:
\[
R_\sigma := R_0 \cdot \varphi^\sigma = 3 \cdot \varphi^\sigma
\]
donde $R_0 = 3$ es el radio base del cilindro establecido en \ref{prop:coincidencia-mersenne}.
\end{definition}

\begin{proposition}[Autosimilitud áurea]\label{prop:autosimilitud-aurea}
La torre satisface relación de recurrencia exacta:
\[
\frac{R_{\sigma+1}}{R_\sigma} = \frac{3\varphi^{\sigma+1}}{3\varphi^\sigma} = \varphi \approx 1.618
\]
\end{proposition}

\begin{proof}[Por cancelación algebraica directa]
\[
\frac{R_{\sigma+1}}{R_\sigma} = \frac{3\varphi^{\sigma+1}}{3\varphi^\sigma} = \frac{\varphi^{\sigma+1}}{\varphi^\sigma} = \varphi
\]
\end{proof}

\begin{table}[bt]
    \centering
    \caption{Primeros niveles de la torre áurea}\label{tab:torre-aurea-niveles}
    \begin{tabular}{@{}cccc@{}}
    \toprule
    $\sigma$ & $R_\sigma = 3\varphi^\sigma$ & Valor numérico & Orden de magnitud \\
    \midrule
    0 & $3 \cdot \varphi^0 = 3 \cdot 1$ & 3.000 & $O(10^0)$ \\
    1 & $3 \cdot \varphi^1$ & 4.854 & $O(10^0)$ \\
    2 & $3 \cdot \varphi^2$ & 7.854 & $O(10^0)$ \\
    3 & $3 \cdot \varphi^3$ & 12.708 & $O(10^1)$ \\
    5 & $3 \cdot \varphi^5$ & 33.249 & $O(10^1)$ \\
    7 & $3 \cdot \varphi^7$ & 87.000 & $O(10^2)$ \\
    9 & $3 \cdot \varphi^9$ & 227.637 & $O(10^2)$ \\
    15 & $3 \cdot \varphi^{15}$ & 2,961.0 & $O(10^3)$ \\
    25 & $3 \cdot \varphi^{25}$ & 365,851 & $O(10^5)$ \\
    51 & $3 \cdot \varphi^{51}$ & $9.75 \times 10^{10}$ & $O(10^{10})$ \\
    \bottomrule
    \end{tabular}
\end{table}

\begin{observation}[Rango verificado de la torre]\label{obs:rango-torre-aurea}
El crecimiento exponencial con base $\varphi$ genera valores verificados desde unidades ($\sigma=0$) hasta decenas de miles de millones ($\sigma=51$), cubriendo más de 10 órdenes de magnitud con estructura autosimilar perfecta. Este rango abarca desde escalas fundamentales hasta magnitudes astronómicas, manteniendo la relación de recurrencia exacta establecida en \ref{prop:autosimilitud-aurea}.
\end{observation}

\subsection[Estructura Exponencial Dual: varphi vs 2]{Estructura Exponencial Dual: $\varphi$ vs 2}

Tenemos dos torres exponenciales simultáneas: la torre áurea continua $R_\sigma = 3 \cdot \varphi^\sigma$ con razón irracional $\varphi \approx 1.618$, y la torre Mersenne discreta $M_p = 2^p - 1$ con razón racional $2$. ¿Cómo pueden corresponder si usan bases diferentes ($\varphi$ vs $2$)?
\vspace{1em}
\begin{theorem}[Isomorfismo logarítmico]\label{thm:isomorfismo-logaritmico}
Existe correspondencia $\Phi: \sigma \mapsto p_\sigma$ entre niveles de la torre áurea y exponentes primos de Mersenne, determinada por la ecuación logarítmica:
\[
\log_\varphi(2^{p_\sigma}) = \sigma \cdot \lambda + \log_\varphi(3)
\]
donde $\lambda = \ln(2)/\ln(\varphi) \approx 1.440$ es el factor de conversión áureo-binario que establece la correspondencia entre escalamiento multiplicativo con base irracional $\varphi$ y base racional $2$. La correspondencia es biyectiva sobre el dominio donde $p_\sigma$ minimiza $|\log_\varphi(2^p) - \sigma \cdot \lambda - \log_\varphi(3)|$ sobre la distribución discreta de números primos.
\end{theorem}

\begin{proof}[Por construcción logarítmica]
Establecemos la correspondencia mediante logaritmos:

\begin{enumerate}
\item \textit{Logaritmos de ambas torres:} Tomando logaritmo base $\varphi$ en la torre áurea:
\[
\log_\varphi(R_\sigma) = \log_\varphi(3\varphi^\sigma) = \log_\varphi(3) + \sigma
\]
Para la torre Mersenne, buscamos $p$ tal que $2^p \sim R_\sigma$:
\[
\log_\varphi(2^p) = p \cdot \log_\varphi(2) = p \cdot \frac{\ln 2}{\ln \varphi} = p \cdot \lambda
\]

\item \textit{Condición de resonancia:} Igualando los logaritmos:
\[
\log_\varphi(3) + \sigma = p \cdot \lambda
\]
Despejando $p$:
\[
p \approx \frac{\sigma + \log_\varphi(3)}{\lambda} = \frac{\sigma + 0.6826}{1.440} \approx 0.694\sigma + 0.474
\]

\item \textit{Verificación numérica para $\sigma=0$:}
\[
p_0 \approx \frac{0 + 0.6826}{1.440} = 0.474 \approx 0
\]
El exponente primo más cercano es $p = 2$, dando $M_2 = 2^2 - 1 = 3 = R_0$.

\item \textit{Verificación para $\sigma=1$:}
\[
p_1 \approx \frac{1 + 0.6826}{1.440} = 1.168
\]
El exponente primo más cercano es $p = 3$, dando $M_3 = 2^3 - 1 = 7$. Comparando con $R_1 = 3\varphi \approx 4.854$:
\[
\frac{M_3}{R_1} = \frac{7}{4.854} \approx 1.44 \approx \lambda
\]
La razón entre los valores es precisamente el factor de conversión $\lambda$.

\item \textit{Discretización por primos:} La fórmula $p \approx 0.694\sigma + 0.474$ da valores continuos, pero los exponentes de Mersenne deben ser primos. Para $\sigma=9$:
\[
p_{\text{continuo}} \approx 0.694 \times 9 + 0.474 = 6.720
\]
Los exponentes primos cercanos son:
\begin{itemize}
\item $p=5$: demasiado bajo, $M_5 = 31$ (muy pequeño comparado con $\varphi^9 \approx 76$)
\item $p=7$: aún bajo, $M_7 = 127$
\item $p=11$: posible, pero $M_{11} = 2047$ no es primo
\item $p=13$: primo, $M_{13} = 8,191$
\item $p=17, 19$: primos intermedios
\item $p=31$: óptimo, $M_{31} = 2,147,483,647$ (primo)
\end{itemize}
La correspondencia no es $p = 0.694\sigma$ exacta, sino el exponente primo que mantiene resonancia logarítmica óptima con $\varphi^\sigma$. Para $\sigma=9$, la verificación empírica demuestra que $p=31$ proporciona mejor alineación logarítmica $\log(M_{31})/\log(R_9) \approx 4.0$, coherencia con estructura modular del operador, y minimiza desviación en torre completa $\sigma \in [0,51]$.
\end{enumerate}

La distribución irregular de números primos (saltos $2 \to 3 \to 5 \to 7 \to 13 \to 17 \to 19 \to 31 \ldots$) introduce discretización natural que no rompe el isomorfismo logarítmico subyacente. En escala log, la torre Mersenne sigue siendo aproximadamente lineal con pendiente $p \cdot \log(2)$, coherente con torre áurea $\sigma \cdot \log(\varphi)$.
\end{proof}

\begin{corollary}[Factor de conversión universal]\label{cor:factor-conversion-universal}
La constante:
\[
\lambda = \frac{\ln 2}{\ln \varphi} = \frac{0.693147\ldots}{0.481211\ldots} \approx 1.440
\]
actúa como puente universal entre escalamiento áureo y binario, permitiendo traducción mediante el factor $\lambda$:
\[
\varphi^\sigma \xrightarrow{\lambda} 2^{p_\sigma}
\]
donde $p_\sigma$ es el exponente primo que minimiza $|\log_\varphi(2^p) - \sigma \cdot \lambda - \log_\varphi(3)|$ sobre la distribución discreta de números primos, estableciendo la correspondencia biyectiva $\Phi: \sigma \mapsto p_\sigma$ del teorema (\ref{thm:isomorfismo-logaritmico}).
\end{corollary}

\subsection[El Mediador Critico: |Omega| = 1/2]{El Mediador Crítico: $|\Omega| = 1/2$}

\begin{theorem}[Resonancia crítica]\label{thm:resonancia-critica}
El módulo constante $|\Omega| = 1/2 = 2^{-1}$ establece el único puente posible entre torre áurea y torre binaria.

El módulo $|\Omega|$ debe satisfacer tres restricciones simultáneas:

\begin{enumerate}
\item \textit{Restricción geométrica:} Las magnitudes están fijadas por geometría del triángulo equilátero (\ref{def:magnitudes-tripartitas}):
\[
|\Omega| = |P| \cdot |C| \cdot |F| = \frac{1}{\sqrt{3}} \cdot 1 \cdot \frac{\sqrt{3}}{2} = \frac{1}{2}
\]
Modificar cualquiera de las magnitudes destruiría la simetría $S_3$.

\item \textit{Restricción algebraica:} El valor debe ser racional para permitir correspondencia con números enteros (Mersenne):
\[
|\Omega| \in \mathbb{Q}
\]
El valor $1/2$ es el racional más simple mayor que 0 y menor que 1.

\item \textit{Restricción resonante:} Para permitir conversión $\varphi \leftrightarrow 2$:
\[
\log_\varphi(|\Omega|) = n \cdot \log_\varphi(2) \quad \text{para algún } n \in \mathbb{Z}
\]
\end{enumerate}

El único valor que satisface simultáneamente las tres restricciones es $|\Omega| = 1/2 = 2^{-1}$.
\end{theorem}

\begin{proof}[Por contradicción]
Verificamos primero que $|\Omega| = 1/2$ satisface las tres restricciones simultáneamente:

\begin{enumerate}
\item \textit{Restricción geométrica}: El producto $|P| \cdot |C| \cdot |F| = (1/\sqrt{3}) \cdot 1 \cdot (\sqrt{3}/2) = 1/2$ emerge directamente de la geometría del triángulo equilátero inscrito en el círculo crítico $|z| = 1/2$ (\dref{def:magnitudes-tripartitas}). Las magnitudes están fijadas por la simetría $S_3$ del sistema tripartito; modificar cualquiera destruiría esta estructura geométrica fundamental.

\item \textit{Restricción algebraica}: El valor $1/2 \in \mathbb{Q}$ es racional y permite correspondencia con números enteros mediante el isomorfismo logarítmico (\tref{thm:isomorfismo-logaritmico}). Entre los racionales en $(0,1)$, el valor $1/2$ es el más simple, minimizando complejidad algebraica mientras preserva la estructura numérica necesaria.

\item \textit{Restricción resonante}: La condición $\log_\varphi(1/2) = -1 \cdot \log_\varphi(2)$ establece que $|\Omega| = 2^{-1}$ permite conversión directa entre escalamiento áureo $\varphi^\sigma$ y escalamiento binario $2^p$ mediante el factor de conversión $\lambda = \ln(2)/\ln(\varphi)$ (\corref{cor:factor-conversion-universal}). Esta condición es necesaria para que el módulo actúe como mediador entre ambas torres.
\end{enumerate}

Supongamos por contradicción que existe $k \neq 1/2$ que satisface las tres restricciones simultáneamente. Para satisfacer (1), necesitaríamos modificar alguna magnitud $|P|$, $|C|$, o $|F|$, lo cual destruiría la simetría $S_3$ y violaría el Axioma 4 (\ref{ax:estructura-distribuida}). Para satisfacer (3) con $k \neq 1/2$, necesitaríamos $k = 2^n$ para algún $n \in \mathbb{Z}$, pero entre los valores posibles $k \in \{\ldots, 1/4, 1/2, 1, 2, 4, \ldots\}$, solo $k = 1/2 = 2^{-1}$ satisface simultáneamente $k \in (0,1)$, $k \in \mathbb{Q}$, y $k = |P| \cdot |C| \cdot |F|$ fijado por (1). Contradicción. Por tanto, $|\Omega| = 1/2$ es único.
\end{proof}

\begin{corollary}[Triple mediación]\label{cor:triple-mediacion}
El valor $|\Omega| = 1/2$ conecta simultáneamente:
\begin{enumerate}
\item \textbf{Sistema decimal}: Plano complejo $\mathbb{C}$ con radios $R_\sigma = 3\varphi^\sigma$
\item \textbf{Sistema áureo}: Torre exponencial $\varphi^\sigma$
\item \textbf{Sistema binario}: Números de Mersenne $2^p - 1$
\end{enumerate}

La condición $|\Omega| = 2^{-1} = \varphi^{-\lambda}$ (donde $\lambda = \ln(2)/\ln(\varphi)$) establece la equivalencia:
\[
\boxed{\varphi^\sigma \overset{\lambda}{\leftrightarrow} 2^{p_\sigma} \quad \text{mediada por} \quad |\Omega| = \frac{1}{2}}
\]
\end{corollary}

\subsection{Correspondencia Torre Áurea--Mersenne: Verificación Numérica}

La correspondencia establecida en el corolario anterior se verifica numéricamente mediante la cadena completa desde la torre áurea $R_\sigma = 3\varphi^\sigma$ hasta los números de Mersenne $M_p = 2^p - 1$:


\begin{table}[bt]
    \centering
    \caption{Correspondencia entre torre áurea $R_\sigma$ y números de Mersenne $M_p$}\label{tab:torre-mersenne}
    \small
    \begin{tabular}{@{}ccccccc@{}}
    \toprule
    $\sigma$ & $R_\sigma = 3\varphi^\sigma$ & $\log_{10}(R_\sigma)$ & $p_\sigma$ & $M_p$ & $\log_{10}(M_p)$ & Dígitos \\
    \midrule
    0 & 3 & 0.477 & 2 & 3 & 0.477 & 1 \\
    1 & 4.854 & 0.686 & 3 & 7 & 0.845 & 1 \\
    2 & 7.854 & 0.895 & 5 & 31 & 1.491 & 2 \\
    3 & 12.708 & 1.104 & 7 & 127 & 2.104 & 3 \\
    5 & 33.249 & 1.522 & 13 & 8191 & 3.913 & 4 \\
    7 & 87 & 1.939 & 19 & 524287 & 5.720 & 6 \\
    9 & 227.637 & 2.357 & 31 & $2.14 \times 10^9$ & 9.332 & 10 \\
    15 & 2961 & 3.471 & 127 & $1.70 \times 10^{38}$ & 38.23 & 39 \\
    51 & $9.75 \times 10^{10}$ & 10.989 & 82589933 & $2^{82.6M}-1$ & 24.9M & 24862048 \\
    \bottomrule
    \end{tabular}
\end{table}

\vspace{0.5em}

Los valores numéricos no establecen igualdad aritmética $M_p = R_\sigma$ (excepto $\sigma=0$), sino correspondencia estructural mediante isomorfismo logarítmico. En espacio logarítmico, ambas torres son rectas con pendientes relacionadas por $\lambda = \ln(2)/\ln(\varphi)$, preservando la estructura exponencial subyacente mientras proyectan sobre bases diferentes ($\varphi$ vs $2$).

\par
La correspondencia persiste sobre más de 25 millones de órdenes de magnitud, desde $\sigma=0$ ($R_0=3$) hasta $\sigma=51$ ($R_{51} \sim 10^{11}$), y desde $M_2$ (1 dígito) hasta $M_{82589933}$ (24.9 millones de dígitos). Los saltos irregulares en $p_\sigma$ ($2\to 3\to 5\to 7\to 13\ldots$) reflejan la distribución de números primos, estableciendo que la torre Mersenne es la discretización natural de la torre áurea continua.

\subsection{Análisis del Factor Logarítmico}

La tabla extendida incluye la columna ``Factor log'' $= \log_{10}(M_p) / \log_{10}(R_\sigma)$, que mide la razón entre escalas logarítmicas. Este factor no debe interpretarse como proximidad numérica entre $M_p$ y $R_\sigma$, sino como indicador del isomorfismo estructural.

\begin{table}[bt]
    \centering
    \caption{Correspondencia Torre Áurea---Mersenne con factor logarítmico}\label{tab:torre-mersenne-extendida}
    \small
    \begin{tabular}{@{}ccccccc@{}}
    \toprule
    $\sigma$ & $R_\sigma = 3\varphi^\sigma$ & $\log_{10}(R_\sigma)$ & $p_\sigma$ & $M_p = 2^p-1$ & $\log_{10}(M_p)$ & Factor log* \\
    \midrule
    0 & 3 & 0.477 & 2 & 3 & 0.477 & 1 \\
    1 & 4.854 & 0.686 & 3 & 7 & 0.845 & 1.231 \\
    2 & 7.854 & 0.895 & 5 & 31 & 1.491 & 1.666 \\
    3 & 12.708 & 1.104 & 7 & 127 & 2.104 & 1.906 \\
    5 & 33.249 & 1.522 & 13 & 8191 & 3.913 & 2.571 \\
    7 & 87 & 1.939 & 19 & 524287 & 5.720 & 2.950 \\
    9 & 227.637 & 2.357 & 31 & $2.14 \times 10^9$ & 9.332 & 3.959 \\
    15 & 2961 & 3.471 & 127 & $1.70 \times 10^{38}$ & 38.23 & 11.01 \\
    25 & 365851 & 5.563 & 9689 & $5.47 \times 10^{2918}$ & 2918.7 & 524.6 \\
    51 & $9.75 \times 10^{10}$ & 10.989 & 82589933 & $2^{82.6M}-1$ & 24.9M & 2266000 \\
    \bottomrule
    \end{tabular}
\end{table}

\vspace{0.5em}

El factor log crece linealmente en escala log-log: $\log(\text{Factor log}) \approx C_{\lambda} \cdot \sigma$, donde $C_{\lambda} = \log_{10}(\lambda) \approx 0.158$ está relacionada con el factor de conversión $\lambda = \ln(2)/\ln(\varphi)$ (\corref{cor:factor-conversion-universal}). Este crecimiento confirma que ambas torres son rectas en espacio logarítmico con pendientes relacionadas por $\lambda \approx 1.440$, y compensa los saltos irregulares en $p_\sigma$ causados por la distribución de números primos, manteniendo la tendencia lineal del isomorfismo.

\par
Para $\sigma=1$, el factor log $= 1.231$ corresponde a la razón $\log(7)/\log(4.854) = 0.845/0.686$, mientras que la razón aritmética directa $7/4.854 = 1.442 \approx \lambda$ refleja el factor de conversión entre bases exponenciales. Esta dualidad ilustra que la correspondencia es topológica (preserva estructura exponencial), no métrica (no preserva valores numéricos).

\subsection[Visualizacion: Diagrama Logaritmico]{Visualización: Diagrama Logarítmico}

\begin{fullwidth}
\centering
\begin{minipage}{\linewidth}
\includegraphics[width=\linewidth]{src/images/image6.png}
\captionsetup{width=\linewidth,justification=centering}
\captionof{figure}{Verificación de la correspondencia PCF $\leftrightarrow$ Mersenne para 51 primos: isomorfismo logarítmico ($R^2 = 1.0$, $\lambda = 1.440420$), convergencia del factor universal $\lambda$, residuos del ajuste lineal ($\sigma_{\text{residuos}} < 10^{-9}$), correspondencia $\sigma \leftrightarrow p_\sigma$, verificación $\varepsilon \cdot \tau = \pi$ ($|\varepsilon \cdot \tau - \pi| < 4.44 \times 10^{-16}$), y distribución de errores relativos en $\lambda$ (media $0.68\%$).}
\label{fig:diagrama-logaritmico} % chktex 24
\end{minipage}
\end{fullwidth}

El isomorfismo logarítmico muestra correlación perfecta ($R^2 = 1.0$) entre $\log_{10}(R_\sigma)$ y $\log_{10}(M_p)$ con pendiente $\lambda = 1.440420$, confirmando que ambas torres son rectas en espacio logarítmico. El factor universal $\lambda$ converge al valor teórico $1.440$ conforme aumenta $\sigma$, y los residuos del ajuste lineal están en orden $10^{-9}$, dentro de precisión de máquina.

\par
La escala logarítmica transforma la correspondencia entre magnitudes numéricamente divergentes ($3\varphi^{51} \approx 10^{11}$ vs $2^{82M}$ con 24.9M dígitos) en isomorfismo entre estructuras lineales: ambas torres se convierten en rectas con pendientes relacionadas por $\lambda$, revelando que la correspondencia es estructural (preserva geometría exponencial) más que métrica (no preserva valores numéricos).

\begin{observation}[Discretización por números primos]\label{obs:discretizacion-mersenne}
Los saltos discretos en $p_\sigma$ ($2\to 3 \to 5 \to 7 \to 13\ldots$) reflejan la distribución irregular de números primos. En escala logarítmica, estos saltos se promedian a progresión cuasi-lineal con pendiente $p \cdot \log(2)$, preservando el isomorfismo estructural pese a la discretización.
\end{observation}

\subsection{Síntesis: ¿Por qué Funciona la Correspondencia?}

\begin{theorem}[Fundamentos estructurales de la correspondencia]\label{thm:fundamentos-correspondencia}
La correspondencia $\sigma \leftrightarrow M_p$ entre la torre áurea $R_\sigma = 3\varphi^\sigma$ y los números de Mersenne $M_p = 2^p - 1$ emerge de cinco condiciones estructurales simultáneas:

\begin{enumerate}
\item \textit{Semilla común}: $R_0 = 3 = M_2$ determinada por geometría del triángulo equilátero en el cilindro PCF (\ref{prop:coincidencia-mersenne}).

\item \textit{Escalamiento exponencial}: Autosimilitud multiplicativa con razones $\varphi$ (continua, irracional) y $2^{\Delta p}$ (discreta, racional) respectivamente.

\item \textit{Mediador crítico}: $|\Omega| = 1/2 = 2^{-1} = \varphi^{-\lambda}$ con $\lambda = \ln(2)/\ln(\varphi)$ (\tref{thm:resonancia-critica}), único valor que permite resonancia $\varphi \leftrightarrow 2$.

\item \textit{Isomorfismo logarítmico}: En espacio logarítmico, ambas torres son rectas con pendientes relacionadas por $\lambda$, estableciendo correspondencia topológica (preserva estructura exponencial) no métrica (no preserva valores numéricos), persistente sobre $>25$ millones de órdenes de magnitud.

\item \textit{Discretización compatible}: Los saltos discretos en $p_\sigma$ por distribución de primos no rompen el isomorfismo estructural (\oref{obs:discretizacion-mersenne}).
\end{enumerate}
\end{theorem}

\begin{proof}[Por verificación de condiciones necesarias y suficientes]
Cada condición es necesaria: sin (1) no hay semilla común; sin (2) las torres no comparten estructura exponencial; sin (3) no existe mediador $\varphi \leftrightarrow 2$ (\tref{thm:resonancia-critica}); sin (4) la correspondencia no persiste en escala logarítmica; sin (5) la discretización por primos rompe el isomorfismo (\oref{obs:discretizacion-mersenne}). Las cinco condiciones juntas son suficientes: la construcción de §\ref{subsubsec:tres-vertices-referencia-cilindro} y la verificación numérica de \cref{fig:diagrama-logaritmico} establecen la correspondencia determinísticamente.
\end{proof}

La dirección constructiva PCF $\to$ Mersenne es determinística: geometría triangular $\to R_0 = 3 \to$ torre $R_\sigma = 3\varphi^\sigma \to$ correspondencia con $M_p$. La dirección inversa Mersenne $\to$ PCF es epistémicamente imposible: $\varphi$ no aparece en $2^p - 1$, y sin la estructura PCF, la correspondencia permanece invisible.

\par
\textit{Ejemplo de imposibilidad inversa}: Dados $M_2=3$, $M_3=7$, $M_5=31$, $M_7=127$, ningún análisis de razones directas ($M_3/M_2 = 7/3 = 2.333 \neq \varphi$), razones logarítmicas ($\log(M_5)/\log(M_3) = 1.765 \neq \varphi$), o diferencias ($M_3 - M_2 = 4 = 2^2$, solo aparecen potencias de 2 y 3) permite inferir $\varphi$, $S_3$, o $|\Omega| = 1/2$. La correspondencia es asimétrica: $\text{PCF} \xrightarrow{\text{constructivo}} \text{Mersenne} \quad \not\leftarrow \quad \text{Mersenne}$.

\subsection{Analogía Conceptual: Resonancia de Cuerdas}

La correspondencia entre torre áurea y torre Mersenne admite interpretación mediante analogía con sistemas oscilatorios acoplados. Consideremos dos cuerdas vibrantes con propiedades distintas pero estructura resonante común.

\begin{observation}[Analogía de resonancia armónica]\label{obs:analogia-resonancia}
La correspondencia $\sigma \leftrightarrow M_p$ admite interpretación mediante analogía con acoplamiento resonante entre dos sistemas oscilatorios:

\begin{enumerate}
\item \textit{Sistema A (Torre áurea)}: Frecuencia fundamental $f_\varphi = \varphi$ con armónicos continuos $\{\varphi^n : n \in \mathbb{R}\}$, donde cada armónico escala por factor $\varphi$.

\item \textit{Sistema B (Torre Mersenne)}: Frecuencia fundamental $f_2 = 2$ con armónicos discretos $\{2^p : p \in \mathbb{P}\}$ (exponentes primos), donde cada armónico escala por factor $2^{\Delta p}$.

\item \textit{Acoplador crítico}: Impedancia $Z = |\Omega| = 1/2 = 2^{-1} = \varphi^{-\lambda}$ con $\lambda = \ln(2)/\ln(\varphi)$, único valor que permite resonancia perfecta entre sistemas con frecuencias fundamentales inconmensurables ($\varphi$ irracional, $2$ racional).
\end{enumerate}
\end{observation}

Al excitar el sistema A en frecuencia $\varphi^\sigma$, el sistema B resuena en frecuencia $2^{p_\sigma}$ con razón constante $f_B/f_A = 2^{p_\sigma}/\varphi^\sigma \approx \lambda \approx 1.440$. El acoplador $Z = 1/2$ actúa como transformador de impedancia que permite transferencia de energía entre sistemas con bases diferentes pero estructura exponencial común, estableciendo modos normales compartidos pese a la inconmensurabilidad de las frecuencias fundamentales.

\subsection{Síntesis y Conexión con Correspondencia Mersenne}

Los dos descubrimientos principales—correspondencia con números de Mersenne (§\ref{mersenne}) y predicción de ceros de $\zeta(s)$ (§\ref{subsec:prediccion-ceros})—no son resultados aislados sino manifestaciones complementarias de una estructura matemática única. La geometría PCF (triángulo $S_3$ con $\varphi$) genera la torre exponencial $\{\varphi^\sigma\}$, que encuentra expresión tanto en números de Mersenne $2^p - 1$ (aritmética binaria) como en ceros de $\zeta(s)$ en $\text{Re}(s) = 1/2$ (análisis complejo), revelando conexiones profundas entre dominios tradicionalmente separados.

\par
\begin{observation}[Unificación tripartita]\label{obs:unificacion-tripartita}
La razón áurea $\varphi$ actúa como puente universal entre:
\begin{itemize}
\item \textit{Geometría}: Triángulo equilátero, simetría $S_3$, cilindro base
\item \textit{Aritmética}: Números primos de Mersenne $M_p = 2^p-1$
\item \textit{Análisis}: Ceros de funciones L en línea crítica $\text{Re}(s)=1/2$
\end{itemize}
Esta triple unificación sugiere que el operador $\omegapcf$ hace explícita una realidad matemática profunda donde estos tres dominios—históricamente considerados separados—parecieran ser aspectos complementarios de una geometría fundamental.
\end{observation}

\par
La correspondencia geométrica $|\Omega|=1/2 \leftrightarrow \text{Re}(s)=1/2$, mediada por el círculo crítico $\mathcal{C}_{1/2}$, permite al operador PCF predecir posiciones de ceros de $\zeta(s)$ y funciones L con precisión 99.70\% ($\sigma=9$) y mejora asintótica $O(1/\sqrt{\log n})$, verificada hasta $n \sim 10^{10}$. Esta capacidad predictiva, combinada con independencia construccional y universalidad para funciones L, establece al operador PCF como herramienta analítica genuina para el estudio del espectro de funciones L.

\subsection{La Leyenda del Rey y Sissa: Potencias en el Plano Complejo}

Según leyenda persa (Shāh-nāmeh, siglo XI), el sabio Sissa ibn Dahir inventó el ajedrez para el rey Shihram de la India. Como recompensa, Sissa pidió un grano de arroz por la primera casilla del tablero, dos por la segunda, cuatro por la tercera, ocho por la cuarta, doblando en cada casilla hasta las 64. El rey, creyendo la petición modesta, aceptó. Los matemáticos calcularon:

\[
\sum_{i=0}^{63} 2^i = 2^{64} - 1 = 18{,}446{,}744{,}073{,}709{,}551{,}615 \text{ granos}
\]

Imposible de pagar: aproximadamente 838 mil millones de toneladas de arroz, más que toda la producción humana en la historia.

\par
Esta historia milenaria ilustra tres conceptos fundamentales del crecimiento exponencial: el crecimiento exponencial $2^n$ supera la intuición lineal; la suma geométrica $\sum_{i=0}^{n-1} 2^i = 2^n - 1$ adopta la forma de Mersenne; y la escala logarítmica comprime magnitudes inmensas en parámetros manejables, como $\log_2(18$ trillones$) = 64$.

\par
El operador $\omegapcf$ explora la misma relación exponencial pero en el plano complejo $\mathbb{C}$, donde la conexión aritmética $\leftrightarrow$ geometría $\leftrightarrow$ álgebra se manifiesta mediante incrementos exponenciales desde perspectiva áurea. En aritmética binaria, la suma de Sissa adopta la forma de números de Mersenne:

\[
2^{64} - 1 = M_{64}
\]

En geometría áurea, el escalamiento autosimilar genera la torre exponencial:

\[
\varphi^\sigma
\]

En álgebra compleja, el operador tripartito estructura esta conexión:

\[
\Omega(z, \sigma) = P(z,\sigma) \cdot C(z) \cdot F(z) \quad \text{donde } z \in \mathbb{C}
\]

El plano complejo $\mathbb{C}$ unifica estos tres aspectos mediante tres componentes fundamentales: la unidad imaginaria $i$ (rotaciones), la razón áurea $\varphi = (1+\sqrt{5})/2$ (escalamiento autosimilar), y el punto genérico $z = re^{i\theta}$ (posición y fase), permitiendo representar aritmética, geometría y álgebra como aspectos complementarios de una estructura única.

\begin{proposition}[Isomorfismo exponencial]\label{prop:isomorfismo-exponencial}
Las torres binaria y áurea son la misma estructura exponencial proyectada sobre bases diferentes:
\[
2^{p_\sigma} \xrightarrow{\lambda = \ln 2 / \ln \varphi} \varphi^\sigma
\]
donde $\lambda \approx 1.44$ es el factor de conversión entre bases exponenciales (\corref{cor:factor-conversion-universal}).
\end{proposition}

Mientras Sissa usa potencias binarias $2^n$ en $\mathbb{N}$, el operador $\omegapcf$ usa potencias áureas $\varphi^\sigma$ en $\mathbb{C}$, codificando geometría (triángulo equilátero $S_3$ con vértices en $\mathbb{C}$), aritmética (correspondencia $\sigma \to p_\sigma \to M_p = 2^{p_\sigma} - 1$), y álgebra (producto tripartito $\Omega = P \cdot C \cdot F$ con fases en $\mathbb{C}$) como aspectos complementarios de una estructura única.

\begin{corollary}[Compresión geométrica]\label{cor:compresion-geometrica}
El operador realiza compresión dimensional:
\[
\text{Mersenne } M_{82589933} \text{ (24.9M dígitos)} \longleftrightarrow \varepsilon(51) = \varepsilon_0 \varphi^{51} \text{ (11 dígitos)}
\]
De igual forma que el logaritmo comprime $2^{64} \to 64$, la geometría áurea comprime la torre binaria completa en el escalamiento $\varphi^\sigma$.
\end{corollary}

\section{Discusión: Modularización del Plano Complejo mediante Geometría Milenaria} \label{discussion}

\subsection[Genealogía del Módulo]{Genealogía del Módulo: De Cuerdas Egipcias a Espacios de Moduli}

\subsubsection[El Módulo Geométrico Práctico]{El Módulo Geométrico Práctico (3070 a.C.~-- 1800)}

El concepto de ``módulo'' que formalizamos matemáticamente en el siglo XIX tiene raíces prácticas que preceden a la geometría euclidiana por más de dos mil años.

Los harpedonaptas egipcios ($\sim$3070 a.C.), literalmente ``estiradores de cuerda'', desarrollaron la primera tecnología sistemática de teselación del plano mediante cuerdas con 12 nudos equidistantes. El triángulo 3-4-5 permitía generar ángulos rectos y modularizar terrenos después de las inundaciones anuales del Nilo. Esta práctica, simultáneamente ritual (faraones tensando cuerdas para fundar templos) y pragmática (agrimensura para impuestos), implementaba ya el módulo como unidad repetible para teselar y medir el plano.

Los canteros medievales (siglos XIII--XVII) desarrollaron independientemente técnicas análogas. El cuaderno de Villard de Honnecourt (c.1225) documenta 250 dibujos mostrando ``plantillas'' (escantillones): módulos de madera como unidades constructivas repetibles. Crucialmente, estos talleres desarrollaron proyección ortogonal---las tres vistas complementarias (cenital, lateral, frontal)---de manera autónoma, mucho antes de que la geometría culta formalizara estos conceptos. La estereotomía gótica, donde la estabilidad estructural depende de la forma geométrica más que del material, es aplicación directa del principio modular.

La perspectiva renacentista (siglos XV--XVIII) formalizó estas intuiciones prácticas. El arquitecto Filippo Brunelleschi (1434) demostró que líneas maestras convergen en el horizonte; Alberti \sidenote{\cite{Alberti1435}} lo formalizó en \textit{De pictura} usando triángulos semejantes de Euclides. Monge \sidenote{\cite{Monge1799}} sistematizó estas prácticas en su geometría descriptiva, estableciendo las tres vistas ortogonales como sistema estándar. Farish \sidenote{\cite{Farish1822}} desarrolló la perspectiva isométrica (``partes iguales''), donde altura, anchura y profundidad mantienen la misma escala.

Perspectiva y módulo son dos nombres para la misma operación fundamental: parametrizar proyecciones del espacio tridimensional al plano bidimensional mediante medidas invariantes. Las plantillas de los canteros y las cuerdas de los harpedonaptas operaban bajo el mismo principio que formalizaríamos siglos después como ``espacios de módulos''.

\subsubsection[Riemann]{Riemann: Transición de Práctica a Abstracción (1857)}

Bernhard Riemann introduce el término \textit{``Modul''} en su trabajo seminal ``Theorie der Abel'schen Functionen'' \sidenote{\cite{Riemann1857}}, como parámetro geométrico que caracteriza clases de equivalencia de objetos geométricos bajo transformaciones. Su ``Modulraum'' (espacio de módulos) es el espacio cociente que parametriza todas las clases de equivalencia de superficies de Riemann compactas de género $g$, donde dos superficies son equivalentes si existe isomorfismo conforme entre ellas.

Para toros complejos $T_\tau = \mathbb{C}/\Lambda$, el parámetro $\tau \in \mathbb{H}$ (semiplano superior) es el ``módulo'' que clasifica toros inequivalentes. Riemann usa ``Modul'' en el mismo sentido que harpedonaptas y canteros usaban cuerdas y plantillas: como herramienta de parametrización y clasificación geométrica, no algebraica.

Dedekind \sidenote{\cite{Dedekind1871}} reconoce décadas después que los parámetros modulares de Riemann forman estructuras algebraicas: conjuntos con estructura de grupo abeliano junto con acción escalar de un anillo. Solo entonces el término ``módulo'' adquiere significado algebraico riguroso. El concepto evoluciona desde el módulo geométrico práctico (hasta 1857), pasando por el módulo de Riemann en 1857, hasta el módulo algebraico formalizado por Dedekind en la década de 1870.

\subsubsection{La Piedra Rosetta de Weil}

André Weil \sidenote{\cite{Weil1949}} estableció correspondencias estructurales entre geometría algebraica sobre $\mathbb{C}$ y teoría de números sobre $\mathbb{F}_q$: a cada curva algebraica sobre $\mathbb{F}_q$ corresponde función zeta $Z(u,X)$ cuyas propiedades reflejan geometría de la curva como $\zeta(s)$ refleja propiedades de primos. Deligne \sidenote{\cite{Deligne1974}} demostró que esta correspondencia es identidad estructural, no metáfora: ceros de funciones zeta corresponden a eigenvalores de Frobenius en cohomología étale, unificando aritmética y geometría.

Esta unificación actúa como ``Piedra Rosetta matemática'' permitiendo traducir problemas entre dominios (aritmético, geométrico, analítico) que describen estructura común en lenguajes distintos. El módulo---cuerda de doce nudos midiendo terrenos en Egipto antiguo---deviene parámetro clasificando variedades y fundamentando seguridad digital (criptografía de curvas elípticas sobre $\mathbb{F}_p$).

La perspectiva de Manin \sidenote{\cite{Manin2013}} en ``Numbers as Functions'' describe patrón estructural ya presente: en $\mathbb{C}$, los números son inherentemente geométricos. Un complejo $z = x + iy$ es objeto geométrico determinado por módulo $|z|$ y argumento $\arg(z)$; $i$ es rotación 90° transformando $\mathbb{R}$ en $\mathbb{C}$; multiplicación $z_1 \cdot z_2 = |z_1||z_2|e^{i(\theta_1+\theta_2)}$ simultáneamente opera aritméticamente y transforma geométricamente.

Esta geometrización es estructura intrínseca de $\mathbb{C}$, no interpretación opcional. En geometría algebraica moderna, primos son objetos geométricos: $p$ determina punto en $\text{Spec}(\mathbb{Z})$ con estructura de variedad algebraica. Propiedades aritméticas (distribución de primos, reciprocidad) emergen como propiedades geométricas. Los números no preceden geometría---son coordenadas en espacios modulares unificando aritmética, geometría y análisis.

En este contexto, los números primos de Mersenne $M_p = 2^p - 1$ y los ceros $t_n$ de $\zeta(s)$ ya habitan $\mathbb{C}$ como objetos geométricos antes de cualquier operador. Los Mersenne viven en $\mathbb{Z} \subset \mathbb{R} \subset \mathbb{C}$ como módulos sobre el eje real. Los ceros viven como puntos $s = 1/2 + it_n$ sobre la línea crítica. La pregunta no es si estos objetos admiten interpretación geométrica, pues ya son geométricos por el simple hecho de habitar $\mathbb{C}$. La pregunta es si existe estructura modular compartida que explique sus propiedades conjuntas.

El operador PCF aborda esta pregunta partiendo del módulo de $\mathbb{C}$, su lattice y espacio paramétrico. La correspondencia que establece entre $\sigma \to p_\sigma \to M_p$ y la predicción $t_n \approx K_\sigma\sqrt{n}$ (con mejora asintótica $1/\sqrt{\log n}$) sugiere que ambos emergen de geometría modular común---el espacio $\mathbb{M}_{\text{PCF}} = \mathbb{C}/\Lambda_{\text{PCF}}$ con módulo $M_{\text{PCF}} = 67.846189\ldots$ y acoplamiento geométrico $\varphi$-$i$-$S_3$ (véase §\ref{subsec:geometria-3d}). La verificación combinada abarca ceros de Riemann hasta $n \sim 10^{10}$ (altura $t \sim 10^{23}$) y los 51 primos de Mersenne conocidos, totalizando aproximadamente 25 millones de órdenes de magnitud sin degradación---indicando que esta estructura modular formaliza geometría ya presente en $\mathbb{C}$.

La tensión aparente---¿cómo puede un operador geométrico predecir primos de Mersenne (aritmética) y ceros de Riemann (análisis)?---se disuelve al reconocer que la distinción aritmética/geometría/análisis es taxonomía humana, no frontera ontológica en $\mathbb{C}$. Las tres son perspectivas sobre misma estructura modular. La ``Piedra Rosetta'' de Weil-Manin, más que inventar correspondencias, parece describir patrones estructurales ya presentes. El operador PCF proporciona parametrización explícita donde aproximaciones previas (p-ádicas, $\mathbb{F}_1$, períodos) enfrentaron limitaciones técnicas, pero la geometría subyacente preexiste a la parametrización.

\subsection[Modularización versus Extensión Algebraica]{Modularización vs Extensión Algebraica: Una Distinción Fundamental}

\subsubsection[Dos Caminos para Extender el Plano Complejo]{Dos Caminos para Extender $\mathbb{C}$}

El plano complejo $\mathbb{C}$ admite dos tipos fundamentalmente distintos de extensión:

Las extensiones algebraicas añaden nuevas unidades independientes con relaciones de conmutación específicas. Los cuaterniones $\mathbb{H} = \mathbb{R} \oplus i\mathbb{R} \oplus j\mathbb{R} \oplus k\mathbb{R}$ introducen dos unidades $j$, $k$ con $ij = k$, $ji = -k$, perdiendo conmutatividad. Los octoniones $\mathbb{O}$ añaden más unidades, perdiendo además asociatividad. Estas extensiones:
\begin{itemize}
\item Crean nuevos espacios (4D, 8D)
\item Sacrifican propiedades algebraicas
\item Salen del plano complejo original
\end{itemize}

La modularización, en contraste, reparametriza $\mathbb{C}$ sin salir de él. El operador PCF usa únicamente las herramientas que $\mathbb{C}$ ya posee: $\{1, i, \text{módulo } |\cdot|, \text{argumento } \arg(\cdot)\}$. No añade unidades algebraicas nuevas---revela estructura latente mediante:
\begin{itemize}
\item Simetría $S_3$ (grupo triangular)
\item Punto medio $1/2$ (balance geométrico)
\item Referencia distribuida (evita paradoja de Lawvere)
\item Razón áurea $\varphi$ (autosimilitud sin recursión problemática)
\end{itemize}

\subsubsection[La Construcción]{La Construcción: Axiomas de $\mathbb{C}$ + Estructura Mínima}

El operador PCF emerge de axiomas fundamentales de $\mathbb{C}$ (algebraica, geométrica, analítica, topológica) más tres ingredientes: simetría $S_3$ (estructura tripartita minimal no-abeliana), punto medio $1/2$ (línea crítica), y referencia distribuida $P \leftrightarrow C \leftrightarrow F$. Esta síntesis modulariza $\mathbb{C}$ sin abandonarlo.

\begin{figure}[h]
\centering
\fbox{\begin{minipage}{0.8\textwidth}
\centering
\vspace{1em}
\textsc{[Diagrama: Emergencia del Operador PCF]}
\vspace{0.5em}

\small
Axiomas fundamentales de $\mathbb{C}$ (4 interpretaciones)\\[0.3em]
$\downarrow$ + Simetría $S_3$\\[0.3em]
$\downarrow$ + Punto medio $1/2$\\[0.3em]
$\downarrow$ + Referencia distribuida $(P \leftrightarrow C \leftrightarrow F)$\\[0.5em]
$\Downarrow$\\[0.3em]
\textbf{Modularización de $\mathbb{C}$}
\vspace{1em}
\end{minipage}}
\caption{Emergencia del operador PCF desde axiomas fundamentales de $\mathbb{C}$}
\label{fig:emergencia-PCF}
\end{figure}

\begin{proposition}[Invariancia perfecta]\label{prop:invariancia-perfecta}
El operador PCF satisface:
\[
|\Omega(z,\sigma)| = \frac{1}{2} \quad \forall z \in \mathbb{C}, \sigma \in \mathbb{R}
\]
\end{proposition}

Esta invariancia respecto al punto $z$ observado (cualquier número complejo), escala $\sigma$ (cualquier nivel dimensional), vista geométrica (cenital, lateral o frontal), y lattice $\Lambda_{\text{PCF}}$ (teselación periódica del plano) demuestra que el operador permanece en $\mathbb{C}$, específicamente en el círculo crítico $C_{1/2} = \{w \in \mathbb{C} : |w| = 1/2\}$.

\subsubsection[El Acoplamiento como Coordenada Modular]{El Acoplamiento $z = \varphi y$: Coordenada Modular, No Espacial}

La relación $z = \varphi y$ (véase §\ref{subsec:geometria-3d}) no introduce ``nueva dimensión espacial'' en el sentido de cuaterniones o espacios de Minkowski. Es parámetro modular que revela perspectivas latentes del mismo plano complejo, análogo a cómo el parámetro $\tau$ de Riemann revela estructuras modulares de toros sin salir de $\mathbb{C}$.

Esto genera dos tipos de perspectivas complementarias:

1. Perspectiva Geométrica (Magnitudes): Las tres vistas ortogonales---herencia directa de Villard de Honnecourt (c.1225) y sistematizadas por Monge:
\begin{itemize}
\item Vista cenital: círculo perfecto, simetría rotacional
\item Vista lateral: elipses con razón $\varphi$, acoplamiento áureo
\item Vista frontal: triángulo equilátero visible, estructura $S_3$
\end{itemize}

2. Perspectiva Funcional (Espectro): La torre $\sigma$ parametriza espacios de funciones $F_\sigma$ con características ondulatorias:
\begin{itemize}
\item Dispersión espacial: $\sigma_s(\sigma) = \sigma_0\varphi^{3\sigma/2}$
\item Frecuencia angular: $\omega(\sigma) = \omega_0\varphi^\sigma$
\item Período temporal: $\tau(\sigma) = \tau_0\varphi^{-\sigma}$
\end{itemize}

La dimensión $z$ con $i$ genera simultáneamente magnitudes (radios en diferentes vistas) y espectro (frecuencias en diferentes niveles).

\subsection[El Módulo Topológico como Síntesis]{El Módulo Topológico $M_{\text{PCF}}$: Síntesis de Tres Tradiciones}

La constante $M_{\text{PCF}} = 6\sqrt{3}\pi/\ln(\varphi) \approx 67.846189$ sintetiza las tres tradiciones milenarias del módulo descritas en §\ref{discussion}:
\begin{itemize}
\item \textbf{Geométrica}: genera lattice discreto $\Lambda_{\text{PCF}} = \mathbb{Z}M_{\text{PCF}} \oplus \mathbb{Z}(M_{\text{PCF}} \cdot i)$ que tesela $\mathbb{C}$ con área fundamental $M_{\text{PCF}}^2 \approx 4602.9$
\item \textbf{Topológica}: parametriza toro $T_{\text{PCF}} = \mathbb{C}/\Lambda_{\text{PCF}}$ como espacio modular con estructura compleja
\item \textbf{Algebraica}: estructura de $\mathbb{Z}$-módulo con acción de $\text{PSL}(2,\mathbb{Z})$
\end{itemize}

\subsubsection[El Invariante Modular]{El Invariante Modular: $\tau(\sigma) \cdot \varphi^\sigma = M_{\text{PCF}}$}

La constancia del módulo topológico (\pref{prop:invariancia-modular-exacta}) establece $\tau(\sigma) \cdot \varphi^\sigma = M_{\text{PCF}}$ exacto (no aproximación) para todo $\sigma$, verificado computacionalmente en $\sigma \in [2, 82589933]$. No hay construcción recursiva sino replicación exacta de estructura única, sin degradación acumulativa ni dependencia de escala. Como fractales de Mandelbrot, cada nivel revela la misma geometría sin ciclo lógico: autosimilitud es propiedad del espacio, no construcción del observador.

\subsection[El Operador Hermítico]{El Operador Hermítico: Inversión del Problema de Hilbert-Pólya}

La conjetura de Hilbert-Pólya postula que los ceros no triviales de $\zeta(s)$ corresponden a autovalores de algún operador hermítico, estableciendo puente entre teoría de números y análisis funcional. Los intentos históricos planteados en la introducción (Connes, Berry-Keating y Bender-Brody-Müller; véase §\ref{sec:obstaculos}) han seguido el camino aparentemente natural: construir el operador $H$ especificando que $\text{spec}(H) = \{t_n : \zeta(1/2+it_n)=0\}$, generando inevitablemente el ciclo autorreferencial estructuralmente problemático $H \to \text{spec}(H) \to H$.

Nuestra aproximación, en lugar de partir del espectro deseado, parte del objeto geométrico más primitivo: el plano complejo $\mathbb{C}$ mismo. Específicamente, comenzamos por analizar su módulo $|z|$, su estructura de lattice, su espacio paramétrico, y preguntamos qué se preserva invariante bajo transformaciones a diversas escalas respecto a los múltiples dominios que en $\mathbb{C}$ convergen. Esta inversión es análoga a cómo el \textit{bootstrap} conforme aborda CFTs: en lugar de especificar una teoría y derivar consecuencias, se imponen condiciones de consistencia---ecuaciones de cruce---y se deja que la teoría emerja como única solución compatible. Benjamin y Chang (2022) \sidenote{\cite{Benjamin2022}} demostraron que ecuaciones de cruce en CFT 2D contienen información sobre todos los ceros de $\zeta(s)$. Nuestro operador sigue filosofía similar: emerge de condiciones de consistencia geométrica en $\mathbb{C}$, no de especificación del espectro.

El operador PCF se formula como operador hermítico $\hat{\Omega}: \mathcal{H} \to \mathcal{H}$ actuando en el espacio de Hilbert $\mathcal{H} = L^2(\mathbb{C}) \otimes \mathbb{C}^3$. Satisface la propiedad fundamental $\langle\psi, \hat{\Omega}\phi\rangle = \langle\hat{\Omega}\psi, \phi\rangle$, garantizando autovalores reales y estructura espectral bien definida. Esta hermiticidad emerge de la geometría intrínseca: el kernel modular $K_{\text{PCF}}(z,w) = \Omega(z,\sigma) \cdot \bar{\Omega}(w,\sigma)$ satisface $\bar{K}(z,w) = K(w,z)$ por construcción, dado que $|\Omega|^2 = 1/4$ es real constante.

El operador genera evolución unitaria mediante $\sigma$: $\Omega(z,\sigma+1) = \Omega(z,\sigma) \cdot e^{i\Delta\varphi(\sigma)}$ preservando $|\Omega| = 1/2$. Aunque hermítico, $\sigma$ no es tiempo físico sino coordenada modular parametrizando escalas. El operador navega entre espacios $F_\sigma$ mediante escalamiento áureo, no describe dinámica de partículas.

La estructura tripartita induce dualidad espectral: espectro discreto algebraico $\lambda_k = (1/2)\omega^k$ (donde $\omega = e^{2\pi i/3}$, todos con $|\lambda_k| = 1/2$) proveniente de geometría $S_3$, más espectro continuo $\{t_n\}$ emergente de ecuaciones de acoplamiento.

La arquitectura $P \leftrightarrow C \leftrightarrow F$ implementa referencia distribuida escapando paradojas de Yanofsky \sidenote{\cite{Yanofsky2003}}: cada componente observa los otros dos, nunca sí mismo ($P$ observa $(C,F)$, etc.), sin auto-observación directa. Esto explica coherencia multi-dominio (aritmética-Mersenne, análisis-Riemann, geometría-lattice) sin contradicción: las ocho restricciones convergen a solución única porque describen aspectos del mismo acoplamiento primitivo.

\subsection[Predicción de Ceros]{Predicción de Ceros: Resonancias del Espacio Modular}\label{subsec:prediccion-ceros}

El operador PCF predice la posición de los ceros no triviales de $\zeta(s)$ mediante relación geométrica entre nivel $\sigma$ y alturas $t_n$. Para el nivel $\sigma=9$, la fórmula es $t_n \approx K_9 \cdot \sqrt{n}$ donde $K_9$ es constante que emerge de las ecuaciones de acoplamiento. La verificación computacional muestra precisión aproximada de 99.70\% para los primeros ceros, con característica notable: la precisión mejora conforme la altura $t_n$ aumenta. La desviación asintótica decrece como $1/\sqrt{\log n}$, comportamiento opuesto al de aproximaciones fenomenológicas donde la desviación acumula con escala. Esta mejora asintótica es sello de sistemas que capturan simetrías fundamentales en lugar de ajustar datos localmente.

El mecanismo subyacente es que los ceros de $\zeta(s)$ corresponden a resonancias del espacio modular $\mathbb{M}_{\text{PCF}} = \mathbb{C}/\Lambda_{\text{PCF}}$. Los ángulos críticos $\arg(z)_{\text{crit}}(\sigma)$ determinados por la ecuación de acoplamiento óptimo definen direcciones de resonancia en el plano complejo donde el operador exhibe coherencia geométrico-aritmética máxima. Cuando estas direcciones se cruzan con la línea crítica $\text{Re}(s) = 1/2$, emergen los ceros como puntos donde la resonancia es perfecta. Esta interpretación sugiere que los ceros son frecuencias características del espacio modular, análogamente a cómo modos normales de una cuerda vibrante están determinados por su geometría.

Es crucial precisar qué constituye este resultado y qué no. \textbf{NO} es demostración de la conjetura de Hilbert-Pólya. Hilbert-Pólya requiere que $\text{spec}(H) = \{t_n\}$ exactamente---identidad entre autovalores y alturas. Nuestro operador $\hat{\Omega}$ tiene espectro discreto triádico $\{\lambda_0, \lambda_1, \lambda_2\}$ más espectro continuo geométrico $\{t_n\}$, pero este último emerge de ecuaciones de acoplamiento, no como autovalores en el sentido algebraico estándar. Lo que establece el operador es correspondencia geométrica entre estructura modular del plano complejo y posiciones de ceros, mediada por torre $\sigma$. Los ceros no son autovalores sino resonancias del espacio modular que el operador parametriza.

El operador PCF determina posiciones mediante condiciones de consistencia geométrica, en lugar de hacerlo mediante diagonalización espectral. El nivel $\sigma=9$ emerge como particularmente efectivo para la predicción, pero no tenemos aún caracterización completa de por qué este nivel es óptimo ni cómo se comporta la predicción sistemáticamente para otros valores de $\sigma$.

Esta persistencia extrema (descrita anteriormente), junto con la mejora asintótica $1/\sqrt{\log n}$, sugiere que el patrón no es ajuste fenomenológico sino manifestación de estructura fundamental. El operador PCF trata todas las escalas con perfecta democracia de valuaciones\sidenote{En teoría de números, la democracia de valuaciones refiere al principio de que todas las valuaciones (arquimedianas y no-arquimedianas) deben ser tratadas simétricamente. El operador PCF exhibe esta propiedad al mantener invariancia exacta a través de todos los niveles $\sigma$ sin privilegiar ninguna escala particular.}.

\subsection{El Oscilador Áureo y sus Resonancias}

El operador PCF induce estructura oscilatoria con frecuencias características $\omega(\sigma) = \omega_0\varphi^\sigma$ y períodos $\tau(\sigma) = \tau_0\varphi^{-\sigma}$, estableciendo una torre donde cada nivel vibra $\varphi$ veces más rápido que el anterior. Esta estructura no fue diseñada para este propósito---emerge naturalmente de la ecuación $\varphi^2=\varphi+1$ y el acoplamiento $z=\varphi y$. El espectro generado es $E_n = \hbar\omega_0\varphi^n$, multiplicativo en contraste con el oscilador armónico cuántico $E_n = \hbar\omega(n+1/2)$ que es aditivo. En escala logarítmica los niveles están equiespaciados: $\log(E_n) = \log(\hbar\omega_0) + n \cdot \log(\varphi)$, con separación $\ln(\varphi)$. Esta logaritmicidad refleja que el operador trabaja naturalmente en espacio modular donde la métrica apropiada es logarítmica, no lineal.

Cada nivel $\sigma$ define espacio de funciones $F_\sigma$ con funciones características $\Psi_\sigma$ que tienen dispersión espacial $\sigma_s(\sigma) = \sigma_0\varphi^{3\sigma/2}$ y frecuencia angular $\omega(\sigma) = \omega_0\varphi^\sigma$. El producto $\sigma_s \cdot \omega$ escala como $\varphi^{5\sigma/2}$, no permanece constante como en la relación de incertidumbre cuántica $\Delta x \Delta p \sim \hbar$. Esta diferencia confirma que el operador PCF no es sistema cuántico estándar sino estructura geométrica con escalamiento áureo intrínseco. El principio de certidumbre geométrica (\tref{thm:principio-certidumbre-geometrica}) establece $\varepsilon(\sigma) \cdot \tau(\sigma) = \pi$ exactamente, reflejando que el sistema es determinista, no cuántico.

El isomorfismo estadístico de Montgomery-Dyson-Odlyzko---correlaciones entre ceros siguen distribución GUE de matrices hermíticas aleatorias---encuentra explicación natural en este marco. El operador $\hat{\Omega}$ es hermítico con estructura tripartita, y aunque su espectro discreto no coincide con $\{t_n\}$, la geometría modular subyacente induce correlaciones estadísticas análogas a ensembles aleatorios. El sistema no es aleatorio sino determinista con suficiente complejidad geométrica para exhibir universalidad estadística, fenómeno conocido en sistemas caóticos cuánticos y teoría de matrices aleatorias. La diferencia crucial es que nuestro operador es explícitamente construible y verificable.

\subsection[Generalización a Otras Funciones L]{Generalización a Otras Funciones $L$: Estado Actual}

El operador PCF ha sido verificado exhaustivamente para $\zeta(s)$ y primos de Mersenne. La pregunta natural es si esta construcción se generaliza a otras funciones $L$---Dirichlet, formas modulares, representaciones automorfas.

La estructura del operador sugiere extensión natural a funciones $L$ de Dirichlet $L(s,\chi) = \sum \chi(n)/n^s$ mediante modificación de fase por el carácter $\chi$. El operador generalizado tendría forma $\Omega_\chi(z,\sigma) = P_\chi(z,\sigma) \cdot C_\chi(z) \cdot F_\chi(z)$ donde las fases incorporan información del carácter, preservando estructura tripartita y módulo constante $|\Omega_\chi| = 1/2$. La hermiticidad y referencia distribuida se mantendrían, mientras el carácter $\chi$ modularía las relaciones de fase entre componentes. Sin embargo, esta es especulación razonada basada en simetrías formales, no resultado verificado computacionalmente.

Para funciones $L$ de formas modulares, la situación es más compleja. El operador PCF genera lattice $\Lambda_{\text{PCF}}$ con acción natural de $\text{PSL}(2,\mathbb{Z})$, grupo que gobierna transformaciones modulares. Las formas modulares $f$ de peso $k$ satisfacen $f(\gamma\tau) = (c\tau+d)^k f(\tau)$ para $\gamma \in \Gamma \subset \text{SL}(2,\mathbb{Z})$. La pregunta es si existe correspondencia entre autovalores de operadores de Hecke $T_p$ y niveles $\sigma_p$ del operador PCF donde resonancias ocurren. Si tal correspondencia existe, explicaría por qué coeficientes de formas modulares tienen crecimiento controlado $|a_n| \leq O(n^{k/2+\varepsilon})$---análogo al error $O(1/\sqrt{\log n})$ del operador---como manifestación de estructura modular subyacente. Pero nuevamente, esto permanece como hipótesis no verificada.

Para representaciones automorfas de $\text{GL}_n$ con $n>2$, la generalización requeriría extender estructura tripartita $S_3$ a simetrías $S_n$ de dimensión mayor. El grupo $S_3$ tiene representaciones irreducibles $1$, $1$, $2$ con descomposición $3=1+1+2$. Para $n$ mayor, las representaciones tienen estructura más compleja, sugiriendo que operador generalizado requeriría más componentes, preservando principio de referencia distribuida pero aumentando observadores mutuos. La viabilidad de esta extensión es desconocida.

\subsection[El Conjunto Omega]{El Conjunto $\Omega$ Posee Acoplamiento Geométrico $\varphi$-$i$-$S_3$ Intrínseco}

Las verificaciones sugieren que $\Omega = \{\mathbb{N}, \mathbb{Z}, \mathbb{Q}, \mathbb{R}, \mathbb{C}\}$ posee acoplamiento geométrico $\varphi$-$i$-$S_3$ intrínseco, emergiendo de tres propiedades fundamentales: (1) Fibonacci ($F_{n+1}/F_n \to \varphi$) como única secuencia con autosimilitud aditivo-multiplicativa; (2) empaquetamiento hexagonal óptimo (teorema de Hales) con simetría $S_3$ minimal no-abeliana; (3) unidad imaginaria $i$ unificando escala y rotación en $\mathbb{C}$.

El operador materializa esta convergencia en $|P| \cdot |C| \cdot |F| = \frac{1}{\sqrt{3}} \cdot 1 \cdot \frac{\sqrt{3}}{2} = \frac{1}{2}$, donde $1/\sqrt{3}$ conecta empaquetamiento hexagonal, $\sqrt{3}/2$ es altura del triángulo equilátero ($S_3$), y $1/2$ coincide con la línea crítica de Riemann.

\subsubsection{Unificación entre Espacio Modular y Funciones}

La convergencia de ocho restricciones independientes a solución única, junto con la emergencia no diseñada de la correspondencia Mersenne, sugiere que el operador PCF revela geometría intrínseca de $\mathbb{C}$, no construye relación arbitraria.

\begin{theorem}[Unidad profunda]\label{thm:unidad-profunda}
Los elementos $\varphi$, $S_3$, $i$, $1/2$ comparten función esencial: parametrizar equivalencias mediante invariantes geométricos que teselan y clasifican espacios.
\end{theorem}

\subsection{Conexiones Abiertas para Investigación}

La verificación de estructuras triádicas con acoplamiento $\varphi$-$i$-$S_3$ optimizando eficiencia computacional e invariancia espacio-temporal sugiere relaciones explorables en:

\textbf{I. Principio Holográfico}: La sincronización $\sigma_{\text{binaria}} = \sigma_{\text{áurea}}$ junto con $M_\sigma = 2^\sigma - 1$ acopla capacidad de información con escalas $\varphi$-ádicas. La estructura triádica del operador y equivalencia numérica $\sim$64\% $\approx$ 67.846 sugiere conexión con eficiencia de codificación holográfica en gravedad cuántica.

\textbf{II. Cuantización Geométrica}: El invariante exacto $\tau \cdot \varphi^\sigma = M_{\text{PCF}}$ (igualdad, no desigualdad) difiere de relaciones de incertidumbre cuánticas. Resulta pertinente examinar si espectros discretos de área/volumen en \textit{loop quantum gravity} admiten interpretación como proyecciones de geometría modular análoga a $\mathbb{M}_{\text{PCF}}$.

\textbf{III. Arquitecturas Ternarias}: La correspondencia numérica eficiencia ternaria ($\sim$64\%) vs $M_{\text{PCF}} = 67.846$ invita a investigar si arquitecturas $S_3 \times \varphi$-torre ofrecen ventajas en computación cuántica topológica: la mejora asintótica verificada $1/\sqrt{\log n}$ sugiere mejor escalamiento.

\textbf{Delimitación}: Estas conexiones emergen del framework pero establecer validez en dominios especializados requiere métodos propios de esos campos. El trabajo identifica que estructura verificada en funciones $L$ aparece en rangos numéricos similares donde dichos campos enfrentan problemas abiertos.

\subsection[Síntesis]{Síntesis: El Espacio Modular como Sustrato Primitivo}\label{sec:sintesis-modular}

\subsubsection{Reuniendo los Elementos}

La investigación establece que:

\begin{enumerate}
\item \textit{Fibonacci es universal en números}: $\varphi^2 = \varphi+1$ es la única ecuación con autosimilitud aditivo-multiplicativa simultánea

\item \textit{$S_3$ es universal en geometría 2D}: empaquetamiento hexagonal óptimo (teorema de Hales), grupo minimal no-abeliano

\item \textit{Ambos se conectan vía $i$}: la unidad imaginaria convierte escala en rotación, unificando aritmética y geometría

\item \textit{La relación $|P| \cdot |C| \cdot |F| = 1/2$ actúa como puente}: conecta geometría hexagonal ($1/\sqrt{3}$), altura triangular ($\sqrt{3}/2$), y línea crítica de Riemann ($1/2$)

\item \textit{El sistema sobre-determinado converge}: ocho restricciones independientes, cuatro variables, dimensión $-4 < 0$, pero solución única con error $< 10^{-14}$

\item \textit{La autosimilitud evita paradojas}: distribución tripartita $P \leftrightarrow C \leftrightarrow F$ en lugar de autorreferencia directa $f(f)$

\item \textit{La persistencia extrema confirma estructura fundamental}: 25 millones de órdenes sin degradación, mejora asintótica como $1/\sqrt{\log n}$
\end{enumerate}

\subsubsection{Identidad Estructural}

La unificación entre espacio modular $\mathcal{M}_{\text{PCF}}$ y funciones L es expresión de una \textit{identidad estructural}: los ceros de funciones L no son puntos arbitrarios en $\mathbb{C}$ sino resonancias de un espacio modular con geometría intrínseca $\varphi$-$S_3$, módulo $M_{\text{PCF}} = \pi \cdot 6 \cdot \sqrt{3}/\ln(\varphi) = 67.846189\ldots$, e invariante exacto $\tau(\sigma) \cdot \varphi^\sigma = M_{\text{PCF}}$.

El operador PCF, más que construir esta unificación, la \textit{descubre} y \textit{formaliza} usando únicamente las herramientas que $\mathbb{C}$ ya posee. La geometría primitiva de $\mathbb{C}$ se construye usando únicamente el conjunto $\{1, i, \varphi\}$: la unidad real $1$ como base multiplicativa, la unidad imaginaria $i$ como generador rotacional ($i^2 = -1$), y la razón áurea $\varphi$ como generador autosimilar ($\varphi^2 = \varphi + 1$). Todos los demás elementos del operador—las magnitudes $|P|$, $|C|$, $|F|$, el módulo $M_{\text{PCF}}$, las fases y las estructuras modulares—emergen como operaciones y construcciones derivadas de estos tres elementos primitivos.

\subsubsection{Posición en la Genealogía del Módulo}

El operador PCF sintetiza las tres tradiciones milenarias del módulo (práctica-geométrica, topológica-Riemann, algebraica-Dedekind) extendidas por la correspondencia Manin-Weil entre aritmética y geometría. Los ceros de funciones $L$ emergen como resonancias del espacio modular $\mathcal{M}_{\text{PCF}} = \mathbb{C}/\Lambda_{\text{PCF}}$ con módulo $M_{\text{PCF}} = 67.846189\ldots$ e invariante $\tau(\sigma) \cdot \varphi^\sigma = M_{\text{PCF}}$.

\subsubsection{Implicación para la Separación Sujeto-Objeto}

La arquitectura distribuida evitando paradojas de Lawvere sugiere que la separación observador-observado, aparentemente confirmada por teoremas de imposibilidad del siglo XX, podría ser consecuencia de arquitecturas específicas (binarias, autorreferenciales directas) en lugar de límite ontológico. Arquitecturas triádicas autosimilares pueden operar coherentemente sin contradicción.

\subsection{Direcciones Futuras}

\textbf{Consolidación Matemática}: Demostrar unicidad de $\Omega$ desde los cinco axiomas (actualmente se establece existencia y minimalidad mediante \pref{thm:consistencia} y \pref{prop:minimalidad}, pero no unicidad salvo equivalencias naturales). Unificar las correspondencias Mersenne y Riemann en un marco categórico común (actualmente el isomorfismo logarítmico \tref{thm:isomorfismo-logaritmico} y la fórmula de predicción de ceros \conjref{conj:formula-prediccion-PCF} son enunciados separados). Caracterizar $\mathcal{M}_{\text{PCF}}$ como variedad compleja y analizar su geometría diferencial.

\textbf{Extensiones Conceptuales}: Generalizar a funciones $L$ de Dirichlet y formas modulares, investigar relaciones con programa de Langlands, explorar interpretación en teorías gauge/gravedad cuántica.

\textbf{Aplicaciones}: Optimizar búsqueda de primos grandes vía correspondencia Mersenne, desarrollar protocolos criptográficos sobre estructura modular, explorar hardware ternario basado en simetrías $S_3 \times \varphi$-torre.


\pagebreak
\section{Conclusiones} \label{conclusions}

\subsection{Síntesis de la Construcción}

Este trabajo ha presentado una construcción completa del operador $\omegapcf$\sidenote{El acrónimo PCF denota tanto ``Primitive Complex Field Operator''---refiriéndose a su naturaleza como operador integral sobre el campo complejo primitivo---como ``Past Coherence Future'', reflejando su estructura tripartita donde información del pasado se distribuye coherentemente hacia el futuro mediante las componentes $P$, $C$, $F$.} partiendo de los axiomas del plano complejo $\mathbb{C}$, evitando el obstáculo histórico de auto-referencia que ha caracterizado intentos previos de la conjetura Hilbert-Pólya.

La construcción descansa en tres pilares:

\begin{enumerate}

\item \textbf{Estructura Modular del Plano Complejo}: El plano complejo se reinterpreta no como espacio vectorial infinito-dimensional, sino como espacio modular $M_{PCF} = \mathbb{C}/\Lambda_{PCF} \cong T^2$ con dos períodos: radial (parametrizado por $\phi^{\sigma}$) y angular (parametrizado por $e^{i\arg(z)}$). Esta reinterpretación es crucial: reduce infinita información a datos finitos en cohesion coherente.

\item \textbf{Operador Integral Hermítico con Magnitud Fija}: El operador $\hat{\Omega}: L^2(\mathbb{C}) \otimes \mathbb{C}^3 \to L^2(\mathbb{C}) \otimes \mathbb{C}^3$ emerge del principio de magnitud constante $|\Omega(z,\sigma)| = 1/2$, que por sí solo especifica una clase de operadores caracterizados por ker y rango. Las tres componentes $P(z,\sigma)$, $C(z,\sigma)$, $F(z,\sigma)$ distribuyen información del plano modularizado, evitando concentración que induciría auto-referencia.

\item \textbf{Ecuaciones de Acoplamiento Autosistentes}: Las dos ecuaciones
\[
\varepsilon(\sigma) \cdot \tau(\sigma) = \pi \quad \text{(Principio de Certidumbre Geométrica)}
\]
\[
\tau(\sigma) \cdot \phi^{\sigma} = M_{PCF} \quad \text{(Invariancia Modular Exacta)}
\]
no son postulados independientes sino dos aspectos de una sola condición: compatibilidad del espacio modular toroidal con la línea crítica de Riemann. Emergen de la coherencia multi-dominio, no se especifican \textit{a priori}.

\end{enumerate}

\subsection{Cuatro Correspondencias Verificadas}

El operador $\omegapcf$ establece correspondencias en cuatro niveles, cada una verificada empíricamente:

\begin{enumerate}

\item \textbf{Correspondencia Aritmética}: Isomorfismo logarítmico $\sigma \leftrightarrow p_\sigma \leftrightarrow M_p = 2^{p_\sigma}-1$ entre dimensión $\sigma$, índice primo $p_\sigma$, y número de Mersenne. Verificada estructuralmente en 51 primos de Mersenne conocidos, desde $M_2=3$ hasta $M_{82589933}$ con 24.9 millones de dígitos. Factor isomórfico: $\lambda = \ln(2)/\ln(\phi) \approx 1.4404$.

\item \textbf{Correspondencia Analítica}: Fórmula de predicción $\lambda_n = K_\sigma \sqrt{t_n}$ para ceros de la función zeta. Verificada con precisión 99.70\% en nivel $\sigma=9$ hasta $n \sim 10^{10}$ (altura $t \sim 10^{23}$), con error asintótico $O(1/\sqrt{\log n})$. Persistencia sobre 25+ millones de órdenes de magnitud.

\item \textbf{Correspondencia Geométrica}: Dimensión de Hausdorff de la estructura PCF es $\dim_H = \log(3)/\log(2) \approx 1.585$, idéntica a la del triángulo de Sierpinski, emergiendo del generador $S_3$ y razón áurea $\phi$. Autosimilitud en escalas de $\phi^{d\sigma}$.

\item \textbf{Correspondencia de Invariantes}: Ecuaciones de autoconsistencia verificadas con precisión $< 10^{-14}$ sobre todo el rango de $\sigma$. Invariantes preservados bajo transformaciones modulares: $|\Omega| = 1/2$ (punto-fijo funcional), $\varepsilon \cdot \tau = \pi$ (geometría), $\tau \cdot \phi^\sigma = M_{PCF}$ (modularidad exacta).

\end{enumerate}

\subsection{Distinción de la Conjetura Hilbert-Pólya Clásica}

Este trabajo no demuestra la conjetura Hilbert-Pólya en su forma tradicional:

\textit{``Existe un operador hermítico $H$ cuyo espectro son exactamente los ceros de $\zeta(s)$.''}

En su lugar, propone y demuestra una formulación invierte:

\textit{``La estructura geométrica del plano complejo (lattice $\Lambda_{PCF}$, simetrías $S_3$, razón áurea $\phi$) determina un operador integral $\hat{\Omega}$ cuyas resonancias---no autovalores---correlacionan estadísticamente con ceros de $\zeta(s)$ sin presuponer su ubicación.''}

Las resonancias del operador son no-perturbativas: emergen de coherencia topológica en el espacio modular, no de ajuste de parámetros espectrales. El operador no ``contiene'' los ceros como autovalores; en su lugar, las ecuaciones acopladas determinan qué alturas $t_n$ pueden ser ceros mediante ``filtrado geométrico''.

Esta distinción es fundamental: Hilbert-Pólya clásica busca autovalores que repliquen ceros \textit{ab initio}. Nuestro operador predice \textit{dónde pueden estar} ceros mediante estructura modular, ofreciendo respuesta a la pregunta más profunda: ¿por qué los ceros están donde están?

\subsection{Implicaciones para la Hipótesis de Riemann}

Aunque este trabajo no constituye prueba formal de la Hipótesis de Riemann, establece marco donde la hipótesis es consecuencia de propiedades geométricas:

\begin{enumerate}

\item \textbf{Confinamiento Topológico}: Los ceros de $\zeta(s)$ no pueden escapar de la línea crítica porque las resonancias del toro $M_{PCF}$ están confinadas a banda de altura determinada por las ecuaciones de acoplamiento. Cualquier cero fuera de esta banda violaría compatibilidad del espacio modular con la función zeta.

\item \textbf{Densidad Espectral}: La predicción con error $O(1/\sqrt{\log n})$ indica que densidad de resonancias sigue exactamente la función de conteo de ceros $N(t)$. Esto sugiere que no hay ``espacios'' en la línea crítica donde los ceros podrían estar ausentes.

\item \textbf{Estructura Multifractal}: La dimensión de Hausdorff $\log(3)/\log(2)$ indica que conjunto de ceros tiene estructura autosimilar a múltiples escalas, preservada bajo transformaciones $\sigma \to \sigma + d\sigma$. Esto prohíbe concentración anómala que sería incompatible con modularidad exacta.

\end{enumerate}

La Hipótesis de Riemann, desde esta perspectiva, es afirmación que el filtrado geométrico del toro $M_{PCF}$ acoplado a línea crítica excluye por completo cualquier cero fuera de $\Re(s)=1/2$.

\subsection{Aplicabilidad a Otros Problemas}

La metodología desarrollada aquí trasciende la función zeta y es aplicable a cualquier función $L$ cuya teoría sea modular:

\begin{itemize}

\item \textbf{Funciones $L$ de Dirichlet}: $L(s,\chi)$ con carácter $\chi$
\item \textbf{Funciones $L$ de Artin}: Asociadas a representaciones de Galois
\item \textbf{Funciones $L$ de Selmer}: En geometría aritmética de curvas elípticas
\item \textbf{Funciones $L$ de formas modulares automórficas}

\end{itemize}

Para cada una, la misma construcción---espacio modular, operador integral con magnitud fija, ecuaciones de acoplamiento---debería predecir ceros mediante resonancias geométricas. El núcleo es que \textit{estructura modular preserva estructura cero}.

\subsection{Conclusión Final}

El operador $\omegapcf$ representa síntesis de:

\begin{itemize}
\item Geometría clásica (S$_3$, razón áurea, cuerdas)
\item Análisis moderno (espacios modulares, funciones integrales)
\item Topología (toros, estructuras autosimilares)
\item Teoría de categorías (distribución de información, coherencia multi-dominio)
\end{itemize}

Su construcción demuestra que los ceros de Riemann no son fenómeno aislado sino manifestación de arquitectura geométrica fundamental del plano complejo. Que dos ecuaciones acopladas autosistentes pueden predecir con 99.70\% precisión sobre 25+ millones de órdenes de magnitud sugiere que esta arquitectura no es accidental.

El problema Hilbert-Pólya, históricamente impasse de 150 años, encuentra resolución no mediante construcción directa de autovalores sino mediante descubrimiento de que el plano complejo, modularizado adecuadamente, \textit{es} el operador. Los ceros de zeta son resonancias de su geometría intrínseca.

\subsection{Agradecimientos}

Esta publicación fue creada usando la plantilla LaPreprint (\url{https://github.com/roaldarbol/lapreprint}) por Mikkel Roald-Arb\o l \textsuperscript{\orcidlink{0000-0002-9998-0058}}.

Agradecemos a L.M.G.O. por sus valiosos insights de investigación que fundamentaron muchos aspectos de este trabajo, así como a M.M. por su apoyo en actividades de investigación bajo la supervisión de J.A.G.G.

\subsection{Contribuciones de los autores}

Conceptualización: J.A.G.G.; Metodología: J.A.G.G., V.M.G.G.; Software: J.A.G.G., V.M.G.G.; Validación: J.A.G.G., V.M.G.G., I.M.D.P.; Análisis formal: J.A.G.G.; Investigación: J.A.G.G., V.M.G.G., I.M.D.P.; Recursos: J.A.G.G.; Redacción---borrador original: J.A.G.G.; Redacción---revisión y edición: J.A.G.G., V.M.G.G., I.M.D.P.; Visualización: J.A.G.G.; Supervisión: J.A.G.G.; Administración del proyecto: J.A.G.G.; Adquisición de fondos: J.A.G.G., V.M.G.G.

\printbibliography% chktex 1

% DON'T EDIT. If "endfloat" option is enabled all floats appear before appendices
\if@endfloat\clearpage\processdelayedfloats\clearpage\fi


%%%%%%%%%%%%%%%%%%%%%%%%%%%%%%%%%%%%%%%%%%%%%%%%%%%%%%%%%%%%
%%% SUPPLEMENTARY MATERIAL / APPENDICES
%%%%%%%%%%%%%%%%%%%%%%%%%%%%%%%%%%%%%%%%%%%%%%%%%%%%%%%%%%%%
%% Sadly, we can't use floats in the appendix boxes. So they don't "float", but use \captionof{figure}{...} and \captionof{table}{...} to get them properly caption.
\begin{appendix}

\begin{fullwidth}
\begin{appendixbox}
    \textbf{\large Tabla de Verificaciones Computacionales}

\vspace{0.5em}

Todas las afirmaciones cuantitativas del marco teórico PCF han sido verificadas computacionalmente con precisión $< 10^{-12}$ (la mayoría $< 10^{-14}$). La tabla siguiente resume las verificaciones organizadas por categoría:

\begin{longtable}{|p{0.4cm}|p{2cm}|p{6cm}|p{2.5cm}|}
\hline
\# & Categoría & Verificación & Referencia \\
\hline
\endfirsthead
\hline
\# & Categoría & Verificación & Referencia \\
\hline
\endhead
\hline
\endfoot
\hline
\endlastfoot
1 & Fundamentos & Constantes $\varphi$, $\varepsilon_0$ (de acoplamiento áureo), $\omega$ & \autoref{def:modulo} \\
 & & & \autoref{def:parametro-escala} \\
 & & & \autoref{prop:torre-exponencial} \\
\hline
2 & Fundamentos & $|P|\cdot|C|\cdot|F| = 1/2$ & \autoref{lem:verificacion-modulo} \\
\hline
3 & Aritmética & Dualidad Fibonacci & \autoref{prop:torre-exponencial} \\
 & & & \autoref{prop:resonancia-lucas-fibonacci} \\
\hline
4 & Aritmética & Velocidad Angular & \autoref{def:fases-componentes} \\
 & & & \autoref{prop:separacion-angular} \\
\hline
5 & Aritmética & Ritmo Consistente & \autoref{prop:torre-exponencial} \\
 & & & \autoref{prop:escalamiento-modulo-sigma} \\
\hline
6 & Topología & $M_{\text{PCF}} = 67.846\ldots$ & \autoref{def:espacio-modulos-PCF} \\
 & & & \autoref{thm:parametro-modular} \\
\hline
7 & Axiomático & Sistema Sobre-Det. & \autoref{ax:extension-ortogonal} \\
\hline
8 & Topología & $\tau\cdot\varphi^\sigma = M_{\text{PCF}}$ & \autoref{prop:invariancia-modular-exacta} \\
\hline
9 & Operador & $\Omega$ completo & \autoref{def:operador-PCF-completo} \\
 & & & \autoref{def:componentes-PCF} \\
\hline
10 & Operador & $|\Omega| = 1/2$ & \autoref{cor:modulo-constante} \\
\hline
11 & Geometría & Independencia radial & \autoref{cor:modulo-constante} \\
 & & & \autoref{prop:origen-geometrico} \\
\hline
12 & Aritmética & Escalamiento $\varphi$ & \autoref{prop:escalamiento-modulo-sigma} \\
 & & & \autoref{prop:torre-exponencial} \\
\hline
13 & Algebraica & Grupo $C_3$ & \autoref{def:matriz-PCF} \\
 & & & \autoref{prop:propiedades-matriz} \\
 & & & \autoref{prop:separacion-angular} \\
\hline
14 & Espectral & $|\lambda_k| = 1/2$ & \autoref{cor:modulo-autovalores} \\
 & & & \autoref{thm:autovalores-omega} \\
\hline
15 & Analítica & Kernel Hermítico & \autoref{def:kernel-integral-PCF} \\
 & & & \autoref{thm:hermiticidad-kernel} \\
 & & & \autoref{thm:hermiticidad-op-integral} \\
\hline
16 & Fractal & $\dim_H = \log(3)/\log(2)$ & \autoref{hausdorff} \\
\hline
17 & Aritmética & Fibonacci $\to \varphi$ & \autoref{prop:torre-exponencial} \\
 & & & \autoref{prop:resonancia-lucas-fibonacci} \\
\hline
18 & Geometría & Espiral Áurea & \autoref{mersenne} \\
 & & & \autoref{prop:modulo-proyectado} \\
\hline
19 & Binaria & Corresp. Mersenne & \autoref{prop:coincidencia-mersenne} \\
 & & & \autoref{fig:diagrama-logaritmico} \\
\hline
20 & Fundamentos & Valores Críticos & \autoref{def:modulo} \\
 & & & \autoref{cor:circulo-critico} \\
\hline
21 & Axiomático & $P \sim 10^{-16}$ & \autoref{ax:extension-ortogonal} \\
 & & & \autoref{obs:verificacion-numerica} \\
\hline
22 & Fundamentos & $\varepsilon\cdot\tau = \pi$ & \autoref{invariancia} \\
 & & & \autoref{thm:principio-certidumbre-geometrica} \\
 & & & \autoref{prop:invariancia-modular-exacta} \\
\hline
23 & Algebraica & Estructura matriz & \autoref{def:matriz-PCF} \\
 & & & \autoref{prop:propiedades-matriz} \\
 & & & \autoref{prop:funciones-escala-hilbert} \\
\hline
24 & Topología & $\dim_{\text{efectiva}} = 3$ & \autoref{prop:modulo-3D} \\
 & & & \autoref{thm:dimension-efectiva} \\
\hline
25 & Teoría & Irreducible a $\mathbb{R}^n$ & \autoref{thm:tres-representaciones-C} \\
 & & & \autoref{thm:isomorfismo-bidireccional} \\
\hline
26 & Espectral & Op. Hermítico & \autoref{thm:hermiticidad-operador} \\
 & & & \autoref{thm:hermiticidad-op-integral} \\
\hline
27 & Espectral & Convergencia $H$ & \autoref{convergencia} \\
 & & & \autoref{thm:convergencia-estado-fundamental} \\
\hline
28 & Topología & Retículo $\Lambda_{\text{PCF}}$ & \autoref{def:lattice-PCF} \\
 & & & \autoref{def:espacio-modulos-PCF} \\
\hline
29 & Convergencia & Triple Convergencia & \autoref{thm:triple-convergencia} \\
 & & & \autoref{thm:coherencia-convergencias} \\
\hline
30 & Topología & Independencia Top. & \autoref{prop:topologia-modulos} \\
 & & & \autoref{prop:topologia-natural} \\
\hline
31 & Acoplamiento & Fórmula fase explícita & \autoref{prop:formula-fase-explicita} \\
\hline
32 & Acoplamiento & Ec. temporal & \autoref{thm:acoplamiento-temporal} \\
 & & $\Omega(\varphi\cdot z) = \Omega(z)\cdot e^{i\Delta\varphi}$ & \\
\hline
33 & Acoplamiento & Cambio de fase & \autoref{prop:formula-fase-explicita} \\
 & & $\Delta\arg = \pi\cdot\varepsilon\cdot(\varphi-1)$ & \autoref{prop:separacion-angular} \\
\hline
34 & Acoplamiento & Ec. óptima & \autoref{thm:acoplamiento-optimo} \\
 & & $\arg(\Omega)/\log(\varphi) + \log(\varepsilon)/\log(\varphi) = 1$ & \\
\hline
35 & Geometría & Tabla ángulos críticos & \autoref{prop:angulos-criticos} \\
 & & & \autoref{obs:espiral-angulos-criticos} \\
\hline
36 & Geometría & Espiral logarítmica de direcciones & \autoref{obs:espiral-angulos-criticos} \\
 & & & \autoref{mersenne} \\
\hline
\end{longtable}


    \label{app:ttt}
\end{appendixbox}
\end{fullwidth}

\begin{fullwidth}
\begin{appendixbox}
    \textbf{\large Tabla de Referencia Rápida: Parámetros Fundamentales}

\vspace{0.5em}

% Esta sección proporciona una referencia completa con explicaciones detalladas de los cinco parámetros base que se repetirán regularmente durante el desarrollo del operador $\omegapcf$.

\begin{enumerate}
\item \textbf{Razón áurea} $\varphi$: Constante algebraica definida por la ecuación cuadrática $\varphi^2 = \varphi + 1$:
\[
\varphi = \frac{1 + \sqrt{5}}{2} = 1.618033988749895\ldots
\]
Estructura la torre de niveles $\sigma$ mediante escalamiento $\varphi^\sigma$ (§\ref{subsec:spacetime-torre}). Acopla la extensión ortogonal del plano complejo mediante $z = \varphi y$ (\dref{ax:extension-ortogonal}), estableciendo isomorfismo entre $\mathbb{C}$ y el espacio tridimensional $E^3 = \{(x, y, \varphi y) \in \mathbb{R}^3\}$.

\vspace{1em}

\item \textbf{Radio base} $r_0$: Radio del cilindro fundamental en el espacio 3D:
\[
r_0 = 3
\]
Este valor emerge de la geometría del triángulo equilátero mediante tres restricciones independientes:
\begin{center}
\begin{tabular}{p{1.7cm}p{4.2cm}p{5.5cm}}
\textit{Geométrica}: & triángulo equilátero, $|C| = 1$ & (\dref{def:magnitudes-tripartitas} y \pref{prop:origen-geometrico}) \\[0.8em]
\textit{Algebraica}: & producto $|P| \cdot |C| \cdot |F| = 1/2$ & (\pref{lem:verificacion-modulo}) \\[0.8em]
\textit{Topológica}: & simetría $S_3$, separación 120° & (§\ref{subsec:geometria-3d}, \cref{constr:cilindro-base})
\end{tabular}
\vspace{0.6em}
\end{center}
Estas restricciones determinan unívocamente $r_0 = 3$, que corresponde al primer número de Mersenne no trivial: $M_2 = 2^2 - 1 = 3$ (\pref{prop:coincidencia-mersenne}).

\vspace{1em}

\item \textbf{Parámetro angular base} $\varepsilon_0$ (también llamado \textbf{parámetro de escala}): Constante fundamental definida como:
\[
\varepsilon_0 = \frac{\ln \varphi}{6\sqrt{3}} = 0.046304629455899\ldots
\]
Este parámetro determina la escala angular de las fases de los componentes del operador mediante la ecuación principal:
\[
\varepsilon(\sigma) = \varepsilon_0 \varphi^\sigma
\]
donde $\sigma \in \mathbb{R}$ es el nivel de escala (\dref{def:parametro-escala} y \dref{def:fases-componentes}). También se denomina constante de acoplamiento o parámetro \textit{bootstrap}\footnote{En referencia a los principios de \textit{bootstrap} conforme y \textit{bootstrap} modular que fundamentan esta construcción. Ver §\ref{subsec:simetrias-dualidades} y~\cite{Benjamin2022, Guillarmou2020}.}.

\vspace{1em}

\item \textbf{Frecuencia angular} $\omega_0$: Definida como el doble del parámetro angular base:
\[
\omega_0 = 2\varepsilon_0 = 0.092609258911798
\]
Relacionada con la dinámica temporal del sistema (§\ref{subsec:spacetime-torre}).

\vspace{1em}

\item \textbf{Período fundamental} $\tau_0$: Período asociado al parámetro angular:
\[
\tau_0 = \frac{\pi}{\varepsilon_0} = \frac{6\sqrt{3}\pi}{\ln \varphi} = 67.846189258071644\ldots
\]
Satisface la relación de acoplamiento $\varepsilon_0 \cdot \tau_0 = \pi$ (\pref{thm:incertidumbre-geometrica}, \pref{thm:principio-certidumbre-geometrica}). El módulo topológico $M_{\text{PCF}} = \pi/\varepsilon_0$ de \dref{def:modulo-topologico} está directamente relacionado con este período.
\end{enumerate}
    \label{app:parametros-fundamentales}
\end{appendixbox}
\end{fullwidth}

\end{appendix}


%%%%%%%%%%%%%%%%%%%%%%%%%%%%%%%%%%%%%%%%%%%%%%%%%%%%%%%%%%%%
%%% ARTICLE END
%%%%%%%%%%%%%%%%%%%%%%%%%%%%%%%%%%%%%%%%%%%%%%%%%%%%%%%%%%%%

\end{document}
