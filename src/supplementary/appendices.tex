\textbf{\large Tabla de Verificaciones Computacionales}

\vspace{0.5em}

Todas las afirmaciones cuantitativas del marco teórico PCF han sido verificadas computacionalmente con precisión $< 10^{-12}$ (la mayoría $< 10^{-14}$). La tabla siguiente resume las verificaciones organizadas por categoría:

\begin{longtable}{|p{0.4cm}|p{2cm}|p{6cm}|p{2.5cm}|}
\hline
\# & Categoría & Verificación & Referencia \\
\hline
\endfirsthead
\hline
\# & Categoría & Verificación & Referencia \\
\hline
\endhead
\hline
\endfoot
\hline
\endlastfoot
1 & Fundamentos & Constantes $\varphi$, $\varepsilon_0$ (de acoplamiento áureo), $\omega$ & \autoref{def:modulo} \\
 & & & \autoref{def:parametro-escala} \\
 & & & \autoref{prop:torre-exponencial} \\
\hline
2 & Fundamentos & $|P|\cdot|C|\cdot|F| = 1/2$ & \autoref{lem:verificacion-modulo} \\
\hline
3 & Aritmética & Dualidad Fibonacci & \autoref{prop:torre-exponencial} \\
 & & & \autoref{prop:resonancia-lucas-fibonacci} \\
\hline
4 & Aritmética & Velocidad Angular & \autoref{def:fases-componentes} \\
 & & & \autoref{prop:separacion-angular} \\
\hline
5 & Aritmética & Ritmo Consistente & \autoref{prop:torre-exponencial} \\
 & & & \autoref{prop:escalamiento-modulo-sigma} \\
\hline
6 & Topología & $M_{\text{PCF}} = 67.846\ldots$ & \autoref{def:espacio-modulos-PCF} \\
 & & & \autoref{thm:parametro-modular} \\
\hline
7 & Axiomático & Sistema Sobre-Det. & \autoref{ax:extension-ortogonal} \\
\hline
8 & Topología & $\tau\cdot\varphi^\sigma = M_{\text{PCF}}$ & \autoref{prop:invariancia-modular-exacta} \\
\hline
9 & Operador & $\Omega$ completo & \autoref{def:operador-PCF-completo} \\
 & & & \autoref{def:componentes-PCF} \\
\hline
10 & Operador & $|\Omega| = 1/2$ & \autoref{cor:modulo-constante} \\
\hline
11 & Geometría & Independencia radial & \autoref{cor:modulo-constante} \\
 & & & \autoref{prop:origen-geometrico} \\
\hline
12 & Aritmética & Escalamiento $\varphi$ & \autoref{prop:escalamiento-modulo-sigma} \\
 & & & \autoref{prop:torre-exponencial} \\
\hline
13 & Algebraica & Grupo $C_3$ & \autoref{def:matriz-PCF} \\
 & & & \autoref{prop:propiedades-matriz} \\
 & & & \autoref{prop:separacion-angular} \\
\hline
14 & Espectral & $|\lambda_k| = 1/2$ & \autoref{cor:modulo-autovalores} \\
 & & & \autoref{thm:autovalores-omega} \\
\hline
15 & Analítica & Kernel Hermítico & \autoref{def:kernel-integral-PCF} \\
 & & & \autoref{thm:hermiticidad-kernel} \\
 & & & \autoref{thm:hermiticidad-op-integral} \\
\hline
16 & Fractal & $\dim_H = \log(3)/\log(2)$ & \autoref{hausdorff} \\
\hline
17 & Aritmética & Fibonacci $\to \varphi$ & \autoref{prop:torre-exponencial} \\
 & & & \autoref{prop:resonancia-lucas-fibonacci} \\
\hline
18 & Geometría & Espiral Áurea & \autoref{mersenne} \\
 & & & \autoref{prop:modulo-proyectado} \\
\hline
19 & Binaria & Corresp. Mersenne & \autoref{prop:coincidencia-mersenne} \\
 & & & \autoref{fig:diagrama-logaritmico} \\
\hline
20 & Fundamentos & Valores Críticos & \autoref{def:modulo} \\
 & & & \autoref{cor:circulo-critico} \\
\hline
21 & Axiomático & $P \sim 10^{-16}$ & \autoref{ax:extension-ortogonal} \\
 & & & \autoref{obs:verificacion-numerica} \\
\hline
22 & Fundamentos & $\varepsilon\cdot\tau = \pi$ & \autoref{invariancia} \\
 & & & \autoref{thm:principio-certidumbre-geometrica} \\
 & & & \autoref{prop:invariancia-modular-exacta} \\
\hline
23 & Algebraica & Estructura matriz & \autoref{def:matriz-PCF} \\
 & & & \autoref{prop:propiedades-matriz} \\
 & & & \autoref{prop:funciones-escala-hilbert} \\
\hline
24 & Topología & $\dim_{\text{efectiva}} = 3$ & \autoref{prop:modulo-3D} \\
 & & & \autoref{thm:dimension-efectiva} \\
\hline
25 & Teoría & Irreducible a $\mathbb{R}^n$ & \autoref{thm:tres-representaciones-C} \\
 & & & \autoref{thm:isomorfismo-bidireccional} \\
\hline
26 & Espectral & Op. Hermítico & \autoref{thm:hermiticidad-operador} \\
 & & & \autoref{thm:hermiticidad-op-integral} \\
\hline
27 & Espectral & Convergencia $H$ & \autoref{convergencia} \\
 & & & \autoref{thm:convergencia-estado-fundamental} \\
\hline
28 & Topología & Retículo $\Lambda_{\text{PCF}}$ & \autoref{def:lattice-PCF} \\
 & & & \autoref{def:espacio-modulos-PCF} \\
\hline
29 & Convergencia & Triple Convergencia & \autoref{thm:triple-convergencia} \\
 & & & \autoref{thm:coherencia-convergencias} \\
\hline
30 & Topología & Independencia Top. & \autoref{prop:topologia-modulos} \\
 & & & \autoref{prop:topologia-natural} \\
\hline
31 & Acoplamiento & Fórmula fase explícita & \autoref{prop:formula-fase-explicita} \\
\hline
32 & Acoplamiento & Ec. temporal & \autoref{thm:acoplamiento-temporal} \\
 & & $\Omega(\varphi\cdot z) = \Omega(z)\cdot e^{i\Delta\varphi}$ & \\
\hline
33 & Acoplamiento & Cambio de fase & \autoref{prop:formula-fase-explicita} \\
 & & $\Delta\arg = \pi\cdot\varepsilon\cdot(\varphi-1)$ & \autoref{prop:separacion-angular} \\
\hline
34 & Acoplamiento & Ec. óptima & \autoref{thm:acoplamiento-optimo} \\
 & & $\arg(\Omega)/\log(\varphi) + \log(\varepsilon)/\log(\varphi) = 1$ & \\
\hline
35 & Geometría & Tabla ángulos críticos & \autoref{prop:angulos-criticos} \\
 & & & \autoref{obs:espiral-angulos-criticos} \\
\hline
36 & Geometría & Espiral logarítmica de direcciones & \autoref{obs:espiral-angulos-criticos} \\
 & & & \autoref{mersenne} \\
\hline
\end{longtable}

