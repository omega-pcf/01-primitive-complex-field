\textbf{\large Tabla de Referencia Rápida: Parámetros Fundamentales}

\vspace{0.5em}

% Esta sección proporciona una referencia completa con explicaciones detalladas de los cinco parámetros base que se repetirán regularmente durante el desarrollo del operador $\omegapcf$.

\begin{enumerate}
\item \textbf{Razón áurea} $\varphi$: Constante algebraica definida por la ecuación cuadrática $\varphi^2 = \varphi + 1$:
\[
\varphi = \frac{1 + \sqrt{5}}{2} = 1.618033988749895\ldots
\]
Estructura la torre de niveles $\sigma$ mediante escalamiento $\varphi^\sigma$ (§\ref{subsec:spacetime-torre}). Acopla la extensión ortogonal del plano complejo mediante $z = \varphi y$ (\dref{ax:extension-ortogonal}), estableciendo isomorfismo entre $\mathbb{C}$ y el espacio tridimensional $E^3 = \{(x, y, \varphi y) \in \mathbb{R}^3\}$.

\vspace{1em}

\item \textbf{Radio base} $r_0$: Radio del cilindro fundamental en el espacio 3D:
\[
r_0 = 3
\]
Este valor emerge de la geometría del triángulo equilátero mediante tres restricciones independientes:
\begin{center}
\begin{tabular}{p{1.7cm}p{4.2cm}p{5.5cm}}
\textit{Geométrica}: & triángulo equilátero, $|C| = 1$ & (\dref{def:magnitudes-tripartitas} y \pref{prop:origen-geometrico}) \\[0.8em]
\textit{Algebraica}: & producto $|P| \cdot |C| \cdot |F| = 1/2$ & (\pref{lem:verificacion-modulo}) \\[0.8em]
\textit{Topológica}: & simetría $S_3$, separación 120° & (§\ref{subsec:geometria-3d}, \cref{constr:cilindro-base})
\end{tabular}
\vspace{0.6em}
\end{center}
Estas restricciones determinan unívocamente $r_0 = 3$, que corresponde al primer número de Mersenne no trivial: $M_2 = 2^2 - 1 = 3$ (\pref{prop:coincidencia-mersenne}).

\vspace{1em}

\item \textbf{Parámetro angular base} $\varepsilon_0$ (también llamado \textbf{parámetro de escala}): Constante fundamental definida como:
\[
\varepsilon_0 = \frac{\ln \varphi}{6\sqrt{3}} = 0.046304629455899\ldots
\]
Este parámetro determina la escala angular de las fases de los componentes del operador mediante la ecuación principal:
\[
\varepsilon(\sigma) = \varepsilon_0 \varphi^\sigma
\]
donde $\sigma \in \mathbb{R}$ es el nivel de escala (\dref{def:parametro-escala} y \dref{def:fases-componentes}). También se denomina constante de acoplamiento o parámetro \textit{bootstrap}\footnote{En referencia a los principios de \textit{bootstrap} conforme y \textit{bootstrap} modular que fundamentan esta construcción. Ver §\ref{subsec:simetrias-dualidades} y~\cite{Benjamin2022, Guillarmou2020}.}.

\vspace{1em}

\item \textbf{Frecuencia angular} $\omega_0$: Definida como el doble del parámetro angular base:
\[
\omega_0 = 2\varepsilon_0 = 0.092609258911798
\]
Relacionada con la dinámica temporal del sistema (§\ref{subsec:spacetime-torre}).

\vspace{1em}

\item \textbf{Período fundamental} $\tau_0$: Período asociado al parámetro angular:
\[
\tau_0 = \frac{\pi}{\varepsilon_0} = \frac{6\sqrt{3}\pi}{\ln \varphi} = 67.846189258071644\ldots
\]
Satisface la relación de acoplamiento $\varepsilon_0 \cdot \tau_0 = \pi$ (\pref{thm:incertidumbre-geometrica}, \pref{thm:principio-certidumbre-geometrica}). El módulo topológico $M_{\text{PCF}} = \pi/\varepsilon_0$ de \dref{def:modulo-topologico} está directamente relacionado con este período.
\end{enumerate}