\section{Introducción}

\subsection{La Conjetura de Hilbert-Pólya y el Isomorfismo de Montgomery-Dyson-Odlyzko}

Las conjeturas atribuidas a Hilbert y Pólya\sidenote{\cite{Polya1926}} postularon a principios del siglo XX que la Hipótesis de Riemann podría abordarse mediante traducción entre dominios, utilizando los autovalores de un operador hermítico para atacar un problema de teoría de números. Esta conjetura permaneció como especulación teórica hasta que Montgomery identificó que las correlaciones de pares entre ceros consecutivos seguían una distribución específica\sidenote{\cite{Montgomery1973}}. Dyson reconoció esta función como idéntica a la del Gaussian Unitary Ensemble (GUE) de matrices hermíticas aleatorias\sidenote{\cite{Dyson1962}}, estableciendo una conexión inesperada entre teoría analítica de números y física estadística. Odlyzko verificó computacionalmente esta correspondencia mediante el cálculo de más de $10^{13}$ ceros con precisión sin precedentes\sidenote{\cite{Odlyzko1987}}, mientras que trabajos posteriores extendieron estas verificaciones y exploraron sus implicaciones teóricas\sidenote{\cite{Odlyzko1989}}.

El isomorfismo estadístico establece que las funciones de correlación de pares de ceros sucesivos $\{t_n\}$ en la línea crítica $\text{Re}(s) = 1/2$ satisfacen:
\[
R_2(s) = 1 - \left[\frac{\sin({\pi s})}{{\pi s}}\right]^2 % chktex 3
\]
idéntica a la distribución de eigenvalores en GUE\@. Sin embargo, este resultado, aunque proporciona información estadística sobre el conjunto de ceros, no establece correspondencias determinísticas para ceros individuales.

\subsection{Obstáculos Históricos y Limitaciones Estructurales}\label{sec:obstaculos}

Pese a la validación empírica del isomorfismo Montgomery-Dyson-Odlyzko, consideramos que la construcción explícita del operador Hilbert-Pólya ha enfrentado dos obstáculos fundamentales que han persistido a través de décadas de intentos\@.

\textbf{Obstáculo I: Autorreferencia}% chktex 13

Diversos intentos exhiben una estructura circular característica. Se parte del conocimiento del espectro deseado:
\[
\text{spec}(H) = \{t_n : \zeta(1/2+it_n)=0\}
\]
se utiliza esta información para construir el operador $H$, luego se diagonaliza $H$ para obtener sus eigenvalores, y finalmente se verifica que estos eigenvalores coinciden con $\text{spec}(H)$ original. Este ciclo presupone conocimiento \textit{a priori} de aquello que pretende descubrir.

La geometría no conmutativa de Connes\sidenote{\cite{Connes2000}} requiere información \textit{a priori} sobre los ceros para definir el espacio de fases donde el operador actuaría. El operador $H = xp$ de Berry-Keating\sidenote{\cite{BerryKeating1999}} necesita regularización espectro-dependiente. Las simetrías PT de Bender-Brody-Müller\sidenote{\cite{BenderBrodyMuller2002}} requieren ajustar parámetros mediante conocimiento previo del espectro buscado. Los trabajos recientes de Yakaboylu\sidenote{\cite{Yakaboylu2022,Yakaboylu2024}} continúan limitados por condiciones de confinamiento o de frontera que presuponen información sobre los ceros\@.

\textbf{Obstáculo II: Degradación Asintótica}% chktex 13

Aun cuando se evite la autorreferencia directa, otros operadores propuestos exhiben limitaciones predictivas sistemáticas. Los métodos basados en aproximaciones locales predicen con precisión decreciente conforme la altura $t$ aumenta, teniendo como efecto que lo que funciona para los primeros ceros, falla para $n > 10^6$. Las periodicidades artificiales emergen en construcciones que no capturan la estructura cuasi-periódica genuina de $\zeta(s)$. Esta degradación asintótica sugiere que las aproximaciones no acceden a la geometría fundamental subyacente.

Entre las aproximaciones más significativas que implican períodos se encuentran las de Vinogradov\sidenote{\cite{Vinogradov1958}} en 1958 que, paralela y simultáneamente que Korobov\sidenote{\cite{Korobov1958}} (bajo ideas previas de Vinogradov), estableció regiones libres de ceros que restringen dónde pueden estar los ceros no triviales:
\[
\zeta(\sigma + it) \neq 0 \quad \text{para} \quad \sigma \geq 1 - \frac{C}{(\log{|t|})^{2/3}(\log\log{|t|})^{1/3}} % chktex 3
\]

Esta restricción implica que cualquier operador propuesto debe predecir ceros sólo en la región permitida. Los métodos que predicen con precisión decreciente para $t$, violan implícitamente estas cotas\@.

\subsection{Traducción entre Dominios: Perspectiva Histórica}% chktex 13

Pese a los obstáculos del programa Hilbert-Pólya, la traducción entre dominios ha persistido como aproximación fértil. Manin\sidenote{\cite{Manin2013}}, en su artículo \textit{Numbers as Functions}, compila las principales líneas convergentes, revelando una arquitectura común: estructuras locales (p-ádico, $\mathbb{F}_1$, ciclos) que ascienden a propiedades globales mediante invariantes preservadas:

Buium\sidenote{\cite{Buium1995}} construyó ecuaciones diferenciales p-ádicas mediante el cociente de Fermat $\delta_p(a) = (a^p - a)/p$, extendiendo la analogía clásica entre números y funciones a espacios jet aritméticos. Los representantes de Teichmüller (raíces $p$-ésimas de unidad) juegan el rol de constantes que, en ausencia de uniformización, complican la teoría más allá del caso clásico.

Borger\sidenote{\cite{Borger2009}} estableció que las lambda-estructuras---codificando sistemas coherentes de levantamientos de Frobenius---son datos de descenso sobre el campo con un elemento $\mathbb{F}_1$. Tanto la construcción p-ádica de Buium como el enfoque de Borger enfrentan un obstáculo estructural común: la asimetría del primo arquimediano impide traducción completa entre la geometría finita y la infinita.

Kontsevich-Zagier\sidenote{\cite{Kontsevich2001}} definieron el anillo de períodos $P \subset \mathbb{C}$ como valores de integrales $\int_\gamma \omega$ con datos algebraicos sobre $\mathbb{Q}$, estableciendo cómo estructuras topológicas (ciclos geométricos) se traducen a objetos analíticos (integrales) que a su vez satisfacen ecuaciones algebraicas (ecuaciones de Picard-Fuchs). Esta traducción tripartita permite que propiedades geométricas se expresen analíticamente y luego se codifiquen algebraicamente. Aunque incluye $\pi$, $\log(2)$ y valores zeta múltiples, permanece abierto---contra intuición inicial---si $1/\pi$, $e$ o la constante de Euler $\gamma$ son períodos, ni siquiera períodos exponenciales.

Complementando la compilación de Manin, la correspondencia estadística establecida por Montgomery y Dyson---espaciamientos entre ceros de $\zeta(s)$ siguen distribución GUE de matrices aleatorias---y verificada computacionalmente por Odlyzko hasta $10^{13}$ ceros (véase §\ref{sec:obstaculos}), establece una traducción entre teoría de números y física cuántica mediante operadores hermíticos. El obstáculo central permanecía irresoluto: ningún operador hermítico explícito había sido construido, dejando la conjetura de Hilbert-Pólya como principio heurístico más que construcción matemática.

\subsection{Simetrías y Dualidades como Diccionarios Universales}\label{subsec:simetrias-dualidades}

El análisis de estas aproximaciones revela, más allá de un tipo de matemática específica, simetrías y dualidades como fundamento común. Este principio ha demostrado éxito en múltiples contextos matemáticos y físicos, más allá de la Hipótesis de Riemann.

Múltiples construcciones matemáticas y físicas ejemplifican este principio. La transformada de Fourier establece correspondencia entre espacio de posición y espacio de momento, preservando la norma $L^2(\mathbb{R})$: $\|f\|_2 = \|\hat{f}\|_2$. La dualidad AdS/CFT\sidenote{\cite{Maldacena1998}} constituye una equivalencia completa entre teoría de gravedad en $d+1$ dimensiones y teoría de campos conforme en $d$ dimensiones, preservando funciones de partición $Z_{\text{CFT}}[J] = Z_{\text{gravity}}[\varphi_0=J]$. Esta dualidad fuerte-débil permite traducir problemas intratables en un dominio a problemas tratables en el otro.

En el contexto del \textit{bootstrap} modular escalar, Benjamin y Chang\sidenote{\cite{Benjamin2022}} demostraron que ecuaciones de cruce en CFT 2D contienen información sobre todos los ceros de $\zeta(s)$, reformulando la Hipótesis de Riemann como afirmación sobre densidad de operadores. Desde una perspectiva unificadora, Baez y Stay\sidenote{\cite{Baez2009}} mostraron que física cuántica, topología, lógica y computación comparten estructura de categorías monoidales simétricas cerradas, permitiendo traducción entre dominios mediante funtores naturales.

\textbf{Evitando Auto-Referencia}

Yanofsky\sidenote{\cite{Yanofsky2003}}, siguiendo a Lawvere\sidenote{\cite{Lawvere1969}}, formalizó que paradojas auto-referenciales emergen de argumentos diagonales donde sistemas intentan describir sus propias propiedades. Las traducciones exitosas evitan esto mediante distribución de información entre múltiples dominios con invariantes preservados. La estructura circular $D_1 \to D_2 \to D_1$ genera auto-referencia, mientras que coherencia multi-dominio $D_1 \leftrightarrow D_2 \leftrightarrow D_3 \leftrightarrow D_4$ establece determinación mutua sin ciclos directos. En síntesis, §\ref{subsec:simetrias-dualidades} establece que traducciones exitosas entre dominios evitan autorreferencia mediante distribución de información entre múltiples dominios con invariantes preservados.

\subsection{El Dilema Fundamental: La Consecuencia Algebraica de la Dimensionalidad}

El punto de partida del programa espectral para la Hipótesis de Riemann reside en una tensión irresoluble entre la necesidad lógica y la restricción algebraica.

La necesidad de evitar la circularidad funcional (la Paradoja de la Autorreferencia) impone un requisito de dimensionalidad mínima. La teoría de puntos fijos para sistemas recursivos establece que para romper la tautología (por ejemplo, ``Esta oración es falsa'') se requiere un marco con una estructura tripartita mínima (P-C-F) que no puede resolverse en un espacio binario o bidimensional sin auto-contradicción\sidenote{\cite{Yanofsky2003}}. Esto implica que la construcción del operador debe operar en un espacio con una dimensionalidad efectiva de tres o más.

Sin embargo, al intentar construir esta dimensionalidad mediante extensiones algebraicas del plano complejo ($\mathbb{C}$), se entra en conflicto con el Teorema de Frobenius\sidenote{\cite{Frobenius1877}}. Este teorema establece que las únicas álgebras de división reales, asociativas y de dimensión finita son los números reales ($\mathbb{R}$), los complejos ($\mathbb{C}$) y los cuaterniones ($\mathbb{H}$):

\begin{itemize}
\item La extensión a $\mathbb{H}$ (dim 4) sacrifica la conmutatividad ($ij \neq ji$).
\item La extensión a los Octoniones ($\mathbb{O}$, dim 8, conforme al teorema de Hurwitz-Zorn\sidenote{\cite{Hurwitz1923,Zorn1933}}) sacrifica adicionalmente la asociatividad ($(a \cdot b) \cdot c \neq a \cdot (b \cdot c)$).
\end{itemize}

La teoría espectral clásica requiere tanto la conmutatividad como la asociatividad para garantizar la diagonalización de los operadores y, por ende, que el espectro (los autovalores) sea puramente real y discreto. La extensión de la teoría de operadores a álgebras no conmutativas o no asociativas (el contexto de las Álgebras de Von Neumann o $C^*$-álgebras\sidenote{\cite{Dixmier1981,Arveson2002}}) es sustancialmente más compleja y no proporciona el resultado de espectro real que requiere la solución espectral a la Hipótesis de Riemann. Por lo tanto, cualquier extensión algebraica para ganar dimensionalidad implica un sacrificio de las propiedades esenciales para la coherencia espectral.

\subsubsection{El Hamiltoniano como Mecanismo de Conversión \texorpdfstring{$\mathbb{C} \to \mathbb{R}$}{C a R}}\label{subsubsec:hamiltoniano-conversion-C-R}

El Hamiltoniano es el Generador Infinitesimal de la Evolución Temporal\sidenote{\cite{Shankar1994}}. Formalmente, $H$ es el elemento del Álgebra de Lie que genera el grupo unitario de la evolución $U(t)$:
\[
U(t) = e^{-iHt/\hbar}
\]

La acción de $H$ sobre el vector de estado $\psi \in \mathcal{H}$ (el espacio de Hilbert, que es un espacio vectorial complejo $\mathbb{C}^n$) induce una rotación de fase constante en el plano complejo, donde la energía ($E$) se interpreta como la frecuencia o velocidad angular de dicha rotación. La intención de $H$ es, por lo tanto, unificar la Geometría (el flujo del sistema), el Álgebra (la estructura de conmutación de los operadores) y la Aritmética (el espectro discreto).

El propósito de $H$ es garantizar que las cantidades físicas medibles sean reales. La conversión de la naturaleza compleja del espacio de Hilbert ($\mathbb{C}$) a un espectro puramente real ($\mathbb{R}$) se logra mediante una imposición algebraica: el operador $H$ debe ser Hermítico (o auto-adjunto), $H = H^\dagger$. Esta propiedad, postulada como axioma en la mecánica cuántica\sidenote{\cite{VonNeumann1932}}, garantiza que la solución de la ecuación de autovalores $H\psi = E\psi$ produzca autovalores $E$ necesariamente reales ($E \in \mathbb{R}$). Esta coherencia algebraica solo está garantizada dentro de $\mathbb{C}$; cualquier extensión dimensional que viole Frobenius destruye precisamente el mecanismo que convierte espectro complejo en espectro real.

\subsection{Solución Propuesta: Modularización Geométrica}

En lugar de extender $\mathbb{C}$ mediante álgebra, lo extendemos mediante modularización geométrica y simetría.

\subsubsection{Extensión Algebraica}

Una extensión algebraica de un cuerpo $K$ construye un cuerpo mayor $L \supset K$ añadiendo raíces de polinomios. El ejemplo canónico es $\mathbb{C} = \mathbb{R}[x]/(x^2+1)$: se añade un elemento nuevo $i$ que satisface $i^2 = -1$. La extensión $\mathbb{R} \to \mathbb{C}$ añade elementos; la extensión $\mathbb{C} \to \mathbb{H}$ añade $j, k$ con relaciones $j^2 = k^2 = -1$, $ij = k = -ji$. Cada extensión algebraica introduce nuevos generadores con nuevas relaciones algebraicas, y son estas relaciones las que destruyen conmutatividad o asociatividad.

\subsubsection{Modularización Geométrica}

Una modularización geométrica no añade elementos. Dado un lattice $\Lambda \subset \mathbb{C}$, el espacio modular es el cociente:
\[
M_\Lambda = \mathbb{C}/\Lambda = \{[z] : z \sim z + \lambda, \; \lambda \in \Lambda\}
\]

La modularización identifica puntos equivalentes bajo traslación por el lattice. Topológicamente, $\mathbb{C}/\Lambda \cong T^2$ (toro). Algebraicamente, las operaciones de $\mathbb{C}$ se heredan a funciones $\Lambda$-periódicas: si $f(z) = f(z + \lambda)$ para todo $\lambda \in \Lambda$, entonces $f$ vive naturalmente sobre el toro. La modularización reorganiza sin añadir; preserva todas las propiedades algebraicas de $\mathbb{C}$ en el espacio de funciones periódicas.

\subsubsection{Acoplamiento por Simetría: Estructura Tripartita y Dualidad Lattice}

La construcción utiliza simetría $S_3$ (grupo simétrico de orden 3) codificada mediante las raíces cúbicas de la unidad $\omega = e^{2\pi i/3}$. Los tres componentes $(P,C,F)$ se disponen con separación angular de $2\pi/3$, formando un triángulo equilátero en el plano complejo---estructura tipo Eisenstein.

Sin embargo, el lattice que el operador genera no es hexagonal sino rectangular:
\[
\Lambda_{\text{PCF}} = \mathbb{Z}M_{\text{PCF}} \oplus \mathbb{Z}(M_{\text{PCF}} \cdot i)
\]

con base $\{M, Mi\}$ y ángulo $90°$---estructura tipo Gauss.

La razón áurea $\varphi$ media entre ambas estructuras: entrada tripartita ($\omega^3 = 1$), salida rectangular ($i^2 = -1$), mediador $\varphi^2 = \varphi + 1$. El operador no elige entre Eisenstein o Gauss---mantiene coherencia entre ambos mediante el invariante $|\Omega| = 1/2$.

\subsubsection{Acoplamiento Fibonacci}

A la simetría geométrica se une un acoplamiento aritmético mediante la razón áurea:
\[
z = \varphi y \quad \text{donde} \quad \varphi = \frac{1+\sqrt{5}}{2}
\]

Este acoplamiento reduce los 3 grados de libertad nominales de la estructura $S_3$ a 2 grados de libertad efectivos, estableciendo una dependencia funcional que elimina un grado de libertad sin destruir la información tripartita. La razón áurea $\varphi$ satisface $\varphi^2 = \varphi + 1$, lo que genera autosimilaridad bajo escalamiento---propiedad esencial para la coherencia multi-escala del operador.

\subsubsection{Matriz Normal}

Con esta estructura, construimos la matriz generadora:
\[
\hat{\Omega} = [\omega_1, \omega_2, \omega_3]^T \quad \text{donde} \quad \omega_j = e^{2\pi i{(j-1)}/3} % chktex 3
\]

Esta matriz es normal pero no hermítica:
\[
\hat{\Omega}\hat{\Omega}^\dagger = \hat{\Omega}^\dagger\hat{\Omega} \quad \text{pero} \quad \hat{\Omega} \neq \hat{\Omega}^\dagger
\]

La normalidad garantiza que $\hat{\Omega}$ es diagonalizable con base ortonormal de eigenvectores\sidenote{\cite{Halmos1958}}. La no-hermiticidad refleja la asimetría geométrica inherente a la estructura tripartita $S_3$: los tres componentes $(P, C, F)$ están relacionados cíclicamente, no simétricamente.

\subsubsection{Kernel Hermítico}

La matriz normal $\hat{\Omega}$ se acopla a un kernel integral $K_{\text{PCF}}$ mediante integración sobre el dominio fundamental del toro:
\[
K_{\text{PCF}}(x, y) = \int_{\mathcal{F}} \hat{\Omega}(x, t) \overline{\hat{\Omega}(y, t)} \, dt
\]

Este kernel es hermítico por construcción:
\[
K_{\text{PCF}}(x, y) = \overline{K_{\text{PCF}}(y, x)}
\]

La hermiticidad emerge del proceso de integración---la simetrización ocurre en el espacio de funciones $L^2$, no en la matriz $\hat{\Omega}$. El operador integral $T_K f(x) = \int K_{\text{PCF}}(x,y) f(y) \, dy$ hereda la hermiticidad del kernel.

\subsubsection{Espectro Real}

Por el teorema espectral para operadores integrales hermíticos compactos\sidenote{\cite{Riesz1955}}, el operador $T_K$ tiene espectro puramente real y discreto:
\[
\text{Spec}(T_K) \subset \mathbb{R}, \quad \text{Spec}(T_K) = {\{\lambda_n\}}_{n=1}^{\infty}
\]

Este es el resultado buscado: partiendo de una estructura tripartita en $\mathbb{C}^3$ (que evita autorreferencia), mediante modularización geométrica y acoplamiento $\varphi$-$i$-$S_3$ (que preserva propiedades de $\mathbb{C}$), llegamos a un operador hermítico con espectro real (que satisface los requisitos espectrales para la Hipótesis de Riemann).

La cadena completa es:
\[
\mathbb{C}^3 \xrightarrow{\text{simetría } S_3} \hat{\Omega} \text{ (normal)} \xrightarrow{\text{acoplamiento } \varphi} \mathbb{C}^2_{\text{eff}} \xrightarrow{\text{modularización}} T^2 \xrightarrow{\text{kernel}} K_{\text{PCF}} \text{ (hermítico)} \xrightarrow{\text{espectral}} \lambda_n \in \mathbb{R}
\]

\subsection{Fundamento y Alcance del Presente Trabajo}

En lugar de construir un operador especificando el espectro de ceros de zeta---ciclo $H \mapsto \text{spec}(H) \mapsto H$---reinterpretamos el plano complejo como espacio modular $M_{\text{PCF}} = \mathbb{C}/\Lambda_{\text{PCF}} \cong T^2$ con lattice $\Lambda_{\text{PCF}}$ determinado por periodicidades geométricas. Esta reinterpretación reduce información infinita a datos finitos aglutinados mediante coherencia estructural\sidenote{La identificación $z \sim z + \lambda$ para $\lambda \in \Lambda_{\text{PCF}}$ forma clases $[z] \in \mathbb{C}/\Lambda_{\text{PCF}}$. La coherencia estructural preserva invariantes (módulo $|z|$, fase $\arg(z)$, estructura algebraica) que reconstruyen propiedades globales desde datos locales finitos, siguiendo el principio donde representaciones equivalentes privilegian aspectos particulares (\cite{Manin2013})---principio formalizado en §\ref{prop:equivalencia-definiciones}.\label{note:aglutinacion-modular}}, evitando el problema de auto-referencia descrito en §\ref{sec:obstaculos} donde $H$ requiere conocer $\text{spec}(H)$ \textit{a priori}.

Integramos los axiomas de $\mathbb{C}$ con principios de autoconsistencia---formalizados en \textit{bootstrap} conforme por Guillarmou \textit{et al.}\sidenote{\cite{Guillarmou2020}} y en \textit{bootstrap} modular por Benjamin-Chang (véase §\ref{subsec:simetrias-dualidades})---mediante coherencia multi-dominio $D_1 \leftrightarrow D_2 \leftrightarrow D_3 \leftrightarrow D_4$ donde la información se distribuye entre múltiples dominios que se determinan mutuamente sin autoobservación directa. Siguiendo el análisis de Yanofsky sobre ciclos prohibidos (véase §\ref{subsec:simetrias-dualidades}), el operador emerge de propiedades geométricas determinadas por la estructura de $\mathbb{C}$ mismo mediante kernel modular $K(z,w)$, no de especificación de un Hamiltoniano microscópico. La emergencia hermítica en espacios adjuntos es consecuencia de esta construcción fundamental, no su punto de partida.

\subsection{Verificación Computacional}

Esta construcción exhibe dos observables clave. Una correspondencia aritmética mediante isomorfismo logarítmico vincula $\sigma \to p_\sigma \to M_p = 2^{p_\sigma}-1$, verificada en 51 primos de Mersenne desde $M_2 = 3$ hasta $M_{82589933}$ con 24.9 millones de dígitos. Por otra parte, se proporciona predicción analítica de ceros de $\zeta(s)$ y análogamente de otras L-funciones. La verificación alcanza precisión de máquina (véase \ref{def:precision-computacional}) para $n \sim 10^{10}$ (altura $t \sim 10^{23}$). % chktex 2

El operador no requiere conocer estos ceros para su construcción y su espectro exhibe correlación con ellos \textit{a posteriori}.

\subsection{Estructura del Presente Trabajo}

Este documento desarrolla la construcción completa del operador $\omegapcf$ y verifica sus propiedades estructurales y numéricas. \autoref{sec:plano-complejo-modulos}\ introduce el plano complejo como espacio de módulos, incluyendo espacios paramétricos adjuntos (\autoref{subsec:espacios-adjuntos}\ ). \autoref{sec:operador-PCF}\ desarrolla el operador $\omegapcf$ mediante construcción axiomática (\autoref{subsec:axiomas}\ ), construcción desde el módulo con ecuaciones de acoplamiento (\autoref{subsec:construccion-modulo}\ ), y geometría asociada incluyendo estructura 3D (\autoref{subsec:geometria-3d}\ ), torre de escalas (\autoref{subsec:spacetime-torre}\ ), espacio-tiempo pentadimensional (\autoref{subsec:spacetime-pentadimensional}\ ) y funcionalización en espacio de Hilbert (\autoref{subsec:funcionalizacion}\ ). La necesidad del toro complejo y estructura tensorial se justifica en \autoref{subsec:toro-lattice}.

Las secciones subsecuentes establecen propiedades espectrales y geométricas. \autoref{convergencia}\ analiza convergencia espectral en espacio de Hilbert. \autoref{invariancia}\ demuestra invariancia modular exacta y su relación con el principio de certidumbre. \autoref{hausdorff}\ estudia la dimensión de Hausdorff de la estructura fractal. \autoref{triple}\ establece coherencia triple en espacios inequivalentes. \autoref{mersenne}\ expone correspondencias aritméticas: números de Mersenne, espiral áurea y estructura logarítmica. Apéndice~\ref{app:ttt}\ contiene la tabla completa de verificaciones computacionales.

