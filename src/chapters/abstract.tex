\begin{abstract}

Mediante principios \textit{bootstrap} de coherencia multi-dominio, trascendemos problemas de autorreferencia tipo Lawvere-Yanofsky que han limitado varios intentos previos de construcción del operador Hilbert-Pólya. En lugar de extender $\mathbb{C}$ mediante álgebra (que enfrenta restricciones del teorema de Frobenius), desarrollamos modularización geométrica que preserva todas las propiedades algebraicas de $\mathbb{C}$ mientras revela estructura toroidal subyacente\sidenote{Mediante lattice $\Lambda_{\text{PCF}}$ y módulo $M_{\text{PCF}} = \mathbb{C}/\Lambda_{\text{PCF}} \cong T^2$, a diferencia de extensiones algebraicas clásicas que añaden nuevos elementos y pueden perder propiedades fundamentales (e.g., octoniones $\mathbb{O}$ pierden asociatividad). Nuestra modularización reorganiza sin añadir, preservando todas las propiedades de $\mathbb{C}$ mediante acoplamiento geométrico $\varphi$-$i$-$S_3$.\label{note:octonions}}. La construcción parte de estructura tripartita $(P,C,F)$ tipo Eisenstein en $\mathbb{C}^3$ mediante simetría $S_3$, genera lattice rectangular tipo Gauss mediante acoplamiento $\varphi$, y produce operador hermítico con espectro real mediante kernel modular sobre el toro. La matriz generadora $\hat{\Omega}$ en $\mathbb{C}^3$ es normal pero no hermítica\sidenote{La matriz $\hat{\Omega}$ opera en espacio de componentes $\mathbb{C}^3$ codificando direccionalidad de la estructura tripartita $(P, C, F)$ mediante simetría $S_3$ del triángulo equilátero; su no-hermiticidad refleja geometría del sistema, no defecto algebraico.}, mientras que la hermiticidad del operador integral en $L^2(\mathbb{R})$ emerge del mecanismo de construcción mediante kernel simetrizado\sidenote{El kernel se construye mediante términos $\delta(x-y) + \varepsilon(x,y)$ que introducen simetrización, permitiendo que hermiticidad emerja aunque $\hat{\Omega}$ no sea hermítica. Ver §\ref{subsubsec:emergencia-hermiticidad}.}, no de propiedades algebraicas de $\hat{\Omega}$. El acoplamiento $\varphi$-$i$-$S_3$ genera esta hermiticidad emergente con magnitud constante $|\Omega| = 1/2$, estableciendo correspondencia entre escalas autosimilares del plano complejo\sidenote{El módulo constante $|\Omega| = 1/2$ actúa como punto fijo funcional que ancla toda la construcción mediante auto-referencia distribuida en estructura tripartita $P \leftrightarrow C \leftrightarrow F$ que evita ciclos prohibidos $D_1 \to D_2 \to D_1$ identificados por Lawvere y Yanofsky. Esta estrategia evita autorreferencia mediante coherencia multi-dominio con invariantes preservados, formalizada en bootstrap conforme por Guillarmou \textit{et al.} y en bootstrap modular por Benjamin-Chang. Ver §\ref{subsec:simetrias-dualidades} y §\ref{sec:obstaculos}.\label{note:distributed-reference}}.

El análisis del operador revela dos correspondencias estructurales fundamentales. Primero, isomorfismo logarítmico entre torre áurea continua $R_\sigma = 3\varphi^\sigma$ y torre Mersenne discreta $M_p = 2^p-1$ mediante factor de conversión $\lambda = \ln(2)/\ln(\varphi) \approx 1.440$, donde ambas torres son rectas en espacio logarítmico con pendientes relacionadas---correspondencia topológica (preserva estructura exponencial) no métrica, verificada sobre más de 25 millones de órdenes de magnitud desde $M_2$ hasta $M_{82589933}$, mediada por módulo crítico $|\Omega| = 1/2 = 2^{-1}$ que establece el único puente posible entre escalamiento áureo y binario. Segundo, predicción espectral de ceros de $\zeta(s)$ mediante fórmula $\lambda_n = K_\sigma \sqrt{t_n}$ donde $K_\sigma = M_{\text{PCF}}/\varphi^\sigma$, con precisión que mejora asintóticamente conforme aumenta altura $t$ (discrepancias $< 10^{-14}$ en primeros 100 ceros, verificada hasta $n \sim 10^{10}$ con altura $t \sim 8.3 \times 10^{23}$). El operador se construye independientemente de $\zeta(s)$; su espectro exhibe correlación estructural con ceros \textit{a posteriori}.

La verificación numérica confirma robustez estructural del operador: discrepancias observadas reflejan límites de precisión computacional (véase \ref{def:precision-computacional}), no deficiencia matemática. El operador mantiene integridad incluso al manipular números enteros de magnitud extrema (primos de Mersenne con millones de dígitos), preservando invariantes bajo la acción del acoplamiento $\varphi$-$i$-$S_3$ a través de todas las escalas autosimilares. Esta robustez—donde discreto y continuo coexisten coherentemente en el espectro—demuestra que el operador captura invariantes matemáticos fundamentales de $\mathbb{C}$.

\textbf{Keywords:} Riemann hypothesis, Hilbert-Pólya conjecture, L-functions, Self-adjoint operators, Random matrix theory, Zeta function zeros, Mersenne primes, Modular spaces.

\end{abstract}