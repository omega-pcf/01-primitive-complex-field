\documentclass{article}
\usepackage{amsmath,amsthm,amssymb}
\usepackage[spanish]{babel}

% Definir entornos de teoremas
\newtheorem{proposition}{Proposición}[section]
\newtheorem{theorem}{Teorema}[section]

% Macros de notación (solo los específicos del documento)
\newcommand{\omegapcf}{\Omega_{\text{PCF}}}
\newcommand{\omegahat}{\hat{\Omega}}

\begin{document}

\section{Ejemplo Comparativo: Pruebas}

\subsection{Forma Recomendada: Solo \texttt{\textbackslash begin\{proof\}...\textbackslash end\{proof\}}}

\begin{proposition}[Invariancia rotacional]\label{prop:invariancia-rotacional}
$|e^{i\theta} z| = |z|$ para todo $\theta \in \mathbb{R}$ y $z \in \mathbb{C}$.
\begin{proof}
Por definición del módulo, tenemos:
\[
|e^{i\theta} z| = |e^{i\theta}| \cdot |z| = 1 \cdot |z| = |z|
\]
donde hemos usado que $|e^{i\theta}| = 1$ para todo $\theta \in \mathbb{R}$.
Por lo tanto, el módulo es invariante bajo rotación.
\end{proof}
\end{proposition}

\subsection{Forma NO Recomendada: \texttt{\textbackslash qed} manual dentro de \texttt{proof}}

\begin{proposition}[Verificación del cilindro]\label{prop:verificacion-cilindro}
Los tres vértices satisfacen la ecuación del cilindro.
\begin{proof}
Verificamos cada vértice:
\begin{align*}
\sqrt{x_P^2 + y_P^2} &= \sqrt{3^2 + 0^2} = 3 \\
\sqrt{x_C^2 + y_C^2} &= \sqrt{{(-1.5)}^2 + {(2.598)}^2} = \sqrt{2.25 + 6.75} = 3 \\
\sqrt{x_F^2 + y_F^2} &= \sqrt{{(-1.5)}^2 + {(-2.598)}^2} = \sqrt{2.25 + 6.75} = 3
\end{align*}
\qed % ← REDUNDANTE: amsthm ya coloca el QED automáticamente
\end{proof}
\end{proposition}

\subsection{Prueba que termina con ecuación display}

\begin{proposition}[Módulo al cuadrado]
Para $z = x + iy \in \mathbb{C}$, se tiene $|z|^2 = z \cdot \bar{z}$.
\begin{proof}
Por definición del módulo y propiedades de la conjugación:
\[
|z|^2 = (x^2 + y^2) = (x + iy)(x - iy) = z \cdot \bar{z}
\]
Por lo tanto, $|z|^2 = z \cdot \bar{z}$.
\end{proof}
\end{proposition}

\subsection{Prueba larga con múltiples pasos}

\begin{theorem}[Estructura isomorfa]
La estructura PCF es isomorfa al plano complejo.
\begin{proof}
La demostración procede en tres pasos.

\textit{Paso 1}: Verificamos que la estructura es cerrada bajo operaciones.
Para $z_1, z_2 \in \mathbb{C}$, tenemos $z_1 + z_2 \in \mathbb{C}$ y $z_1 \cdot z_2 \in \mathbb{C}$ por las propiedades del campo complejo.

\textit{Paso 2}: Mostramos que preserva la métrica.
La distancia euclidiana $|z_1 - z_2|$ es invariante bajo traslación y rotación, lo cual preserva la estructura métrica.

\textit{Paso 3}: Concluimos que es isomorfismo.
La biyección $f: \mathbb{C} \to \mathbb{C}$ definida por $f(z) = z$ es claramente un isomorfismo que preserva todas las estructuras.

Por lo tanto, la estructura es isomorfa al plano complejo.
\end{proof}
\end{theorem}

\end{document}

